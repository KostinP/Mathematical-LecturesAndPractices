\documentclass[main]{subfiles}

\begin{document}
    \subsection{Метод резолюций для исчисления аргументов}

    \begin{enumerate}
        \item Алфавит: алфавит для проп. перем. $\rceil,\ \vee$
        \item Ф-лы: предположения (элементарные дизъюнкции)
        \item Акс: -
        \item Правила резолюции
        %картинка от Ильи со схемой
        %слева $D_1 \vee x \vee D_2$ справа $D_3 \vee \rceil x \vee D_4$, они ведут вниз к $D_1 \vee D_2 \vee D_3 \vee D_4$

    \end{enumerate}
%фффффффффффффффффффффффффффффффффффффффффффффффффффффффффффффффффффффффффффффффффффффффффффффффффффффффффффффффффффффффффффффффффффффффффффффффффффффффффффффффффффффффффффффффффффффффффффффффффффффффффффффффффффффффффффффффффффффффффффффффффффффффффффффффффффффффффффффффффффффффффффффффффффффффффффффффффффффффффффффффффффффффффффффффффффффффффффффффффффффффффффффффффффффффффффффффффффффффффффффффффффффффффффффффффффффффффффффффффффффффффффффффффффффффффффффффффффффффффффффффффффффффффффффффффффффффффффффффффффффффффффффффффффффффффффффффффффффффффффффффффффффффффффффффффффффффффффффффффффффффффффффффффффффффффффффффффффффффффффффффффффффффффффффффффффффффффффффффффффффффффффффффффффффффффффффффффффффффффффффффффффффффффффффффффффффффффффффффффффффффффффффффффффффффффффффффффффффффффффффффффффффффффффффффффффффффффффффффффффффффффффффффффффффффффффффффффффффффффффффффффффффффффффффффффффффффффффффффффффффффффффффффффффффффффффффффффффффффффффффффффффффффффффффффффффффффффффффффффффффффффффффффффффффффффффффффффффффффффффффффффффффффффффффффффффффффффффффффффффффффффффффффффффффффффффффффффффффффффффффффффффффффффффффффффффффффффффффффффффффффффффффффффффффффффффффффффффффффффффффффффффффффффффффффффффффффффффффффффффффффффффффффффффффффффффффффффффффффффффффффффффффффффффффффффффффффффффффффффффффффффффффффффффффффффффффффффффффффффффффффффффффффффффффффффффффффффффффффффффффффффффффффффффффффффффффффффффффффффффффффффффффффффффффффффффффффффффффффффффффффффффффффффффффффффффффффффффффффффффффффффффффффффффффффффффффффффффффффффффффффффффффффффффффффффффффффффффффффффффффффффффффффффффффффффффффффффффффффффффффффффффффффффффффффффффффффффффффффффффффффффффффффффффффффффффффффффффффффффффффффффффффффффффффффффффффффффффффффффффффффффффффффффффффффффффффффффффффффффффффффффффффффффффффффффффффффффффффффффффффффффффффффффффффффффффффффффффффффффффффффффффффффффффффффффффффффффффффффффффффффффффффффффффффффффффффффффффффффффффффффффффффффффффффффффффффффффффффффффффффффффффффффффффффффффффффффффффффффффффффффффффффффффффффффффффффффффффффффффффффффффффффффффффффффффффффффффффффффффффффффффффффффффффффффффффффффффффффффффффффффффффффффффффффффффффффффффффффффффффффффффффффффффффффффффффффффффффффффффффффффффффффффффффффффффффффффффффффффффффффффффффффффффффффффффффффффффффффффффффффффффффффффффффффффффффффффффффффффффффффффффффффффффффффффффффффффффффффффффффффффффффффффффффффффффффффффффффффффффффффффффффффффффффффффффффффффффффффффффффффффффффффффффффффффффффффффффффффффффффффффффффффффффффффффффффффффффффффффффффффффффффффффффффффффффффффффффффффффффффффффффффффффффффффффффффффффффффффффффффффффффффффффффффффффффффффффффффффффффффффффффффффффффффффффффффффффффффффффффффффффффффффффффффффффффффффффффффффффффффффффффффффффффффффффффффффффффффффффффффффффффффффффффффффффффффффффффффффффффффффффффффффффффффффффффффффффффффффффффффффффффффффффффффффффффффффффффффффффффффффффффффффффффффффффффффффффффффффффффффффффффффффффффффффффффффффффффффффффффффффффффффффффффффффффффффффффффффффффффффффффффффффффффффффффффффффффффффффффффффффффффффффффффффффффффффффффффффффффффффффффффффффффффффффффффффффффффффффффффффффффффффффффффффффффффффффффффффффффффффффффффффффффффффффффффффффффффффффффффффффффффффффффффффффффффффффффффффффффффффффффффффффффффффффффффффффффффффффффффффффффффффффффффффффффффффффффффффффффффффффффффффффффффффффффффффффффффффффффффффффффффффффффффффффффффффффффффффффффффффффффффффффффффффффффффффффффффффффффффффффффффффффффффффффффффффффффффффффффффффффффффффффффффффффффффффффффффффффффффффффффффффффффф
    \begin{definition}
        Множество предположений называется {\bf неудовлетворимым}, если из него выводимо пустое предложение nill
        \[\us{\text{мн-во}}\nothing, \q \us{\text{слово}}{\wedge},\q \us{\text{предл.}}{nill}\] %не ведге, а скорее _ведге_ внешне
    \end{definition}

    \begin{utv}
        Множество предложений неудовлетворимо тогда и только тогда, когда их конъюнкция является противоречеием
        %картинка от Ильи
        %В листах деревьев D_1,... они все ведут к nill
    \end{utv}

    Обоснование дерева
    \[A_1... A_m \RA B_1...B_k\]
    \[(*) \LRA A_1 & ... & A_m \ra B_1, ..., B_k \text{ --- тавт.} \LRA\]
    \[\LRA \rceil A_1 \vee ... \vee \rceil A_m \vee B_1 \vee ... B_k \text{ --- тавт.} \LRA\]
    \[\LRA A_1 \& ... \& A_m \& \rceil B_1 \& ... \& \rceil B_k \text{ --- противор.} \os{C_p \text{ --- КНФ для P}}{\LRA}\]
    \[\LRA C_{A_1} \& ... \& C_{A_m} \& C_{\rceil B_n} \& ... \& C_{\rceil B_k} \text{ --- против.}\]
    Т.о. для того чтобы д-ть (*) достаточно:
    \begin{enumerate}
        \item Каждую из ф-л $A_1, ..., A_m,\q \rceil B_1,..., \rceil B_k$\\
        привести к КНФ и удалить $\&$
        \item Док-ть выводимость nill из полученного мн-ва предложений
    \end{enumerate}

    \subsection{Формулы исчисления предикатов}

\end{document}
