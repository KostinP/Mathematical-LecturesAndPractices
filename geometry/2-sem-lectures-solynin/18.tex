\documentclass[geometry.tex]{subfiles}

\begin{document}
  \section{Компоненты Связности.}

  \begin{definition}
      X - топ. пр-во\\
      Компонентой связности т. $x_0 \in X$ назыв. наиб. по включению
      связное множество, ее содерж.\\
      %\[K_{x_0} = \cup \{M \in 2^X  \mid x_0 \in M \text{ - связ.}\}\]
  \end{definition}

  \begin{definition}[другое определение]
      A - компонента связности $\lra$
      \begin{enumerate}
        \item A связно
        \item $\forall B \us{\neq}{\supset} A \Ra B$ - несвязно
      \end{enumerate}
  \end{definition}

  \begin{example}
      *здесь когда-нибудь будет пример*
  \end{example}

  \begin{consequence}
    Компоненты связности могут не быть открытыми
  \end{consequence}

  \begin{theorem}
      \begin{enumerate}
          \item $\forall x, y \in X \q K_x = K_y$ или $K_x \cap K_y = \varnothing$
          \item компоненты связности - замк.
  %        \item Для любого связ. мн-ва $\exists$ компонента связности, в которой оно
  %        целиком содержится\\
  %        $\forall M \subseteq X \ (M - \text{связ.} \ra \exists x \in X: M \subseteq K_x)$
  %        \item $\forall x, y, z \in X \ (x, y \in K_z \rla \exists M \text{ - связ.}:
  %        x, y \in M \text{ и } z \in M)$
      \end{enumerate}
  \end{theorem}

  \begin{proof}
      *здесь когда-нибудь будет док-во*
  \end{proof}

  \begin{ddefinition}
      X - топ. пр-во назыв. вполне несвязным, если $\forall x \in X: K_x = \{x\}$
  \end{ddefinition}
\end{document}
