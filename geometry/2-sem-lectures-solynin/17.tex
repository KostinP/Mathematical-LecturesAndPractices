\documentclass[geometry.tex]{subfiles}

\begin{document}
  \section{Связность произведения пространств}

  \begin{ttheorem}
      $\{X_i\}_{i \in I}$ - топ. пр-во
      \[\Ra \forall i \q X_i \text{ - св. } \lra \prod_{i \in I} X_i \text{ - связ.}\]
  \end{ttheorem}

  \begin{theorem}
      X, Y - топ. пр-ва\\
      \[X \times Y \text{ - связн. $\rla$ X, Y - связн.}\]
  \end{theorem}

  \begin{remark}
      Любое конечное произведение связных топ. пр-в связно
  \end{remark}

  \begin{proof}
      ($\Rightarrow$):
\[p: X \times Y \ra X \text{ - непр., сюръект.}\]
\[(x,\ y) \ra x\]
\[X \times Y \text{ - св. } p(X \times Y) = X \text{ - св. (из теоремы в прошлом билете)}\]
      ($\Leftarrow$):
\[\letus X,\ Y \text{ - связны, } X \times Y \text{ - несв.}\]
\[\text{По следствию из прошлого билета } \exist \text{непр. } f: X \times Y \ra \{0,\ 1\}\]
\[f(x_0,\ y_0) = 0\]
\[f(x_1,\ y_1) = 1\]
\[f(x_0, y_1) = ? \text{. Разберём случай если 0 (1 аналогично)}\]
\[\text{Рассмотрим }i: X \ra X \times Y \qq i(x) = (x,\ y_1)\]
\[\text{$i$ - непр. Таких отображений много}\]

\[\text{Почему непрерывно?}\]
\[U \subset X, \qq V \subset Y \text{ - откр.}\]
\[i^{-1}(\bigcup U_j \times V_j) = \bigcup i^{-1} (U_j \times V_j)\]
\[f \circ i: x \os{i}{\ra} X \times Y \os{f}{\ra} \{0.\ 1\}\]
\[x_0 \ra (x_0,\ y_1) \ra 0\]
\[x_1 \ra (x_0,\ y_1) \ra 1\]
\[text{Сюръект. непр., но $X$ - св. по усл. Противоречие.}\]
  \end{proof}
\end{document}
