\documentclass[geometry.tex]{subfiles}

\begin{document}
  \section{Полнота и вполне ограниченность метрических пространств.}

  \begin{definition}
      Фунд. послед.\\
      $\{X_n\}$ - фунд., если $\forall \mathcal{E} > 0 \q \exists N: \forall n, m > N: \rho(X_n, X_m) < \mathcal{E}$
  \end{definition}

  \begin{definition}
      X назыв. полным, если $\forall$ фунд. послед. сходится
  \end{definition}

  \begin{example}
      *здесь когда-нибудь будет пример*
  \end{example}

  \begin{definition}
      $\{X_i\}_{i \in I}$ - $\mathcal{E}$-сеть, если $\forall x \q \exists x_i: \rho(x, x_i) < \mathcal{E}$
  \end{definition}

  \begin{definition}
      X назыв. вполне огранич., если $\forall \mathcal{E} > 0 \q \exists$ конечная $\mathcal{E}$-сеть
  \end{definition}

  \begin{theorem}
      $X$ - метр. $\Ra$ сепар. $\lra$ $\RNUmb{2}$ А.С.
  \end{theorem}

  \begin{proof}
      *здесь когда-нибудь будет док-во*
  \end{proof}
\end{document}
