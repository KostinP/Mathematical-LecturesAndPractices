\documentclass[geometry.tex]{subfiles}

\begin{document}
  \section{Гомеоморфизм.}

  \begin{definition}
      $f: X \rightarrow Y$ - гомеоморфизм $(X \simeq Y)$, если:
      \begin{enumerate}
          \item f - непр.
          \item f - биекция
          \item $f^{-1}$ - непр.
      \end{enumerate}
      $X$ и $Y$ называются гомеоморфными
  \end{definition}

  \begin{examples}
    \begin{enumerate}
      \item $(-\dfrac{\pi}{2};\ \dfrac{\pi}{2}) \simeq \R \qq (f(x) = \tg x)$
      \item Не гомеоморфизм (1 и 2 есть, 3 нет):
      \[[0,\ 2\pi) \os{f}{\ra} S' = \{z \in \sigma \ |\ |z| = 1\}\]
      \[f(t) = e^{it} = \cos t + i \sin t,\q f \text{ - непр. и биект.}\]\
      \item Ещё контрпример:
      \[f: (\R, \text{ дискр.}) \ra (\R, \text{ обычн.})\]
      \[f(x) = x \text{ - непр., тк. . из дискретной, биективна - очевидно}\]
      \[f: (X,\ \Omega_1) \ra (X,\ \Omega_2) \qq f^{-1} \text{ - разрывна}\]
      \[\text{непр.} \lra \Omega_1 \supset \Omega_2\]
    \end{enumerate}
  \end{examples}

  \begin{hypothesis}
      $\simeq$ - отношение эквив.
  \end{hypothesis}

  \begin{theorem}
      Если $(X, \Omega_X) \simeq (Y, \Omega_Y)$, то:\\
      $f_*: \Omega_X \rightarrow \Omega_Y$ - биекция, $f_*(U) = f(U)$
  \end{theorem}

  \begin{proof}
    Почему $f_{*}(U) \in \Omega_Y?$\\
    Образ открытого - открыт?\\
    $g = f^{-1}$ ($f$ - гомеом. $\Rightarrow \exist g$)\\
    $g^{-1}(U)$ - прообраз $U$ откр., т.к. $g$ - непр.\\
    Почему биекция?\\
    $g_{*}: \Omega_Y \ra \Omega_X$\\
    $f_{*}$ и $g_{*}$ - вз. обр.
  \end{proof}
\end{document}
