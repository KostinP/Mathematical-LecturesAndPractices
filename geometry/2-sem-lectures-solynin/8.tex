\documentclass[geometry.tex]{subfiles}

\begin{document}
  \section{Равносильные определения непрерывности.}

  \begin{definition}
      $(X, \rho); \q (Y, d)$ - метр. пр-ва $\q f: X \rightarrow Y$\\
      f - назыв. непрерывной в т. $x_0$, если:
      \[\forall \mathcal{E} > 0 \q \exists \  \delta > 0 :
      \text{если } \rho(x, x_0) < \delta \Ra d(f(x), f(x_0)) < \mathcal{E}\]
      f - непрерывна, если она непр. в каждой точке
  \end{definition}

  \begin{theorem}
      f - непр. в $x_0 \rla$
      \[\forall U - \text{откр.} \subset Y: U \ni f(x_0) \q \exists V \subset X - \text{откр.}:  \q x_0 \in V \text{ и } f(V) \subset U\]
  \end{theorem}

  \begin{proof}
      f - непр. в $x_0$
      \[\Ra \forall \mathcal{E} > 0 \q \exists \delta > 0: f(B(x_0, \delta)) \subset B(f(x_0), \mathcal{E})\]
      \[\Ra \forall U -$ откр. $\subset Y: \q f(x_0) \in U \Ra \exists \mathcal{E} > 0:\]
      \[f(x_0) \in B(f(x_0), \mathcal{E}) \subset U \Ra \exists \delta > 0:\]
      \[f(B(x_0, \delta)) \subset B(f(x_0), \mathcal{E}) \subset U \q B(x_0, \delta) = V\]
      $\la \forall$ обрывается
  \end{proof}

  \begin{definition}
    $X, Y$ - топологические пр-ва, $x_0 \in X$, $f: X \ra Y$
    \[\text{f назыв. непр. в т. $x_0$, если $\forall$откр. $U \ni f(x_0)$:}\]
    \[\e \text{откр. $V$: } x_0 \in V \text{ и } f(V) \subset U\]
  \end{definition}

  \begin{theorem}
    $X,Y$ - метрич. (тополог.), $f: X \ra Y$. f - непр $\lra$
    \[\forall U \text{откр. в }Y \q \us{\text{откр. в X}}{f^{-1}(U)} = \{x: f(x) \in U\}\]
  \end{theorem}

  \begin{proof}
    $\Rightarrow:$
	\[\forall U \subset Y \text{ - откр, } \forall x_0 \in f^{-1} (U)\]
	\[\text{f - непр.}\Rightarrow f(x_0) \in  U \Rightarrow \extst V_{x_0} \text{ - откр.: } x_0 \in V_{x_0} \text{ и } f(V_{x_0}) \subset U\]
	\[\bigcup V_{x_0} = f^{-1} (U)\]
	\[x_0 \in f^{-1} (U) \text{ - откр. мн-во}\]
	\[\subset: f(\bigcup V_{x_0}) = \bigcup f(V_{x_0}) \subset U\]
	\[x \in f^{-1} (U)\]
	\[\supset: \forall x_0 \in f^{-1}(U) \qq x_0 \in V_{x_0} \qq x_0 \in \text{объединению}\]
    $\Leftarrow:$
	\[\forall x_0 \qq \forall U \subset Y: f(x_0) \in U\]
	\[\text{Хотим построить } x_0 \in V \qq f(V) \subset U\]
	\[V := f^{-1} (U) \text{ - откр.}\]
	\[x_0 \in f^{-1} (U) = V\]
	\[f(f^{-1}(U)) \subset U\]
  \end{proof}
	$NB: f(f^{-1}(A)) \subset A \qq \qq f^{-1}(f(B)) \supset A$
	

  \begin{eexample}
    $\R$\\
    $\Omega_1$ - антидискр. \\
    $\Omega_2$ - топология конеч. дополнений \\
    $\Omega_3$ - стандартная. \\
    $\Omega_4$ - дискретная. \\
    $f(x) = x$ \\
    $f$ - непр., если $i > j$ \\
    Отдельная ситуация: $X, \ra (Y, \text{ антидискр.}) \Rightarrow f$ - непр.\\
    $f: (X, \text{ дискр.}) \ra (Y, \text{ какая угодно}) \Rightarrow f$ - непр. \\
    $\Omega_4$ - дискретная. \\
  \end{eexample}
\end{document}
