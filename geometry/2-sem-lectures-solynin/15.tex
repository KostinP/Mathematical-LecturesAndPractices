\documentclass[geometry.tex]{subfiles}

\begin{document}
  \section{Связность замыкания. Связность объединения.}

  \begin{theorem}
      $(X, \Omega)$ - топ. пр-во, $A \subset X$ - связно
      \[\Ra \forall B: \q A \subset B \subset \Cl A \q \Ra B \text{ - связно}\]
  \end{theorem}

  \begin{consequence}
      Если A - связ., то ClA - связ.
  \end{consequence}

  \begin{proof}
    Допустим, $B$ - несв. $\Rightarrow \exist U_1,\ U_2:$\\
    \begin{enumerate}
          \item $B \subset U_1 \cup U_2$
          \item $\varnothing = U_1 \cap U_2$
          \item $U_1 \cap  B \neq \varnothing, \qq U_2 \cap B \neq \varnothing$
    \end{enumerate}\\
    $A \subset B \subset U_1 \subset U_2 \lra A \subset U_1 \subset U_2$\\
    $U_1 \cap U_2 \cap A = \varnothing$\\
    Меняем условия над меньшим мн-вом\\
    $A$ - связно $\Rightarrow U_1 \cap A = \varnothing$ (иначе меняем $U_1$ и $U_2$)\\
Нужно использовать, что $B$ достаточно мало, $B \in \Cl A$\\
$F_1 := X \setminus U_1$ - замкн. $F_1 \supset A$\\
$\Rightarrow F_1 \supset \Cl A \lra U_1 \cap \Cl A = \varnothing$\\
$U_1 \cap B \neq \varnothing$??\\
$B \cap \Cl A$?? 
  \end{proof}

  \begin{theorem}
      $(X, \Omega)$ - топ. пр-во, $A, B \subset X$ - связны,
      \[A \cap B \neq \varnothing \Ra A \cup B \text{ - связно}\]
  \end{theorem}

  \begin{proof}
    Допустим, $\exist U_1, U_2$ - откр. в $X$
    \begin{enumerate}
          \item $U_1 \cup U_2 \supset A \cup B$
          \item $U_1 \cap U_2 \cap (A \cup B) = \varnothing$
          \item $U_1 \cap   U_2 \cap (A \cup B) = \varnothing$
    \end{enumerate}\\
$U_1 \cap (A \cup B) \neq \varnothing$\\
$U_2 \cap (A \cup B) \neq \varnothing$\\

\begin{enumerate}
          \item $U_1 \cup U_2 \supset A$ (из 1)\\
$U_1 \cap U_2 \cap A = \varnothing$ (из 2)\\
$\Rightarrow U_1 \cap A = \varnothing$\\
$x_0 \in A \cap B \qq x_0 \notin U_1 \qq x_0 \in U_2$
          \item $U_1 \cup U_2\supset B$ (из 1)\\
$U_1 \cap U_2 \cap B = \varnothing$ (из 2)\\
$x_0 \in  U_2 \cap B$\\
$\Rightarrow  U_1 \vap B = \varnothing$
    \end{enumerate}\\
Противоречие
  \end{proof}
\end{document}
