\documentclass[geometry.tex]{subfiles}

\begin{document}
  \section{Топологические пространства. Примеры.}

  \begin{definition}
      X - мн-во\\
      $\Omega \subset 2^X = \{A \subset X\}$ - мн-во подмн-в X\\
      $(X, \Omega)$ - назыв. топологическим пр-вом, если:
      \begin{enumerate}
          \item $\forall \{U_i\}_{i \in I} \in \Omega \Rightarrow \us{i \in I}{\cup} U_i \in \Omega$
          \item $U_1, U_2, ..., U_n \Rightarrow U_1 \cap U_2 \cap ... \cap U_n \in \Omega$
          \item $\varnothing; \  X \in \Omega$\\\\
          $\Omega$ - топология (топологическая структура) на X\\
          $U \in \Omega$ - называется открытым мн-вом
      \end{enumerate}
  \end{definition}

  \begin{definition}
      $(X, \Omega)$ - топ. пр-во; $F \subset X$ \\
      F - называется замкнутым, если $X \setminus F \in \Omega$
  \end{definition}

  \begin{theorem}
      \begin{enumerate}
          \item $\us{i \in I}{\cap} F_i \text{ - замкн., если } F_i - \text{замкн.}$
          \item $F_1 \cup F_2$ - замкн., если $F_1, F_2$ - замкн.
          \item $\varnothing, X$ - замкн.
      \end{enumerate}
  \end{theorem}

  \begin{examples}
      \begin{enumerate}
          \item $(X, \rho)$ - топ. пр-во
	\item $(\R, \Omega)$, где $\Omega)$ - совокупность объединений всевозможных семейств открытых интервалов. Это пространство называется обычно вещественной прямой, а топологическую структуру называют канонической или стандартной топологией в $\R$
          \item Дискретное пр-во: $\Omega = 2^X$\\
          Нетрудно заметить, что все его элементы открыты по определению (можно сравнить с мешком гороха, где каждая горошина сама по себе). Также они замкнуты
          \item Антидискретное пр-во: $\Omega = \{\varnothing, X\}$\\
          (можно сравнить с запутанным клубком ниток)\\
          Замкнуты только $x$ и $\varnothing$

      \begin{definition}
          $(X, \Omega)$ - метризуемо, если $\exists$ метрика $\rho: X \times X \rightarrow \R_X$\\
          $\Omega = $ мн-во откр. подмн. в $\rho$\\
          Антидискретное - не метризуемо, если |X| > 1
      \end{definition}
          \item Стрелка\\
                $X = \R  \ $ или $\   \R_+ = \{x \geq 0\}$\\
                $\Omega = \{(a, +\infty)\} \cup \{\varnothing\} \cup \{X\}$
          \item Связное двоеточие\\
                $X = \{a, b\}$\\
                $\Omega = \{\varnothing, X, \{a\}\}$
          \item Топология конечных дополнений (Зариского)\\
                X - беск. мн-во\\
                Замкнутые конечные мн-ва и X \\
                $\Omega = \{A \  | \  X \setminus A \text{ конечно}\}$
          %здесь пошла необязательная часть билета
          \begin{uutv}
            Вариации топологии Зарицкого:
            \begin{enumerate}
              \item $\CC^n = \{(z_1,...,z_n)\ |\ z_i \in CC\}$\\
              $F \subset \CC^n$ - замкн., если $F$ является мн-вом решений системы:\\ \ \\
              $\begin{cases}
                f_1(z_1,...,z_n) = 0\\
                f_2(z_1,...,z_n) = 0\\
                ...\\
                f_k(z_1,...,z_n) = 0
              \end{cases}$\\
              $f_1,...,f_k$ - мн-ны от $n$ переменных
              \[\ub{f}{\frac{x^2}{a^2} + \frac{y^2}{b^2} - 1} = 0 \text{ - эллипс}\]
              \[\frac{x^2}{a^2} + \frac{y^2}{b^2} + 1 = 0 \text{ - в $\CC$ непусто, поэтому используем их}\]
              Любое пересечение замкнутых замкнуто?\\ \ \\
              $F \longleftrightarrow \begin{cases}
              f_1(z_1,...,z_n) = 0\\
              f_2(z_1,...,z_n) = 0\\
              ...\\
              f_k(z_1,...,z_n) = 0
            \end{cases} \qq
            G \longleftrightarrow \begin{cases}
              g_1(z_1,...,z_n) = 0\\
              g_2(z_1,...,z_n) = 0\\
              ...\\
              g_k(z_1,...,z_n) = 0
              \end{cases}$\\
              $F \cup G \longleftrightarrow \begin{cases}
              f_1(z_1,...,z_n) = 0\\
              f_2(z_1,...,z_n) = 0\\
              ...\\
              f_k(z_1,...,z_n) = 0\\
              g_1(z_1,...,z_n) = 0\\
              g_2(z_1,...,z_n) = 0\\
              ...\\
              g_k(z_1,...,z_n) = 0
              \end{cases}$\\

              \begin{ttheorem}[Гильберта о базисе]
                Мн-во решений бесконечной системы равносильно мн-ву решений конечной системы
              \end{ttheorem}
              Система ур-ий может быть неудобно
              $(**)\begin{cases}
		(*) \begin{cases}
		x^2 + y^2 + z^2 - 1 = 0 \text{ - сфера}\\
              	x + y + z = 0  \text{ - пл-ть}\\
            	\end{cases}\\
		x^2 + y^2 + z^2 + x + y + z - 1 = 0
	\end{cases}$
	Решение (**) не отличимо от реш. (*). Можем добавить ещё $(x + y + z) (x^2 - 1) = 0$ и беск. много ур-ий не влияющих на решения\\ \ \\
	Наша система эквивалентна. Мн-во мн-ов, $F$ - мн-во решений $\forall$ из мн-ов. Получили идеал
              \begin{ttheorem}[Гильберта]
                Любой идеал можно представить как конечную систему мн-ов
              \end{ttheorem}
              $F \cap G \cap ...$ беск. пересеч. C-м (возможно беск. с-ма ур-ий) представляется как идеал, а он представл. как кон. система многочленов \\ \ \\
	$F \cup G ?$\\ Если  $f_1=0$, то $\exist G$; если  $g_1=0$, то $\exist F$ \\
	$ F \cup G  \longleftrightarrow\begin{cases}
		f_1 * g_1 = 0\\
		f_1 * g_2 = 0\\
		...\\
		f_1 * g_l = 0\\
		f_2 * g_1 = 0\\
		...\\
		f_k * g_l = 0\\
            	\end{cases} $\\
	(или как-то по-другому, но похоже)\\
	$x^n + y^n - z^n = 0$ над $\Q$\\
	$(\frac{a}{b})^n + (\frac{c}{d})^n + (\frac{e}{f})^n = 0$ домн. на знам-ли\\
	$(adf)^ n + (cbf)^n = (ebd)^n$ почти Великая теорема Ферма, но у неё $\N$, а у нас $\Z$. Хочется ввести стр-ру $K^n$ ($K$ - поле), описывающую с-му ура-ий. Уже ввели её только что
              \begin{ttheorem}[Гильберта о нулях]
                $K$ - алгебраически замкнутое поле $\Ra$ замкнутые мн-ва в $K^n$ - идеалы в $K[x_1,...,x_n]$ - биекция
              \end{ttheorem}
	Зачем нужна алг. замкнутость? Если не алг. замкнуто: $K \in \R \qq I_1 = \R [x_1, ..., x_n]$ - множество корней пусто.\\
	$I_2 = \{ f: x^2 + y^2 + 1 \}$ - мн-во корней пусто, ему соотв. $\empty$ биекция
            \end{enumerate}
          \end{uutv}
      \end{enumerate}
  \end{examples}
\end{document}
