\documentclass[geometry.tex]{subfiles}

\begin{document}
  \section{Топологические пространства. Примеры.}

  \begin{definition}
      X - мн-во\\
      $\Omega \subset 2^X = \{A \subset X\}$ - мн-во подмн-в X\\
      $(X, \Omega)$ - назыв. топологическим пр-вом, если:
      \begin{enumerate}
          \item $\forall \{U_i\}_{i \in I} \in \Omega \Rightarrow \us{i \in I}{\cup} U_i \in \Omega$
          \item $U_1, U_2, ..., U_n \Rightarrow U_1 \cap U_2 \cap ... \cap U_n \in \Omega$
          \item $\varnothing; \  X \in \Omega$\\\\
          $\Omega$ - топология на X\\
          $U \in \Omega$ - называется открытым мн-вом
      \end{enumerate}
  \end{definition}

  \begin{definition}
      $(X, \Omega)$ - топ. пр-во; $F \subset X$ \\
      F - называется замкнутым, если $X \setminus F \in \Omega$
  \end{definition}

  \begin{theorem}
      \begin{enumerate}
          \item $\us{i \in I}{\cap} F_i \text{ - замкн., если } F_i - \text{замкн.}$
          \item $F_1 \cup F_2$ - замкн., если $F_1, F_2$ - замкн.
          \item $\varnothing, X$ - замкн.
      \end{enumerate}
  \end{theorem}

  \begin{examples}
      \begin{enumerate}
          \item $(X, \rho)$ - топ. пр-во
          \item Дискр. пр-во: $\Omega = 2^X$\\
          Нетрудно заметить, что все его элементы открыты по определению (можно сравнить с мешком гороха, где каждая горошина сама по себе). Также они замкнуты
          \item Антидискр. пр-во: $\Omega = \{\varnothing, X\}$\\
          (можно сравнить с запутанным клубком ниток)\\
          Замкнуты только $x$ и $\varnothing$

      \begin{definition}
          $(X, \Omega)$ - метризуемо, если $\exists$ метрика $\rho: X \times X \rightarrow \R_X$\\
          $\Omega = $ мн-во откр. подмн. в $\rho$\\
          Антидискретное - не метризуемо, если |X| > 1
      \end{definition}
          \item Стрелка\\
                $X = \R  \ $ или $\   \R_+ = \{x \geq 0\}$\\
                $\Omega = \{(a, +\infty)\} \cup \{\varnothing\} \cup \{X\}$
          \item Связное двоеточие\\
                $X = \{a, b\}$\\
                $\Omega = \{\varnothing, X, \{a\}\}$
          \item Топология конечных дополнений (Зариского)\\
                X - беск. мн-во\\
                Замкнутые конечные мн-ва и X \\
                $\Omega = \{A \  | \  X \setminus A \text{ конечно}\}$
          %здесь пошла необязательная часть билета
          \begin{uutv}
            Вариации топологии Зарицкого:
            \begin{enumerate}
              \item $\CC^n = \{(z_1,...,z_n)\ |\ z_i \in CC\}$\\
              $F \subset \CC^n$ - замкн., если $F$ является мн-вом решений системы:\\ \ \\
              $\begin{cases}
                f_1(z_1,...,z_n) = 0\\
                f_2(z_1,...,z_n) = 0\\
                ...\\
                f_k(z_1,...,z_n) = 0
              \end{cases}$\\
              $f_1,...,f_k$ - мн-ны от $n$ переменных
              \[\ub{f}{\frac{x^2}{a^2} + \frac{y^2}{b^2} - 1} = 0 \text{ - эллипс}\]
              \[\frac{x^2}{a^2} + \frac{y^2}{b^2} + 1 = 0 \text{ - в $\CC$ непусто, поэтому используем их}\]
              Любое пересечение замкнутых замкнуто?\\ \ \\
              $F \longleftrightarrow \begin{cases}
              f_1(z_1,...,z_n) = 0\\
              f_2(z_1,...,z_n) = 0\\
              ...\\
              f_k(z_1,...,z_n) = 0
            \end{cases} \qq
            G \longleftrightarrow \begin{cases}
              g_1(z_1,...,z_n) = 0\\
              g_2(z_1,...,z_n) = 0\\
              ...\\
              g_k(z_1,...,z_n) = 0
              \end{cases}$\\
              $F \cup G \longleftrightarrow \begin{cases}
              f_1(z_1,...,z_n) = 0\\
              f_2(z_1,...,z_n) = 0\\
              ...\\
              f_k(z_1,...,z_n) = 0\\
              g_1(z_1,...,z_n) = 0\\
              g_2(z_1,...,z_n) = 0\\
              ...\\
              g_k(z_1,...,z_n) = 0
              \end{cases}$\\

              \begin{ttheorem}[Гильберта о базисе]
                Мн-во решений бесконечной системы равносильно мн-ву решений конечной системы
              \end{ttheorem}
              *здесь когда-нибудь возможно будет алгебраическая формулировка с примером*
              \begin{ttheorem}[Гильберта]
                Любой идеал можно представить как конечную систему мн-ов
              \end{ttheorem}
              *здесь когда-нибудь возможно будет дополнение*
              \begin{ttheorem}[Гильберта о нулях]
                $K$ - алгебраически замкнутое поле $\Ra$ замкнутые мн-ва в $K^n$ - идеалы в $K[x_1,...,x_n]$ - биекция
              \end{ttheorem}
            \end{enumerate}
          \end{uutv}
      \end{enumerate}
  \end{examples}
\end{document}
