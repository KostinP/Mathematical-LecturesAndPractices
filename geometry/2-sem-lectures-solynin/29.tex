\documentclass[geometry.tex]{subfiles}

\begin{document}
  \section{Первая аксиома счётности.}

  \begin{definition}
      База окр-тей точки:\\
      $\forall x \q \exists \{U_{x_i}\}_{i \in I_x}$
      \begin{enumerate}
          \item $U_{x_i} \in \Omega; \q x \in U_{x_i}$
          \item $\forall U \in \Omega \ : \ x \in U \q \exists U_{x_i} \ : \ x \in U_{x_i} \subset U$
      \end{enumerate}
  \end{definition}

  \begin{remark}
      Если выделить мн-во всех окрестностей точки, то это будет база топологии
  \end{remark}

  \begin{definition}
      Если $\exists$  база окр-тей:\\
      $\forall x \ \{U_{x_i}\}_{i \in I_x}$ не более чем счетное $\Ra$ X удовл. I А.С.
  \end{definition}

  \begin{examples}
      *здесь когда-нибудь будут примеры*
  \end{examples}
\end{document}
