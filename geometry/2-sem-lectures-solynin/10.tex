\documentclass[geometry.tex]{subfiles}

\begin{document}
  \section{Инициальная топология. Топология произведения как инициальная.}

  \begin{definition}
      $\forall i \in I \q f_i: X \rightarrow Y_i$\\
      $(Y_i, \Omega_i)$ - топ. пр-во
      \[\{f^{-1}_{i1} (U_1) \cap f^{-1}_{i2}(U_2) \cap ... \cap f^{-1}_{ik}(U_k) \ |\
      \us{j = 1, ..., k \in \N}{U_j \in \Omega_{ij}}\} \text{ - база нек. топологии} \]
      $\Omega_X$ - соотв. топология (инициальная топология)
  \end{definition}

  \begin{definition}
      $\{f_i^{-1}(U)\}$ - предбаза топологии
  \end{definition}

  \begin{example}
    $(X,\ \Omega_X),\ (Y,\ \Omega_Y)$ - топ. пр-ва\\
    Введём топологию на $X \times Y$:\\
    База $\mathcal{B} = \{ U \times V | U \in \Omega_X, \ V \in \Omega_Y \}$\\
    $(U_1 \times V_1) \supset (U_2 \times V_2) = (U_1 \times U_2) \times (V_1 \supset V_2)$ ф-ла для проверки\\
    По-другому:\\
    $(Y,\ \Omega_Y) \overset{\la}{P_Y} X \times Y \overset{\ra}{P_x} (X,\ \Omega_X) $\\
    &P_X(X, Y) = X& - проекции $X \times Y$ на $X$ и $Y$\\
    $P_Y (X, Y) = Y$
  \end{example}

  \begin{theorem}
      Топология произведения совпадает с инициальной топологией
  \end{theorem}

  \begin{proof}
    Докажем, что $f^{-1}_{i_1} (U_{1}) \supset f_{i_2}^{-1} (U_2)$ и $(U_1 \times V_1) \supset (U_2 \times V_2) = (U_1 \times U_2) \times (V_1 \supset V_2)$ - одно и то же\\ 
    $U \in \Omega_X \qq V \in \Omega_Y$\\
    $P_X^{-1} (U) = U \times Y$\\
    $P_Y^{-1}(V) = X \times V$\\
    $U \times Y \supset X \times V = U \times V$\\
    $U_1,\ U_2 \in \Omega_Y$\\
    $P_X^{-1} (U_1) \supset P_X^{-1}(U_2) = P_X^{-1} (U_1 \supset U_2)$\\
    Итак, база состоит из $f^{-1}_{i_1} (U_{1}) \supset f_{i_2}^{-1} (U_2)$ и $(U_1 \times V_1) \supset (U_2 \times V_2) = (U_1 \times U_2) \times (V_1 \supset V_2) \Rightarrow$ топологии совпадают
  \end{proof}

  \begin{Definition}
      \[\prod_{i \in I} x_i = \{f: I \rightarrow \bigcup_{i \in I} x_i \ | \ f(i) \in X_i \}\]
      \[p_k : \prod_{i \in I} x_i \rightarrow X_k \q k \in I\]
      \[p_k(f) = f(k)\]
      \[\Ra \text{если } X_i \text{- топ.} \ra \prod_{i \in I} X_i - \text{топ.}\]
      Таким образом введём инициальную топологию\\ \ \\
      Топология на $X_1 \times X_2 \times ...$\\
      Со счётным понятнее, хотя бы ясно из чего состоит (из последовательностей)\\
      База: $\{U_1 \times U_2 \times ... \times U_k \times X_{K+1} \times  X_{k+2} \times ...\}$\\
      Начиная с некоторого места открытые мн-ва замыкаются на всё пространство. Почему так?\\
      $U \supset X \times Y \times Z \qq P_X^{-1} (U) = U \times Y \times Z$\\
      В случае большого к-ва будет также\\
      $P = U \times .. \times \text{все пр-ва}$\\
      Когда берем пересечение, какое-то кол-во $U$ заменяем на откртые мн-ва, а остальные на полные пр-ва
  \end{Definition}

  \begin{example}
    $\{0, 2\}$ - дискр. пр-во\\
    $\text{дискр. пр-во} \times \text{дискр. пр-во} = \text{дискр. пр-во}$\\
    А если возьмём беск. произв., то дискр. пр-во не получится\\
    $\{0, 2\} \times \{0, 2\} \times ... = \{ \{ x_i \}_{i=1}^{\infinity} | x_i = \text{0 или 2}\}$\\
    Каждая точка в $\{0, 2\}$ октрыта\\
    База:  $\{ \{a_1\} \times \{a_2\} \times ... \times \{a_k\} \times \{0, 2\}  \times \{0, 2\}  \times ...\} = $\\
    $a_1 ... a_k = \text{0 или 2}$\\
    $= \{(a_1 a_2 ... a_k ...)\}$\\
    Первые $k$ членов посл-фи фиксируем, остальные не трогаем, берем все такие мн-ва, получаем базу
  \end{example}

  \begin{example}[индуцированная $\neq$ инициальная]
    $(X,\ \Omega_X)$ и $A \subset X$\\
     Если $X$ - метр. пр-во, то на $A$ метрика вводится элементарно. Индуцированная топология на $A$:\\
    $\Omega_A = \{U \cap A | U \in \Omega_X\}$\\
    $[0,\ \frac{1}{2}) = (- \frac{1}{2},\ \frac{1}{2}) \cap [0,\ 1]$ - индуц. топология
    \begin{example}
      $X = \R;\ A = [0,\ 1]$\\
      $B_A(0,\ \frac{1}{2})=[0,\ \frac{1}{2}]$ - открытое в $A$, не открыто в $X$\\
Упражнение: $i: A \lra X \qq i^{-1}(U) = U \cap A$\\
$i(x) = x$
    \end{example}
  \end{example}
\end{document}
