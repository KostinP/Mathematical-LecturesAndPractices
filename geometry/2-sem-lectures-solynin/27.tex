\documentclass[geometry.tex]{subfiles}

\begin{document}
  \section{Вторая аксиома счётности и сепарабельность.}

  \begin{definition}
      X - обл. \RNumb{2} А.С., если в X $\exists$ счетная база
  \end{definition}

  \begin{definition}
      X - назыв сепараб., если $\exists A \subset X$:\\
      $|A| \leq \aleph_0$ и $Cl A = X$
  \end{definition}

  \begin{definition}
      A - всюду плотно, если $\Cl A = X$
  \end{definition}

  \begin{examples}
      *здесь когда-нибудь будут примеры*
  \end{examples}

  \begin{theorem}
      X - \RNumb{2} А.С. $\ra$ X - сепараб.
  \end{theorem}

  \begin{proof}
      *здесь когда-нибудь будет док-во*
  \end{proof}

  \begin{upr}
      $\RNumb{2}$ А.С. и сепарабельность - топологические св-ва
  \end{upr}
\end{document}
