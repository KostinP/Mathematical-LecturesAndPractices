\documentclass[geometry.tex]{subfiles}

\begin{document}
  \section{Открытые и замкнутые множества. Свойства}

  \begin{definition}
      Открытый шар с центром в $x_0$ и радиусом $\mathcal{E}$ (окр.  $x_0$):
      \[B(x_0, \mathcal{E}) = \{x \in X \ | \ \rho(x, x_0) < \mathcal{E}\}\]
  \end{definition}

  \begin{definition}
      $U \subset X, \quad U$ - открыто, если:
      \[\forall x \in U \quad \exists \mathcal{E}\text{: } B(x, \mathcal{E}) \subset U\]
  \end{definition}

  \begin{example}
      В 1 (манхэтенская метрика) - квадрат, в 2 ($\rho_{\infty}$) тоже квадрат, 6,7 - открытые
  \end{example}

  \begin{definition}
      $Z \subset X \quad Z -$ замкнуто, если:
      \[X \setminus Z \text{ - открытое мн-во}\]
  \end{definition}

  \begin{theorem}[св-ва откр. мн-в]
      \begin{enumerate}
          \item $\{ U_\alpha \}_{\alpha \in A}$ - семейство откр. мн-в
                 \[\Rightarrow \bigcup_{\alpha \in A}U_\alpha - \text{откр.}\]
          \item $U_1,...,U_n$ - откр.(конеч. число) \[\Rightarrow \bigcap_{i = 1}^n U_i - \text{откр.}\]
          \item $\varnothing,\ X - $ откр.
      \end{enumerate}
  \end{theorem}
  \begin{proof}
      \begin{enumerate}
          \item $\forall x \in \bigcup\limits_{\alpha \in A} U_\alpha \Rightarrow \exists \alpha_0\text{: } x \in U_{\alpha_0}$
                 \[U_{\alpha_0} - \text{откр.}\Rightarrow \exists \mathcal{E}\text{: } B(x, \mathcal{E}) \subset U_{\alpha_0}\]
                 \[B(x, \mathcal{E}) \subset \bigcup_{\alpha \in A} U_\alpha \Rightarrow
                 \bigcup_{\alpha \in A} U_\alpha - \text{откр.}\]
          \item $\forall x \in \bigcap\limits_{i = 1}^n U_i \Rightarrow \forall i \q x \in U_i$
                \[\exists \mathcal{E}_i\text{: } B(x, \mathcal{E}_i) \subset U_i\]
                \[\mathcal{E} = \min_{i = 1,..., n}\{\mathcal{E}_i\} \quad B(x, \mathcal{E}) \subset B(x, \mathcal{E}_i) \subset U_i\]
                \[B(x, \mathcal{E}) \ \subset\  \bigcap\limits_{i=1}^n U_i\  \Rightarrow\  \bigcap\limits_{i = 1} ^ n U_i - \text{откр}\]
      \end{enumerate}
  \end{proof}

  \begin{Example}
    \[U_i = \left(- \frac{1}{i}, \frac{1}{i}\right)\]
    \[\bigcap_{i = 1}^\infty U_i = \{0\} \text{ - объясняет, почему должно быть конечное число в пересечении} \]
  \end{Example}

  \begin{lemma}
      $B(x_0, r) - $ открыто $\forall$ метр. пр-ва $X \quad \forall x_0 \q \forall r > 0$
  \end{lemma}
  \begin{Proof}
      \[x \in B(x_0, r) \Ra \rho(x_0, x) = d < r\]
      \[\text{Возьмём }\mathcal{E}=\frac{r-d}{2}\]
      \[B(x, \mathcal{E}) \subset B(x_0, r) ?\]
      $\text{*/ Здесь очень внимательно надо смотреть на предположение,}\\
      x_1 \text{ лежит в предполагаемой области за пределами шарика } B(x_0, r)\text{ */}$
      \[\sqsupset \exists x_1 \in B(x, \mathcal{E}) \setminus B(x_0, r)\]
      \[\rho(x_1, x) < \mathcal{E} = r - d\]
      \[\rho(x_0, x) = d\]
      \[\rho(x_1, x_0) \geq r\]
      \[rho(x_1, x_0) \geq  \rho(x_1, x) + \rho(x, x_0)\]
      \[\rho(x_1, x_0) \geq r \quad \text{и} \quad \rho(x_1, x) + \rho(x, x_0) < r\]
      противореч. нер-ву $\triangle$
  \end{Proof}
  \begin{theorem}[св-ва замкнутых мн-в]
      \begin{enumerate}
          \item $\{F_i\}_{i \in A} - $ замкн.
          \[\Rightarrow \bigcap_{i \in A} F_i - \text{замкн.}\]
          \item $F_1, ..., F_n - $ замкн.
          \[\Rightarrow \bigcup_{i = 1}^n F_i - \text{замкн.}\]
          \item $\varnothing$ и $X$ замкн.
      \end{enumerate}
  \end{theorem}

  \begin{Proof}[1]
      \[F_i = X \setminus U_i, \quad U_i \text{ - откр.}\]
      \[\bigcap F_i = \bigcap (X \setminus U_i) = X \setminus \bigcup U_i\]
  \end{Proof}
\end{document}
