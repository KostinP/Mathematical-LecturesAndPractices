\documentclass[geometry.tex]{subfiles}

\begin{document}
  \section{Топология произведения пространств.}

  \begin{example} [конструкция]
      Даны $(X, \Omega_X); \q (Y, \Omega_Y)$ - топ. пр-ва\\
      Введем базу топ. на $X \times Y$:\\
      \[\B = \{U \times V \ |\ U \in \Omega_X; \q V \in \Omega_Y\}\]
      Это топология:
      \[\Omega_{X \times Y} = \{\bigcup_{i \in I} U_i \times V_i \ | \ U_i \in \Omega_X; \q V_i \in \Omega_Y\}\]
      Для объединения - очевидно, для пересечения:
      \[(\bigcup_{i \in I} U_i \times V_i) \cap (\bigcup_{j \in J} S_j \times T_j) =
      \bigcup_{i \in I \  j  \in J}
      (
          \ub{\in \Omega_X}{(U_i \cap S_j)}
          \times
          \ub{\in \Omega_Y}{(V_i \cap T_j
      )} \in \Omega_{XXY}\]
  \end{example}
\end{document}
