\documentclass[geometry.tex]{subfiles}

\begin{document}
  \section{Связность топологического пространства и множества.}

  \begin{definition}
    X называется несвязным, если $\e$откр. $U_1, U_2 \neq \varnothing \in X:$
    \[X = U_1 \cup U_2,\qq U_1 \cap U_2 = \varnothing\]
  \end{definition}

  \begin{upr}
    Написать определение связного, как не несвязного
  \end{upr}

  \begin{definition}
    $A \subset X$, $A$ называется связным, если $A$ связно как топол. пр-во с индуцированной топологией\\ \ \\
    $A$ несв., если $\e$ открытые $U_1, U_2 \subset X$:
    \[\begin{matrix}
      (U_1 \cup A) \cap (U_2 \cup A) = A\\
      (U_1 \cup A) \cup (U_2 \cup A) = \varnothing\\
      U_1 \cup A \neq \varnothing\\
      U_2 \cup A \neq \varnothing
    \end{matrix} \os{\text{или}}{\lra} \begin{matrix}
      U_1 \cap U_2 \supset A\\
      U_1 \cup U_2 \cup A = \varnothing\\
      U_1 \cup A \neq \varnothing\\
      U_2 \cup A \neq \varnothing
    \end{matrix}\]
  \end{definition}
    В чем трудность док-ва того что пространство связное? Нужно доказывать, что нет разрывов.
\end{document}
