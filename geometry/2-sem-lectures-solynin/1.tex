\documentclass[geometry.tex]{subfiles}

\begin{document}
  \section{Метрические пространства. Примеры.}

  \begin{definition}
      X - мн-во (X $\neq$ $\varnothing$) \\
      $\rho: X \times X \rightarrow \R$ (метрика)\\ \\
      Пара $(X, \rho)$ называется метрическим пространством, если:
      \begin{enumerate}
          \item $\rho(x, y) \geq 0$
          \item $\rho(x, y) = 0 \Leftrightarrow x = y$
          \item $\rho(x, y) = \rho(y, x)$
          \item нер-во $\bigtriangleup$ \\ $\rho(x, z) \leq \rho(x, y) + \rho(y, z)$
      \end{enumerate}
  \end{definition}

  \begin{examples}
      \begin{enumerate}
          \item $\R, \R^2, \R^3$ со станд. $\rho$
          \item На $\R^2$
          \begin{enumerate}
              \item $\rho_1((x_1, y_1), (x_2, y_2)) = |x_1 - x_2| + |y_1 - y_2|$ - манхэттенская метрика
              \item $\rho_\infty = max\left\{|x_1 - x_2|, |y_1 - y_2|\right\}$
              \item $\rho_p = (|x_1 - x_2|^p + |y_1 - y_2|^p)^{\frac{1}{p}}$
              \item $\rho_2 \text{ - евклидова метрика}$
          \end{enumerate}
          \item X - город без односторонних дорог, $\rho(A, B)$ - min время, за которое можно добраться от A до B
          \item $X = \Z$, p - простое, $a = p^k a', \q a' \not \devides p$, $\rho(a, 0) = p^{-k}$
          \[a = 0 = p^{+\infty} 0\]
          \[\rhp(-a) = \rho(a) \qq \rho(a,b) = \rho(b,a)\]
          (p-адическая метрика)
          \item M - мн-во фигур на пл-ти
          \[\rho(A, B) = S_{A \bigtriangleup B} \qq (A \bigtriangleup B = A\setminus B \cup B \setminus A)\]
          Если есть точка, то $S_{A \bigtriangleup B} = 0$, но фигуры не совпадают (отличаются одной точкой)\\
          Метрика Хаусдорфа (даны две кривые $l_1, l_2$. Подбираем $\E$ так, чтобы $\forall y \in l_1 \q \e x \in l_2: \rho(x,y)< e$)
          \[\w{\rhp}(l_1, l_2) = \inf\{\E\}\]
          \[\rho(l_1, l_2) = \max\{\w{\rho}(l_2,l_1), \w{\rho}(l_1,l_2)\}\]
          \item X - мн-во\\
              \[\rho(a, b) =
              \begin{cases}
                  0, &a = b\\
                  1, &a \neq b
              \end{cases}\text{ - дискретная метрика}\]
          \item Алфавит $\sigma$, X - мн-во слов
          \[\rho(a_1,...,a_k;b_1,...,b_n) = \text{кол-во разрывов, в которых слова различ.}\]
      \end{enumerate}
  \end{examples}

  \begin{upr}
    Проверить, что это метрики
  \end{upr}
\end{document}
