\documentclass[12pt, fleqn]{article}

\usepackage{../../template/template}
\usepackage{../../template/fortickets}
%\usepackage{../../template/KillTableofcontents}

\begin{document}
  \subfile{preamble.tex}

  %\section{Метрические пространства. Примеры.}
  \subfile{1.tex}

  %\section{Открытые и замкнутые множества. Свойства}
  \subfile{2.tex}

  %\section{Внутренность и вшеность множества.}
  \subfile{3.tex}

  %\section{Замыкание множества.}
  \subfile{4.tex}

  %\section{Топологические пространства. Примеры.}
  \subfile{5.tex}

  %\section{База топологии. Критерий базы.}
  \subfile{6.tex}

  %\section{Топология произведения пространств.}
  \subfile{7.tex}

  %\section{Равносильные определения непрерывности.}
  \subfile{8.tex}

  %\section{Прообраз топологии. Индуцированная топология.}
  \subfile{9.tex}

  %\section{Инициальная топология. Топология произведения как инициальная.}
  \subfile{10.tex}

  %\section{Финальная топология. Фактортопология. Приклеивание.}
  \subfile{11.tex}

  %\section{Гомеоморфизм.}
  \subfile{12.tex}

  %\section{Связность топологического пространства и множества.}
  \subfile{13.tex}

  %\section{Связность отрезка.}
  \subfile{14.tex}

  %\section{Связность замыкания. Связность объединения.}
  \subfile{15.tex}

  %\section{Связность и непрерывные отображения.}
  \subfile{16.tex}

  %\section{Связность произведения пространств}
  \subfile{17.tex}

  %\section{Компоненты Связности.}
  \subfile{18.tex}

  %\section{Линейная связность}
  \subfile{19.tex}

  %\section{Компактность. Примеры.}
  \subfile{20.tex}

  %\section{Простейшие свойства компактности.}
  \subfile{21.tex}

    %\section{Компактность произведения пространств.}
    \subfile{22.tex}

    %\section{Компактность и хаусдорфовость}
    \subfile{23.tex}

    %\section{Лемма Лебега. Компактность отрезка.}
    \subfile{24.tex}

    %\section{Критерий компактности подмножеств евклидова пространства.}
    \subfile{25.tex}

    %\section{Теорема Вейерштрасса. Примеры.}
    \subfile{26.tex}

    %\section{Вторая аксиома счётности и сепарабельность.}
    \subfile{27.tex}

    %\section{Теорема Линделёфа.}
    \subfile{28.tex}

    %\section{Первая аксиома счётности.}
    \subfile{29.tex}

    %\section{Из компактности следует секвенциальная компактность (с первой АС).}
    \subfile{30.tex}

    %\section{Из секвенциальной компактности следует компкатность (со второй АС).}
    \subfile{31.tex}

    %\section{Полнота и вполне ограниченность метрических пространств.}
    \subfile{32.tex}

    %\section{Из полноты и вполне ограниченности следует компактность}
    \subfile{33.tex}

    %\section{Аксиомы отделимости.}
    \subfile{34.tex}

    %\section{Нормальность матрического пространства.}
    \subfile{35.tex}

    %\section{*Задачи из практик}
    \subfile{practice.tex}
\end{document}
