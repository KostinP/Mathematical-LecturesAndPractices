\documentclass[11pt]{article}
\usepackage[english,russian]{babel}
\usepackage{url}
%для поддержки русского
\usepackage{graphicx,DCCN2019_ru}
%пакет с нужными мне штуками
%для красивых символов
\usepackage{upgreek}
\usepackage{dsfont}
\usepackage{amssymb}
\usepackage{tipa}
\usepackage{ wasysym }
\usepackage{gensymb} %градусы
%зачеркивание текста
\usepackage{cancel}
%для цветного текста
\usepackage[usenames]{color}
%для ссылок
\usepackage{xcolor}
\usepackage{hyperref}
% Цвета для гиперссылок
\definecolor{linkcolor}{HTML}{0000ff} % цвет ссылок
\definecolor{urlcolor}{HTML}{0000ff} % цвет гиперссылок
\hypersetup{pdfstartview=FitH,  linkcolor=linkcolor,urlcolor=urlcolor, colorlinks=true}
%Для вставки картинок
\graphicspath{{pictures/}}
\DeclareGraphicsExtensions{.pdf,.png,.jpg}

%команды
\usepackage{amsmath,amsthm,amssymb,amsfonts, enumitem, fancyhdr, color, comment, graphicx, environ}

%настройки текста
\pagestyle{fancy} 
\fancyhead{} 
\fancyfoot{} 
\usepackage[utf8]{inputenc}

%настройки страницы
\makeatletter
\fancyhead[R]{\small Павел Костин}
\fancyhead[L]{\small Матмех СПбГУ}
\fancyhead[C]{\small Билеты, мат. анализ, 2 семестр, 2019}
\pagestyle{fancy}
\fancyfoot[R]{\thepage}

%команды
\usepackage{amsmath,amsthm,amssymb,amsfonts, enumitem, fancyhdr, color, comment, graphicx, environ}
%римские цифры
\newcommand{\RNumb}[1]{\uppercase\expandafter{\romannumeral #1\relax}}
%команды для ускорения набора
\newcommand{\R}{\mathds{R}}
\newcommand{\Q}{\mathds{Q}}
\newcommand{\Z}{\mathbb{Z}}
\newcommand{\B}{\mathcal{B}}
\newcommand{\CC}{\mathds{C}}
\newcommand{\N}{\mathds{N}}
\newcommand{\ra}{\Rightarrow}
\newcommand{\la}{\Leftarrow}
\newcommand{\rla}{\Leftrightarrow}
\newcommand{\lra}{\Leftrightarrow}
\newcommand{\e}{\exists}
\newcommand{\E}{\mathcal{E}}
\newcommand{\q}{\quad}
\newcommand{\devides}{\mathop{\raisebox{-2pt}{\vdots}}}

%вёрстка
\newenvironment{solutions}[1][]
{\begin{trivlist}\item{\underline{\bfseries #1}}}{\end{trivlist}\newpage}
\newenvironment{definition}[1][Опр.]
{\begin{trivlist}\item{\underline{\bfseries #1} }}{\end{trivlist}}
\newenvironment{definition2}[2][Опр]
{\begin{trivlist}\item[\underline{{\bfseries #1}} {\bfseries #2.}]}{\end{trivlist}}
\newenvironment{instance}[1][Пример.]
{\begin{trivlist}\item{\underline{\bfseries #1} }}{\end{trivlist}}
\newenvironment{instances}[1][Примеры.]
{\begin{trivlist}\item{\underline{\bfseries #1} }}{\end{trivlist}}
\newenvironment{statement}[1][Утв.]
{\begin{trivlist}\item{\underline{\bfseries #1} }}{\end{trivlist}}
\newenvironment{lemma}[1][Лемма.]
{\begin{trivlist}\item{\underline{\bfseries #1} }}{\end{trivlist}}
\newenvironment{lemma2}[2][Лемма]
{\begin{trivlist}\item[\underline{{\bfseries #1}} {\bfseries #2.}] \hspace{0pt} \\}{\end{trivlist}}
\newenvironment{comments}[1][Замечание.]
{\begin{trivlist}\item{\underline{\bfseries #1} }}{\end{trivlist}}
\newenvironment{theorem}[1][Теорема.]
{\begin{trivlist}\item{\underline{\bfseries #1} }}{\end{trivlist}}
\newenvironment{theorem2}[2][Теорема]
{\begin{trivlist}\item[\underline{{\bfseries #1}} {\bfseries #2.}] \hspace{0pt} \\}{\end{trivlist}}
\newenvironment{reminder}[2][Напоминание]
{\begin{trivlist}\item[\underline{{\bfseries #1}} {\bfseries #2:}] \hspace{0pt} \\}{\end{trivlist}}
\newenvironment{proofs}[1][Доказательство.]
{\begin{trivlist}\item{\bfseries #1} }{\end{trivlist}}
\newenvironment{proofs2}[2][Доказательство]
{\begin{trivlist}\item[\underline{{\bfseries #1}} {\bfseries #2.}]\hspace{0pt}} {\end{trivlist}}
\newenvironment{proofByDisagreement}[1][Доказательство (от противного). ]
{\begin{trivlist}\item{\bfseries #1}}{\end{trivlist}}
\newenvironment{proofByInduction}[1][Доказательство (по индукции). ]
{\begin{trivlist}\item{\bfseries #1}}{\end{trivlist}}
\newenvironment{properties}[1][Свойство.]
{\begin{trivlist}\item{\underline{\bfseries #1} }}{\end{trivlist}}
\newenvironment{properties2}[2][Свойства]
{\begin{trivlist}\item[\underline{{\bfseries #1}} {\bfseries #2.}]\hspace{0pt}} {\end{trivlist}}
\newenvironment{consequence}[1][Cледствие.]
{\begin{trivlist}\item{\underline{\bfseries #1} }}{\end{trivlist}}
\newenvironment{consequence2}[2][Cледствие]
{\begin{trivlist}\item[\underline{\bfseries #1} {\bfseries #2.}]}{\end{trivlist}}

%фикс отступа
\usepackage{tocloft}
\setlength{\cftbeforetoctitleskip}{1em}

%сам документ
\begin{document}
\begin{center}
  \huge Лекции по дифференциальным уравнениям
  (читает Звягинцева Т. Е.)
\end{center}
Данный документ неидеальный, прошу сообщать о найденных недочетах в \href{https://vk.com/drab_existence_a}{вк}
\tableofcontents
\newpage

%Замечания
\section{Введение}
\subsection{Литература}\ \\
Учебник Бибиков "Обыкновенные дифферинциальные уравнения"\\
Филиппов - задачи\\
"Методы интегрирования"\\
Каддинктон Ливенгсон "Обыкновенные дифференциальные уравнения"\\
Яругии
\\
\subsection{Введение}\ \\
$F(x,y,y',y'',...,y^{(n)})=0$\\
$x$ - неизвестная переменная\\
$y=y(x)$ - неизвестная функция лалалалалалала

\begin{definition}
Порядок уравнения - порядок старшей производной
\end{definition}
\\
Кроме того, $x= \frac{dx}{dt}$, $x^{(k)}= \frac{d^k x}{dt^k}$
\\
\subsection{Применение}\ \\
1) механика\\
2) электротехника\\
3) физика: $\dot{Q}=k Q$, $Q=Q_0 e^{kt}$\\
4) упр. движением\\
5) биология, экология\\
Пример из биологии:

x - хищник

y - жертва

\begin{equation*}
 \begin{cases}
   \dot{x} = -ax+cxy\\
   \dot{y}=by-dxy
 \end{cases}
\end{equation*}
$$a,b,c,d > 0,\ x,y>0$$

\section{Дифферинциальные уравнения первого порядка}
\subsection{Введение}\ \\
(1) $\dot{x}=X(t,x)$
\\
$X(t,x) \in C(G)$, G - обл, $G \subset \R^2$
\\
Но чаще будем $\in C(D)$ $D \subset \R^2$

\begin{definition}
Решение (1) - функция $x=\upvarphi(t)$, $t \in <a,b>:\ \dot{\upvarphi}(t) \equiv X(t,\upvarphi(t))$ на <a,b>
\end{definition}

1) $\forall t \in <a,b>$ $(t, \upvarphi(t)) \in D$

2) $\upvarphi(t)$ - дифф на $<a,b>$

3) $\upvarphi(t)$ - непр. дифф. (X- непр на D)

\begin{definition}
(2) Задача Коши - задача нахождения решения (1) $x=\upvarphi(t):\ \upvarphi(t_0)=x_0$ $((t_0, x_0) \in D)$
\end{definition}
\\
Геометрический смысл уравнения первого порядка - уравнение 1 задаёт поле направлений на множестве G

\begin{definition}
График решения называется интегральной кривой
\end{definition}
\\
В каждой точке задано направление, которое совпадает с касательной в этой точке к интегральной кривой

$\dot{\upvarphi}(t) |_{t=t_0} = X(t_0, x_0)$
\\
\subsection{Метод изоклин}\
\begin{definition}
Изоклина - это кривая, на которой поле направлений постоянно
\end{definition}
\\
Уравнение изоклин $X(t,x)=c$, где $c=const$

\begin{instance}\ \\
$\dot{x}=-\frac{t}{x}$ ($x=\upvarphi(t)$)\\
$-\frac{t}{x}=tg \alpha$\\
$x=-\frac{1}{c} t$, $c \neq 0$\\
$c=1\ (\alpha=\frac{\pi}{4})$ $x=-t$ - уравнение изоклин\\
$c=-1\ (\alpha=-\frac{\pi}{4})$ $x=t$\\
Решение задачи Коши (1, 1) - это $x=\sqrt{2-t^2}$\\
Решение задачи Коши (1,-1) - это $x=-\sqrt{2-t^2}$
\end{instance}

\subsection{Теорема Пеано}\ \\
(1) $\dot{x} = X(t,x)$, $X \in C(D)$\\
$D=\{(t,x):|t-t_0| \leqslant ... \leqslant |x-x_0| \leqslant b \}$\\
(2) $(t_0,x_0)$\\
По теореме Вейерштрасса $\exists M:\ |X(t,x)| \leqslant M\ \forall(t,x) \in D$\\
$h=min(a,\frac{b}{M})$

\begin{theorem2}{(Пеано)}
$\exists$ реш. задачи К. (1), (2) $x=\upvarphi(t)$ опр-е на $[t_0-h,\ t_0+h]$ - отрезок Пеано
\end{theorem2}

\begin{definition}
$\{\upvarphi_k(t)\}_{k=1}^\infty$, $t \in [c,d]$

1) $\upvarphi_k(t)$ - равномерно ограничена на $[c,d]$, если $\exists N:\ |\upvarphi_k(t)| \leqslant N$ $\forall k \in \mathds{N}$, $\forall t \in [c,d]$

2) $\upvarphi_k(t)$ - равностепенно непр на [c,d],  если $\forall \E > 0$ $\exists \delta > 0:$ $\forall t_1, t_2 \in [c,d]$ $|t_1-t_2| < \delta$ $\ra$ $|\upvarphi_k(t_1)-\upvarphi_k(t_2)| < \E$ $\forall k \in \mathds{N}$
\end{definition}

\begin{lemma2}{(Арцелло - Асколи)}
$\upvarphi_k(t)$, $k\in \mathds{N}$, равномерно огр. и равностепенно непр на $[c,d]$ $\ra$ $\exists$ подпосл $\upvarphi_k_j(t):$ $\upvarphi_k_j(t) \overset{[c,d]}{\underset{j \ra + \infty}{\rightrightarrows}} \upvarphi(t)$
\end{lemma2}
\end{document}