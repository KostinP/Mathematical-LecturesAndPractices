\documentclass[main]{subfiles}

\begin{document}
    \Date{24.10.19}
    \subsection{Первая фундаментальная форма}

    \begin{Example}
      \[F: U \subset \R^2 \ra \R^3\]
      \begin{multline*}
        $$\det \begin{pmatrix}
          <\dfrac{\d F}{\d t}, \dfrac{\d F}{\d t}> & <\dfrac{\d F}{\d s}, \dfrac{\d F}{\d t}>\\
          \\
          <\dfrac{\d F}{\d t}, \dfrac{\d F}{\d s}> & <\dfrac{\d F}{\d s}, \dfrac{\d F}{\d s}>
        \end{pmatrix} = \\ \\
          = <\frac{\d F}{\d t}, \frac{\d F}{\d t}> <\frac{\d F}{\d s}, \frac{\d F}{\d s}> - <\frac{\d F}{\d s}, \frac{\d F}{\d t}> <\frac{\d F}{\d t}, \dfrac{\d F}{\d s}> =\\ \\
         =\abs{\dfrac{\d F}{\d t}}^2 \abs{\dfrac{\d F}{\d s}}^2 - \abs{\dfrac{\d F}{\d s}}^2 \abs{\dfrac{\d F}{\d t}}^2 \cos^2 t = \abs{\dfrac{\d F}{\d t}}^2 \abs{\dfrac{\d F}{\d s}}^2
        =
        \begin{vmatrix}
          \dfrac{\d F}{\d t} \times \dfrac{\d F}{\d s}
        \end{vmatrix}^2$$
      \end{multline*}
    \end{Example}

    \begin{Remark}
      \[A(S)=\sum A(\square)\]
      \[A(\square) \approx \abs{\dfrac{\d F}{\d t} \times \dfrac{\d F}{\d s}} \Delta t \Delta s\]
      \[\RNumb{1}(F)= \begin{pmatrix}
        <\dfrac{\d F}{\d t}, \dfrac{\d F}{\d t}> & <\dfrac{\d F}{\d s}, \dfrac{\d F}{\d t}>\\
        \\
        <\dfrac{\d F}{\d t}, \dfrac{\d F}{\d s}> & <\dfrac{\d F}{\d s}, \dfrac{\d F}{\d s}>
      \end{pmatrix}\]
      \[A(S)= \iint \abs{\dfrac{\d F}{\d t} \times \dfrac{\d F}{\d v}} dt ds = \iint \sqrt{\det \RNumb{1}(F)} dt ds\]
    \end{Remark}

    \begin{Example}
      \[F: (0,\ 2\pi) \times (0,\ 2\pi) \ra \R^3\]
      \[(\theta,\ \varphi) \ra (\cos \theta \sin \varphi,\ \cos \theta \cos \varphi,\ \sin \theta)\]
      \begin{enumerate}
        \item Доказать, что образ F находится на сфере радиуса 1
        \item Найти S сферы через $\RNumb{1}(F)$
      \end{enumerate}
    \end{Example}

    \begin{proof}
      \begin{enumerate}
        \item Видно из параметрического уравнения сферы что это сфера, а также понятен радиус и её центр
        \[\begin{cases}
          x = x_0 + R \cdot \sin \theta\cdot \cos \phi,\\
          y = y_0 + R \cdot \sin \theta\cdot \sin \phi,\\
          z = z_0 + R \cdot \cos \theta,\\
        \end{cases}\]
        где $\theta \in [0, \pi]$ и $\phi \in [0, 2\pi)$ (у нас будет сдвиг на угол)
        \item Найдем переменные для $\RNumb{1}(F)$:
        \[<\frac{\d F}{\d \theta},\ \frac{\d F}{\d \theta}> = \sin^2 \theta \sin^2 \varphi + \sin^2 \theta \cos^2 \varphi + \cos^2 \theta = 1\]
        \[<\frac{\d F}{\d \theta},\ \frac{\d F}{\d \varphi}> = 0,\q <\frac{\d F}{\d \varphi},\ \frac{\d F}{\d \theta}> = 0,\q <\frac{\d F}{\d \varphi},\ \frac{\d F}{\d \varphi}> = \cos^2 \theta\]
        \[\Ra \RNumb{1}(F)=\begin{pmatrix}
          1 & 0\\
          0 & \cos^2 \theta
        \end{pmatrix}\]
        \[\Ra A(S) = \iint \sqrt{\det \RNumb{1}(F)} d\theta d\varphi = \int_0^\pi \int_0^{2\pi} |\cos \theta|d\theta d\varphi = \int_0^\pi 4 d\varphi = 4 \pi\]
      \end{enumerate}
    \end{proof}
\end{document}
