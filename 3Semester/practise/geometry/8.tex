\documentclass[main]{subfiles}

\begin{document}
    \subsection{(22.10.19) Завершаем тему}

    \begin{Example}
      \[\varphi: \R^2 \ra \R^3,\q \varphi \text{ - регулярная поверхность}\]
      Такая что $\forall u \subset \R^2$ (отк)\\
      Площадь: $\mathcal{A}(\varphi(u)) = \mathcal{A}(u)$
      \begin{enumerate}
        \item Доказать, что $\det (\RNumb{1}(\varphi)) = 1$
        \item Доказать, что $\varphi$ сохраняет углы и площади $\lra$ $\varphi$ сохраняет расстояние
      \end{enumerate}
    \end{Example}

    \begin{proof}
      \begin{enumerate}
        \item $\iint\limits_U \sqrt{\det \RNumb{1}(\varphi)} du dv = \mathcal{A}(u) = \iint\limits_u du dv \q \forall u \subset \R^2 \text{ отк.}$
        \[\iint_u (\sqrt{\det \RNumb{1}(\varphi)} - 1) du dv = 0 \q \forall u \subset \R^2\]
        Но $\varphi \in C^1 \Ra \sqrt{\det \RNumb{1} (\varphi)} - 1 \text{ непр.} $\\
        Предположим, что $\sqrt{\det \RNumb{1}(\varphi)} - 1 \neq 0 \Ra \e (u_0,\ v_0)$ т.ч. $\sqrt{\det \RNumb{1}(\varphi)_{(u_0,\ v_0)}} - 1 \neq 0$
        \[\Ra \e V \ni (u_0,\ v_0) \text{ т.ч. } \forall (u,\ v) \in V,\q \sqrt{\det \RNumb{1}(\varphi)} - 1 \neq 0\]
        \[\forall (u,\ v) \in V \q \sqrt{\det \RNumb{1}(\varphi)} - 1 > 0\]
        Тогда $\iint\limits_V \sqrt{\det \RNumb{1}(\varphi)} - 1 > 0$ - противоречие\\
        Значит, что $\det \RNumb{1}(\varphi) = 1$
        \item ???
      \end{enumerate}
    \end{proof}

    \begin{remark}
      Есть такая теорема:
      \[\varphi: (0,\ 2\pi) \times (0,\ 2\pi) \ra TOR \subset \R^3\]
      $\varphi \in C^1$ т.ч. $\RNumb{1}(\varphi) = \begin{pmatrix}
        1 & 0\\
        0 & 1
      \end{pmatrix}$\\
    \end{remark}

    \begin{Definition}
      \[SU(2) = \left\{ \begin{pmatrix}
        \alpha & - \ol{\beta}\\
        \beta & \ol{\alpha}
      \end{pmatrix},\q \alpha,\beta \in \CC,\q |\alpha|^2 + |\beta|^2 = 1 \right\} \subset \CC^4 \cong \R^8\]
    \end{Definition}

    \begin{Definition}
      \[S^3 = \{(x,y,z,t) \in \R^4\ |\ x^2+y^2+z^2+t^2 =1 \}\]
    \end{Definition}

    \begin{example}
      Доказать, что $\R^4 \supseteq S^3 \cong SU(2)$
    \end{example}

    \begin{proof}
      Мы можем перейти $SU(2) \ra S^3$, расписав через $\real$ и $\imag$. Получится подобное уравнение как в $S^3$, аналогично назад:
      \[\varphi: S^3 \ra SU(2) \subset \R^8\]
      \[x,y,z,t  \ra \begin{pmatrix}
        x + iy & -z + it\\
        z + it & x - iy
      \end{pmatrix}\]
      \[\w{\varphi}: \R^4 \ra \R^8\]
      Непрерывная функция компактна, значит Хаусдорф
    \end{proof}
\end{document}
