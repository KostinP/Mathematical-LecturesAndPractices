\documentclass[12pt, fleqn]{article}
\usepackage{../../../template/template}

%сам документ
\begin{document}
\begin{center}
  \huge Практика по геометрии

  (преподаватель Амрани И. М.)

  \large Записал Костин П.А.
\end{center}

Данный документ неидеальный, прошу сообщать о найденных недочетах в \href{https://vk.com/drab_existence_a}{вконтакте}
\tableofcontents
\newpage

\section{Дифференциальная геометрия}
\subsection{(03.09.2019) Кривые и поверхности}
\begin{Example}
    \[\gamma: \R \ra \R^3,\q \gamma \in C^2$,\text{ т.ч.}\q $|\gamma(t)|=1\ \forall t \in \R\]
    \[\text{Д-ть, что } \gamma'(t) \bot \gamma''(t)$ $\forall t \in \R\]
\end{Example}

\begin{Proof}
    \[|\gamma'|=1 \lra \sqrt{<\dot{\gamma},\dot{\gamma}>}=1 \lra <\dot{\gamma},\dot{\gamma}>=1\]
    \[(<\dot{\gamma},\dot{\gamma}>)'=(1)' \Ra 2<\dot{\gamma},\ddot{\gamma}> = 0\]
    Вообще очевидно, но если нет, то:
    \[(<\dot{\gamma},\dot{\gamma}>)'=(\sum_{i=1}^3 \dot{\gamma_i}^2)' = \sum_{i=1}^3 2 \dot{\gamma_i} \ddot{\gamma_i} = 2<\dot{\gamma},\ddot{\gamma}>\]
\end{Proofs}

\begin{Example}
    \[\gamma: \R \ra \R^3,\q \gamma \in C^3,\q |\gamma'|=1,\q \gamma'' \neq 0\]
    \[T(t)=\gamma'(t),\q B(t)=T(t) \times N(t),\q N(t)=\frac{\gamma''(t)}{|\gamma''(t)|}\]
    \begin{enumerate}
      \item Д-ть, что $\{T(t), N(t),B(t) \}$ - ОНБ
      \item Найти координаты $\dfrac{dT}{dt}$, $\dfrac{dN}{dt}$, $\dfrac{dB}{dt}$ в базисе $\{T,N,B\}$
    \end{enumerate}
\end{Example}

\begin{proof}
  \begin{enumerate}
    \item Очевидно, $B(t) = \us{=1}{T} \cdot \us{=1}{N} \sin \angle (T,N)$
    \[T \bot N \ (\text{по пред. задаче}),\q B \bot N,\q B \bot T\ (\text{по опр. вект. произв.})\]

    \item По определению "взятием производной"\,получаем:
    \[\dfrac{dT}{dt} = 0T + |\ddot{\gamma}|N + 0B\]
    \[<N, T> = 0 \Ra <\frac{d N}{dt}, T> + <N, \frac{d T}{dt}> = 0\]
    \[\text{Аналогично } 0 = <\frac{d T}{dt},B> = - <\frac{d B}{dt}, T>\]
    \[|\ddot{\gamma}| = <\frac{d N}{dt}, T> = -<N, \frac{d T}{dt}>\]
    \[\frac{d N}{dt} = -|\ddot{\gamma}|T + 0N + \varphi(t)B\]
    \[\frac{d B}{dt} = 0T - \varphi(t)N + 0B\]

  \end{enumerate}

\end{proof}

\subsection{(01.10.2019) Первая и вторая фундаментальные формы }

F

\end{document}
