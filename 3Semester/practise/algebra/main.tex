\documentclass[12pt, fleqn]{article}
\usepackage{../../../template/template}

%сам документ
\begin{document}
\begin{center}
  \huge Практика по матану, 3 сем

  \Large (преподаватель Демченко О. В.)

  \large Записал Костин П.А.
\end{center}

Данный документ неидеальный, прошу сообщать о найденных недочетах в \href{https://vk.com/drab_existence_a}{вконтакте}
\tableofcontents
\newpage

%билеты
\section{Теория групп}
\subsection{Жордановы формы, 03.09.2019}

\begin{utv}
    Пусть $A \in M_n(\CC)$, $U \in GL_n(\CC)=\{U \in M_n(\CC): |U| \neq 0\}$

    Сопряжение матрицы A с помощью U: $A \longmapsto U^-1 A U$
\end{utv}

\begin{theorem}[Жордана, матрич. форма]
    $\forall A \e U: U^{-1} A U = J$\\
    Пусть $U^{-1} A U = J$, $V^{-1} A V = I$ - совпадают с точностью до перестановки жардановых блоков
\end{theorem}

\begin{example}
    $A_1 \in M_n(K)$, $A_2 \in M_m(K)$\\
    $\begin{pmatrix}
      A_1 & 0\\
      0 & A_2
    \end{pmatrix}$
    $\begin{pmatrix}
      A_2 & 0\\
      0 & A_1
    \end{pmatrix}$\\
    С помощью какой матрицы можно поулчить сопряжением другую
\end{example}

\begin{theorem}[Жордана, операт. форма]
    Пусть $L \in \mathscr{L}(V)$ (оператор на V), V - конечномерное пр-во над $\CC$. Тогда $\e \{e_1,...,e_n\}$ (жарданов базис) - базис V. $[L]_e=J$\\
    Единственность: если есть два базиса, то матрицы можно получить перестановкой
\end{theorem}
-------------- тут не хватает чего-то
\subsection{Собственные вектора, 10.09.2019}
------------- что-то пропущено\\
\subsection{Жордановы матрицы, 17.09.2019}

\begin{Example}
    \[A \in M_n(\CC)\]
    \[X^2=A=C^{-1}J C\]
    Пример $J=\begin{pmatrix}
      \lambda & 1\\
      0 & \lambda
    \end{pmatrix}$, $Y^2=J$, $Y=\begin{pmatrix}
      \sqrt{\lambda} & ?\\
      0 & \sqrt{\lambda}
    \end{pmatrix}$, ? - из уравнения
\end{Example}

Как найти J и C?

1) Находим все ссобственные числа матрицы A

Если все с.ч. равны, то J без единичек

Если одно собственное число
а) диагонализируема
$\begin{pmatrix}
\lambda & 0 & 0\\
0 & \lambda & 0\\
0 & 0 & \lambda
\end{pmatrix}$

б) блоки 2 и 1
$\begin{pmatrix}
\lambda & 1 & 0\\
0 & \lambda & 0\\
0 & 0 & \lambda
\end{pmatrix}$

в) $\begin{pmatrix}
\lambda & 1 & 0\\
0 & \lambda & 1\\
0 & 0 & \lambda
\end{pmatrix}$

\begin{example}
    Найдём, сколько собственных вектор-столбцов

    Первая матрица:
    $\begin{pmatrix}
    \lambda & 0 & 0\\
    0 & \lambda & 0\\
    0 & 0 & \lambda
    \end{pmatrix}
    \begin{pmatrix}
    x_1\\
    x_2\\
    x_3
    \end{pmatrix}
    =
    \begin{pmatrix}
    \lambda x_1\\
    \lambda x_2\\
    \lambda x_3
    \end{pmatrix}
    \Ra
    \begin{cases}
    \lambda x_1=\lambda x_1\\
    \lambda x_2=\lambda x_2\\
    \lambda x_3=\lambda x_3
    \end{cases}
    \Ra
    x_1,x_2,x_3 \in R
    $ - три л.н. переменные

    Для второго решение:
    $\begin{pmatrix}
    \lambda x_1\\
    0\\
    \lambda x_3
    \end{pmatrix}$ - 2 собственных вектор-столбца
\end{example}


\begin{example}
    Пусть у нас матрица 4*4, 2 собственных л.н. столбца (два блока)
\end{example}

\begin{utv}
    G,H - изоморфны, G - комм. $\Ra$ H - комм.
\end{utv}

\begin{proof}
    $\e \varphi: G \ra H: \varphi$ - биекция и $\varphi(g_1 g_2)=\varphi(g_1) \varphi(g_2)$, кроме того, $g_1 g_2=g_2 g_1$ $\forall g_1, g_2 \in G$, применим $\varphi$ к последнему выражению

    $h_1 h_2=\varphi(g_1) \varphi(g_2)=\varphi(g_1 g_2)=\varphi(g_2 g_1)=\varphi(g_2) \varphi(g_1)= h_2 h_1$
\end{proof}

\begin{homework}
    $X^2=\begin{pmatrix}
    0 & 1\\
    -1 & -1
    \end{pmatrix}$
\end{homework}

Дз: G,H - изоморфны, G - цикл. $\Ra$ H - цикл.
\begin{sol}
    Грппа G - цикл $\lra \e g \in G: \forall g' \in G \q \e k \in \Z$
    \[G \text{ - цикл., }G \cong H \Ra \e \varphi: G \ra H\]
    \[\forall h' \in H \q \e g' \in G: h'=\varphi(g')=\varphi(g^k)=\varphi(\underbrace{g...g}_k)=\underbrace{\varphi(g)...\varphi(g)}_k=\underbrace{h...h}_k=h^k\]
\end{sol}

Чтобы доказать, что две группы не изоморфны, можно доказать что у одной из них свойство выполняется, а у другой нет

\begin{example}
    \begin{enumerate}
        \item $\Z /_{6 \Z}$, $D_3$ - коммуннитативность
        \item $\Z /_{4 \Z}$, $\Z_{ /2 \Z} \times \Z /_{2 \Z}$ - цикличность
        \item $\Z /_{8 \Z}$, $\Z /_{4 \Z} \times \Z /_{2 \Z}$ - дз
        \item $\Z /_{4 \Z} \times \Z /_{2 \Z}$, $\Z /_{2 \Z} \times \Z /_{2 \Z} \times \Z /_{2 \Z}$ - порядки элементов
        \item $\Z$, $\Z \times \Z$ - цикличность?
    \end{enumerate}
\end{example}

\subsection{В ожидании кр..., 24.09.2019}

\begin{Example}
    \[A \in M_n(\CC),\q A=C^{-1}J C,\q C \in \Gl_n(\CC)\]
    \[J=
    \begin{pmatrix}
        \lambda& 1 &\ldots & 0\\
        0& \lambda &\ldots & 0\\
        \vdots& \vdots &\ddots & \vdots\\
        0&0 &\ldots & \lambda
    \end{pmatrix}\]
\end{Example}

1) находим все с.ч.

2) для каждого с.ч. находим л.н. уравнение

3) решаем систему линейных уравнений

...ЗДЕСЬ ЧТО-ТО ПРОПУЩЕНО, СМ. ТЕТРАДЬ

\begin{Example}
    \[D_3=\{e,l,r,s_1,s_2,s_3\}\]
    \includegraphics[scale=0.3]{triangle_d_3}
    \[H_1=\{e,r,l\}\]
    \[H_2=\{e,s_1\}\]
    1) Разбить по подгруппам, по левым и правым классам. Какая нормальная, какая нет?\\
    2) Найти g,G. Чтобы произведение не лежало в $H_2$
\end{Example}

Дз: $D_4=\{...\}$, $H_1=\{e,s_2\}$, $H_2=\{e, r^2\}$

Дз: $K(D_3)$ - найти коммутант для $D_3$

\subsection{Комутаторы и комутанты, 01.10.2019}

\begin{example}
  Дз (прошлое): $G=D_4$
  \[H=\{e,r^2\}\]
  \[H \triangleleft G\]
  \[G/_H\]
\end{example}

Дз (новое):
\begin{enumerate}
  \item Чему изоморфно $G/_H$? $\Z/_{4 \Z}$, $\Z/_{2 \Z} \times \Z/_{2 \Z}$
  \item $|G|=4 \ \Ra \
  \left[
  \begin{array}{ccc}
     G & \cong & \Z/_{4 \Z} \\
     G & \cong & \Z/_{2 \Z} \times \Z/_{2 \Z} \\
  \end{array}
\right.$
\end{enumerate}

\begin{example}[я не знаю, что это было]

\end{example}

\begin{example}
  \begin{enumerate}
    \item $\CC^* \ra \R^* \q (z \mapsto |z|)$
    \item $\CC^* \ra \CC^* \q (z \mapsto z^4)$

  \end{enumerate}
  Что получается при применении основной теоремы о гомоморфизме? (найти ядро образ, факторизовать, д-ть, что изморфна образу)
\end{example}

\begin{sol}
  \begin{enumerate}
    \item $\CC^*/_{\{Z \in \CC:\ |z|=1\}} \cong \R^*_{>0}$
    \item ДЗ
  \end{enumerate}
\end{sol}

\begin{example}
  ДЗ: $\GL_n(\R)/_{\{A \in \GL_n(\R):\ \det A = \pm 1 \}} \cong ?$

  Как это сделать? Нужно найти $\varphi: \GL_n(\R) \ra H$ - гомоморфизм: $\Ker \varphi = \{ A \in \Gl_n(\R):\ \det A = \pm 1\}$
\end{example}

\begin{Sol}
  \[\varphi(A)=|\det A|\]
  \[\varphi(A)=(\det A)^2\]
\end{Sol}

ДЗ: $\GL_n(\R)/_{\{A \in \GL_n(\R):\ \det A = \pm 1,\ \pm i \}} \cong ?$

\subsubsection{Действие группы на множество}

\begin{Example}
  \[D_4\]
  Написать разбиение этого множества из 16 эл-ов на орбиты. Сколько орбит?
\end{Example}

\begin{Sol}
  
\end{Sol}



\end{document}
