\documentclass[12pt, fleqn]{article}
\usepackage{../../template/template}

\begin{document}
\begin{example}
  Докажите неприводимость над $\Q$ многочлена $x^5 + 4x^4 - 2x^2 - 3x + 1$
\end{example}

\begin{proof}
  Перейдем к $Z$. Привожимость над $\Z \lra$ приводимости над $\Q$\\
  Предположим, что f - приводимый\\
  В доказательстве редукционного критерия (билет 33) у Всемирнова мы доказали, что если многочлен приводим над $\Z$ то он приводим над $\Z/p\Z$, p - простое. И раскладывается в произведение таких же степеней, что и в $\Z$. Значит если он разложился по-разному по двум разным модулям он неприводим.\\ \\
  Попробуем $\mod 2$:
  \[f \os{\mod 2}{=} x^5 - x + 1, \qq f(0) = 1\q f(1) = 1\]
  Не делится на линейные. То есть $f = g \cdot h \q \deg g = 2,\q \deg h = 3$ \\ \\
  %Неприводимые в $\Z/2$ второй степени:
  %\[x^2+x+1\]
  %$(x^5 - x + 1) = (x^2+x+1)(x^3 - x^2 + 1) = (x^5 + x^4 + x^3) - (x^4 + x^3 + x^2) + (x^2 + x + 1) = x^5 + x + 1$\\ \\
  %Попробуем $\mod 3$:
  %\[f \os{\mod 3}{=} x^5 + x^4 + x^2 + 1, \qq f(0) = 1 \q f(1) = 1\]
  %\[f(2) = 2^5 + 2^4 + 2^2 + 1 = 32 + 16 + 4 + 1 = 53 = 2\]
  %Не делится на линейные\\ \\
  %Неприводимые в $\Z/3$ второй степени:
  %\[x^2+1\qq x^2+x+2 \qq x^2+2x+2\]
  %\begin{enumerate}
  %  \item $x^5 + x^4 + x^2 + 1 = (x^2 + 1)(x^3 + x^2 - x) + x + 1 = $
  %  \[=(x^5 + \cancel{x^3}) + (x^4 + x^2) - (\cancel{x^3} + \cancel{x}) + \cancel{x} + 1\]
  %  \item $x^5 + x^4 + x^2 + 1 = (x^2+x+2)(x^3 - 2x + 2) =$
  %  \[= (x^5 + x^4 + \cancel{2x^3}) - (\cancel{2x^3} + \cancel{2x^2} + \cancel{x}) + (\cancel{2x^2} + \cancel{2x} + 1)\]
  %\end{enumerate}\\
  Попробуем $\mod 5$:
  \[f \os{\mod 5}{=} x^5 - x^4 + 3x^2 + 2x + 1\]%, \qq f(0) = 1 \q f(1) = 1\]
  %\[f(2) = 2^5 - 2^4 + 3 \cdot 2^2 + 2 \cdot 2 + 1 = = 32 - 16 + 12 + 4 + 1 = 2 - 1 + 2=3\]
  %\[f(3) = 3^5 - 3^4 + 2 \cdot 3^2 + 2 \cdot 3 + 1 = 243 - 81 + 18 + 6 + 1 = 3 - 1 + 3 + 1 + 1 = 2\]
  \[f(4) = 4^5 - 4^4 + 3 \cdot 4^2 + 2 \cdot 4 + 1 = 1024 - 256 + 48 + 8 + 1 = 4 - 1 + 3 + 3 + 1 = 0\]
  Делится на линейный. То есть $\w{f} = \w{g} \cdot \w{h} \q \deg \w{g} = 1,\q \deg \w{h} = 4$ \\ \\
  Разложился по разным степеням, ч.т.д.

\end{proof}

\end{document}
