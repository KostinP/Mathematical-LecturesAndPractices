\documentclass[main.tex]{subfiles}

\begin{document}
    \newpage
    \section{Конечные поля}
    Кольцом $R$ будем называть ассоциативное коммутативное кольцо с 1
    \subsection{Идеал кольца, примеры. Идеал, порожденный набором элементов. Главный идеал. Делимость и ассоциированость на языке главных идеалов. Идеалы кольца целых чисел и кольца многочленов}

    \begin{definition}
        $I \subset R$ --- \ul{идеал}, если:
        \begin{enumerate}
            \item $\forall a,b \in I \q a + b \in I$
            \item $\forall a \in I,\ r \in R \q ra \in I$
        \end{enumerate}
    \end{definition}

    \begin{example}
        Четные числа --- идеал кольца целых чисел
    \end{example}

    \begin{remark}
        %Идеал --- аддитивная группа кольца

        Идеал --- подгруппа аддитивной группы
    \end{remark}

    \begin{Definition}[конструкция]
        \[a_1,..., a_n \in R\]
        \[(a_1,...,a_n) = \{ r_1 a_1 + ... + r_n a_n,\q r_i \in R\} \]
    \end{Definition}

    \begin{utv}
        Это множество является идеалом
    \end{utv}

    \begin{example}
        Четные числа --- идеал $(2)$
    \end{example}

    \begin{definition}
        Идеал, порожденный одним элементом называется \ul{главным идеалом}
        \[(a) = \{ ra,\q r \in R\}\]
    \end{definition}

    \begin{properties}
        \begin{enumerate}
            \item $a \devides b \lra (a) \subset (b)$
            \item $a \sim b \lra (a) = (b)$
        \end{enumerate}
    \end{properties}

    \begin{proof}[1]
        $(\La)$:
        \[a \devides b \RA a = bc\]
        \[ra = rcb\]
        $(\Ra)$:
        \[(a) \subset (b) \RA a \in (b)\]
        \[\Ra a = bc \RA a \devides b\]
    \end{proof}

    \begin{theorem}
        Любой идеал $\Z$ (и $K[x]$) --- главный
    \end{theorem}

    \begin{proof}[для $\Z$]
        $I$ --- идеал в $\Z$\\
        Пусть $a$ --- минимальный положительный элемент этого идеала\\
        \[b \in I\]
        Поделим $b$ на $a$ с остатком:
        \[b = aq + c,\q 0 \leq c < a\]
        \[a \in I \RA aq \in I\]
        \[b \in I \RA b-aq \in I \RA c \in I\]
        Значит $c \in I$ и $0 \leq c < a \RA c = 0$\\
        Значит любой элемент делится нацело на $a$\\
        Доказали, что $I \subset (a)$\\
        Но $a \in I \RA ar \in R$, доказали
    \end{proof}

    \begin{proof}[для $K[x$] %добавить ] (сделать K[x])
        Как доказать для кольца многочленов?\\ %написать строго
        Вместо минимального положительного возьмем многочлен минимальной степени, который лежит в идеале. Дальше также. Берем любой, делим на мн-н минимальной степени. Степень остатка меньше степени исходного мн-на
    \end{proof}

    \begin{theorem}
        В $\Z$ (в $R[x]$)
        \[(a,b) = (\gcd(a,b))\qq \gcd(a,b) = d\]
    \end{theorem}

    \begin{proof}[$\Z$]
        $(a,b) \subset (\gcd(a,b))$:
        \[ra + sb = xd \in (d)\]
        \[\text{Возьмем }xd\]
        По теореме о линейном представлении: $t_1 a + t_2 b = d$
        \[\Ra xd = (t_1 x)a + (t_2 x)b \in (a,b)\]
    \end{proof}

    \begin{proof}[в $R[x$] %добавить ]
        Аналогично
    \end{proof}

    \newpage
    \subsection{Факторкольцо. Построение конечного поля, примеры}
    \begin{definition}
        $I \subset R$\\
        Идеал является подгруппой аддитивной группы кольца, которая коммутативна.\\
        Профакторизуем: $R\big/_{\displaystyle I}$ (фактор-группа по сложению)\\
        Сложение такое же.\\
        Умножение: $\ol{a} \cdot \ol{b} \os{def}{=} \ol{ab}$

        \[\left.\begin{matrix}
            \ol{a} = \ol{a'}\\
            \ol{b} = \ol{b'}
        \end{matrix} \right| \os{?}{\RA} \ol{ab} = \ol{a'b'}\]
        \[\begin{matrix}
            a - a' \in I\\
            b - b' \in I
        \end{matrix} \q
        \begin{matrix}
            a' = a + s,\q s \in I\\
            b' = b + t,\q t \in I
        \end{matrix}\]
        Перемножим равенства:
        \[a'b' - ab = at + sb + st \os{\text{т.к. каждый}\in I}{\in}I\]
    \end{definition}

    \begin{utv}
        $R \big/_{\displaystyle I}$ --- кольцо (ком., асс., с 1)
    \end{utv}

    \begin{remark}
        Достаточно д-ть:
        \begin{enumerate}
            \item $(\ol{a} \ol{b}) \ol{c} = \ol{a} (\ol{b} \ol{c})$
            \item $\ol{a} \ol{b} = \ol{b} \ol{a}$
            \item $\ol{1} \ol{a} = \ol{a}$
            \item $\ol{a}(\ol{b} + \ol{c}) = \ol{a} \ol{b} + \ol{a} \ol{c}$
        \end{enumerate}
    \end{remark}

    \begin{proof}
        Докажем комутативность:
        \[\ol{a} \cdot \ol{b} = \ol{ab} = \ol{ba} = \ol{b} \cdot \ol{a}\]
        (остальные аналогично)
    \end{proof}

    У нас получилось новое кольцо, которое мы будем называть фактор-кольцом ($\R\big/_{\displaystyle I}$) по идеалу $I$

    \begin{Reminder}
        \[\Z\big/_{\displaystyle p\Z} \text{ --- поле (было)}\]
    \end{Reminder}

    \begin{utv}
        $K[x]\big/_{\displaystyle (f)}$ --- поле (f --- непр.)
    \end{utv}

    \begin{proof}
        Достаточно доказать, что любой $\ol{g} \neq \ol{0} \q g \in K[x]$ --- обратим
        \[\lra g \neq (f)\]
        Рассмотрим $(g,f)$, f --- неприводим, значит либо $f | g$, либо $\gcd = 1$

        Но первый вариант не может быть, значит $(g,f) = 1$

        Значит существует линейное представление:
        \[gh_1 + f h_2 = 1,\q h_1, h_2 \in K[x]\]
        Обратно перейдем в фактор-кольцо
        \[gh_1 - 1 \in I\]
        \[\lra \ol{gh_1} = \ol{1}, \text{ но } \ol{gh_1} = \ol{g}\ol{h_1}\]
        Нашли обратный
    \end{proof}

    \newpage
    \subsection{Расширение полей, примеры, степень расширения. Степень расширения для башни расширений. Количество элементов конечного поля}

    \begin{reminder}
        Характеристика поля 0 или простое число
    \end{reminder}

    \begin{definition}
        $K'\big/_{\displaystyle K}$, $K \subset K'$, $K,\ K'$ --- поля.

        Называем $K'\big/_{\displaystyle K}$ --- \ul{расширением полей} (это не факторизация!)
    \end{definition}

    \begin{Example}
        \[\CC\big/_{\displaystyle \R}\]
        \[\R\big/_{\displaystyle \Q}\]
    \end{Example}

    \begin{definition}
        $[K', K]$ --- \ul{степень расширения} $K'\big/_{\displaystyle K}$
    \end{definition}

    Пусть $K'\big/_{\displaystyle K}$, $K \subset K'$

    Рассмотрим $\CC$ как векторное пр-во над $\R$

    \begin{remark}
        Степень расширения --- размерность $K'$, рассмотренного как векторное пр-во над $K$
        \[[K',K] = \dim_K K'\]
    \end{remark}

    \begin{example}
        Степень расширения $\CC$ над $\R$ - 2\\
        Степень расширения $\Q$ над $\R$ - $+\infty$ \\
        (не существует конечного набора над $\R$
        такого, чтобы любое другое являлось комбинацией этих коэф. из $\Q$)
    \end{example}

    \begin{utv}
        Рассмотрим $|K| < \infty$
        \begin{enumerate}
            \item $\char K \neq 0$ ($\Ra \char K = p$)\\
                Т.к. когда-то $\ub{n}{1+...+1} = \us{m}{1+...+1} \q m > n$, т.к. поле конечно
                \[\Ra \ub{m+n}{1+...+1} = 0\]
                Значит конечная ненулевая характеристика

                \[\letus \char K = p\]
                \[\Omega = \{0,1, 1+1,...,1+1+1,...\} \subset K\]
                \[\hat \Omega = \{0,1,1+1,...,\ub{p-1}{1+...+1}\}\]
                \[\hat \Omega \subset \Omega\]
                \begin{enumerate}
                    \item Докажем, что в них нет совпадающих элементов. Пусть это не так
                        \[\ub{n}{1+1+...+1} = \ub{m}{1+...+1} \q 0 \leq n < m \leq p+1\]
                        \[\ub{m-n}{1+....+1} \q p-1 \geq m-n > 0\]
                        Но $\char = p$, а тут не так
                    \item Любой элемент из $\Omega$ лежит в $\hat \Omega$\\
                        Возьмем $\ub{n}{1+...+1}$

                        Поделим с остатком:
                        \[\ub{n}{1+...+1} = (\os{=0}{\ub{p}{1+...+1}})(\ub{s}{1+...+1})+(\ub{q}{1+...+1})\]
                        \[n = ps + q,\q 0 \leq p < p\]
                    \item Хотим д-ть, что $\Omega$ --- поле
                    \begin{enumerate}
                        \item $(1+...+1) + (1+...+1) = 1+...+1$ (замкнутость относительно сложения)
                        \item $(1+...+1) \cdot (1+...+1) = 1+...+1$
                        \item $0 \in \Omega$
                        \item $1 \in \Omega$
                        \item $|\Omega| = p$
                        \item $\ub{n}{1+...+1} + \ub{\leq p-n}{1+...+1} = (\ub{s}{1+...+1})(\os{=0}{\ub{p}{1+...+1}})$
                        \[sp - n \geq 0\]
                        \item $1+...+1 \neq 0\ \lra \ n \not \devides p$
                            \[(n,p) = 1\]
                            \[ns - pq = 1, \text{ либо } pq - ns = 1\]
                            В первом случае:
                            \[(\ub{p}{1+...+1})(\ub{s}{1+...+1}) = 1 + (\os{=0}{\ub{p}{1+...+1}})(\ub{q}{1+...+1})\]
                            Во втором случае:
                            \[1 + (\ub{p}{1+...+1})(\ub{s}{1+...+1}) = (\os{=0}{\ub{p}{1+...+1}})(\ub{q}{1+...+1})\]
                            Получилось:
                            \[(\ub{n}{1+...+1})(\ub{s}{1+...+1}) = -1\]
                            \[(\ub{n}{1+...+1})(\ub{pt-s}{1+...+1})=1,\q pt > s\]
                    \end{enumerate}
                \end{enumerate}
        \end{enumerate}
    \end{utv}
\end{document}
