\documentclass[main]{subfiles}

\begin{document}
\begin{lect} {2019-09-24}
	\begin{Reminder}
		\[G / K(G) \text{ - коммутативна}\]
	\end{Reminder}

	\begin{Utv}
		\[H \triangleleft G \q G /_H \text{ - комм}\]
		\[\forall g_1, g_2 \in G \q (g_1H)(g_2H) = (g_2H)(g_1H)\]
		\[[g_1, g_2] = g^{-1}_1 g^{-1}_2 g_1 g_2 \in H \Ra K(G) \subset H\]
	\end{Utv}

	\hsubsection{1.10}{Свойства гомоморфизма}
	\begin{Properties} [гомоморфизма]
		\[f \in \Hom(G, H)\]
		\begin{enumerate}
			\item $f(e_G) = e_H \q\q f(e) = f(e \cdot e) = f(e) \cdot f(e)$
			\item $f(a^{-1}) = f(a)^{-1}$
				\[f(a)f(a^{-1}) = f(aa^{-1}) = f(e) = e\]
			\item Композиция гомоморфизмов --- гомоморфизм
		\end{enumerate}
	\end{Properties}

	\begin{definition}
		$f \in \Hom(G, H)$
		\[\Ker f = \{g \in G : \ f(g) = e\} \subset G\]
		\[\im f = \{f(g) : \ g \in G\} \subset H\]
	\end{definition}

	\begin{utv}
		$\Ker $ --- подгруппа G \qq $\Im$ --- подгруппа $H$
	\end{utv}

	\begin{proof}
		\begin{enumerate}
			\item $f(g_1) = f(g_2) = e \Ra f(g_1g_2) = f(g_1)f(g_2) = e \cdot e = e$
			\item $f(e) = e$
			\item $f(g) = e \Ra f(g^{-1}) = f(g)^{-1} = e^{-1} = e$
		\end{enumerate}
		\begin{enumerate}
			\item $f(g_1) \cdot f(g_2) = f(g_1g_2)$
			\item $e = f(e)$
			\item $f(g)^{-1} = f(g^{-1} ) $
		\end{enumerate}
	\end{proof}

	\begin{utv}
		$\Ker$ --- нормальная подгруппа G
	\end{utv}

	\begin{proof}
		$\Ker f \triangleleft G ?$
		\[g \in G \q\q a \in \Ker f\]
		\[f(g^{-1} a g) = f(g)^{-1} \underbracket{f(a)}_{= e} f(g) = e\]
	\end{proof}

	\hsubsection{1.11}{Основная теорема о гомоморфизме}
	\begin{Utv} [основная теорема о гомоморфизме]
		\[G/_{\Ker f} \cong \im f \]
	\end{Utv}

	\begin{proof}
    	Докажем, что это корректное отображение:
		\[\Ker f = K\]
		\[\varphi(gK) \os{def}{=} f(g) \q\q \varphi: G/_{\Ker f} \to \im f\]
		\[gK = g'K \os{?}{\Ra} f(g) = f(g')\]
		\[g' = g \cdot a, \q a \in K \q\q f(g') = f(g) \cdot \underbracket{f(a)}_{= e} = f(g) \]
    	Докажем, что $\varphi$ --- гомоморфизм:
		\[f(g_1)f(g_2) = \varphi(g_1K) \varphi(g_2K) \os{?}{=} \varphi(g_1 K g_2 K) = \varphi((g_1g_2)K) =
		f(g_1g_2)\]
		\[\varphi(g_1K) = \varphi(g_2K) \os{?}{\Ra} g_1K = g_2 K\]
		Докажем, что это биекция. Что сюръекция --- очевидно
		\[f(g_1) = f(g_2) \RA g_1 g_2^{-1} \in K \RA g_1K = g_2 K \]
		Почему? Потому что:
		\[e g_1 = g_1 \in K g_1 \q (e \in K)\]
		\[g_1 g_2^{-1} g_2 = g_1 \in K g_2 \q (g_1 g_2^{-1} \in K)\]
		Классы $K g_1$ и $K g_2$ либо не пересекаются, либо совпадают ($K$ --- подгруппа $G$)$\RA K g_1 = K g_2$
		\[\underbracket{f(g1)f(g_2)^{-1}}_{= f(g_1)f(g_2^{-1})}  = e \]
	\end{proof}

	\begin{Reminder}
		\[\SL_N(K) \text{ --- квадратные матрицы с det = 1}\]
	\end{Reminder}

	\begin{example}
		\begin{enumerate}
			\item $\det: \GL_n(K) \ra K^* \q \text{матрица - определитель}$\\
				Но это отображение --- сюръекция, а значит:
				\[\SL_n(K) \text{ - матрицы, $\det=1$}\]
				\[\SL_n(K) \text{ - ядро}\]
				\[\GL_n(K) \big/_{\SL_n(K)} \cong K^*\]
				\[\SL_n(K) = \{A \in M_n(K) : \abs{A} = 1\}\]
			\item $S_n \to \{\pm 1\}$
				\[S_n / _{A_n} \cong \{\pm 1\}\q (\cong \Z/_{2\Z} ) \]
				\[(A_n \text{ --- чётные перестановки})\]
			\item $G \times H \to G$
				\[(g_1 h) \to g\]
				\[G \times H \big/_{e \times H} \cong G\]
		\end{enumerate}
	\end{example}

	\subsection{Действие группы на множестве}
	\begin{definition}
		$M \text{ --- множество }, G \text{ --- группа}$
		\[G \times M \to  M\]
		\[(g, m) \to gm\]
		\begin{enumerate}
			\item $g_1(g_2m) = (g_1g_2)m \q \forall g_1 g_2 \in G, \q m \in M$
			\item $em = m \q \forall m \in  M$
		\end{enumerate}
		Если задано такое отображение, то говорим, что группа G действует на множестве M
	\end{definition}

	\begin{examples}
		\begin{enumerate}
			\item $A = k^n \q\q (A, v) \to A_v$
			\[G = \GL_n(K)\]
			\[A(B_v) = (AB)_v\]
			\[E_v = v\]
			\item M = \{количество раскрасок вершин квадрата в два цвета\}
			\[G = D_4\]
			\[ \begin{align}
					&\text{ч} & \text{ч}\\
					&\text{б} & \text{ч}
			\end{align} = \begin{align}
				  &\text{ч} & \text{б}\\
				  &\text{ч} & \text{ч}
			\end{align} \]
			\item $M = G$
			\[gm = gm\]
		\end{enumerate}
	\end{examples}

	\hsubsection{1.13}{$\Stab$ и $\Orb$}
	\begin{definition}
		$m \in M$
		\[\Stab \ m = \{g \in G: gm = m\} \text{ --- стабилизатор}\]
		\[\Orb \ m = \{gm,\  g \in G\} \text{ --- орбита}\]
	\end{definition}

	\begin{Utv}
		\[\Stab \ m < G\]
	\end{Utv}

	\begin{proof}
		Доказательство того, что стабилизатор --- подгруппа:
	    \begin{enumerate}
	    	\item $g_1, g_2 \in Stab \ m$
				\[(g_1 g_2)m = g_1(\underbracket{g_2m}_{= m } ) = g_1m = m\]
			\item $e \cdot m = m$
			\item $gm = m \os{?}{\Ra} g^{-1}m = m $
				\[gm = m\]
				\[\underbracket{g^{-1}gm}_{= (g^{-1}g)m = em = m}  = g^{-1}m \]
	    \end{enumerate}
	\end{proof}

	\begin{Utv}
		\[m_1, m_2 \in M\]
		\[m_1 \sim m_2 \text{, если }\exists g \in G: gm_1 = m_2\]
		\[\Ra \sim \text{ --- отношение эквив.}\]
	\end{Utv}

	\begin{Proof}
		\[\text{(симм.) }gm_1 = m_2 \Ra g^{-1}m_2 = m_1 \q g^{-1} \in G \]
		\[\text{(рефл.) }em = m, \q e \in G\]
		\[\text{(тран.) }\begin{align}
			gm_1 = m_2 \\
			g'm_2 = m_3
		\end{align}
		\bigg| \Ra (g'g)m_1 = g'(gm_1) = g'm_2 = m_3\]
	\end{Proof}

	\begin{Utv}
		\[\abs{\Orb \ m} \cdot \abs{\Stab \ m} = \abs{G}\]
	\end{Utv}

	\begin{Proof}
		\[\Stab \ m = H\]
		\[\{gH, \ g \in G\} \to Orb \ m\]
		\[gH \to gm\]
		Хотим доказать, что это корректно
		\[gH = g'H \os{?}{\Ra} gm = g'm\]
		\[g' = ga, \q a \in H\]
		\[g'm = (ga)m = g(am) = gm\]
		Хотим доказать биективность. Сюръективность --- очев. Инъективность:
		\[gm = g'm \Ra gH = g'H\]
		\[m = em =(g^{-1}g)m= g^{-1}(gm) = g^{-1}(g'm) = (g^{-1}g')m\]
		\[\Ra g^{-1}g' \in H \Ra gH = g'H\]
	\end{Proof}

	\hsubsection{1.14}{Лемма Бернсайда}
	\begin{Lemma}[Бернсайда]
		\[\text{Кол-во орбит } = \frac{1}{\abs{G}} \sum_{g \in G} \abs{M^g}\]
		\[M^g = \{m \in M: gm = m\}\]
	\end{Lemma}
\end{lect}
\end{document}
