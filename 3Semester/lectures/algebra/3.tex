\documentclass[main]{subfiles}

\begin{document}
\begin{lect} {2019-09-17}
	\begin{utv}
	    $|G|=p$,\q $p$ --- простое $\RA G \cong \Z /p \Z$
	\end{utv}

	\begin{proof}
	    $g \in G$, $g \neq e$, $\ord g=p$

	    $\Ra G=\{e=g^0,g,...,g^{p-1}\}$
	\end{proof}

	\begin{utv}
	    $H,G$ --- группы, $\varphi: G \ra H$ --- изоморфизм $\Ra$ $n=\ord g=\ord \varphi(g)$
	\end{utv}

	\begin{proof}
	    Пусть $g^n=e,\ \varphi(g^n)=\varphi(e)\os{?}{=}e$
	    \[\varphi(e)^2=\varphi(e^2)=\varphi(e)\]
		Домножим на обратный:
		\[\varphi(e) = e\]
	    Теперь докажем, что меньшего нет
	    \[\varphi(g)^m = e,\ m \in \N \os{?}{\Ra} m \geqslant n\]
		\[\varphi(g^m) = \varphi(g)^m = e = \varphi(e) \q \Ra g^m = e \Ra m \geq n\]
	\end{proof}

	\hsubsection{1.6}{Сопряжение элемента}
	\begin{definition}
	    $H<G$, тогда H --- нормальная подгруппа, если:
		\[\forall h \in H, g \in G \Ra g^{-1}h g \in H\]
		$g^{-1}h g$ --- сопряжение элемента h с помощью элемента g, обозначается: $H \triangleleft G$
	\end{definition}

	\begin{remark}
	    Элементы подгруппы при сопряжении переходят в элементы подгруппы
	\end{remark}

	\begin{remark}
	    Подгруппа любой коммутативной группы нормальна
	\end{remark}

	\begin{example}
	    $D_3$ --- 6 элементов, 3 поворота и 3 симметрии

	    \includegraphics[width = 5cm]{pics/triangle_d_3.png}

	    $\{e,l,r\}$ --- нормальная

	    $\{e, s_1\}$ --- не нормальная
	\end{example}

	\begin{utv}
	    $H \triangleleft G$ $\lra$ разбиение на Л и П классы смежности по H совпадают
	    \[\forall g \q gH = Hg\]
	\end{utv}

	\begin{proof}
	    $(\Ra):$\\
	    \[h \in H \q g h \in g H\]
	    \[g h = \underbrace{(g^{-1})^{-1} h g^{-1}}_{\in H} g = h_1 g\]
	    $(\La):$\\
	    \[g \in G,\q h \in H,\q g^{-1}h g=h_1\]
	    \[h g \in H g = g H \Ra g h_1, h_1 \in H\]
	\end{proof}

	\hsubsection{1.7}{О классах смежности}
	\begin{Definition}[умножение классов смежности]
	    \[H \triangleleft G\]
	    \[g_1 H * g_2 H \eqdef g_1 g_2 H\]
	\end{Definition}

	\begin{proof}[корректности]
	    Хотим проверить, что
	    \[\w{g}_1 H = g_1 H,\q \w{g}_2 H = g_2 H \os{?}{\Ra} \w{g_1}\w{g_2}H = g_1 g_2 H\]
	    Аналогично прошлому доказательству
	    \[g_2^{-1}h_1 g_2 = h_3 \in H \]
	    \[\widetilde{g_1}\widetilde{g_2}h = g_1 h_1 g_2 h_2 h = g_1 g_2 (\us{= h_3}{g_2^{-1}h_1 g_2})h_2 h\]
	    \[\widetilde{g_1}H = g_1 H \Ra \widetilde{g}_1 = g_1 h_1\]
	    \[\widetilde{g_2}H = g_2 H \Ra \widetilde{g_2} = g_2 h_2\]
	    Не использовали условие $g_2^{-1} h_1 g_2 = h_3 \in H$
	    \[\w{g_1} \w{g_2} H = g_1 h_1 g_2 h_2 h = g_1 g_2 \us{=h_3}{(g_2^{-1} h_1 g_2) }h_2 h\]
	    Осталось доказать, что получается группа
	    \[\text{1) Нейтральный}\q e H=H,\q e H * g H = (e g) H = g H\]
	    \[\text{2) Ассоциативность}\q (g_1 H * g_2 H)*g_3 H \os{?}{=} g_1 H*(g_2 H * g_3 H)\]
	    \[(g_1 g_2)H * g_3 H = (g_1 g_2)g_3 H\]
	    \[\text{3) Обратный}\q gH * g^{-1}H = (g g^{-1})H = eH \]
	\end{proof}

	\begin{Remark}
	    \[G/H\]
	    \[\text{Была эквивалентность: }a \sim b \rla a - b \ \vdots \ h\]
	    \[G = \Z\]
	    \[H=n \Z,\q g_1 g_2^{-1} \in H\text{ --- мульт. запись },\q g_1-g_2 \in n \Z\text{ --- адд. запись}\]
	    \[[a] + [b] = [a + b]\]
	    Аддитивная группа кольца классов вычетов --- это то же самое, что фактор группа группы $\Z$ по подгруппе $n\Z$
	\end{Remark}

	\hsubsection{1.8}{Про коммутанты}
	\begin{definition}
	    Как в произвольной группе найти подгруппу?

	    $[g,h]=g h g^{-1} h^{-1}$, $g,h \in G$ --- коммутатор элементов $h,g \in G$

	    Коммутант --- мн-во произведений всех возможных коммутаторов

	    Обозначается $K(G)=\{[g_1,h_1]...[g_n,h_n],\ g_i,h_i \in G\}$
	\end{definition}

	\begin{proof}[коммутант --- подгруппа]
	    $K(G)<G$\\
		Замкнутость относительно операций?
		\[\text{произв. $n$ комм. * $m$ комм. = произв. $(n+m)$ комм.}\]
	    Нейтральный элемент:
		\[[e,e]=e\]
	    Обратный элемент?
		\[[g_1,h_1]...[g_n,h_n]\]
	    Как его найти?
		\[[g,h]^{-1}=(g h g^{-1} h^{-1})^{-1}=h g h^{-1} g^{-1}=[h,g]\]
	    %\[([g_1,h_1]...[g_n,h_n])^{-1}=[g_1,h_1]...[g_n,h_n]\]
        \[([g_1, h_1]...[g_n, h_n])^{-1}  = [h_n, g_n]...[h_1, g_1] \]
	    Значит это подгруппа. Нормальная ли?
		\[g^{-1}[g_1,h_1]...[g_n,h_n]g\]
		\[g^{-1} [g_1,h_1] g (g^{-1} [g_2,h_2]g)...(g^{-1} [g_n, h_n] g)\]
	    Нужно доказать, что сопряжение коммутатора лежит в коммутанте
		\[g^{-1} g_1 h_1 g_1^{-1} h_1^{-1} g =
        \underbrace{g^{-1} g_1 h_1 g_1^{-1} g h_1^{-1}}_{=[g^{-1} g_1,h_1]}
        \underbrace{h_1 g^{-1} h_1^{-1} g}_{=[h_1,g^{-1}]}\]
	\end{proof}

	\begin{utv}
	    Фактор-группа ($G / K(G)$) по коммутанту --- коммутативна
	\end{utv}

	\begin{Proof}
	    \[g_1, g_2 \in G \q\q g_1 K(G) g_2 K(G) \os{?}{=} g_2 K(G) g_1 K(G)\]
        \[g_1 K(G) g_2 K (G) = g_1 g_2 K(G) \q\q g_2 K(G) g_1 K(G) = g_2 g_1 K(G)\]
		\[[g_1, g_2] = g_1 g_2 (g_2 g_1)^{-1} \in K(G) \]
		\[g_1 g_2 g_1^{-1} g_2^{-1} \cdot K(G) = K(G) \text{ (по лемме $aH = H \q \forall a \in H$)}\]
		Домножим справа на $g_2 g_1$:
		\[g_1 g_2 g_1^{-1} g_2^{-1} K(G) g_2 g_1 = K(G) g_2 g_1\]
		Т.к. $K(G)$ - норм. подгруппа $G$, то $K(G) = g_2 g_1 K(G)$
		\[\Ra \ g_1 g_2 K(G) = g_2 g_1 K(G)\]
	\end{Proof}

	\begin{Utv}
	    \[\Z_n \times \Z_m \simeq \Z_{mn} \text{, если } (m, n) = 1 \]
	\end{Utv}

	\begin{proof}
	    Нужно построить изоморфизм $[a]_{m n} \mapsto    ([a]_n,[a]_m)$
		\[[a]_{m n} = [a']_{m n} \Ra [a]_n = [a']_n$, $[a]_m=[a']_m\]
	    Теперь нужно проверить биекцию

	    Сюръективность:
		\[\forall b,c \in \Z$ $\e x \in \Z: \begin{cases}[x]_n=[b]_n\\ [x]_m=[c]_m \end{cases},\text{ по КТО всё хорошо}\]
	    Инъективность:
		\[\left[\begin{matrix}
			[a]_n = [b]_n\\
			[a]_m = [b]_m
		\end{matrix}\right. \RA [a]_{m n} = [b]_{m n}\]
	    На языке сравнений:
		\[\begin{matrix}
			a \equiv b(n)\\
			a \equiv b(m)
		\end{matrix} \RA a \equiv b (m n)\]
	    На самом деле достаточно было проверить одно
	\end{proof}

	\hsubsection{1.9}{Гомоморфизм}
	\begin{Definition}
		\[\varphi : G \to H \text{ --- гомоморфизм, если } \varphi(g_1 g_2) = \varphi(g_1) \varphi(g_2)\]
		\[\text{изоморфизм = гомоморфизм + биекция}\]
		\[\varphi \in \text{Hom}(G, H) \text{ --- множество гомоморфизмов}\]
	\end{Definition}

	\begin{examples}
		\begin{enumerate}
			\item $\CC^* \to \R^*$
			\[z \to |z|\]
			\item $GL_n(K) \to K^*$
			\[A \to \det A\]
			\item $S_n \to  \{\pm 1\}$
			\[\sigma \to \left\{ \begin{align}
				&+1,& &\text{ если } \sigma \text{ --- четн.}\\
				&-1,& &\text{ если } \sigma \text{ --- неч.}
			\end{align}\]
			\item $a \in G \q G \to G$
			\[g \to a^{-1}g a\]
			\[(a^{-1}g a)(a^{-1}g_1a) = a^{-1} g g_1 a\]
		\end{enumerate}
	\end{examples}
\end{lect}
\end{document}
