\documentclass[main]{subfiles}

\begin{document}
	\begin{Property}
	    \[(U^{\bot})^{\bot} = U\]
	\end{Property}

	\begin{Proof}
		\[\left. \begin{align}
			\dim U^{\bot} + \dim U = \dim V\\
			\dim(U^{\bot})^{\bot} + \dim U^{\bot} = \dim V
		\end{align} \right|
		\Ra \dim(U^{\bot})^{\bot} = \dim U\]
		Докажем, что $U \subset (U^{\bot})^{\bot}$
		\[(U^{\bot})^{\bot} = \{ v \in V: (v,\ w) = 0 \q \forall w \in U^{\bot}\}\]
		Возьмём $v \in U$, тогда $\forall v \in U$ это будет выполняться. С учетом равенства $\dim$ пр-ва равны
	\end{Proof}

	\begin{Definition}
		\[U < V, \q v \in V\]
		\[U \oplus U^{\bot} = V\]
		\[\Ra \e! u \in U,\ w \in U^{\bot}: v = u + w\]
		u называется \ul{ортогональной проекцией}
		\[\text{Обозначение: } \pr_U v \eqdef u\]
		%картинка про ортогональную проекцию
		\[v = \pr_U v + w \Ra (v,u) = (\pr_U v, u)\]
	\end{Definition}

	\begin{properties}[орт. проекции]
		\begin{enumerate}
			\item $\pr_U (v+v') = \pr_U v + \pr_U v'$
				\[v = u + w,\ u \in U, w \in U^{\bot}\]
				\[v' = u' + w',\ u \in U,\ w' \in U^{\bot}\]
				\[v+v' = \us{\in U}{(u+u')} + \us{\in U^{\bot}}{(w+w')}\]
				%картинка про проекции 2%
			\item $\Abs{v - \pr_U v} \leqslant \Abs{v - u}\q \forall u \in U$
				\[\Abs{v - u}^2 = \us{\in U^{\bot}}{\Abs{v - \pr_U v}^2} + \us{\in U}{\Abs{\pr_U v - u}^2}\]
				(по т. Пифагора $U \bot U^{\bot}$)
		\end{enumerate}
	\end{properties}

	\begin{definition}
		$e_1,...,e_n$ --- базис V
		\[\text{Базис называется \ul{ортогональным}, если }(e_i, e_j)=0 \q \forall i \neq j\]
		\[(e_i,e_j) = \delta_{i,j} = \left[ \begin{align}
			0, & i \neq j\\
			1, & i = j
		\end{align} \right.\]
	\end{definition}

	\newpage
	\subsection{Процесс ортогонализации Грама-Шмидта. Ортогональные и ортонормированные базисы, примеры. Ортогональная проекция в ортонормированном базисе. Коэффициенты Фурье}

	\begin{alg}
		Процесс ортогонализации Грамма-Шмидта:
		\[e_1,...,e_n \text{ --- базис}\]
		\begin{multline*}
			\text{Хотим ортонормированный} f_1,...,f_n:\\
			<f_1,...,f_k> \ = \ <e_1,... e_k>\q \forall 1 \leqslant k \leqslant n:
		\end{multline*}
		Строим по индуции:\\
		Б.И. k=1:
		\[f_1 = \dfrac{1}{\Abs{e_1}} e_1\]
		И.П. $k-1 \ra k$:
		\[f_k = e_k + \sum_{i=0}^{k-1} \lambda_i f_i\]
		\[\text{Хотим это }(f_k, f_j) \os{?}{=} 0 \q 1 \leqslant j \leqslant k-1\]
		\[(f_k, f_j) = (e_k, f_j) + \sum_{i=1}^{k-1} \lambda_i \us{= \lambda_j}{(f_i, f_j)}\]
		\[\text{Тогда сделаем }\lambda_j = - (e_k, f_j) \q \forall 1 \leqslant j \leqslant k-1\]
		Ортонормируем $f_k$, чтобы $(f_k, f_k)=1$
	\end{alg}

	\begin{utv}
		Если $e_1,...,e_n$ --- ОНБ U
		\[\pr_U v = \sum_{i=1}^n (v, e_i) e_i\]
	\end{utv}

	\begin{proof}
		Хотим доказать $v - \sum_{i=1}^n (v, e_i) e_i \in U^{\bot}$\\
		Достаточно доказать, что вектор ортогонален любому
		\[\us{1 \leqslant j \leqslant n}{(v - \sum_{i=1}^n (v, e_i) e_i) e_j} = (v, e_j) - \sum_{i=1}^n (v, e_i) (e_i, e_j)\]
	\end{proof}

	\begin{Example}
		\[\R^n\]
		\[(x; y) = \sum x_i y_i\]
		\[e_i = (0, 0, ..., \us{i}{1}, ..., 0)\]
	\end{Example}

	\begin{Example}
		\[T_n = \{a_0 + \sum_{k = 1}^n a_k \cos kx + \sum^n_{k = 1} b_k \sin kx\}\]
		\[(f; g) = \int_0^{2 \pi} fg dx \]
		\[\left\{\frac{1}{\sqrt{2 \pi}}, \frac{1}{\sqrt{\pi}} \cos kx \us{k = 1,..., n}{}, \
		\frac{1}{\sqrt{\pi}} \sin kx \us{k = 1, ..., n}{} \right\}\]
		\begin{multline*}
		\pr_{T_n} f = \frac{1}{2\pi} \int_0^{2\pi} f(x)dx  +
		\frac{1}{\pi} \sum^n_{k = 1} \left(\int_0^{2\pi} f(x) \cos (kx) dx \right) \cdot
		\cos kx  +\\ \frac{1}{\pi} \sum^n_{k = 1} \left(\int_0^{2\pi} f(x) \sin(kx)dx \right) \cdot \sin kx
		\end{multline*}
	\end{Example}

	\newpage
	\subsection{Группа ортогональных матриц. Ортогональные операторы, эквивалентные определения, примеры}
	\begin{Definition}
		\[A \in M_n(K) \text{ назыв. \ul{ортогональной}, если } A^TA = E\]
		\[O_n(K) \text{ --- множество орт. матриц}\]
	\end{Definition}

	\begin{Utv}
		\[O_n(K) \text{ --- группа по умножению}\]
	\end{Utv}

	\begin{Proof}
	    \[\begin{align}
    		A^T A = E\\
			B^T B = E
		\end{align} \  \bigg| \Ra (AB)^T AB = B^T \underbrace{A^TA}_E B = B^T B = E\]
		\[A^T A = E \Ra A^{-1} = A^T \]
		\[(A^{-1})^T A^{-1} \os{?}{=} E \]
		\[(A^T)^T A^{-1}  = AA^{-1}  = E\]
	\end{Proof}

	\begin{Utv}
		\[L \in \mathscr{L}(V) \text{ (пр-во лин. функционалов)}\]
		Следуюшие утверждения равносильны:
		\begin{enumerate}
			\item $(Lv,\ Lv') = (v,\ v') \q \forall v, v' \in V$
			\item $\Abs{Lv} = \Abs{v} \q\q \forall v \in V$
			\item $[L]_e \in O_n(\R), $ если $e$ --- ортонорм. базис
		\end{enumerate}
	\end{Utv}

	\begin{proof}
		$(2 \Ra 1)$:
		\[\Abs{v+v'}^2 = \Abs{v}^2 + 2(v,\ v')+\Abs{v}^2\]
		\[(v,\ v') = \frac{1}{2}(\Abs{v + v'}^2 - \Abs{v}^2 - \Abs{v'}^2)\]
		$(3 \Ra 2)$:
		\[[L v]_e = [L]_e [v]_e\]
		\[\Abs{L v}^2 = (L v, L v) = [L v]^T_e \Gamma_e [L v]_e = [L v]^T_e [L v]_e = \]
		\[= [v]_e^T \underbrace{[L]_e^t [L]_e}_{= E} [v]_e  = [v]^T_e [v]_e =
		[v]^T_e \Gamma_e [v]_e = (v,\ v) = \Abs{v}^2\]
		$(1 \Ra 3)$:
		\[\mathcal{E}_i^T [L]_e^T [L]_e \mathcal{E}_j\]
		\[\mathcal{E}_i = (0, ..., \us{i}{1}, ..., 0)\]
		\[\mathcal{E}_i^T A \mathcal{E}_j = a_{ij} \]
		\[\mathcal{E}_i = [e_i]_e\]
		\[\mathcal{E}_j = [e_j]_e\]
		\[[e_i]_e^T [L]_e^T [L]_e [e_j]_e = [L {e_i}]_e^T [L {e_j}]_e = [L {e_i}]_e^T \Gamma_e
		[L {e_j}]_e = (L {e_i},\ L {e_j}) = (e_i,\ e_j) = \delta_{ij} \]
	\end{proof}
\end{document}
