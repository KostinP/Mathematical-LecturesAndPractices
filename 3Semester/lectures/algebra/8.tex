\documentclass[main]{subfiles}

\begin{document}

    \subsection{Ортогональные операторы на плоскости и в пространстве}

    \begin{definition}
        L --- орт. оператор на плоскости, $\det L = 1$, тогда L --- поворот
    \end{definition}

    e --- ортонорм. базис, $[L]_e = \begin{pmatrix}
        a & b\\
        c & d
    \end{pmatrix}$
    \[\begin{pmatrix}
        a & c\\
        b & d
    \end{pmatrix} \begin{pmatrix}
        a & b\\
        c & d
    \end{pmatrix} = \begin{pmatrix}
        1 & 0\\
        0 & 1
    \end{pmatrix}\]
    \[\begin{cases}
        a^2 + c^2 = 1\\
        b^2 + d^2 = 1\\
        ab + cd = 0\\
        ad - bc = 1
    \end{cases}\]
    \[a = \cos \varphi,\q c = \sin \varphi\]
    \[b = \sin \psi,\q d = \cos \psi\]
    \[\us{= \sin(\varphi + \psi)}{\cos \varphi \sin \psi + \sin \varphi \cos \psi} = 0\]
    \[\us{= \cos(\varphi + \psi)}{\cos \varphi \cos \psi - \sin \varphi \sin \psi} = 0\]
    \[\Ra \varphi + \psi = 0\]
    \[\begin{pmatrix}
        \cos \varphi & - \sin \varphi\\
        \sin \varphi & \cos \varphi
    \end{pmatrix}\]

    \begin{definition}
        Если L --- ортогональный оператор на пл-ти, $\det L = -1$\\
        S --- какая-то осевая симметрия\\
        Тогда:
        \begin{enumerate}
            \item $L = S \circ R_{\psi}$
            \item $L = R_{\varphi} \circ S$
        \end{enumerate}
    \end{definition}

    Рассмотрим $S^{-1} \circ L$ --- ортогональный оператор с определителем 1, значит по предыдущему определению $S^{-1} \circ L = R_{\varphi}$

    \begin{utv}[теорема Эйлера]
        В трехмерном пространстве ортогональное отображение с определителем 1 является поворотом относительно некоторой оси
    \end{utv}

    Следствие: берем две прямые. Поворачиваем сначала относительно одной, потом относительно другой. И их композицией будет поврот

    \begin{proof}[теоремы Эйлера]
        L --- орт. оператор в пр-ве
        \[\det L = 1\]
        \[\chi_L(t) \in \R[x],\q \deg \chi_L = 3\]
        $\lambda_1, \lambda_2, \lambda_3$ --- корни
        \[|\lambda_1| = |\lambda_2| = |\lambda_3| = 1\]
        Два варианта:
        \begin{enumerate}
            \item $\lambda_1, \lambda_2, \lambda_3 \in \R$
            \item $\lambda_1 \in \R$, $\lambda_2 = \ol{\lambda_3}$
        \end{enumerate}
        В 1 случае одно из $\lambda$ равно 1, пусть $\lambda_1$\\
        Во 2 случае $\lambda_1 = 1$ т.к. $\lambda_1 \lambda_2 \lambda_3 = \lambda_1 \ol{\lambda_2} \lambda_3 = \lambda_1 |\us{=1}{\lambda_3}|^2 = \lambda_1$\\
        С.в. остается неподвижным при повороте. Ось тоже. Значит собственный вектор  при повороте и есть ось\\
        Осталось д-ть, что ортогональное дополнение есть вращение. Тогда докажем, что наш исходный оператор --- вращение относительно оси
        \[\letus Lv = v\]
        \[v^{\bot}\]
        Докажем, что эта плоскость --- инвариантное подпространство. Нужно доказать:
        \[(u,v) = 0 \ra (Lu, v) = 0\]
        То есть результат будет тоже из ортогонального дополнения
        \[(Lu,v) = (Lu, Lv) = (u,v) = 0 \text{ ч.т.д.}\]
        Так как инвариантное подпространство, можем сузить L. Оно является плоскостью. Т.к. L --- орт. оператор, значит он сохраняет расстояние. Т.к. S тоже сохраняет расстояние, значит L является ортоганальным оператором на плоскости. Осталось убедиться, что модуль равен 1. Если исходный оператор сохраняет расстояние, то и его сужение сохраняет ориентацию. Другой способ: построим матрицу L в базисе: V, \{два ортогональных вектора на плоскости\}, матрица L будет такой:\\
        \[[L] = \begin{pmatrix}
            1 & 0 & 0\\
            0 & ? & ?\\
            0 & ? & ?
        \end{pmatrix}\]
        Вместо ? будет матрица сужения. Мы должны доказать, что это матрица поворота. Определитель большой матрицы равен определителю маленькой, но т.к. большая 1, то и он 1.

        По предыдущим рассуждениям --- это поворот. То есть у нас есть пространство с осью, на которую оператор действует тождественно, а на другое он действует как поворот.
    \end{proof}

    \begin{utv}
        Если L --- ортогональный оператор в пре-ве с определитем -1 равен композиции поворота, относительно оси и симметрии, то это поворот.
    \end{utv}

    \begin{proof}
        Аналогично
    \end{proof}

    \subsection{Собственные числа и собственные вектора унитарного оператора. Диагонализируемость унитарной матрицы.}

    \begin{utv}
		\begin{enumerate}
			\item $|\det L| = 1$
			\item Если L --- унитарный, $Lv = \lambda v \us{v \neq 0}{\Ra} |\lambda| = 1$
			\item $Lv = \lambda v \q Lu = \mu u \q \lambda \neq \mu | \Ra (u,\ v) = 0$
		\end{enumerate}
	\end{utv}

	\begin{proof}
		1 и 2:
		\[\Abs{v} = \Abs{Lv} = \Abs{\lambda v} = |\lambda| \Abs{v}\]
		3:
		\[(u,\ L^* v) = (u,\ \ol{\lambda} v) = \lambda (u,\ v)\]
		\[(u,\ L^* v) = (Lu,\ v) = (\mu u,\ v) = \mu(u,\ v)\]
		Хотим доказать: $Lv = \lambda v \Ra L^* v = \ol{\lambda} v$
		\[v = L^* L v = L^* (\lambda v) = \lambda L^* v\]
		Делим на $\lambda$ и туда переносится $\ol{\lambda}$
	\end{proof}

    \begin{theorem}
        Унитарный оператор имеет ортонормированный базис из с.в.
    \end{theorem}

    \begin{proof}
        Индукция по размерности пр-ва.\\
        Пусть одномерное пр-во ($n = 1$) --- очевидно, т.к. оператор-вектор v
        \[Lv = u,\q \Abs{u} = \Abs{v} \Ra u = \lambda v,\q \abs{\lambda} = 1\]
        Значит $Lv = \lambda v$ --- подходит, когда ортонормируем\\
        v --- с.в. L с каким-то $\lambda$
        \[Lv = \lambda v\]
        \[<v>^{\bot}\]
        Хотим доказать, что подпространство инвариантно относительно действия L:
        \[(v,\ u) = 0 \Ra (v,\ Lu) = 0\]
        \[(v,\ Lu) = (L^* v,\ u) \os{(*)}{=} (\ol{\lambda}v,\ u) = \ol{\lambda} (v,u) = 0\]
        $(*)$ т.к. мы доказывали, что у собственного оператора. Если v --- вектор унитарного оператора с с.ч. $\lambda$\\
        Раз исходный оператор унитарный, то сужение тоже унитарно. Значит мы можем применить индукционное предположение к сужению. На этом ортогональном дополнении у оператора есть базис ортогональных векторов. Добавим к нему отнонормированный вектор v. Очевидно, получим ортонормированный базис из собственных векторов всего пр-ва
    \end{proof}

    Переформулируем на языке матриц

    \begin{theorem}
        U --- унитарная матрица, тогда:
        \[U = MDM^{-1},\q D = \begin{pmatrix}
            \lambda_1 & ... & 0\\
            0 & ... & 0\\
            0 & ... & \lambda_k
        \end{pmatrix},\q |\lambda_i| = 1,\q M \text{ --- унитарная}\]
    \end{theorem}

    \begin{Proof}
        \[\CC^n \q Lz=Uz \qq [L]_e = U\]
        \[e\text{ --- есть базис }\CC^n\]
        \[[L^*L]_e = [L^*]_e [L]_e = [L]_e^* [L]_e = U^* U = E\]
        $(*)$ Из какого-то рассуждения получается\\
        $\Ra$ L --- унитарный оператор\\
        По теореме, которую доказали ранее, f --- ортонормированный базис $\CC^n$ из с.в. L
        \[D = [L]_f = M^{-1}_{e \ra f} \us{=U}{[L]_e} M_{e \ra f}\]
        $(*)$ У D --- на диагонали с.ч., по модулю равные 1\\
        Хотим д-ть: у нас есть два ОНБ, тогда матрица перехода между ними будет унитарна
        \[M_{e \ra f} = \{a_{ij}\}\]
        \[f_j = \sum a_{ij} e_i\]
        \[\delta_{jk} = (f_j,\ f_k) = \Br{\sum_i a_{ij} e_{ij},\ \sum_l a_{ij} \ol{a}_{lk} e_l} = \sum_{i,l} a_{ij} \ol{a}_{lk} (e_i,\ e_l) \sum a_{ij} \ol{a}_{ik}\]
    \end{Proof}

    \subsection{Эрмитовы матрицы и самосопряженные операторы. Собственные числа и собственные вектора самосопряженного оператора}
    \begin{definition}
        $A \in M_n (\CC)$ --- эрмитова, если $A^* = A$\\
        $L \in \mathscr{L}(V)$ --- самосопряженный, если $L^* = L$
    \end{definition}

    \begin{properties}
        \begin{enumerate}
            \item L --- самосопряженный, тогда $[L]_e$ --- эрмитова, если e --- ортонормированный
                \[[L]_e^* = [L^*]_e = [L]_e\]
            \item L --- самосопряженный, тогда с.ч.$\in \R$
                \[\letus Lv = \lambda v,\q v \neq 0\]
                \[\lambda(u,\ v) = (Lv,\ v) = (v,\ Lv) = (v,\ \lambda v) = \ol{\lambda} (v,\ v)\]
            \item $Lv = \lambda v\q Lu = \mu u \qq \lambda \neq \mu \Ra (u,\ v) = 0$
                \[\lambda (v,\ u) = (Lv,\ u) = (v,\ Lu) = (v,\ \mu u) = \mu (v,\ u)\]
        \end{enumerate}
    \end{properties}
\end{document}
