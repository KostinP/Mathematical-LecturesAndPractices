\documentclass[12pt, fleqn]{article}

\usepackage{../../../template/template}

 
\begin{document}
 
\begin{lect}{2019-12-17}\\
    \section{кодирование}
    ...\\
    Теперь возьмем поле из конечного кол-ва элементов.

    \begin{definition} [Линейное кодирование]
        Слово $x_0, ..., x_n$
        \[\begin{pmatrix}
            & & \\
            & A &\\
            & & 
        \end{pmatrix} \begin{pmatrix}
            x_0\\
            \vdots\\
            x_{n - 1} 
        \end{pmatrix} = \begin{pmatrix}
            y_0\\
            \vdots\\
            y_{n - 1} 
        \end{pmatrix} \to y_0, ..., y_{n - 1} \]
    \end{definition}

    \begin{Definition}[Полиномиальное кодирование]
        \[(a_0 + ta_1 + ... + t^ka_k)(x_0 + tx_1 + ... + t^{n - 1}x_{n - 1}  ) =
        (y_0 + ty_1 + ... + t^{m - 1}y_{m - 1}  )\]
    \end{Definition}
    
    \subsection{Код Боуза-Чоудхури-Хоквингена}
    \begin{Definition}
        \[\abs{F} = p^n\]
        Фиксируем $d \leq p^n - 1$\\
        Мы будем строить кодовые слова, расстояние между которыми не меньше, чем $d$
        \[N = p^n - 1 \text{ - длина кодовых слов}\]
        Рассмотрим $\alpha$ - примитивный элемент  \q(примитивный - образующий мультипл. группы)
        \[m_i \text{ - мин. мн-н для } \alpha^i \qq m_i \in \Z_{/p}\Z[x] \]
        \[g = \text{НОК}(m_1, ..., m_{d - 1} ) \]
        Утверждается, что код построенный по этому многочлену будет уд. условию, что расстояние будет не меньше, 
        чем $d$\\
        От противного:\\
        Есть два кода, расстояние между которыми не меньше $d$\\
        Рассмотрим разность этих многочленов (они отличаются не меньше, чем в $d$ разрядах)
        \[P(x) = b_1x^{k_1} + ... + b_{d - 1}x^{k_{d - 1} }, \q \text{ - разность }
        \q 0 \leq k_1 < ... < k_{d-1} \leq N = p^{n } - 1     \]
        $P$ делится на $g$
        \[\Ra P(\alpha) =  P(\alpha^2) = ... = P(\alpha^{d - 1} ) = 0\]
        \[\begin{cases}
            b_1\alpha^{k_1} + ... + b_{d-1}\alpha^{k_{d - 1} } = 0 \\
            b_1\alpha^{2k_{1} } + ... + b_{d-1}\alpha^{2k_{d-1} } = 0\\
            ...\\
            b_1\alpha^{(d-1)k_1} + ... + b_{d - 1}\alpha^{(d - 1)k_{d - 1} } = 0
        \end{cases}\]
        \[C = \begin{pmatrix}
            \alpha^{k_1} & \alpha^{k_2} & ... & \alpha^{k_{d - 1} }\\
            \alpha^{2k_1} & \alpha^{2k_2} & ... & \alpha^{2k_{d - 1} }\\
            \\
            \alpha^{(d - 1)k_1} & \alpha^{(d - 1)k_2} & ... & \alpha^{(d - 1)k_{d - 1} }   
        \end{pmatrix} \begin{pmatrix}
            b_1\\
            b_2\\
            \vdots\\
            b_{d - 1} 
        \end{pmatrix} = \begin{pmatrix}
            0\\
            0\\
            \vdots\\
            0
        \end{pmatrix}\]
        \[\det C = \alpha^{k_1 + ... + k_{d - 1} } \prod_{1 \leq i < j \leq d - 1}(\alpha^{k_j} - \alpha^{k_i}) 
        \neq 0\]
        \[\Ra b_i = 0 \ \forall i\]
        Значит, эти два слова должны совпадать
    \end{Definition}

    \begin{Example}
        \[p = 2, \q n = 4 \qq F = \Z_{/2}\Z[t]\big/_{(t^4 + t + 1)}  \]
        \[\alpha = \overline{t}\]
        Рассмотрим порядок $\alpha$, если он не 1, не 3 и не 5, то он 15\\
        Порядок, действительно, 15
        \[m_1(x) = t^4  + t + 1\]
        \[m_2(x) = t^4 + t + 1\]
        \[m_3(x) = t^4 + t^3 + t^2 + t + 1\]
        \[m_5(x) = t^2 + t + 1\]
        \[m_{7}(x) = t^4 + x^3 + 1 \]
        \[m_1 = m_2 = m_4 = m_8 \]
        \[m_3 = m_6 = m_9\]
        \[m_5 = m_{10} \]
        \[m_7 = m_{11}  = m_{13}  = m_{15} \]
        \[d = 2, 3 \q g = x^4 + x + \overline{1}\]
        \[d = 4, 5 \q g = x^8 + x^7 + x^6 + x^4 + \overline{1}\]
        \[d = 6, 7 \q g = x^{10} + x^8 + x^5 + x^4 + x^3 + x^2 + x + \overline{1} \]
        \[d = 8 \ \ \q g = x^14 + ... + \overline{1}\]
    \end{Example}
\end{lect}

\end{document}
