\documentclass[main]{subfiles}

\begin{document}
	\begin{lect} {2019-10-22}
		\hsubsection{2.10}{Унитарные пространства}
		\begin{definition}[унитарного пространства]
			$U$ - в.п. над $\CC$\\
            \[(\cdot,\cdot) : \ U \times U \ra \CC \q \text{ - эрмитово скал. произведение}\]
			\begin{enumerate}
				\item $(u+v,\ w) = (u,\ w) + (v,\ w)\q \forall u,v,w \in U$
				\[(\lambda v,\ w) = \lambda(v,w)\q \forall \lambda \in C,\q v,w \in U\]
				\item $(u,\ v) = \ol{(v,\ u)}$
				\item $(u,\ u) \geqslant 0$
				\item $(u,\ u) = 0 \Ra u = 0$
			\end{enumerate}
            $(U, (\cdot, \cdot))$ \text{ - унитарное пространство}
		\end{definition}

		\begin{Example}
			\[\begin{tabular}{c|c}
				\R^n  & \CC^n\\
				(x,y) = \sum x_i y_i & (x,y) = \sum x_i \ol{y_i}
			\end{tabular}\]
		\end{Example}

		\[e_1,...,e_n \text{ - базис}\]
		\[\Gamma_e = \{(e_i,\ e_j)\}_{i,j} \text{ - матрица грама}\]
		\[(u,v) = [u]_e^T \Gamma_e \ol{[v]}_e\]
		\[\Gamma_f = M_{e \ra f}^T \Gamma_e \ol{M}_{e \ra f}\]

		\[|(u,v)| < \Abs{u} \cdot \Abs{v},\q \Abs{u} = \sqrt{(u,\ u)}\]
		\[\Abs{tu + v}^2 = t^2 \Abs{u} + t(\us{=2\real (u,v)}{(u,\ v) + (v,\ u)}) + \Abs{v}^2\]
		\[\real (u,v) \leqslant \Abs{u}^2 \Abs{v}^2\]
		\[(u,\ v) = |(u,\ v)| \cdot z |\Ra |z| = 0\]
		\[\real (\frac{1}{z} u,\ v) \leqslant \Abs{\frac{1}{z} u}^2 \Abs{v}^2 = \Abs{u} \Abs{v}\]
		Напоминание: $\Abs{\lambda u} = \sqrt{(\lambda u,\ \lambda u)} = \sqrt{\lambda \ol{\lambda} (u,u)}
        =\abs{\lambda} \Abs{u}$
		\[\real \frac{1}{z}(u,\ v) = \real \abs{(u,v)} = \abs{(u,\ v)}\]
		Доказали КБШ

		\hsubsection{2.11}{Сопряжение?}
		\begin{definition}
			V - в.п. над K
			\[V^* = \mathscr{L}(V,\ K) \text{ - двойственное пр-во}\]
		\end{definition}
		\begin{Example}
			\[v \in V \text{ - евклидово пр-во (унитарное)}\]
			\[\varphi_v(w) = (w,\ v) \q \varphi_v: V \ra \R (\CC)\]
		\end{Example}
		Хотим доказать: $\varphi \in V^* \Ra \e! v\in V: \varphi = \varphi_v$
		\begin{proof}
			$e_1,...,e_n$ - ОНБ V
			\[v = \sum \lambda_i  e_i\]
			Нужно$\q \forall w \in V \q (w,\ v) = \varphi(w)$, т.к. $\varphi$ - линейный функционал
			\[\lra \forall j \q (e_j,\ v) = \varphi(e_j)\]
			\[\q (e_j,\ \sum \lambda_i e_i) = \sum_i \ol{\lambda}_i (e_j,\ e_i)\]
		\end{proof}

		\hsubsection{2.12}{Сопряженная матрица}
		\begin{definition}
			$A \in M_n (\CC)$
			\[A^* = \ol{A}^T \text{ - эрмитово-сопряженная матрица}\]
		\end{definition}

		\begin{properties}
			\begin{enumerate}
				\item $A^{**} = A$
				\item $(\lambda A)^* = \ol{\lambda} A*$
				\item $(A+B)^* = A^* + B^*$
				\item $(AB)^* = B^* A^*$
				\item $(A^{-1})^* = (A^*)^{-1}$
			\end{enumerate}
		\end{properties}

		\begin{utv}
			V - унитарное пр-во, $L \in \mathscr{L}(V)$, $u \in V$
			\[\varphi_n (v) = (Lv,\ u) \in V^*\]
			\[\Ra (Lv,\ u) = (v,\ w_u)\]
			\[\e! w_u \in V:\q (v,\ u) = (v,\ w_u)\]
			\[u \ra w_u\]
			Утверждается, что отображение линейно
		\end{utv}

		\begin{Proof}
			\[\begin{tabular}{c|c}
				(Lv,\ u) = (v,\ w_u)  & (Lv,\ u+u') = (Lv,\ u) + (Lv,\ u') =\\
				(Lv,\ u') = (v,\ w_{u'}) & = (u\ w_u) + (v,\ w_{u'}) = (v,\ w_u + w_{u'}) = (v,\ w_{u + u'})
			\end{tabular}\]
			\[(L v,\ \lambda u) = \ol{\lambda} (Lv,\ u) = \ol{\lambda} (v,\ w_u) = (v,\ \us{= w_{\lambda u}}{\lambda w_u})\]
			\[L^* u = w_u \q (Lv,\ u) = (v,\ L^* u)\]
		\end{Proof}

		\hsubsection{2.13}{Эрмитов сопряженный оператор}
		\begin{definition}
			$L^*$ - эрмитов сопряженный оператор
		\end{definition}

		\begin{properties}
			\begin{enumerate}
				\item $L^{**} = L$
				\[(L^* v,\ u) = (v,\ L^{**} u)\]
				\[(L^* v,\ u) = \ol{(u,\ L*v)} = \ol{(Lu,\ )} = (v,\ Lu)\]
				\[\Ra L^{**} u =Lu \q \forall u \in V\]
				Почему так? $(v,\ w) = (v,\ w')\q \forall v \Ra w = w'$
				\[(v,\ w-w') = 0\]
				\[v = w-w'\]
				\[\Abs{w-w'}^2 = 0\]
				\[\Ra w-w'=0\]
				\item $(\lambda L)^* = \ol{\lambda} L^*$
				\[(\lambda  L) v,\ u) = (v,\ (\lambda L)^* u)\]
				\[(\lambda  L) v,\ u) = (\lambda \cdot L v,\ u) = \lambda(Lv,\ u) = \lambda(v,\ L^*u) = (v,\ \ol{\lambda} L^* u)\]
				\item $(L+L')^* = L^* + L'^* \text{ аналогично}$
				\item $(LNv,\ u) = (v,\ (LN)^* u)$
				\[(LNv,\ u) = (v,\ N^* L^* u) \text{ и то же, что делали раньше}\]
				\item $[L]_e^* = [L^*]_e$, если e - ОНБ
				\[L e_i = \sum a_{li} e_l \q [L]_e = \{a_{ij}\}\]
				\[L e_j = \sum b_{kj} e_k \q [L]_e = \{b_{kj}\}\]
				\[(\us{=a_{ij}}{L e_i, e_j}) = (\us{=\ol{b}_{ij}}{e_i,\ L^* e_j})\]
			\end{enumerate}
		\end{properties}

		\hsubsection{2.14}{.}
		\begin{definition}
			$A \in M_n(\CC)$
			\[A \text{ - унитарная, если }A^*A=E\]
			\[U_n = \{ A \in M_n(\CC): \text{(то что сверху)} \}\]
		\end{definition}

		\begin{Proof}[что это группа по умножению]
			\[\begin{tabular}{c|}
				A^*A = R \\
				B^*B=E
			\end{tabular} \Ra (AB)^* AB = B^* \ub{=E}{A^* A} B = E\]
			\[(A^{-1})^* A^{-1} \os{?}{=} E\]
			\[\lra (A^{-1})^* = A\]
			\[\lra (A^*)^{-1} = (A^{-1})^{-1}\]
			Докажем, что любая унитарная матрица обратима и модуль определителя равен единице
			\[A^* A = E\]
			\[\ol{\det A} \cdot \det A = 1\]
			\[|\det A|^2 = 1\]
		\end{Proof}

		\hsubsection{2.15}{Унитарный оператор}
		\begin{Utv}
			\[L \in \mathscr{L}(V)\]
			Следующие условия равносильны:
			\begin{enumerate}
				\item $\Abs{Lv} = \Abs{v} \q \forall v$
				\item $(Lv,\ Lu) = (v,u) \q \forall v,u$
				\item $[L]_e \in U_n,\q e \text{ - ортонорм.}$
				\item $L^*L = \id_V$
			\end{enumerate}
			И оператор, удовлетворяющий этим условиям называется "унитарным" (в евклидовом случае называется "ортогональным")
		\end{Utv}

		\begin{proof}
			($4 \Ra 2$):
			\[\us{=(v,u)}{(v,\ L^*Lu)} = (Lv,\ Lu)\]
			($2 \Ra 4$):
			\[(v,\ L^* L u) = (Lv,\ Lu) = (v,\ u) \]
		    $L^*L = \id_V$
		\end{proof}

		\begin{utv}
			\begin{enumerate}
				\item $|\det L| = 1$
				\item Если L - унитарный, $Lv = \lambda v \us{v \neq 0}{\Ra} |\lambda| = 1$
				\item $Lv = \lambda v \q Lu = \mu u \q \lambda \neq \mu | \Ra (u,\ v) = 0$
			\end{enumerate}
		\end{utv}

		\begin{proof}
			1 и 2:
			\[\Abs{v} = \Abs{Lv} = \Abs{\lambda v} = |\lambda| \Abs{v}\]
			3:
			\[(u,\ L^* v) = (u,\ \ol{\lambda} v) = \lambda (u,\ v)\]
			\[(u,\ L^* v) = (Lu,\ v) = (\mu u,\ v) = \mu(u,\ v)\]
			Хотим доказать: $Lv = \lambda v \Ra L^* v = \ol{\lambda} v$
			\[v = L^* L v = L^* (\lambda v) = \lambda L^* v\]
			Делим на $\lambda$ и туда переносится $\ol{\lambda}$

		\end{proof}
	\end{lect}
\end{document}
