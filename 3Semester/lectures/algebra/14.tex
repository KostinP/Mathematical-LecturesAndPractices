\documentclass[main.tex]{subfiles}

\begin{document}
    \begin{Utv}
        \[K/_\Omega \qq \Char K = p \qq \abs{\Omega} = p\]
        \[\alpha_1, ..., \alpha_n \in K \text{ --- базис } K \text{ над } \Omega\]
        \[\Ra \forall \alpha \in K \q \alpha = \xi_1 \alpha_1 + ... + \xi_n \alpha_n, \q \xi_i \in \Omega\]
    \end{Utv}

    \begin{utv}
        Если есть два поля одинаковой мощности, то они изоморфны
    \end{utv}

    \begin{Utv}
        \[K\big/_{\displaystyle L},\ N\big/_{\displaystyle K} \text{ - конечные расширения}\]
        \[\Ra N\big/_{\displaystyle L} \text{ --- конечно и }[N:L] = [N:K] [K:L]\]
    \end{Utv}

    \begin{Proof}
        \[K\big/_{\displaystyle L} \text{ --- конечно} \RA \e \beta_1,...,\beta_n \in K:
        \forall \beta \in K \q \e! \alpha_1,...,\alpha_n \in L: \]
        \[\beta = \alpha_1 \beta_1 + ... + \alpha_n \beta_n\]
        \[N\big/_{\displaystyle K} \text{ --- конечно} \RA \e \gamma_1,...,\gamma_m \in N:
        \forall \gamma \in N \q \e! \w{\beta}_1,...,\w{\beta}_m \in L: \]
        \[\gamma = \w{\beta}_1 \gamma_1 + ... + \w{\beta}_m \gamma_m\]
        \[\{\beta_i \gamma_j\}_{\begin{matrix}
            1 \leq  i \leq n\\
            1 \leq j \leq m
        \end{matrix}}\]
        Докажем, что это действительно базис $N$:
        \[\text{Возьмём }\gamma \in N\]
        \[\gamma = \sum \w{\beta}_j \gamma_j = \sum(\sum \alpha_{ij} \beta_i) \gamma_j,\q \alpha_{ij} \in L\]
        Теперь нужно док-ть линейную независимость
        \[\sum \delta_{ij} \beta_i \gamma_j = 0 \q \delta_{ij} \in L\]
        \[\sum_j (\sum_i \us{\in K}{\delta_{ij} \beta_i}) \gamma_i = 0\]
        Так как базис, то в каждой скобке стоит ноль, снова применяем это рассуждение
        \[\Ra \delta_{ij} = 0\]
    \end{Proof}

    \begin{Example}
        \[\R/_{\Q} \]
        \[\CC/_\R\]
    \end{Example}

    \begin{remark}
        Такая конструкция называется башней расширения
    \end{remark}

    \begin{Reminder}
        \[G \text{ --- группа},\q \ord a = s \q a^t = e \RA t \devides s\] %три точки
    \end{Reminder}

    \begin{utv}
        $G$ --- абелева ($=$коммутативная) группа
        \begin{enumerate}
            \item $\left. \begin{matrix}
                \ord a = n\\
                \ord b = m\\
                (n,m) = 1
            \end{matrix} \right| \RA \ord ab = nm$
                \[(ab)^{nm} = a^{nm} b^{nm} = e\]
                \[\text{Предположим }(ab)^k = e \RA \us{= a^{nk} b^{nk} = b^{nk}}{(ab)^{nk}} \RA nk \devides m \RA k \devides m\]
                \[\text{Аналогично } k \devides n\]
                \[\left. \begin{matrix}
                  k \devides m\\
                  k \devides n
                \end{matrix} \right| \RA k \devides m\]
            \item $\left. \begin{matrix}
                \ord a = n\\
                \ord b = m
            \end{matrix} \right|\RA \e n',m': \left.
            \begin{matrix}
                n \devides n',\ m \devides m'\\
                (n', m') = 1\\
                n'm' = \text{НОК}(n,m)
            \end{matrix} \right|\RA \e c \in G: $\\
                \[\ord c = \text{НОК}(m,n)\]
                Док-во первой части:
                \begin{enumerate}
                    \item $\text{Пусть } n = p^{\alpha},\ m = \p^{\beta},\q \alpha \geq \beta$
                        \[n' = \p^{\alpha},\q m' = 1\]
                    \item $n = p_1^{\alpha_!} ... p_s^{\alpha_s},\q m = p_1^{\beta_1} ... p_s^{\beta_s}$
                \end{enumerate}
                Док-во второй части: достаточно д-ть, что $\e$ эл-ты порядка $n',m'$
                \[\text{Пусть } n = n' m'\]
                \[a^n = e\]
                \[\RA (a^{s'})^{n'} = e\]
                \[? \ord a^{s'} = n'\]
                \[(a^{s'})^t = e \q (t < e') \RA a^{st} = e \RA st < s'n' = n\]
                Противоречие с порядком\\
                Значит мы нашли эл-т порядка $n'$, аналогично порядка $m'$. Пользуемся предыдущим пунктом и утверждение доказано
        \end{enumerate}
    \end{utv}

    \subsection{Построение неприводимого многочлена над конечным полем заданной степени. Примеры. Количество элементов поля, построенного с помощью неприводимого многочлена}

    Поняли, как строить определенные поля. Как строить любые?
    \begin{Utv}
        \[f \in \Z \big/_{\displaystyle p \Z}[x] \text{ --- непр.}\q \deg f = n\]
        \[\abs{\Z\big/_{\displaystyle p\Z} [x]\big/_{\displaystyle (f)}} = p^n\]
    \end{Utv}

    \begin{Proof}
        \[g \in \Z\big/_{\displaystyle p\Z}[x]\]
        Поделим с остатком на f:
        \[g = fh + r,\q \deg r < n\]
        Утверждается, что в фактор-кольце лежат такие элементы:
        \[\ol{\alpha_1 + \alpha_1 x + ... + \alpha_{n-1} x^{n-1}} \in \Z\big/_{\displaystyle p\Z}[x]\big/_{\displaystyle (f)}\]
        Всего таких классов $p^n$ в силу произвольности выбора
        \begin{enumerate}
            \item Докажем, что любой элемент поля равен одному из них
                \[\ol{g} = \ol{n}, \text{ т.к. }\deg r < n\]
            \item Докажем, что что никакие два элемента не совпадают
                \[\letus \ol{\alpha_1 + \alpha_1 x + ... + \alpha_{n-1} x^{n-1}} = \ol{\beta_1 + \beta_1 x + ... + \beta_{n-1} x^{n-1}}\]
                Рассмотрим $\ol{\alpha_1 + \alpha_1 x + ... + \alpha_{n-1} x^{n-1} - \beta_1 + \beta_1 x + ... + \beta_{n-1} x^{n-1}} =
                0 \RA \alpha_1 + \alpha_1 x + ... + \alpha_{n-1} x^{n-1} - \beta_1 + \beta_1 x + ... + \beta_{n-1} x^{n-1} \in (f)$

                Многочлен $\deg = n$ делится на многочлен $\deg < n$, такое может быть только тогда,
                когда многочлен нулевой
        \end{enumerate}
        Научились строить поля, у которых $p^n$ элементов\\
        Пусть хотим найти многочлен степени 6 над конечным полем\\ \ \\
        Как это сделать?\\
        Рекурсивно. Составляем список унитарных мн-ов степени 2. Вычеркиваем все, у которых есть корень (подставляем элементы нашего конечного поля). Мн-н неприводим, когда у него нет корней.\\
        Дальше составляем список унитарных мн-ов степени 3.\\
        Составляем список мн-ов степени 4. Вычеркиваем все,
        у которых есть корень и которые делятся на мн-ны степени 2...\\
        За конечное время можно получать такие списки
        А можно ли сделать поле из 24 элементов? Нельзя.
    \end{Proof}

    \subsection{Мультипликативная группа конечного поля. Примитивные элементы, примеры}
    \begin{theorem}
        Мультипликативная группа конечного поля циклическая
    \end{theorem}

    \begin{proof}
        Пусть $|K^*| = m$ (мультипликативная группа)
        \[\alpha \in K^* \text{ --- макс. порядка}, \q \ord \alpha = s\]
        По следствию из теоремы Лагранжа $m \geq s$
        \[\beta \in K^* \q \deg \beta = r\]
        \[\e \gamma \in K^*: \ord \gamma = \text{НОК}(s,r) \geq s\]
        \[\Ra \text{НОК}(s,r) = s \RA s \devides r\]
        \[\beta^r = 1 \RA \beta^s = 1\]
        Рассмотрим $x^s - 1 \in K[x]$. Доказали, что каждый ненулевой элемент будет корнем. Значит у него по крайней мере $m$ корней\\ \ \\
        Степень многочлена не превосходит числа корней
        \[\Ra s \geq m \RA s = m\]
        Значит есть образующий элемент и группа циклическая
    \end{proof}

    \begin{Utv}
        \[K\big/_{\displaystyle L} \text{ --- конечно},\q \alpha \in K\]
        \[\Ra \e f \in L[x]: f(\alpha) = 0\]
    \end{Utv}

    \begin{Proof}
        \[\text{Пусть }[K:L] = n\]
        \[\text{Рассмотрим }1,\alpha, \alpha^2,...,\alpha^n \in K\]
        Если рассматривать это как вектора, то они ЛЗ
        \[\Ra \e \gamma_i: \sum \gamma_i \alpha^i = 0\]
        \[f(x) = \sum \gamma_i x^i\]
    \end{Proof}

    \subsection{Минимальный многочлен, существование и единственность, примеры. Степень минимального многочлена. Изоморфность конечных полей одинаковой мощности}
    \begin{definition}
        $f \in L[x]$ --- минимальный мн-н для $\alpha \in K$ (в расширении $K\big/_{\displaystyle L}$), если:
        \begin{enumerate}
            \item $f(\alpha) = 0$
            \item $g(\alpha) = 0 \RA \deg g \geq \deg f$
                \[g \in L[x]\]
        \end{enumerate}
    \end{definition}

    \begin{example}
        Мн-н минимальной степени в $\R$ у которого корень $i$ --- это $x^2 + 1$
    \end{example}

    \begin{properties}
        \begin{enumerate}
            \item $f$ --- минимальный мн-н над $L$ --- неприводим\\
                Док-во:
                \[f = gh,\q g,h \in L[x]: \deg g < deg f,\q \deg h < f\]
                \[\us{=0}{f(\alpha)} = f(\alpha) h(\alpha)\]
                Противоречие
            \item $g(\alpha) = 0,\q g \in L[x] \RA g \devides f$\\
                Док-во:
                \[g = fh + r,\q \deg r < \deg f,\q r \in L[x]\]
                \[\us{=0}{g(\alpha)} = \us{=0}{f(\alpha) h(\alpha)} + r(\alpha)\]
                \[\Ra r \text{ --- тожд. мн-н}\RA \text{$g$ делится на $f$ без остатка}\]
                \begin{consequence}
                    Значит минимальный многочлен единственный с точностью до ассоциированности
                \end{consequence}

                \begin{consequence}
                    Унитарный минимальный многочлен единственный
                \end{consequence}
            \item $[K:L] \devides \deg f$\\
                Рассмотрим $L(\alpha):= \{\lambda_0 + \lambda_1 \alpha + ... + \lambda_s \alpha^s,\q \lambda_i \in L\} \subset K$\\
                Хотим доказать, что это поля. Очевидно кроме
                \[\left. \begin{matrix}
                    \varphi(\alpha) \neq 0\\
                    \varphi \in L(x)
                \end{matrix} \right| \RA \varphi(\alpha)^{-1} \in L(\alpha)\]
                Рассмотрим $(\varphi, f) \os{\text{f --- неприв.}}{=} \left[\begin{matrix}
                    f\\
                    1
                \end{matrix}\right \begin{matrix}
                    \Ra \varphi \devides f \text{ --- невозможно}\\
                    \
                \end{matrix}$
                \[\Ra (\varphi, f) = 1 \RA 1 = \varphi h + fg,\q h,g \in L[x]\]
                \[1 = \varphi(\alpha) h(\alpha) + \us{=0}{f(\alpha) g(\alpha)}\]
                \begin{remark}
                    Получили башню расширения: $K - L(\alpha) - L$
                \end{remark}
                \[[K:L] = [K:L(\alpha)] \us{\os{?}{=} \deg f}{[L(\alpha):L]}\]
                \[\deg f = n\]
                \[1,\alpha,\alpha^2 ... \alpha^{n-1} \text{ --- базис $L(\alpha)$ над $L$?}\]
                \begin{enumerate}
                    \item ЛН?
                        \[\sum_{i=0}^{n-1} c_i \alpha^i = 0,\q c_i \in L\]
                        \[\psi(x) = \sum_{i=0}^{n-1} c_i x^i \in L[x]\]
                        \[\begin{matrix}
                            \psi(\alpha) = 0\\
                            \deg \psi \leq n-1
                        \end{matrix} \RA \psi = 0 \RA c_i = 0\]
                    \item Порождаемость?
                        \[\varphi(\alpha) \in L(\alpha),\q \varphi \in L[x]\]
                        \[\varphi = fg + r,\q \deg f < n,\q r \in L[x]\]
                        \[\varphi(\alpha) = f(\alpha) g(\alpha) + r(\alpha)\]
                        т.е. $r(\alpha)$ --- ЛК базисных векторов
                \end{enumerate}
        \end{enumerate}
    \end{properties}
    
    \begin{utv}
        $F_1, F_2$ --- изоморфны, если $\exists \varphi : F_1 \to F_2:$
        \begin{enumerate}
            \item  $\varphi(a + b) = \varphi(a) + \varphi(b)$
            \item $\varphi(ab) = \varphi(a)\varphi(b)$
            \item $\varphi$ --- биекция
        \end{enumerate}
    \end{utv}

    \begin{Utv}[предложение]
        \[\abs{F_1} = \abs{F_2} = p^n \Ra F_1 \cong F_2\]
    \end{Utv}

    \begin{Proof}
        \[\Z_{/p}\Z[x]\Big/_{(f)}, \q f \in \Z_{/p}\Z[x] \qq \deg f = n  \]
        \[\abs{F} = p^n\]
        \[\overline{x}^{p^n} = \overline{x} \text{ в } \Z_{/p}\Z[x] \Big/_{(f)}   \]
        \[\Ra x^{p^n} - x \devides f \]
        \[\forall \alpha \in F \q \alpha^{p^n} = \alpha \]
        \[x^{p^n} - x \text{ над }F(\text{произв. поле}) \]
        \[x^{p^n} - x = \prod_{\alpha \in F}(x - \alpha)  \]
        \[\Ra \exists \alpha \in F : f(\alpha) = 0\]
        \[\Z_{/p}\Z[x]\bigg/_{(f)} \to F  \]
        \[\varphi(\overline{x}) \to \varphi(\alpha), \q \varphi \in \Z_{/p}\Z[t] \]
        \[\text{Нужно ядро} = 0\]
        \[\varphi(\overline{x}) \q \varphi(\alpha) = 0 \q f(\alpha) = 0 \q f \text{ --- непр}\]
        \[\Ra \varphi \devides f\]
    \end{Proof}

    \subsection{Разложение многочлена xq-x на неприводимые множители, где q – степень p. Количество неприводимых унитарных многочленов данной степени над конечным полем}
\end{document}
