\documentclass[main]{subfiles}


\begin{document}
  \begin{lect} {2019-10-01}
  	\begin{Reminder}
  			\[\text{Кол-во орбит } = \frac{1}{\abs{G}} \sum_{g \in G} \abs{M^g} \]
  			\[M^g = \{m \in M : gm = m^2\}\]
  	\end{Reminder}

  	\begin{Proof}
  		\[\sum_{g \in G} \abs{M^g} = \abs{\{(g, m) \in G \times M : gm = m\}} = \]
  		\[ = \sum_{m \in M} \abs{Stab \ m} = \abs{G} \sum_{m \in M} \frac{1}{\abs{Orb \ m}} =
  		\abs{G} \cdot \text{Кол-во орбит}\]
  	\end{Proof}

  \section{Евклидовы и унитарные пр-ва}
  \hsubsection{2.1}{Скалярное умножение}
  	\begin{Definition}
  	    \[V \text{ - в.п. над } \R\]
  		Введем отображение
  		\[V \times V \to \R \]
  		\[(u, v)\]
  		Свойства этого отображения
  		\begin{enumerate}
  			\item Симметричность
  				\[(u, v) = (v, u) \q \forall u, v \in V\]
  			\item Линейность
  				\[(\lambda u, v) = \lambda(u, v)  \qq \lambda \in \R \q u, v \in V\]
  				\[(u + u', v) = (u, v) + (u', v) \qq u, u', v \in V\]
  			\item $(u, v) \geq 0 \qq \forall u \in V$
  				\[(u, u) = 0 \rla u = 0\]
  		\end{enumerate}
  		Такое пр-во $V$ с введенным на нем таким отображением мы называем \ul{Евклидовым пр-вом},
  		а отображение \ul{скалярным}.
  	\end{Definition}

  	\begin{Reminder}
  		\[C = \{c_{ij}\}_{i, j = 1}^n  \text{ - квадр. матрица}\]
  		\[Tr \ C = \sum_{i = 1}^n c_{ii} \text{ - след (Trace)}  \]
  		(Сумма элементов главной диагонали)
  	\end{Reminder}

  	\begin{examples}
  		\begin{enumerate}
  			\item Школьные вектора
  			\item $\R^n$
  				\[((a_1, ..., a_n), (b_1, ..., b_n)) = \sum^n_{i = 1} a_i b_i \]
  			\item $V = \R[x]_n$ конечномерное пр-во
  				\[(f, g) = \int_a^b fg dx\]
  			\item $V = M_n(\R)$
  				\[(A, B) = Tr \ AB^T\]
  				(См. след в напоминании)
  		\end{enumerate}
  	\end{examples}

    \hsubsection{2.2}{Матрица Грама}
  	\begin{Definition}
  	    \[e = \{e_1, ..., e_n\} \text{ - базис } V\]
  		\[a_{ij} = (e_i, e_j) \]
  		\[\Gamma_e = \{a_{ij}_{i,j = 1}^n  \}\text{ - матрица Грама}\]
  	\end{Definition}

  	\begin{properties} [матрицы Грама]
  		\begin{enumerate}
  			\item Матрица невырожд
  			\item $e, f$ - базисы
  				\[\Gamma_f = M^T_{e \to f} \Gamma_e M_{e \to f}  \]
  			\item $\Gamma_e = \{a_ij\}$
  				\[u = \sum \lambda_i e_i\]
  				\[v = \sum \mu_j e_j\]
  				\[(u, v) = (\sum \lambda_i e_i, \sum \mu_j e_j) = \sum_{i,j} \lambda_i \mu_j
  				(e_i, e_j)\]
  				\[(u, v) = [u]_e^T \Gamma_e [v]_e\]
  		\end{enumerate}
  	\end{properties}

  	\begin{proof}
  	    \begin{enumerate}
  	    	\item $\letus \abs{\Gamma_e} = 0  \Ra \exists \lambda_i \in \R \text{ не все 0}:$
  				\[\sum \lambda_i (e_i, e_j) = 0 \q \forall j\]
  				\[\left(\sum \lambda_i e_i, \  e_j \right) = 0 \q \forall j\]
  				\[\left(\sum_{i} \lambda_i e_i, \ \sum_j \lambda_j e_j \right) = 0 \rla
  				\sum \lambda_i e_i = 0\]
  				противоречие
  			\item $\letus M_{e \to f} = \{a_{ik} \} \qq f_k = \sum a_{ik} e_i  $
  				\[f_l = \sum a_{jl} e_j \]
  				\[(f_k, f_l) = \sum_{i,j} a_{ik}a_{jl} (e_i, e_j)\]
  				\[a_{ik} (e_i, e_j) a_{je}  \]
  				Напоминание: $X, Y$- матр $ \qq X\times Y = Z \q z_{ij}  = \sum x_{is} y_{sj}  $
  	    \end{enumerate}
  	\end{proof}

    \hsubsection{2.3}{Норма}
  	\begin{Definition}
  	    \[V \text{ - в.п. над } \R\]
  		\[V \to \R_{\geq 0} \]
  		\[v \to \Abs{v} \text{ - норма}\]
  		\begin{enumerate}
  			\item $\Abs{\lambda v} = \abs{\lambda} \Abs{v} \q \forall \lambda \in \R \q v \in V$
  			\item Нер-во треугольника
  				\[\Abs{u + v} \leq \Abs{u} + \Abs{v}\]
  			\item $\Abs{u} = 0 \rla u = 0$
  		\end{enumerate}
  		Если такое отобр. существует, то оно называется \ul{нормой}
  	\end{Definition}

  	\begin{Utv}
  		\[(u, v) \text{ - ск. пр-ве}\]
  		\[\Ra \Abs{u} = \sqrt{(u, u)}\]
  	\end{Utv}

  	\begin{Example}
  			\[\R^n\]
  			\[\Abs{x} = \max \abs{x_i}\]
  			\[\Abs{x} = \sum_{i} \abs{x_i}\]
  	\end{Example}

    \hsubsection{2.4}{Нер-во Коши - Буняковского}
  	\begin{Theorem} [Нер-во Коши - Буняковского]
  		\[\abs{(u, v)} \leq \Abs{u} \cdot \Abs{v}\]
  	\end{Theorem}

  	\begin{Proof}
  		\[\varphi(t) = \Abs{u + rv}^2 = (u + tv, u + tv) = \Abs{u}^2 + 2 (u, v)t + t^2 \Abs{v}^2\]
  		\[D = 4(u, v)^2 - 4 \Abs{u}^2 \Abs{v}^2 \leq 0\]
  		\[\Abs{u + v} \leq \Abs{u} + \Abs{v}\]
  		\[(u + v, u + v) \leq \Abs{u}^2 + \Abs{v}^2 + 2 \Abs{u}\Abs{v}\]
  		\[(u + v, u + v) = \Abs{u}^2 + \Abs{v}^2 + 2(u, v)\]
  		\[2(u, v) \leq 2 \Abs{u} \Abs{v}\]
  	\end{Proof}

  	\begin{Utv} [Теорема Пифагора]
  		\[\text{Если } u \perp v \Ra \Abs{u + v}^2 = \Abs{u}^2 + \Abs{v}^2\]
  	\end{Utv}

  	\begin{Proof}
  		\[\Abs{u + v}^2 = \Abs{u}^2 + \Abs{v}^2 + 2 (u, v)\]
  	\end{Proof}

    \hsubsection{2.5}{Ортогональное дополнение}
  	\begin{Definition} [Ортогональное дополнение]
  			\[V \text{ - евкл. пр-во}\]
  			\[U \subset V \qq U^\perp = \{v \in V : (v, u) = 0 \q \forall u \in U\}\]
  			Множество всех векторов, которые ортогональны всем векторам из $U$\\
  			Такое мн-во называется \ul{ортогональным дополнением}
  	\end{Definition}

  	\begin{Utv}
  			\[U^\perp \text{ - под-пр } V\]
  	\end{Utv}

  	\begin{Proof}
  		\[\begin{align}
  				&(v, u) = 0 \q \forall u\\
  				&(v', u) = 0 \q \forall u
  		\end{align}
  		\Ra (v + v', u) = 0 \q \forall u\]

  		\[(v, u) = 0 \q \forall u\]
  		\[\lambda \in \R\]
  		\[(\lambda v, u) = 0 \q \forall u\]
  		Тогда $U^\perp$ дей-во линейное под-прво $V$
  	\end{Proof}

  	\begin{Properties}
  		\[V = U \oplus U^{\perp} \]
  		\[u \in U \cap U^\perp\]
  		\[u \in U \q u \in U^\perp\]
  		\[(u, u) = 0\]

  	\end{Properties}

  	\begin{Proof}
  		\[e_1, ..., e_n \text{ - базис } U  \text{ дополняем до базиса V}\]
  		\[e_1, ..., e_n, f_1, ..., f_n \text{ - базис }V\]
  		\[v \in U^\perp \q v = \sum \lambda_i e_i + \sum \mu_j f_j\]
  		\[v \in U^\perp \rla (v, e_k) = 0 \q \forall 1 \leq k \leq n\]
  		\[(v, e_k) = \sum \lambda_i(e_{i}, e_k) + \sum \mu_j (f_j, e_k) = 0 \q\q \forall 1 \leq k \leq n\]

  		это матрица
  		\[\begin{tabular}{c | c|c}
  				& n & m\\
  				\hline
  			n & \Gamma_e & C\\
  			\hline
  		\end{tabular}
  		\begin{pmatrix}
  			x\\
  			y
  		\end{pmatrix}
  		=
  		\begin{pmatrix}
  			0\\
  			0
  		\end{pmatrix}\]
  		\[\Gamma_e x + C_y = 0\]
  		\[\{(x, y) \in \R^n \times \R^m : \Gamma_e x + C_y = 0\} \text{ - размерность этого }m\]
  		\[(x, y) \to y\]
  		\[\Gamma_e x + C_y = 0\]
  		\[x = -\Gamma^{-1}_e e_y \]
  		\[\dim U + \dim U^\perp = \dim V\]
  	\end{Proof}
  \end{lect}
\end{document}
