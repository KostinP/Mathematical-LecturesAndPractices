\documentclass[main]{subfiles}

%сам документ
\begin{document}
\begin{lect} {2019-09-10}
			\begin{consequence}
			    G - кон. группа,\q $a \in G$, $\ord a = m$, $H=\{a^n:n \in \Z\}$, \ тогда $|H|=m$
	    \end{consequence}

			\begin{proof}
			    $\{a^0=e,a_1,...,a^{m-1}\}$ - подмножество H\\
			    Докажем, что все остальные элементы тоже здесь есть:
			    \[n \in \Z \RA n=m q+r,\q 0 \leqslantr \leqslant m-1\]
			    \[a^n=a^{m q+r}=(a^m)^q a^r=a^r\]
			    \[a^k=a^l$,\q $0 \leqslant k \leqslant l \leqslant m-1,\q \text{умножим на} a^{-k}\]
			    \[e=a^{l-k},$ $\q 0 \leqslant l-k \leqslant m-1\q \text{m - наименьшее $\N$ такое что $a^m=e$}\]
			    \[l-k=0 \RA l=k\]
			    Докажем, что $|H|=m$:
			    \[\Ra |G| \devides m=\ord a\]
					Т.о. в группе порядок эл-та - делитель порядка группы
			\end{proof}

	    \begin{reminder}[теорема Эйлера]
	        $n,a \in \N$,\q $(a,n) = 1$,\q тогда $a^{\varphi(n)} \equiv 1 \ (\text{mod } n)$
	    \end{reminder}

			\begin{proof}
	        Рассмотрим $G=(\Z _{/n} \Z, *)$ $\q |G|=\varphi(n)$
			    \[\ol{a} \in G$, $\ord \ol{a}=k\]
			    \[\varphi(n) \devides k \Ra \varphi(n) = kl\]
			    \[\ol{a}=\ol{1}\]
			    \[\ol{a}^{\varphi(n)}=\ol{1}\]
			\end{proof}

		\hsubsection{1.3}{Циклическая группа}
		\begin{definition}
		    G - циклическая группа, если:
				\[\e g \in G: \forall g' \in G: \e k \in \Z: g'=g^k\]
		    Такой g называется образующим
		\end{definition}

		\begin{definition}
		    $\Z$ (образующий - единица и минус единица)
		\end{definition}

		\begin{remark}
		    Любая циклическая группа - коммутативна
		\end{remark}

		\begin{proof}
		    $g' g'' = g'' g' = g^k g^l = g^l g^k$
		\end{proof}

		\begin{definition}
				Пусть G,H - г	руппы, рассмотрим
				\[G \times H = \{(g,h): g\in G, h\in H\}\]
				Введем операцию
				\[(g,h)*(g',h')\overset{def}{=}(g*_G g', h*_H h')\]
				Докажем, что это группа.\\
		\end{definition}

		\begin{Proof}[ассоциативности]
				\[((g,h)(g',h'))(g'',h'') \overset{?}{=} (g,h)((g',h')(g'',h'')\]
				\[(gg',hh')(g'',h'') \overset{?}{=} (g,h)(g' g'', h' h'')\]
				\[((gg')g'',(hh')h'') \overset{?}{=} (g(g',g''),h(h'h'') \text{ - очевидно}\]
		\end{Proof}

		\begin{Example}
				\[\Z /_2 \Z \times \Z /_2 \Z = \{(\ol{0},\ol{0}),(\ol{0},\ol{1}),(\ol{1},\ol{0}),(\ol{1},\ol{1})\}\]
		\end{Example}

		\begin{definition}
		    Конечная группа порядка n является циклической тогда и только тогда, когда она содержит элемент порядка n
				\[(|G|=n, \text{ G - циклическая }\equiv\ \e g \in G: \ord g = n)\]
		\end{definition}

		\begin{example}
				Рассмотрим $\Z /_2 \Z \times \Z /_3 \Z$ - циклическая
				\[((\ol{1},\ol{1}), (\ol{0}, \ol{2}), (\ol{1}, \ol{0}), (\ol{0}, \ol{1}), (\ol{1},\ol{2}))\]
				Рассмотрим $\Z /_2 \Z \times \Z /_4 \Z$ - не циклическая
		\end{example}

		\begin{definition}
		    $\varphi: G \ra H$ - биекция и $\varphi(g_1,g_2)=\varphi(g_1) \varphi(g_2)\q \forall g_1,g_2 \in G$,\\ тогда $\varphi$ - изоморфизм
		\end{definition}

		\begin{examples}
		    \begin{enumerate}
		        \item $D_3 \ra S_3$
		        \item $U_n=\{z\in \CC: z^n=1\} \la \Z / n \Z$\\
		        \[(\cos \frac{2\pi a}{n}+i \sin \frac{2\pi a}{n} = \varphi \ol{a} \mapsfrom \ol{a})\]
		        \[\ol{a}=\ol{b} \ra \varphi(\ol{a})=\varphi(\ol{b})\]
		        \[\varphi(\ol{a}+\ol{b}) \overset{?}{=} \varphi(\ol{a})\varphi(\ol{b})\]
		        \[\cos \frac{2\pi(a+b)}{n}+i \sin \frac{2\pi(a+b)}{n}=(\cos\frac{2\pi a}{n} + i \sin \frac{2\pi a}{n})\]
		    \end{enumerate}
		\end{examples}

		\hsubsection{1.4}{Изоморфные группы}
		\begin{definition}
		    Две группы называются изоморфными, если между ними существует изоморфизм
		\end{definition}

		\begin{utv}
		    Изоморфизм - отношение эквивалентности
		\end{utv}

		\begin{proof}
		    Т.к. композиция изоморфизмов - изоморфизм $G \overset{e}{\ra} H \overset{\psi}{\ra} H$
		    \[(\psi \circ \varphi)(g_1 g_2) = \psi(\varphi(g_1 g_2) = \psi(\varphi(g_1) \varphi(g_2)) = \]
				\[= \psi(\varphi(g_1)) \psi(\varphi(g_2)) = (\psi \circ \varphi)(g_1) \circ (\psi \circ \varphi)(g_2)\]
		    Рефлексивность - тождественное отображение - изоморфизм\\
		    Транзитивность: $G \underset{\varphi}{\ra} H$, $H \underset{\varphi^{-1}}{\ra} G$
		\end{proof}

		\hsubsection{1.5}{Теорема о циклических группах}
		\begin{theorem}
		    G - циклическая группа
				\begin{enumerate}
						\item $|G|=n \Ra G \cong \Z / n \Z$\\
						\item $|G|=\infty \Ra G \cong \Z$
				\end{enumerate}
		\end{theorem}

		\begin{proof}
				\begin{enumerate}
						\item g - обр. G, значит $G=\{e,g,g^2,...,g^{n-1}\}$\\
						(среди них нет одинаковых)
						\[\text{Построим изоморфизм в $\Z / n \Z$:$\q \varphi(g^k)=\ol{k}$}\]
				    \[\text{Проверим, что $\varphi(g^k g^l)= \varphi(g^k)+\varphi(g^l)=\ol{k}+\ol{l}$}\]
				    \[\text{Левая часть: $\varphi(g^{k+l}=\ol{(k+l) \mod n} = \ol{k}+\ol{l}$}\]
						\item $G=\{...,g^{-1},e,g,g^2,...\}$\\
						(тоже нет совпадающих элементов, иначе $g^k=g^l$, при $k>l$, тогда $g^{k-l}=e$, но тогда конечное число элементов, потому что оно зацикливается через каждые $k-l$ элементов), построим отображение в $\Z$.\\
				    $\varphi(g^n)=n$ - очевидно, биекция.\\
						Нужно доказать, что $\varphi(g^n g^k)=\varphi(g^n)-\varphi(g^k)=n+k$
				\end{enumerate}
		\end{proof}
\end{lect}
\end{document}
