\documentclass[main]{subfiles}

%сам документ
\begin{document}
	\begin{lect} {2019-09-10}

		\begin{consequence}
		    G - кон. группа, $a \in G$, $\ord a = m$, $H=\{a^n:n \in \Z\}$, тогда $|H|=m$\\
		\end{consequence}

		\begin{proof}
		    $\{a^0=e,a_1,...,a^{m-1}\}$ - подмножество H\\
		    Докажем, что все остальные элементы тоже здесь есть\\
		    $n \in \Z \Ra n=m q+r$, $0 \leqslantr \leqslant m-1$\\
		    $a^n=a^{m q+r}=(a^m)^q a^r=a^r$\\
		    $a^k=a^l$, $0 \leqslant k \leqslant l \leqslant m-1$, умножим на $a^{-k}$\\
		    $e=a^{l-k}$ $o \leqslant l-k \leqslant m-1$ m - наименьшее $\N$ такое что $a^m=e$\\
		    $l-k=0 \Ra l=k$\\
		    Докажем, что $|H|=m$\\
		    $\Ra |G| \devides m=\ord a$, т.о. в группе порядок эл-та - делитель порядка группы
		\end{proof}

		Напоминание

		\begin{consequence}[теорема Эйлера]
		    $n,a \in \N$, $(a,n)=1$, тогда $a^{\varphi(n)} \equiv 1 (\mod n)$
		\end{consequence}

		\begin{proof}
		    Рассмотрим $G=(\Z / n \Z)*$ $|G|=\varphi(n)$\\
		    $\overline{a} \in G$, $\ord \overline{a}=k$\\
		    $\varphi(n) \devides k \Ra \varphi(n) = kl$\\
		    $\overline{a}=\overline{1}$\\
		    $\overline{a}^{\varphi(n)}=\overline{1}$
		\end{proof}

	\hsubsection{1.3}{Циклическая группа}
	\begin{definition}
	    G - циклическая группа, если $\e g \in G: \forall g' \in G: \e k \in \Z: g'=g^k$\\
	    Такой g называется образующим
	\end{definition}

	\begin{definition}
	    $\Z$ (образующий - единица и минус единица)
	\end{definition}

	\begin{remark}
	    Любая циклическая группа - коммунитативна
	\end{remark}

	\begin{proof}
	    $g' g'' = g'' g' = g^k g^l = g^l g^k$
	\end{proof}
	\\
	Пусть G,H - группы, рассмотрим $G \times H = \{(g,h): g\in G, h\in H\}$\\
	Введем операцию $(g,h)*(g',h')\overset{def}{=}(g*_G g', h*_H h')$\\
	Докажем, что это группа.\\
	Доказательство ассоциативности:
	$((g,h)(g',h'))(g'',h'') \overset{?}{=} (g,h)((g',h')(g'',h'')$\\
	$(gg',hh')(g'',h'') \overset{?}{=} (g,h)(g' g'', h' h'')$\\
	$((gg')g'',(hh')h'') \overset{?}{=} (g(g',g''),h(h'h'')$ - очевидно\\
	Нейтральный элемент:

	Рассмотрим $\Z /_2 \Z \times \Z /_2 \Z = \{(\overline{0},\overline{0}),(\overline{0},\overline{1}),(\overline{1},\overline{0}),(\overline{1},\overline{1})\}$

	\begin{definition}
	    Конечная группа порядка n является циклической тогда и только тогда, когда она содержит элемент порядка n ($|G|=n$, G - циклическая $\equiv$ $\e g \in G: \ord g = n$)
	\end{definition}
	Рассмотрим $\Z /_2 \Z \times \Z /_3 \Z$ - циклическая\\
	$((\overline{1},\overline{1}), (\overline{0}, \overline{2}), (\overline{1}, \overline{0}), (\overline{0}, \overline{1}), (\overline{1},\overline{2}))$\\
	Рассмотрим $\Z /_2 \Z \times \Z /_4 \Z$ - не циклическая

	\begin{definition}
	    $\varphi: G \ra H$ - биекция и $\varphi(g_1,g_2)=\varphi(g_1) \varphi(g_2)\q \forall g_1,g_2 \in G$, тогда $\varphi$ - изоморфизм
	\end{definition}

	\begin{examples}
	    \begin{enumerate}
	        \item $D_3 \ra S_3$
	        \item $U_n=\{z\in \CC: z^n=1\} \la \Z / n \Z$\\
	        ($\frac{2\pi a}{n}+i \sin \frac{2\pi a}{n} = \varphi \overline{a} \mapsfrom \overline{a}$)\\
	        $\overline{a}=\overline{b} \ra \varphi(\overline{a})=\varphi(\overline{b})$\\
	        $\varphi(\overline{a}+\overline{b}) \overset{?}{=} \varphi(\overline{a})\varphi(\overline{b})$\\
	        $\cos \frac{2\pi(a+b)}{n}+i \sin \frac{2\pi(a+b)}{n}=(\cos\frac{2\pi a}{n} + i \sin \frac{2\pi a}{n}) $
	    \end{enumerate}
	\end{examples}

	\hsubsection{1.4}{Изоморфные группы}
	\begin{definition}
	    Две группы называются изоморфными, если между ними существует изоморфизм
	\end{definition}

	\begin{utv}
	    Изоморфизм - отношение эквивалентности
	\end{utv}

	\begin{proof}
	    т.к. композиция изоморфизмов - изоморфизм $G \overset{e}{\ra} H \overset{\psi}{\ra} H$\\
	    $(\psi \circ \varphi)(g_1 g_2)= \psi(\varphi(g_1 g_2)=\psi(\varphi(g_1) \varphi(g_2))=\psi(\varphi(g_1)) \psi(\varphi(g_2)) = (\psi \circ \varphi)(g_1) \circ (\psi \circ \varphi)(g_2)$\\
	    Рефлексивность - тождественное отображение - изоморфизм\\
	    Транзитивность: $G \underset{\varphi}{\ra} H$, $H \underset{\varphi^{-1}}{\ra} G$
	\end{proof}

	\hsubsection{1.5}{Теорема о циклических группах}
	\begin{theorem}
	    G - циклическая группа\\
	    1) $|G|=n \Ra G \cong \Z / n \Z$\\
	    2) $|G|=\infty \Ra G \cong \Z$
	\end{theorem}

	\begin{proof}\\
	    1) g - обр. G, значит $G=\{e,g,g^2,...,g^{n-1}\}$ (среди них нет одинаковых), построим изоморфизм в $\Z / n \Z$: $\varphi(g^k)=\overline{k}$\\
	    Проверим, что $\varphi(g^k g^l)= \varphi(g^k)+\varphi(g^l)=\overline{k}+\overline{l}$\\
	    Левая часть: $\varphi(g^{k+l}=\overline{(k+l) \mod n} = \overline{k}+\overline{l}$\\
	    2) $G=\{...,g^{-1},e,g,g^2,...\}$ (тоже нет совпадающих элементов, иначе $g^k=g^l$, при $k>l$, тогда $g^{k-l}=e$, но тогда конечное число элементов, потому что оно зацикливается через каждые $k-l$ элементов), построим отображение в $\Z$.\\
	    $\varphi(g^n)=n$ -, очевидно, биекция. И нужно доказать, что $\varphi(g^n g^k)=\varphi(g^n)-\varphi(g^k)=n+k$
	\end{proof}

	\end{lect}
\end{document}
