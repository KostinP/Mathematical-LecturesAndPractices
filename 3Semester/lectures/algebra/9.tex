\documentclass[main]{subfiles}


\begin{document}
  \begin{lect}{2019-10-29}
    \hsubsection{2.23}{Теорема про самосопряженный оператор}
    \begin{Theorem}
        \[L \text{ - самосопр. } \Ra \exists e_1, ..., e_n \text{ - ортнорм. базис из
         с.в. } L\]
         \[Lv = \lambda v\]
         \[(u, v) = 0 \os{?}{\Ra}(Lu, v) = 0\]
         \[(Lu, v) = (u, L^*v) = (u, Lv) = (u, \lambda v) = \lambda(u, v) = 0\]
         Тут мы должны задать вопрос.
    \end{Theorem}

    \begin{Definition}
        \[ A \text{ - эрмитова матрица}\]
        \[\Ra M \text{ - унитарная }\]
        \[D \us{\in \R}{\text{ - диагональная }} : A = MDM^{-1}\]
    \end{Definition}

    \hsubsection{2.23}{Теорема про эрмитову матрицу}
    \begin{Theorem}
        \[A \text{ - эрмитов матрица}\]
        Тогда условия равносильны
        \begin{enumerate}
            \item $\q \us{x \neq 0}{\forall x \in \CC^n} \qq \us{\in \R}{x^* Ax }> 0 \qq
                (x^*Ax)^* = x^*A^*x = x^*Ax$
            \item Все с.ч. $A > 0$
            \item Все гл. миноры $A > 0$ \q(критерий Сильвестра)
                %рисунок1 с главными минорами
            \item $\exists P \text{ - обратимое: } A = P^*P$
        \end{enumerate}
        Если хотя бы одно из них выполняется, то матрица $A$ - положительно опред.
    \end{Theorem}

    \begin{Proof}
        \[4 \to 1\]
        \[A = P^*P\]
        \[x^*Ax = x^*P^*Px = (Px)^*(Px) = <Px, Px>\]
        \[<a, b> = \sum a_i \ol{b}_i \q \text{Стандартное эрмитово скал. произв. в } \CC\]
        \[2 \to 4\] %1
        \[A = MDM^{-1} \q M \text{ - унит} \q D \text{ - диаг. } (\in \R)\]
        %2
        \[D^{\frac{1}{2}} = \begin{pmatrix}
            \sqrt{d_1} & ... & 0\\
                       & \ddots&\\
            0 & ... & \sqrt{d_n}
        \end{pmatrix}  \qq A = (D^{\frac{1}{2}}M^* )^* (D^{\frac{1}{2}}M^*)\]
        \[M \text{ - унитар } \Ra MD^{\frac{1}{2}}D^{\frac{1}{2}}M^* = MDM^{-1} = A\]
        %3
        \[1 \to 2\]%4
        \[Ax = \lambda x\]
        \[\us{>0}{x^*Ax} = x^*\lambda x = \lambda x^* x = \lambda \us{>0}{<x, x>}\]
        \[1 \to 3\]
        Нужно доказать, что все главные миноры больше 0
        \[A = \begin{pmatrix}
            A' & B\\
            C  & D
        \end{pmatrix}\]
        %5
        \[\begin{pmatrix}
            x'\\
            0
        \end{pmatrix}^* \begin{pmatrix}
        A' & B\\
        C & D
        \end{pmatrix} \begin{pmatrix}
            x'\\
            0
        \end{pmatrix} = x'^*A'x' > 0 %6
        \q \forall x' \neq 0\]
        \[\Ra A' \text{ уд первому условию, а еще 4 условию}\]
        \[A' = P*P\]
        \[\det A' = \det P^* \cdot \det P = \overline{\det P} \cdot \det P =
        \abs{\det P}^2 > 0 \q \text{ т.к. P обратим}\]
        \[3 \to 2\]
        Индукция по размеру $A$\\
        Когда матрица $1 \times 1$ очев.\\
        Инд. переход : $n \to n + 1$\\
        Пусть $\lambda $ - с.ч $A$ , \q $\lambda < 0 %7
        \Ra \exists  \mu < 0 $
        \[Ax = \lambda x \qq Ay = \mu y, \q <x, y> = 0\]
        Если $\lambda$ и $\mu$ различные.\\
        Если с.ч. различны, то им соотв. ортогон. с.в
        $\Ra$ у эрмит. матр. ортогон с.в соотв. различным с.ч .

        %8
        У эрмитовой матрицы существует онб из с.в - столбцов.
        В этом базисе будет два вектора, лежащие в одном подпр-ве.

        Что такое собственное под-во?

        Если $\lambda$ и $\mu$ совпадают, то есть два неколл. с.в., мы можем их
        ортогонализировать %9


        \[\exists \alpha, \beta \in \CC : \us{= u}{\alpha x + \beta y} = (u', 0)\]
        \[A = \begin{pmatrix}
            A' & *\\
            *  & *
        \end{pmatrix}\]
        \[u'^*A'u' = u^*Au = \abs{\alpha}^2 x^*Ax + \abs{\beta}^2y^*Ay = \q
        \text{ подставили } u \text{ которое сверху}\]
        \[= \abs{\alpha}^2 \us{<0}{\lambda} \cdot \Abs{x}^2 +
        \abs{\beta}^2 \us{<0}{\mu} \Abs{y}^2 < 0\]
        \[u'^*A'u' < 0\]
        Если бы для матрицы $A'$ выполнялось 3 условие, то должно было бы выполняться
        2 условие, а 1 не выполняется, это значит, что 3 условие не вып.
        Все главные миноры $A'$ - это в частности главные миноры $A$. А 3 выполняется
        для $A$.
        Мы получили противоречие.
    \end{Proof}

    \begin{remark}
        Все то же самое, можно доказать для симм. матрицы.\\
        Пусть след. усл равносильны... для симм. матрицы над $\R$\\
        Только тут будет $P$ над $\R$
    \end{remark}

    КАЖЕТСЯ, ТУТ ЧТО-ТО НЕ ТАК, ЭТО УЖЕ БЫЛО
    \begin{Theorem}
        \[A \text{ - эрмит. матрица}\]
        \[\text{тогда след. условия равносильны}\]
        \begin{enumerate}
            \item $\forall x \in \CC^n \qq \us{\in \R}{ x^*Ax} \geq 0$
            \item Все с.ч. $A \geq 0$
            \item Все гл. миноры $A \geq 0$
            \item $\exists P: \qq A = P^*P$
        \end{enumerate}
        Такая матрица называется положительно полуопред.
    \end{Theorem}

    \begin{proof}
        Доказать дома
    \end{proof}

    \hsubsection{2.24}{Singular value decompostition}
    \begin{Definition} [Singular value decompostition \q SVD]
        \[A \in M_{m, n}(\CC) \Ra \exists \us{m \times m}{U}, \us{n \times n}{V}
        \text{ - унитарные, }  \q S \in M_{m, n}(\R)\]
        $S $ - диаг. насколько это возможно для прямоуг. матрицы, с неотр числами
        на диаг.
        %рисунок прямоугольника
        \[A = USV^*\]
        Поворот, растяжение, поворот
        %ссылка на видео
    \end{Definition}

    \begin{Proof}
        \[m \leq n\]
        \[A^*A \text{ - эрмитова} \qq (A^*A)^* = A^*A \text{ - proof}\]
        \[x^*A^*Ax = (Ax)^*(Ax) \geq 0\]
        Значит эта матрица положительно полуопред.
        \[\exists V \text{ - унитарная: } \q V^*A^*AV = D' \text{ - диаг} \q
        V \in \GL_n(\CC)\]
        т.к. эта матрица положительно полуопред., то у этой матрицы
        на диаг будут стоять неотр. с.ч.
        Переставим с.ч так, что сначала идут положительные, а потом нули
        \[D' = \begin{pmatrix}
            D & 0\\
            0 & 0
        \end{pmatrix} \qq D \in M_k(\R) \q m \geq n \geq k\]
        \[V = (\us{k \text{ столб}}{V_1} \ \us{n - k \text{ столб.}}{V_2}) \q
        V_1 \in M_{n, k}(\CC) \q V_2 \in M_{n, n - k}(\CC)  \]
        \[D' = \begin{pmatrix}
            v_1^*\\
            v_2^*
        \end{pmatrix}A^*A \begin{pmatrix}
        V_1 & V_2
        \end{pmatrix} =
            \begin{pmatrix}
                V_1^*A^*AV_1 & V_1^*A^*AV_2\\
                V^*_2A^*AV_1 & V_2^*A^*AV_2
            \end{pmatrix} = \begin{pmatrix}
                D & 0\\
                0 & 0
            \end{pmatrix}
        \]
        \[\Ra \begin{matrix}
            V_1^*A^*AV_1 = D\\
            V_2^*A^*AV_2 = 0 \Ra AV_2 = 0
        \end{matrix}\]
        \[\begin{pmatrix}
            V_1^*\\
            V_2^*
        \end{pmatrix} \begin{pmatrix}
        V_1 & V_2
        \end{pmatrix} = \begin{pmatrix}
        V_1^*V_1 & V_1^*V_2\\
        V_2^*V_1 & V_2^*V_2
        \end{pmatrix} = \begin{pmatrix}
        E_k & 0\\
        0 & E_{n - k}
        \end{pmatrix}\]
        \[\Ra \begin{matrix}
            V_1^*V_1 = E_k\\
            V_2^*V_2 = E_{n - k}
        \end{matrix} \qq \begin{pmatrix}
        V_1 & V_2
        \end{pmatrix} \begin{pmatrix}
            V_1^*\\
            V_2^*
        \end{pmatrix} = V_1V_1^* + V_2V_2^* = E_n\]

        \[U_1 \os{\det}{=} AV_1D^{-\frac{1}{2}} \in M_{m, k}(\CC)  \]
        \[U_1 D^{\frac{1}{2}}V_{1}^* = AV_1D^{-\frac{1}{2}}D^{\frac{1}{2}} V_1^* = A - AV_2V_2^*  = A \]
    \end{Proof}
  \end{lect}
\end{document}
