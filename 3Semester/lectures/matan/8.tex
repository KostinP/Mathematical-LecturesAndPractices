\documentclass[main]{subfiles}

\begin{document}
	\newpage
	\subsection{Метод множителей Лагранжа. Пример: норма линейного оператора}

	\begin{example}
		\begin{enumerate}
			\item Экстремум кв. формы на ед. сфере
			\[A = \begin{pmatrix}
				a_{11} & ... & a_{1n}\\
				...\\
				a_{n_1} & ... & a_{nn}
			\end{pmatrix} \qq a_{ij} = a_{ji}\]
			\[f : \R^n \to \R \qq f(x) = (Ax,\ x) = \sum^n_{i, j = 1} a_{ij} x_i x_j  \]
			\[\text{При условии } \sum^n_{j = 1} x_j^2 = 1 \RA \Phi(x) = \sum^n_{j = 1} x^2_j - 1\]
			Если в т. $x^*$ --- отн. экстремум, то $\exists \lambda \in \R^n$:
			\[\begin{cases}
				\nabla f(x^*) - \lambda \nabla \Phi(x^*) = 0\\
				\Phi(x^*) = 0
			\end{cases}\]
			\[\nabla f = 2 A x,\q \nabla \Phi = 2x\]
			\[\lambda \in \R\]
			\[\begin{cases}
				A x^* = \lambda x^*\\
				\sum x^{*2} = 1
			\end{cases} \Ra \lambda \text{ --- с.ч. } \q x^* \text{ --- с.в соотв. } \lambda\]
			\[\Abs{x^*} = 1\]
			\[A \begin{pmatrix}
				x_1\\
				\vdots\\
				x_n
			\end{pmatrix} =
		    \lambda \begin{pmatrix}
		    	x_1 \\
				\vdots\\
				x_n
		    \end{pmatrix}
			\]
			Ищем экстр. $f(x) = (Ax, x)$
			\[\Ra f(x^*) = (Ax^*, x^*) = (\lambda x^*, x^*) = \lambda \us{= 1}{(x^*, x^*)} = \lambda\]
			\[\Ra \max \text{ и } \min \text{ знач. кв. ф. на ед. сфере равны }\]
			\[\max \text{ и } \min \text{ с.ч. } A\]
			\item $L \in \LL(\R^n,\ \R^n) \qq (x,\ Ly) = (L^{*}x,\ y)$
			\[\text{Норма } L : \q \Abs{L} = \max_{x \in S} \Abs{Lx} \]
			\[f(x) = \us{x \in S}{\Abs{Lx}}^2 = (Lx,\ Lx) = (L^{*}Lx,\ x)\]
			\[\Abs{L}^2 - \max \text{ с.ч. } (L^*L)\]
		\end{enumerate}
	\end{example}
\end{document}
