\documentclass[main]{subfiles}

\begin{document}
\begin{lect}{2019-09-11}
	\begin{lemma}
		$K \subset \R^n$ - компакт, тогда:
		\begin{enumerate}
			\item K - замкн
			\item K - огр
			\item $\forall D \subset K \q D \text{ - замк } \Ra D \text{ - комп} $
		\end{enumerate}
	\end{lemma}

	\begin{Proof} \
		\center{\includegraphics[width=5cm]{pics/2_1}}
		\begin{enumerate}
			\item $K^c \ni a$
			      \[\forall x \in K \q d(a, x) > 0\]
			      \[r_x = \frac{1}{3} d(x, a)\]
			      \[\forall x \in K \q B(x, r_x) \text{ - откр.}\]
			      \[K \subset \bigcup_{x \in K} B(x, r_x) \text{ - откр. покр. компакта } K\]
			      \[\Ra \e x_1, ..., x_N \in K : K \subset \bigcup_{j = 1}^N B(x_j, r_{x_j})\]
			      \[a \in \bigcap_{k = 1}^N B(a, r_{x_k}) = B(a, r_{\min})\]
			      \[r = \min(r_{x_1},\ ...,\ r_{x_N}) > 0\]
			      \[\text{Причем } \bigcap_{1}^N B(a, r_{x_k}) \text{ не имеет общих точек с } \bigcup_{1}^N B(x_k, r_{x_k})\supset K\]
			      \[\e B(a, e_{mn}) \subset K^c \RA K^c \text{ - откр }\RA \text{K - замкн.} \]
			\item $\text{ комп. - }K \subset \bigcup_{k = 1}^\infty B(0, k) \text{ - откр. покр}$
			      \[\Ra \e k_1, ..., k_n\]
			      \[K \subset \bigcup_{j = 1}^N B(0, k_j) = B(0, \max_{1 \leq j \leq N}(k_j)) \RA K \text{ - огр} \]
			\item $\text{ замкн. - }D \subset K \text{ - комп.}$
			      \[\text{Пусть откр. покр. } D \subset \bigcup_{\alpha \in A} U_\alpha \]
			      \[U^* = D^c \text{ - откр. - добавим к покр. K}\q \{U_\alpha\}_{\alpha \in A}\]
			      \[\Ra \text{ выд. конечн. подпокр. } K \q \{U_{\alpha_j}\}_{j = 1}^N \cup \{U^*\} \]
			      \[\Ra D \subset \bigcup_{j = 1}^N U_\alpha\]
		\end{enumerate}
	\end{Proof}

	\begin{theorem}
		Следующие условия равносильны:
		\begin{enumerate}
			\item K - компакт.
			\item K - замк. и огр.
			\item $\forall \{x_m\}_{m = 1}^\infty \ x_m \in K$
			      \[\e \text{ подпосл } x_{m_k} \to x \in K\]
		\end{enumerate}
	\end{theorem}

	\begin{proof}
		$(1 \Ra 2)$ было\\
		$(2 \Ra 1)$
		\[\text{т.к. } K \text{ - огр} \Ra \e I = \prod_{j = 1}^n [a_j, b_j]\]
		\[\text{замкн - }K \subset I \text{ - комп}\]
		$\Ra$ (лемма) K - комп\\
		$(2 \Ra 3)$
		\[x_m \in K \text{ - замк и огр}\]
		\[\Ra \e x_{m_k} \text{ - сх (пр. выб. Б-В)}\]
		\[x_{m_k} \to x \text{ предпол } x \not \in K\]
		\[x \in K^c \text{ - откр } \Ra \e B_x \subset K^c\]
		\[\text{Но } K \ni d(x_{m_k}, x) \to 0 \text{ противореч } x \in K \]
		$(3 \Ra 2)$
		\[\text{а) предп.} K \text{ не явл. огр.} \]
		\[\forall n \in \N \q \e x_n \in K: d(0, x_n) > n\]
		\[\{x_n\} \text{ не огр} \Ra \text{ не сх.}\]
		\[\Ra K \text{ - огр}\]
		\[\text{б) предп., что } K \text{ - не явл. замкн}\]
		\[K^c \text{ - не откр }\]
		\[\e a \in K^c : \forall \delta > 0 \  B(a, \delta) \cap K \neq \varnothing\]
		\[\e x_n \in B(a, \frac{1}{n}) \cap K\]
		\[x_n \in K\]
		\[0 \leq d(x_n, a) < \frac{1}{n} \to 0 \q x_n \to a; \ x_{n_k} \to x \in K\]
	\end{proof}

	\begin{Upr}
		\[K_1 \supset K_2 \supset ...\]
		\[\text{д-ть } \bigcap_{j \in \N} K_j \neq \varnothing\]
	\end{Upr}

	\section{Отображения в $\R^n$}
	\begin{Definition}
		\[E \subset \R^n,\q f: E \to \R^m \text{ - отобр-е (вект. ф-я)}\]
		\[(m = 1 \text{ - ф-я})\]
		\[f(x) = (f_1(x), ..., f_m(x))\]
		\[x= (x_1, ..., x_n) \q f_j: E \to \R \text{ - коорд. ф-я}\]
	\end{Definition}

	\begin{Definition}
		\[a \in \R^n \text{ - пред. т. E, если:}\]
		\[\forall \delta > 0 \q U()(a, \delta) \cap E \neq \varnothing\]
	\end{Definition}

	\begin{Definition}
		\[f: E \to \R^m, a \text{ - пред. т. E}\]
		\[\lim_{x \to a} f(x) = L \text{, если:}\]
		\[\text{(Коши) } \forall \E > 0 \q \e \delta > 0 : \forall x \in E\]
		\[0 < d(x, a) < \delta \ra d(f(x), L) < \E\]
		\[\text{(Гейне) } \forall \{x_k\}_{k = 1}^\infty \q x_k \in E \setminus \{a\} x_k \to_{k \to \infty} a  \ra F(x_k) \to_{k \to \infty} L \]
	\end{Definition}

	\begin{upr}
		Эквивалентность определений по Коши и по Гейне
	\end{upr}

	\begin{upr}
		Сходимость $\lra$ покоординатная сходимость
	\end{upr}

	\begin{Example}
		\[f(x, y) = \left\{ \begin{align}
				 & \frac{xy}{x^2 + y^2}, & (x,y) \neq (0, 0) \\
				 & 0,                    & (x,y) = (0, 0)
			\end{align}\]
		Повторные пределы
		\[\lim_{x \to 0} \lim_{y \to 0} f(x, y) = 0\]
		\[\lim_{y \to 0} \lim_{x \to 0} f(x, y) = 0 \]
		\[f(\delta, \delta) = \frac{1}{2} \underset{\delta \to 0}{\to}\frac{1}{2}\]
		\[f(\delta, -\delta) = -\frac{1}{2}\]
		\[\text{т.е } \lim_{(x, y) \to (0,0)} f(x, y) \text{ не сущ.} \]
	\end{Example}

	\begin{Theorem}[предел композиции]
		\[E \subset \R^n, \q F \subset \R^m \q\q \R^n \ni a \text{ - пред т. E} \ F \ni b \text{ - пред. т. F}\]
		\[f: E \to F; \q g: F \to \R^l\]
		\[\lim_{x \to a} f(x) = b; \q \lim_{x \to b} g(x) = g(b) \]
		\center{\includegraphics[width=5cm]{pics/2_2}}
		\[\text{ Тогда } \lim_{x \to a} g \circ f = g(b) \]
	\end{Theorem}

	\begin{Theorem}[критерий Коши]
		\[a \text{ - пред т. } E\]
		\[f(x) \text{ имеет предел в т. } a\]
		\[ \rla \forall \E > 0\q \e \delta > 0 :
			\forall x, y \in\ \doted{U}(a, \delta) \cap E \RA d(f(x), f(y)) < \E\]
	\end{Theorem}

	\hsubsection{2.1}{Непрерывные отображения}
	\begin{Definition} [непрерывные отображения]
		\[a \in E \q\q f: E \to \R^m\]
		Если $a$ - изол.$\RA f$ - непр. в $a$\\
		Если $a$ - пред., то $f \text{ - непр. в т. } a \RLA \lim_{x \to a}f(x) = f(a)$
	\end{Definition}

	\begin{Utv}
		\[f \text{ - непр в т.} a \rla f_j \text{ - непр. в т } a \ \forall 1 \leq j \leq m\]
		\[f \text{ - непр в т. } a; g \text{ - непр в } f(a) \RLA g \circ f \text{ - непр в т } a\]
		\[\text{непр сохр. при +, умн. на число}\]
		\[f \text{ - непр на } E \RLA \text{ непр } \forall a \in E\]
	\end{Utv}

	\begin{Theorem}[эквивалентность определений непрерывности]
		\[f: E \to \R^m\]
		\[f \text{ - непр на } E \rla \forall G \subset \R^m \q G \text{ - откр } \RA
			f^{-1}(G) \text{ - откр в } E\]
	\end{Theorem}

	\begin{proof}
		($\Ra$):
		\[G \text{ - откр.},\q f^{-1}(G) \text{ - откр ?}\]
		\begin{figure}[h!]
			\center{\includegraphics[width = 5cm]{pics/2_3}}
		\end{figure}
		\[a \in f^{-1}(G)\]
		\[f(a) \in G \text{ - откр } \Ra \e U(f(a), \E) \subset G\]
		\[\text{т.к. } f \text{ непр. в т. } a \q \e \delta : d(a, x) < \delta \Ra d(f(a), f(x)) < \E\]
		\[f(B(a, \delta)) \subset B(f(a), \E) \subset G\]
		\[\Ra B(a, \delta) \subset f^{-1}(G)\]
		($\La$):
		\[a \in E \os{?}{\RA} f \text{ - непр в т. a}\]
		\begin{figure}[h!]
			\center{\includegraphics[width = 5cm]{pics/2_4}}
		\end{figure}
		\[\forall \E > 0: B(f(a), \E) \text{ - откр в } \R^m \RA f^{-1}(B(f(a), \E) \text{ - откр.}\]
		\[\Ra \e \delta : B(a, \delta) \subset f^{-1} (B(f(a), \E)) \RA f \text{ - непр. в т } a\]
	\end{proof}

	\subsubsection{Локальные св-ва непрерывности}
	\begin{theorem}
		\begin{enumerate}
			\item $f$ - непр. в т. $a \RA f$ - лок. огр.
			\item $f$ - непр. в т. $a$, $g(a) > 0 \RA \e$окр. т. a: $f(x) > 0\q \forall x \in U_a$
			\item $f, g$ - непр. в т. $a$ $f \circ g$ непр в a.
		\end{enumerate}
	\end{theorem}
\end{lect}
\end{document}
