\documentclass[12pt, fleqn]{article}

\usepackage{../../../template/template}

 
\begin{document}
 
\begin{lect} {2019-10-02}
		\begin{Reminder}
			\[f \in C^m(U) \text{ тогда}\]
			\[f(p + h) - T_m(f, p, h) = o(\Abs{h}^m) \q h \to 0\]
		\end{Reminder}

		\begin{proof}
		    По ф-ле Тейлора с остатком в форме Лагранжа
			\[f(p + h) = \sum^{m-1}_{k = 0} \frac{1}{k!} \sum^n_{i_1, ..., i_k = 1} 
			\frac{\partial ^{(k)}f }{\partial x_{i1}  ... \partial x_{ik} } h_{i1} \cdot ... \cdot h_{ik} + R_m\]
			\[R_m = \frac{1}{m!} \sum_{i_1, ..., i_m = 1}^n  \frac{\partial ^m f}
			{\partial x_{i1} \cdot ... \cdot x_{im} \us{\xi \in (0, 1)}}{(p + \xi h)} h_{i1} \cdot h_{im} = \]
			\[f \in C^m(U) \Ra \frac{\partial^m f}{\partial x_{i1} ... \partial x_{im} \text{ - непр в } U }\]
			\[ = \frac{1}{m!} \sum^{n}_{i_1, ..., i_, = 1} \left(\frac{\partial^m f}{
			\partial x_{i1} ... \partial x_{im}  }(p) + \alpha(h)\right) \cdot h_1 ... h_m = \]
			\[\frac{1}{m!} \sum_{i_1, ..., i_m = 1}^n \frac{\partial^m f}{\partial x_{i1}  ... \partial x_{im} }(p)
			\cdot h_1 ... h_m + \frac{1}{m!} \sum^n_{i1, ..., i_m = 1} \alpha(h) \cdot h_{i1} ... h_{im}\]
			\[\frac{\abs{R_m}}{\Abs{h}^m} \leq C \cdot \alpha(h) \to 0 \q h \to 0\]
			\[\frac{\abs{h_{i1} ... h_{im}  }}{\Abs{}} \leq\]
			\[\text{т.е } R_n = o(\Abs{h})\] % тут чего-то не хватает (стерли раньше)
		\end{proof}

		\begin{Theorem} [достаточные условие экстремума]
				\[U \subset \R^n \q f : \us{\text{откр}}{U} \to \R^1 \q a \in U\]
				\[f \in C^2(U) \qq a \text{ - критическая}\]
				\begin{enumerate}
					\item $d^2_a f$ - полож. опр. кв. ф $\Ra$ $a$ - строгий лок. $\min$
					\item $d^2_a f$ - отр. опр $\Ra$ $a$ - стр. $\max$
					\item $d^2_a f$ - неопр $\Ra$ в т. $a$ нет экстрем
				\end{enumerate}
		\end{Theorem}

		\begin{Proof}
			%ЗДЕСЬ ОЧЕНЬ МНОГО ОШИБОК TO-DO ИСПРАВИТЬ!!! 
			%Выглядит очень плохо
			\[p = a\]
			\[f(p + h) = f(p) + \us{= 0}{d_p} f(h) + \frac{1}{2} d^2_p f(h) + \frac{1}{2}\alpha(h) \cdot \Abs{h}^2 \qq 
			\alpha(h) \to 0\]
			\[2(f(p + h) - f(p)) = d_p^2 f (h) + \alpha(h) \cdot \Abs{h}^2\]
			\[d^2_p f \text{ на } S = \{h : \Abs{h} = 1\}\]
			Случай 1: $\letus \ \us{\text{непр как ф от h}}{d_p^2 f(h)} > 0 \q \forall h \neq 0$
			\[m = \min_{h \in S} d_p^2 f(h) = d^2_p f(\xi) \q \xi \in S\]
			\[M = \max_{h \in S} d_p^2 f(h) = d^2_p f(\mu) \q \mu \in S\]
			\[d^2_p f(h) = \sum^n_{i, j= 1} (\frac{\partial^2 f}{\partial x_{i} \partial x_{j} } 
			\frac{h_i}{\Abs{h}} \frac{h_j}{\Abs{h}}) \cdot \Abs{h}^2 =  \]
			\[ = \Abs{h}^2 d^2_p f(\frac{h}{\Abs{h}}) \geq m \cdot \Abs{h} ^2\]
			\[\text{Т.к. } \alpha(h) \to 0 \text{ то}\]
			\[\exists \delta > 0 : \forall \Abs{h} < \delta \q \abs{\alpha(h)} < \frac{m}{2}\]
			\[2(f(p + h) - f(p)) = d_p^2 f(h) + \alpha(h) \Abs{h}^2 = \Abs{h}^2 (d_p^2 f (\frac{h}{\Abs{h}}) + 
			\alpha(h)) \geq \frac{m}{2} \cdot \Abs{h}^2 > 0 \q \forall h \neq 0\]
			\[\abs{\alpha} < \frac{m}{2}\]
			\[\text{т.е } f(p + h) > f(p)\]
			\[\forall \Abs{h} < \delta\]
			т.о $p$ - т. стр. лок $\min$
			Случай 2: Аналогично\\
			Случай 3: \\
			$\displaystyle \letus m = \min_{\Abs{h} = 1} d_p^2 f(h) = d_p^2 f(\xi) < 0$
			\[M = \max_{\Abs{h} = 1} d_p^2 f(h) = d_p^2 f(\mu) > 0 \]
			\[? \exists t :  f(p + t \xi) < f(p)\]
			\[f(p + t \mu) > f(p)\]
			\[f(p - \us{h}{t \cdot \xi}) - f(p) = t^2 (d^2_p f(\xi) + \alpha(h)) \]
			\[\exists \delta > 0 : \Abs{h} < \delta \Ra \abs{\alpha(h)} < \frac{m}{2}\]
			\[\Ra \abs{t} < \delta \q f(p + t\xi) - f(p) < 0 \q \forall t \neq 0\]
			Аналогично $\exists \widetilde{\delta} : \Abs{h} < \widetilde{\delta} \Ra 
			\abs{\alpha(h)} < \frac{m}{2}$
			\[f(p + t\mu) - f(p) > 0 \q \forall h \neq 0\]
		\end{Proof}

		\begin{Example}
			\[f(x, y) = xy + \frac{1}{x} + \frac{1}{y} \qq x > 0 \q y > 0\]
			\[(x, y) = (1, 1) \text{ - крит. точка}\]
			\[\begin{cases}
					f'_x = y - \frac{1}{x^2} = 0\\
					f'_y = x - \frac{1}{y^2} = 0
			\end{cases}\]
			\[f''_{xx} = \frac{2}{x^3} \q f''_{xy}  = 1 \q f''_{yy} = \frac{2}{y^3}\]
			\[d^2_{(1, 1)} f(h) = (h_1, h_2) \begin{pmatrix}
				2 & 1\\
				1 & 2
			\end{pmatrix} 
		   \begin{pmatrix}
		        h_1\\
				h_2
		   \end{pmatrix} 
	   = 2 h_1^2 + 2h_1 h_2 + 2 h_2^2\]
	   \[d^2_{(1,1)} f(h) > 0 \q \forall h \neq 0 \]
	   т.о $(1, 1)$ - т. лок $\min$
	   %Я не из тех, кто чтит попов
	   %Кто безотчетно верит в бога
		\end{Example}

		\begin{Theorem} [Об обратном отображении]
			%самой теоремы тут нет
			\[A \text{ - лин отобр.} \q A \in \L(\R^n, \R^m)\]
			Если $A$ - обратмое отобр $\Ra$ ?
			\[m = n \q \ker A = 0\]
			\[\det A \neq 0  \q A(\R^n) = \R^n\]
			\[f : U \to \R^m \qq U \subset \R^n \text{ предпол., что } f \text{ - диф на }U\]
			\[f \text{ - обратимо и } f^{-1} \text{ - тоже диф-мо} \q f^{-1} : f(U) \to U  \]
			\[(f^{-1} \circ f)(x) = x \]
			\[d_x (f^{-1} \circ f) = d_{f(x)} f^{-1} \us{A}{d_x} f = E_n = \begin{pmatrix}
				1 & ... & 0\\
				0 & \ddots & 0 \\
				0 & ... & 1
			\end{pmatrix} \]
			\[\Ra n = m\]
			\[f : \us{\subset \R^n}{U} \to \R^n\]
		\end{Theorem}

		\begin{Theorem} [Об обратном отобр.]
			\[ \letus\ \us{\text{откр}}{U} \subset \R^n \q f \in C^1 (U) \q a \in U\]
			\[d_a f \text{ - обратим. Тогда } \exists U_a \subset U \text{ (окр. т. $a$)}\]
			\begin{enumerate}
				\item $f \big|_{U_a}$ - инъекция
				\item $f(U_a)$ - откр. \q $V = f(U_a)$
				\item $f^{-1} \in C^1 \q (V \to U_a)$
				\item $d_{f(a)} f^{-1} \circ d_a f = E_n$
			\end{enumerate}
		\end{Theorem}

		\begin{Example}
				%рисунок (парабола) -2	
				\[n = 1\]
				\[f(x) = x^2\]
				\[a = -2\]
				\[d_af(h) = 2a \cdot h = -4h \text{ - обратим}\]
				\[(d_0 f(h) = 0 \cdot h = 0 \text{ - необр})\]
				\[\exists U_a = (-3; -1) : f^{-1}(y) = - \sqrt{y} \]
		\end{Example}

		\begin{Lemma} [1 об операторе, близком к обратимому]
			\[GL(n) \text{ - группа обратимых операторов}\]
			\[\letus\ A \in GL(n)\]
			\[B \in \L(\R^n)\]
			\[\Abs{B - A} < \frac{1}{\Abs{A^{-1}}}\]
			Тогда
			\begin{enumerate}
				\item $B \in GL(n)$
				\item отобр $A \to A^{-1}$ - непр (в операторной норме) 
			\end{enumerate}
		\end{Lemma}

		\begin{Proof}
			\[\alpha = \Abs{A^{-1}} \q \beta = \Abs{B - A}\]
			\[\Abs{Ax} = \Abs{(A + B - B)x} \leq \Abs{Bx} + \Abs{(A - B)x}\]
			\[\Abs{Bx} \geq \Abs{Ax} - \Abs{(A-B)x}\]
			\[\Abs{x} = \Abs{A^{-1}Ax} \leq \us{\text{норма оп. }}{\Abs{A^{-1}}} \us{\text{норма вект}}{\Abs{Ax}} 
			\Ra \Abs{Ax} \geq \frac{1}{\alpha} \Abs{x}\]
			\[\Abs{Bx} \geq \Abs{Ax} - \Abs{(A - B)x} \geq \frac{1}{\alpha} \Abs{x} - \beta \Abs{x}
			= (\frac{1}{\alpha} - \beta) \Abs{x} \q (*)\]
			\[\Abs{Bx} > 0\]
			\[\text{т.е. инъекция} \Ra B \in GL(n)\]

			\[\Abs{A^{-1} - B^{-1}}\]
			\[A^{-1} - B^{-1} = A^{-1}(B - A)B^{-1}\]
			\[\Abs{A^{-1} - B^{-1}} \leq \Abs{A^{-1}} \cdot \Abs{B - A} \cdot \Abs{B^{-1}}\]
			\[\text{Из (1) } \Ra \Abs{Bx} \geq (\alpha^{-1} - \beta) \cdot \Abs{\us{=B^{-1}(y)}{x}}\]
			\[\Abs{By} \geq (\frac{1}{\alpha} - \beta) \Abs{B^{-1} y }\]
			\[\Abs{B^{-1}y } \leq \frac{\Abs{y}}{a - \beta} \qq \frac{1}{\alpha} = a\]
			\[\Abs{B^{-1}} \leq \frac{1}{a - \beta} \]
			\[\Abs{A^{-1} - B^{-1}} \leq \frac{1}{a} \cdot \beta \cdot \frac{1}{a - \beta}\]
			\[\Phi(A) = A^{-1}\]
			\[\Phi \text{ - непр в т. }A\]
			\[\forall \mathcal{E} > 0 \exists \delta > 0 : \forall B : \Abs{B - A} < \delta \Ra 
			\Abs{A^{-1} - B^{-1}} \leq \frac{1}{a} \frac{\beta}{(a - \beta)} < \mathcal{E}\]
			\[a \text{ - фиксир}\]
			\[\beta \to 0\]
		\end{Proof}

		\begin{Lemma} [2 об оценке приращ диф-мого отображения] 
				\[U \subset \R^n\]
				\[f: U \to \R^m \qq f \text{ - диф на } U\]
				%рисунок (отрезок в множестве U)
				\[\letus\ [a, b] \subset U\]
				\[\text{Тогда } \exists\ \Theta \in (0, 1) : \]
				\[c = a + \Theta(b-a)\]
				\[\Abs{f(b) - f(a)} \leq \Abs{d_c f} \cdot \Abs{b - a}\]
		\end{Lemma}

		\begin{Proof}
			\[\varphi(t) = (f(a + t(b - a));\ f(b) - f(a)) \text{ - скал. произв в } \R^n\]
			\[t \in [0, 1] \q \varphi : [, 1] \to \R\]
			Т. Лагранжа для функции $\varphi$
			\[\exists\ \Theta \in (0, 1) : \]
			\[\varphi(1) - \varphi(0) = \varphi'(\Theta) \cdot (1 - 0)\]
			\[= \Abs{f(b) - f(a)}^2\]
			\[\abs{\varphi'(\Theta)} = \abs{ (d_c f(b - a);\ f(b) - f(a))} \os{\text{КБШ}}{\leq} \]
			\[c = a + \Theta(b - a)\]
			\[\leq \Abs{d_c f(b-a)} \cdot \Abs{f(b) - f(a)} \leq \Abs{d_c f} \cdot \Abs{b - a} \cdot 
			\Abs{f(b) - f(a)}\]
			\[\Ra \Abs{f(b) - f(a)} \leq \Abs{d_c f} \cdot \Abs{b - a}\]
		\end{Proof}
		\\
		Договоримся использовать
		\[A = d_a f\]
		\[\lambda = \frac{1}{4 \Abs{A^{-1}}}\]

		\begin{Lemma} [3]
			\[\text{Пусть }f \in C^{1}(U \to \R^n) \q \us{\text{откр}}{U} \subset \R^n \q a \in U\]
			\[\text{и } d_af \text{ - обратим. Тогда } \exists U_a:\]
			\begin{enumerate}
				\item $\forall x \in U_a \q d_xf $- обр.
					%рисунок с окрестностью в множестве 
				\item $\forall x, \ x + h \in U_a$
					\[\Abs{f(x+h) - f(x) - d_a f(h)} \leq 2 \lambda \Abs{h} \q (a)\]
					\[\Abs{f(x + h) - f(x)} \geq 2 \lambda \Abs{h} \q (b)\]
			\end{enumerate}
		\end{Lemma}

		\begin{Proof}
			\[\text{Т.к } f \in C^{1}(U) \rla df \text{ как отобр из } U \text{ в } \L(\R^n, \R^n) \text{ - непр}\]
			\[\mathcal{E} = 2\lambda \Ra \exists  B_a : \forall x \in B_a \q \Abs{d_a f - d_x f} < 2 \lambda\]
			По лемме об операторе близком к обратимому
			\[(A \text{ - обр и } \Abs{A - B} \leq \frac{1}{\Abs{A^{-1} }} \Ra B \text{ - обр})\]
			\[\Abs{d_a f - d_x f} < 2\lambda < 4 \lambda = \frac{1}{\Abs{A^{-1}}} \Ra 
			d_xf \text{ - обратим}\]

			\[F(u) = f(u) - d_a f(u)\]
			\[\forall c \in B_a\] %опять рисунок шарика в области U
			\[\Abs{d_cF} = \Abs{d_cf - d_af} < 2\lambda \q \forall c \in B_a\]
			\[(a) \q\Abs{f(x + h) - f(x) - d_af(h)} = \Abs{F(x + h) - F(x)} \leq \text{(л 2)}\]
			\[\leq \Abs{d_{x+ \Theta h} F} \cdot \Abs{h} < 2 \lambda \Abs{h}\]
			\[(b) \q \Abs{h} = \Abs{A^{-1}Ah} \leq \Abs{A^{-1}} \cdot \Abs{A h}\]
			\[\Abs{Ah} \geq \frac{\Abs{h}}{\Abs{A^{-1}}} = 4 \lambda \Abs{h}\]
			\[\Abs{Ah} \leq \Abs{f(x + h) - f(x) - Ah} + \Abs{f(x + h) - f(x)}\]
			%\[\Abs{f(x + h) - f(x)} \leq \Abs{f(x + h) - f(x) - d_af(h)} + \Abs{d_afh} \os{\text{из (a)}}{ \leq}\]
			\[\Abs{f(x + h) - f(x)} \geq \Abs{Ah} - \Abs{f(x + h) - f(x) - Ah} \geq 
			4\lambda \Abs{h} - 2\lambda \Abs{h} = 2 \lambda \Abs{h} \q (b)\]
		\end{Proof}

		\begin{Proof} [Теоремы об обратном отобр]
			\[1) \ B_a \text{ - из леммы 3} \qq U_a = B_a\]
			Уже знаем, что $\forall x \in B_a \q \Abs{d_x f- d_a f} < 2\lambda$
			\[\text{Из л.3 } \Abs{f(x + h) - f(x)} \geq 2 \lambda \cdot \Abs{h} > 0 \qq \forall h \neq 0\]
			\[\text{т.е. } f \text{ - инъекция}\]
			(пункт 1 доказан)\\

			$2)$
			Докажем, что $f(B_a)$ - откр.
			%рисунок с шариком и отображением
			\[V = f(B_a)\]
			\[\text{Зафиксируем } y_0 \in V\]
			\[y_0 = f(x_0) \q x_o \in B_a\]
			\[\exists r > 0 : \q \ol{B(x_0, r)} \subset B_a \subset U\]
			\[\text{Цель: } B(y_0, \lambda_r) \subset f(B(x_0, r))\]
			\[\text{Рассмотрим } y \in B(y_0, \lambda_r)\]
			\[\Phi(x) = \Abs{f(x) - y}^2 \qq x \in \overline{B(x_0, r)}\]
			\[\Phi(x_0) = \Abs{\us{= y_0}{f(x_0)} - y}^2 < (\lambda r)^2\]
			\[\Phi \text{ - непр на } \overline{B(x_0, r)}\]
			\[\exists x_{\min} \in B(x_0, r):\]
			\[\min_{B(x, r)} \Phi(x) = \Phi(x_{\min} ) \]
			\[\text{Предпол, что } x_{\min} \in \partial B(x_0, r) \qq (\Abs{x_{\min} - x_0 } = r)\]
			\[\sqrt{\Phi(X)} = \Abs{f(x) - y} > \ul{ \Abs{f(x) - y} + \Abs{f(x_0) - y}} - \lambda r \geq\]
			\[\Abs{f(x) - f(x_0)}  - \lambda r \os{\text{по л3(2в)}}{\geq} 2\lambda \Abs{x - x_0} - \lambda r\]
			\[\text{При } x = x_{\min} \in \partial B(x_0, r) \]
			\[\sqrt{\Phi(x_{\min})} > 2\lambda \cdot r - \lambda r = \lambda r \geq \sqrt{\Phi(x_0)}\]
			Противоречие $(\Phi(x_{\min}) \text{ - не минимально})$
			\[\Ra x_{\min} \in B(x_0, r) \]
			%\[\Ra d_{x_{\min}} \Phi = 0 = d_{x_{\min} } (f(x) - y;\ f(x) - y) = 2(d_a f; )\]
			\[d_{x_{\min}} \Phi(h) = 2() \]
			\[\Phi(x) = (f(x) - y; f(x) - y)\]
			\[d_{x_{\min}} \Phi(h) = 2(d_{x_{\min}} fh; f(x_{\min} ) - y  ) = 0 \qq \forall h \in \R^n\]
			\[\text{т.к. } d_{x_{\min}} f \text{ - обратим } \Ra d_{x_{\min}} f(\R^n) = 
			\R^n \Ra \]
			\[f(x_{\min} ) - y \in (\R^n)^{\perp}\]
			\[\text{т.о. } \exists x_{\min} \in B(x_0, r) \ra f(x_{\min} ) = y \]
			\[\Ra y \in f(B(x_0, r))\]

			\[3) \q f \big|_{b_a} \text{ - биекция } \q B_a \to V \Ra \exists \us{= g}{f^{-1}}: V \to B_a\]
			\[y, y + k \in V\]
			%рисунок с отображением g в B_a 
			\[y = f(x) \qq\q x = g(y)\]
			\[y + k = f(x + h) \q x + h = g(y + k)\]
			\[\text{Докажем непр на } V \q \Abs{\us{=h}{g(y + k) - g(y)}} = \Abs{h} \leq \frac{1}{2\lambda} 
			\Abs{\us{=f(x+h) - f(x)}{k}} \Ra g \text{ - непр}\]
			\[d_x f = A\]
			\[k = f(x + h) - f(x) = Ah + \alpha(h) \Abs{h} \q h \to 0\]
			\[A^{-1}k = h + A^{-1} (\alpha(h) \Abs{h})\]
			\[g(y + k) - g(y) - A^{-1}k = -A^{-1}(\alpha(h) \Abs{h}) \]
			\[\Abs{A^{-1}(\alpha(h) \Abs{h}) } \leq \Abs{A^{-1}} \cdot \abs{\alpha} \Abs{h} \leq \]
			\[\leq \mathcal{E} \Abs{A^{-1}} \abs{h} \leq \mathcal{E} \Abs{A^{-1}} \frac{1}{2 \lambda} 
			\Abs{k}\]
			\[\text{т.о.} \lim_{k \to 0} \frac{\Abs{A^{-1}}(\alpha \Abs{h}) }{\Abs{k}}\]
		\end{Proof}
\end{lect}

\end{document}
