\documentclass[12pt, fleqn]{article}

\usepackage{../../../template/template}

 
\begin{document}
 
	\begin{lect}
		\begin{example}
		    Экстремум кв. форму ны ед. сфере
			\[A = \begin{pmatrix}
				a_{11} & ... & a_{1n}\\
				...\\
				a_{n_1} & ... & a_{nn} 
			\end{pmatrix} \qq a_{ij} = a_{ji}\]
			\[f : \R^n \to \R \qq f(x) = (Ax, x) = \sum^n_{i, j = 1} a_{ij} x_i x_j  \]
			\[\text{При условии } \sum^n_{j = 1} x_j^2 = 1 \]
			\[\Phi(x) = \sum^n_{j = 1} x^2_j - 1\]
			Если в т. $x^*$ - отн. экстремум, то 
			\[\exists \lambda \in \R^n\]
			\[\begin{cases}
				\nabla f(x^*) - \lambda \nabla \Phi(x^*) = 0\\
				\Phi(x^*) = 0
			\end{cases}\]
			\[\nabla f = 2 A x\]
			\[\nabla \Phi = 2x\]
			\[\lambda \in \R\]
			\[\begin{cases}
				A x^* = \lambda x^*\\
				\sum x^{*2} = 1 
			\end{cases} \Ra \lambda \text{ - с.ч. } \q x^* \text{ - с.в соотв. } \lambda\]
			\[\Abs{x^*} = 1\]
			\[A \begin{pmatrix}
				x_1\\
				\vdots\\
				x_n
			\end{pmatrix} = 
		    \lambda \begin{pmatrix}
		    	x_1 \\
				\vdots\\
				x_n
		    \end{pmatrix} 
		\]
		Ищем экстр. $f(x) = (Ax, x)$
		\[\Ra f(x^*) = (Ax^*, x^*) = (\lambda x^*, x^*) = \lambda \us{= 1}{(x^*, x^*)} = \lambda\]
		\[\Ra \max \text{ и } \min \text{ знач. кв. ф. на ед. сфере равны }\]
		\[\max \text{ и } \min \text{ с.ч. } A\]
		\end{example}

		\begin{Definition}
			\[L \in \LL(\R^n, \R^n) \qq (x, Ly) = (L^{*}x, y)\]
			\[\text{Норма } L : \q \Abs{L} = \max_{x \in S} \Abs{L_x} \]
			\[f(x) = \us{x \in S}{\Abs{Lx}}^2 = (Lx, Lx) = (L^{*}Lx, x)\]
			\[\Abs{L}^2 - \max \text{ с.ч. } (L^*L)\]
		\end{Definition}

		\section{Теория функций компл. переменного}
		\begin{Reminder}
		    \[z = x + iy \in \CC \qq x, y \in \R\]
			\[i^2 = -1\]
			\[z_1 + z_2 = x_1 + x_2 + i(y_1 + y_2)\]
			\[z_1 \cdot z_2 = x_1 x_2 - y_1 y_2 + i(x_1 y_2  + x_2 y_1)\]
			\[\overline{z} = x - iy \qq \abs{z} = \sqrt{x^2 + y^2}\]
			%рисунок 1 z на плоскости
			\[z \cdot \overline{z} = \abs{z}^2\]
			\[\frac{z_1}{z_2} = \frac{z_1 \cdot \overline{z_2}}{\abs{z_2}^2}\]
			\[k \in \R \Ra \frac{z}{k} = \frac{x}{k} + i \frac{y}{k}\]
			Сложение действует как на векторах, что с умножением?\\
			Перейдем к полярной системе координат
			%рисунок 2 полярные координаты
			\[z = \abs{z} (\cos \varphi + i \sin \varphi)\]
			\[\real z = \abs{z} \cos \varphi\]
			\[\im z = \abs{z} \sin \varphi\]
			\[z_1 = \abs{z_1} (\cos \varphi_1 + i \sin \varphi_1)\]
			\[z_2 = \abs{z_2} (\cos \varphi_2 + i \sin \varphi_2)\]
			\[z_1 \cdot z_2 = \abs{z_1} \cdot \abs{z_2} 
			(\cos \varphi_1 \cos \varphi_2 - \sin \varphi_1 \sin \varphi_2 
		    +i(\cos \varphi_1 \sin \varphi_2 + \sin \varphi_1 \cos \varphi_2)) = \]
			\[ = \abs{z_1} \abs{z_2} (\cos(\varphi_1 + \varphi_2) + 
			i \sin (\varphi_1 + \varphi_2))\]
			%рисунок 3
		\end{Reminder}

		\begin{Theorem} [Ф-ла Муавра]
			\[z^n = \abs{z}^n (\cos{n \varphi} + i \sin n \varphi )\]
		\end{Theorem}

		\begin{Definition} [н-во \bigtriangleup]
			\[\abs{z_1 + z_2} \leq \abs{z_1} + \abs{z_2}\]
		\end{Definition}

		\begin{Definition} [н-во Коши]
		    \[z_j, w_j \in \CC, \q j = 1, ..., n\]
			\[\abs{\sum_{j = 1}^n z_j \cdot w_j}^2 \leq \sum_{j = 1}^n 
			\abs{z_j}^2 \cdot \sum_{j = 1}^n \abs{w_j}^2 \]
		\end{Definition}

		\begin{Proof}
			\[\overline{ab} = \overline{a} \cdot \overline{b} \qq 
			z + \overline{z} = 2 \real z \qq 
			z - \overline{z} = 2 i \im z\]
			%\[0 \leq \sum_{j = 1}^n  \abs{z_j - \lambda w_j}^2 = 
			%\sum_{j = 1}^n  (z_j - \lambda w_j) (\overline{z}_j - 
			%\overline{\lambda} \overline{w}_j) = \sum( \abs{z}^2 + \abs{\lambda}^2 
		    %\abs{w_j}^2 - \]
			%\[- (z_j \overline{\lambda} w_j + \overline{z}_j \lambda w_j) ) = 
			%\sum \abs{z_j}^2 + \sum \abs{\lambda}^2 \abs{w_j}^2 - \sum 2 \real 
			%\underbrace{(\overline{\lambda} \cdot z_j \cdot w_j)}\]
			\[0 \leq \sum_{j = 1}^n  \abs{z_j - \lambda \overline{w}_j}^2 = 
			\sum \abs{z_j}^2 + \abs{\lambda}^2 \sum \abs{w_j}^2 - 2 \real 
		    (\sum_{j = 1}^n z_j \overline{\lambda} w_j)\]
			\[\lambda = \frac{\sum z_j w_j}{\sum \abs{w_j}^2}\]
			\[0 \leq \sum \abs{z_j}^2 + \frac{\abs{\sum z_j w_j}^2}
			{(\sum \abs{w_j}^2)^2} \cdot \sum \abs{w_j}^2 - 
		    2 \real \left[  \frac{\sum \overline{z_j} \cdot \overline{w_j}}
	        {\sum \abs{w_j}^2} \sum_{j = 1}^n z_j w_j \right]\]
			\[\text{hint: } [...] \leq \frac{\abs{\sum z_j w_j}^2}{\sum \abs{w_j}^2}\]
			\[0 \leq \sum \abs{z_j}^2 + \frac{\abs{\sum z_j w_j}^2}{\sum \abs{w_j}} - 
			2 \frac{\abs{\sum z_j w_j}^2}{\sum \abs{w_j}^2}\]
			\[\abs{\sum^n_{j = 1} z_j w_j}^2 \leq \sum^n_{j = 1} \abs{z_j}^2 \cdot 
			\sum_{j = 1}^n \abs{w_j}^2 \]
		\end{Proof}

		\begin{definition}
		    Комплексная последовательность
			\[c_n \in \CC\]
			\[c_n = a_n + i b_n, a_n, b_n \in \R\]
			\[c_n \to c \in \CC \rla \abs{c_n - c} \to 0 \rla \begin{cases}
					a_n \to a\\
					b_n \to b
			\end{cases} \rla \{c_n\}_{n \in \N} \text{ - сх. в себе} \]
			\[\text{при } n \to \infty \qq \text{ т.е } \ \begin{align}
					\real c_n \to \real c\\
					\im c_n \to \im c
			\end{align}\]

		\end{definition}

		\begin{examples} [функций к. п.]
			\begin{enumerate}
				\item $ \us{\text{зафикс}}{a \in \CC}  \qq f(z) = z + a \qq f: \CC \to \CC$\\
					парал. перенос вдоль вектора $\overline{a} = (\real z, \im a)$
					%рисунок 4
				\item $\lambda \in \CC \q \abs{\lambda} = 1 \q \lambda = \cos \Theta + i \sin \Theta \q 
					z = \abs{z}(\cos \varphi + i \sin \varphi)$
					\[f(z) = \lambda z = \abs{z} (\cos(\varphi + \Theta) + i \sin (\varphi + \Theta))\]
					Поворот вокруг O на угол $\Theta$ против часовой стрелки
					%рисунок 5 
				\item $k \in [0, +\infty)$
					\[f(z) = k z = k \cdot \abs{z} (\cos \varphi + i \sin \varphi)\]
					\[\abs{f(z)} = k \abs{z}\]
					Гомотетия с коэф. $k$
				\item $f(z) = z^2 = \abs{z}^2 (\cos 2\varphi + i \sin 2\varphi)$
					%рисунок 6
					\[z :\ \abs{z} < 1 \qq\q \Ra z :\ \abs{z} < 1\]
					\[0 \leq \varphi \leq \frac{\pi}{2} \qq\q 0 \leq \varphi \leq \pi\]
				\item Инверсия (относительно ед. окружности)
					\[f(z) = \frac{1}{z} \qq f: \CC \setminus \{0\} \to \CC\]
					\[f(z) = \frac{\overline{z}}{\abs{z}^2}\]
					%рисунок 7
					Какие точки останутся неподвижными? Их ровно две $-1$ и $1$ 
					$\left(\displaystyle z = \frac{1}{z}\right)$
				\item Дробно-линейные отобр-я (преобр Мёбиуса)
					\[L(z) = \frac{az + b}{cz + d} \qq (c, d) \neq (0, 0)\]
					\[\text{Если } c = 0, \text{ то } L \text{ - афинное преобразование, т.е композиция }\]
					гомотетий, поворотов и пар. переносов
					\[L : \CC \setminus \{- \frac{d}{c}\} \to \CC\]
					\[\text{Если } \begin{vmatrix}
						a & b\\
						c & d
					\end{vmatrix} = 0, \text{ то } L(z) = const\]
					Доопр. инв. $\displaystyle f(z) = \frac{1}{z} \qq f(0) = \infty \qq f(\infty) = 0$
					\[L \text{ - доопределим}\]
					\[L(- \frac{d}{c}) \qq L(\infty) = \frac{a}{c}\]
					\[\text{Тогда } L : \hat{\CC} \to \hat{\CC} \text{ - вз. однозн., если } 
					\begin{vmatrix}
						a & b\\
						c & d
					\end{vmatrix} \neq 0\]
					\[w = \frac{az + b}{cz + d}\]
					\[czw + dw = az + b\]
					\[z(cw - a) = b - dw\]
					\[z = \frac{b - dw}{cw - a} \qq \begin{vmatrix}
						-d & b\\
						c & -a
					\end{vmatrix} = ad - bc \neq 0 \]
					
			\end{enumerate}\\
			\text{ }\\
			\\Сфера римана $\rla \overline{\CC} = \CC \cup \{\infty\}$
		\end{examples}
		\begin{utv}
			Если известно, что $L(z_1) = w_1 \qq L(z_2) = w_2 \qq L(z_3) = w_3$
			\[\Ra \text{ можно восстановить дробно-лин. отобр } L\]
			\[z_1 \neq z_2 \neq z_3 \qq w_1 \neq w_2 \neq w_3\]
		\end{utv}

		\begin{definition}
		    Обобщенная окр-ть $=$ окр-ть или прямая
		\end{definition}

		\begin{utv} [круговое св-во]
			Дробно-лин отобр. переводит обощенные окр. в обобщ. окр.
		\end{utv}

		\begin{proof}
			Дробно-лин. отобр - композиция
			\begin{enumerate}
				\item гомотетий
				\item пар. переносов.
				\item поворотов
				\item инверсий
			\end{enumerate}
			\[1 - 3 \text{ - переводят окр } \to \text{окр} \qq \text{прямые } \to \text{ прямые}\]
			Надо разобраться, что делает инверсия с окр
			\[\alpha \cdot \abs{z}^2 + \beta \real z + \gamma \im z + \delta = 0\]
			\[\alpha, \beta, \gamma, \delta \in \R\]
			\[\alpha(x^2 + y^2) + \beta x + \gamma y + \delta = 0\]
			\[\alpha = 0 \text{ - прямые}\]
			\[\alpha \neq 0 \text{ - окружности}\]
			\[x^2 + y^2 + \frac{\beta}{\alpha} x + \frac{\gamma}{\alpha}y + \frac{\delta}{\alpha} = 0\]
			\[\left(x + \frac{\beta}{2\alpha}\right)^2 + \left(y + \frac{\gamma}{2 \alpha}\right)^2 
			+ \frac{\delta}{\alpha} - \frac{\beta ^2 + \gamma^2 }{4 \alpha^2} = 0\]
			\[4\alpha \delta \leq \beta^2 + \gamma^2\]
			\[z \ra \frac{1}{z} = \frac{\overline{z}}{\abs{z}^2} = \frac{x - iy}{\abs{z}^2}\]
			\[\alpha \cdot \frac{1}{\abs{z}^2} + \beta \frac{\real z}{\abs{z}^2} - 
			\gamma \frac{\im z}{\abs{z}^2} + \delta = 0\]
			\[\alpha + \beta \real z - \gamma \im z + \delta \abs{z}^2 = 0\]
			\[4 \alpha \delta \leq \beta^2 + \gamma^2\]
		\end{proof}

		\begin{Definition} [симметрия отн. окружности]
		    %рисунок8 кружок
			\[\abs{z^* - z_0} \cdot \abs{z - z_0} = R^2\]
			\[z^* \text{ - симметрична } z \text{ отн окр. } \abs{z - z_0} = R\]
			Рассмотрим
			%рисунок 9 отобр из кружка в плоскость
			\[z^* = \frac{1}{\abs{z}}(\cos \varphi + i \sin \varphi) = 
			\frac{1}{\abs{z}} \cdot \frac{z}{\abs{z}}\]
			\[\begin{align}
				&L: & L(y) = \frac{z + b}{cz + d}\\
				&L(0) = i & L(0) = i = \frac{b}{d}\\
				&L(-1) = 0 & L(-1) = \frac{b - 1}{d - c} = 0\\
				&L(1) = \infty & L(1) = \frac{1 + b}{c + d} = \infty\\
			\end{align}\]
			\[b = 1 \qq d = -i \qq \frac{1 + 1}{c - i} = \infty \q c = i\]
			\[L(z) = \frac{z + 1}{iz - i} = -i \frac{z + 1}{z - 1}\]
			\[L(z) = -i \frac{z + 1}{z - 1} \]
			\[L(z^*) = -i \frac{\frac{z}{\abs{z}^2} + 1}{\frac{z}{\abs{z}^2} - 1} = 
			-i \frac{z + \abs{z}^2}{z - \abs{z}^2}\]
			\[\overline{L(z)} = \overline{-i} \frac{(\overline{z} + 1) ^2 z}{(\overline{z} - 1)^2 z} = 
			i \frac{\abs{z}^2 + z}{\abs{z}^2 - z} = L(z^*)\]
			
		\end{Definition}

		\begin{Example}
			\[7) \q f(z) = e^z = e^{x + iy} = e^x (\cos y + i\sin y) \text{ (по ф. Эйлера)}\]
			\[e^{iy} = \cos y + i \sin y \]
			\[e^{i\pi} = -1 \q  \text{ Замечательная формула, которая связывает 3 числа}\]
			\[(x, y) \os{e^z}{\to } (e^x \cos y;\ e^x \sin y)\]
			%рисунок10 преобразование $e^z$ с солнышком
			\[\begin{cases}
				y = 0\\
				0 \leq x < \infty
			\end{cases} \os{e^z}{\to } e^x (\cos 0 + i \sin 0) = e^x \geq 1\]
			\[\begin{pmatrix}
				x = 0\\
				0 \leq y \leq \pi
			\end{pmatrix} \os{e^z}{\to } e^0 (\cos y + i\sin y)\]
			\[\begin{cases}
					y = \pi\\
					0 \leq x < +\infty
				\end{cases} \os{e^z}{\to} e^x (\underbracket{\cos \pi + i \sin \pi}_{-1}) = -e^x \leq -1\]
			hint: "для понимания можно представлять это как веер"
			\[e^z = e^x (\cos y + i \sin y) = e^x (\cos (y + 2 \pi k) + i \sin(y + 2 \pi k)) =\]
			\[ = e^{x + i(y + 2\pi k)} = e^{z + i \cdot 2 \pi k}  \]
			\[\text{Период } e^z \q T = e\pi k i\]
		\end{Example}
	\end{lect}

\end{document}
