\documentclass[12pt, fleqn]{article}

\usepackage{../../../template/template}

 
\begin{document}
 
\begin{lect}
 
\begin{Reminder}
    \[(1) \qq\dot{x} = X(t, x) \qq \us{\text{обл}}{G} \subset \R^{n + 1} \]
    \[X \in C(G), \qq X \in \Lip_x^{loc}(G) \]
\end{Reminder}

%косяк
\begin{Theorem}[3]
    (о поведении решения при приближении к концу макс.\\ промежутка задания)
    \[G \text{ - огр, }\q X \text{ - огр на } G\]
    \[x = \varphi(t) \text{ - реш } (1),\q t \in  (\alpha, \beta) 
    \text{ макс промеж. задания } \varphi \]
    \[\Ra \exists \lim_{t \to \beta-} \varphi(t) = \xi,\q \text{ и }\q 
    (\beta, \xi) \in \d G\]
\end{Theorem}

\begin{Proof}
    \[\delta > 0\]
    \[t_1, t_2 \in (\beta  - \delta, \beta)\]
    \[\varphi(t_1) = x_1 \Ra \varphi(t) \text{ уд З.К. } (t_1, x_1)\]
    \[\Ra \varphi(t) = x_1 + \int_{t_1}^t X(\tau, \varphi(\tau))d\tau \]
    \[\text{В частн., } \varphi(t_2) = x_1 + \int_{t_1}^{t_2}
    X(\tau, \varphi(\tau))d\tau\]
    \[\Ra \abs{\varphi(t_2) - \varphi(t_1)} = \abs{\int_{t_1}^{t_2} 
    X(\tau, \varphi(\tau))d\tau} \leq \abs{\int_{t_1}^{t_2} 
    \abs{X(\us{\leq M}{\tau, \varphi(\tau)})} d\tau }\]
    \[X \text{ - огр на } G \Ra \ \exists \  M : \ \abs{X(t, x)} \leq M, \qq 
    \forall (t, x) \in G\]
    \[\Ra \abs{\varphi(t_2) - \varphi(t_1)} \leq M \cdot \abs{t_2 - t_1} \qq (2)\]
    \[\Ra \forall \mathcal{E} > 0 \q \exists \delta > 0 : \q \abs{t_2 - t_1} 
    < \delta \ \Ra \  \abs{\varphi(t_2) - \varphi(t_1)} < \mathcal{E}\]
    \[\delta \leq \min(\beta - \alpha, \frac{\mathcal{E}}{M})\]
    \[\os{\text{кр. Коши}}{\Ra } \exists  \lim_{t \to \beta-} \varphi(t) = \xi \]
    \[\us{\forall t \in (\alpha, \beta)}{(t, \varphi(t))}
        \in G \ \Ra \ (\beta, \xi) \in \us{\text{замык}}{\overline{G}}\]
    \[\text{Если } (\beta, \xi) \in G \us{\text{Т1}}{\Ra} \varphi(t) \text{ 
    продолж. вправо за $\beta$ противореч.}\]
    \[\Ra (\beta, \xi) \in \overline{G} \setminus G = \d G\]
\end{Proof}

\begin{theorem} [3']
    Аналогичная $(3)$ для левого конца промежутка
\end{theorem}


\begin{Theorem}[о выходе макс. продолж. решения из компакта или Еругина]
    \[(1) \qq \dot{x} = X(t, x)\]
    \[x = \varphi(t) \text{ - реш. } (1) \qq t \in (\alpha, \beta) 
    \text{ - макс. пр-к задания  } \varphi\]
    \[\us{\text{комп}}{D} \subset G\]
    \[\Ra \exists  \delta > 0 : \q\forall t \in (\beta - \delta, \beta) \qq 
    (t, \varphi(t)) \ \cancel{\in }\ D\]
\end{Theorem}

\begin{Proof}[от противного]
    \[\us{\text{комп}}{D} \subset G \text{ - зафиксировали}\]
    \[\forall \delta > 0 \q \exists t \in (\beta - \delta, \beta): \q 
    (t, \varphi(t)) \in D\]
    \[\{\delta_k\}_{k = 1}^\infty \qq \delta_k > 0  \qq \delta_1 > \delta_2 > 
    ... > \delta_k > \delta_{k + 1} > ... \]
    \[\delta_k \us{k \to  + \infty}{\to } 0 \qq \delta_1 < \beta - \alpha\]
    \[\Ra \ \exists t_k : \q t_k \in (\beta - \delta_k, \beta) \text{ и }
    (t_k, \varphi(t_k)) \in \us{\text{комп}}{D}\]
    \[t_k \us{k \to  + \infty}{\to } \beta\]
    \[\exists  \text{ под/послед. } \{(t_k, \varphi(t_k))\}_{k = 1}^\infty,
    \text{ сх-ся  к } (\beta, \xi) \in D \subset G\]
    \[\Ra \exists  \ a > 0, \ b > 0: \]
    \[D_0 = \{(t, x) : \ \abs{t - \beta} \leq 2a, \ \abs{x - \xi} \leq 2b\} 
    \subset G \qq\qq (3)\]
    \[X \in C(D_0) \ \Ra \exists M : \q \abs{X(t, x)} \leq M \q \forall (t, x) 
    \in D_0\]
    \[h = \min\left(a, \frac{b}{M}\right)\]
    \[t_k \us{k \to +\infty}{\to }\beta \ \Ra \ \exists k_1 :\ 
    \forall k > k_1 \qq \beta - h < t_k < \beta \qq\qq(4)\]
    \[\varphi(t_k) \us{k \to  +\infty}{\to } \xi \Ra \exists k_2 : \ 
    \forall k > k_2 \qq \abs{\varphi(t_k) - \xi} < b \qq\qq (5)\]
    \[\text{фикс } k > \max(k_1, k_2) \ \Ra \ \text{ вып } (4), (5)\]
    \[D_k = \{(t, x): \ \abs{t - t_k} \leq a, \ \abs{x - \varphi(t_k)} \leq b\}\]
    \[\text{Докажем: } D_k \subset D_0\]
    %рисунок1 (точка в прямоугольнике D_0)
    \[\text{Взяли произвольную точку } \q (t, x) \in D_k\] 
    \[\abs{t - \beta} \leq \abs{\us{\leq a}{t - t_k}} + \abs{\us{\leq h
    \leq a}{t_k - \beta}} \leq 2a\]
    \[\abs{x - \xi} \leq \abs{\us{\leq b}{x - \varphi(t_k)}}
    + \abs{\us{\leq b}{\varphi(t_k) - \xi}} \leq 2b\]
    \[\Ra (t, x) \in D_0\]
    \[\text{з. Коши } \q (t_k, \varphi(t_k))\]
    \[\exists \text{  реш } x = \psi(t), \text{ опред на } [t_k - h,\ t_k + h]\]
    \[\text{ и реш } x = \varphi(t) \q (t \in (\alpha, \beta)) \q 
    \text{проходит через } (t_k, \varphi(t_k))\]
    %рисунок2 прямая с точками и отрезком
    \[\text{из (4)}: \ \beta < t_k + h\]
    \[x = \begin{cases}
        \varphi(t),& t \in (\alpha, \beta)\\
        \psi(t), & t \in [t_k - h,\ t_k + h]
    \end{cases} \text{ - продолжение } \varphi(t) \text{ вправо за } \beta\]
    противоречие\\
    (опред. корректно: $\varphi(t) \equiv \psi(t)$ на общ мн-ве)
\end{Proof}

\section{Системы сравнимые с линейными}

\begin{Reminder}
    \[(1) \qq  \dot{x} = X(t, x) \qq X \in C(G), \q X \in \Lip_x^{loc}(G) \]
    \[G = \{(t, x) : \q t \in (a, b), \q x \in \R^n\} \qq \text{ м.б }
    a = -\infty \q b = -\infty\]
    \[\abs{x} < +\infty\]
\end{Reminder}

\begin{Definition}
    \[(1) \text{ - сравнима с линейной, если }\]
    \[\exists  \ M(t) \geq 0, \ N(t) \geq 0 \text{ - непрер. на } (a, b)\]
    \[(2) \q \abs{X(t, x)} \leq M(t) \cdot \abs{x} + N(t) \qq \forall t \in
    (a, b)\]
\end{Definition}

\begin{Theorem}
    \[(1) \text{ ср с лин.}\]
    \[x = \varphi(t) \text{ - реш. }(1) \ \Ra \ \varphi(t) \text{ опред на }
    (a, b)\]
\end{Theorem}

\begin{Proof}[от противного]
    \[\letus \ \exists \ \text{ решение (1) } x = \varphi(t), \text{ определена на } 
    (\alpha, \beta) \text{ макс. пром. задания}\]
    \[(\alpha, \beta) \subset (a, b), \text{ но } (\alpha, \beta) \neq (a, b)\]
    \[\text{НУО } \beta < b:\]
    \[t_0 \in (\alpha, \beta) \q \varphi(t_0) = x_0\]
    \[\Ra \varphi(t) = x_0 + \int_{t_0}^t X(\tau, \varphi(\tau))d\tau \qq 
    \forall t \in [t_0, \beta)\]
    \[\abs{\varphi(t)} \leq \abs{x_0} + \int_{t_0}^t \abs{X(\tau,
    \varphi(\tau))d\tau} \leq \abs{x_0} + \int_{t_0}^t N(\tau)d\tau + 
    \int_{t_0}^t M(\tau)\abs{\varphi(\tau)}d\tau \]
    \[[t_0, \beta] \subset(a, b) \q (\beta < +\infty)\]
    \[M \text{ - непр на } [t_0, \beta] \ \Ra \ \exists  L > 0: \q 
    \abs{M(t)} \leq L \qq \forall t_0 \in  [t_0, \beta]\]
    \[\int_{t_0}^t N(\tau)d\tau \leq \int_{t_0}^{\beta} N(\tau)d\tau  \]
    \[\Rightarrow \underset{\forall t \in \ [t_0, \beta)}{|\varphi(t)|} \leqslant
    \underbrace{|x_0|+\int_{t_0}^{\beta} N(\tau)d\tau}_{\text{c - const}} + 
    L\int_{t_0}^{t} \varphi(\tau)d\tau\]
    \[\Rightarrow \text{ (Лемма Гронуолла) } |\varphi(t)| \leqslant ce^{L(t-t_0)} \leqslant
     ce^{L(\beta-t_0)} \ (3)\ \]
    \[D = \{(t, x) : t \in [t_0, \beta], \ |x| \ \leqslant \ ce^{L(\beta - t_0)} \} \text{из (3) следует, что }\]
    \[(t,\varphi(t)) \in \underbrace{D}_{\text{комп}} \forall t \in [t_0, \beta)
    \text{— противоречие с теорией о выходе макс. пр. реш-я.}\]
\end{Proof}

%смерджить
\end{lect}

\end{document}
