\documentclass[main]{subfiles}

\begin{document}
\begin{lect}{2019-10-10}
		\begin{Reminder}
				\[M(x, y)dx + N(x, y) dy = 0 \qq (1)\]
				\[\mu = \mu(x) \qq \frac{1}{\mu}\mu' = \frac{1}{N}(M'_y - N'_x) \qq (11)\]
				\[u(x,y) = \int_{y_0}^y N(x, t)dt + \int_{x_0}^x M(t, y_0)dt \qq (7')\]
		\end{Reminder}

		\begin{Example} [важнейший]
				\[(13) \qq y' = p(x)y + g(x) \qq p(x), g(x) \in C(a, b)\]
				\[(13') \qq (p(x)y + g(x)) dx - dy = 0 \qq (x \not \equiv const)\]
				\[\frac{1}{N}(M'_y - N'_x) = -1 \cdot (p(x) - 0) = -p(x)\]
				\[\exists \mu = \mu(x) : \frac{d\mu}{\mu} = -p(x)dx\]
				\[\mu(x) = e^{- \int_{x_0}^x p(s)ds } \]
				\[e^{-\int_{x_0}^x p(s) ds} (p(x)y + g(x))dx - e^{-\int_{x_0}^x p(s)ds } dy = 0 \qq (14) \]
				Применяем к этому формулу $7'$\\
				полагаем для простоты $y_0$ = 0
				\[u(x, y) = - \int_{0}^{y} e^{-\int_{x_0}^x p(s) ds }dt +
				    \int_{x_0}^x e^{-\int_{x_0}^x p(s)ds } \cdot g(t) dt    \]
				\[\underline{u(x, y) = -c}\]
				\[- y e ^{-\int_{x_0}^x p(s)ds } + \int_{x_0}^x e^{-\int_{x_0}^t p(s)ds} g(t)dt = -c\]
				\[y = c \cdot e ^{\int_{x_0}^x p(s) ds } + e^{\int_{x_0}^x p(s)ds}
				\int_{x_0}^x e^{-\int_{x_0}^t p(s) ds} g(t) dt \qq (15)    \]
				З. Коши $(x_0, y_0)$ \qq $(x_0 \in (a, b))$
				\[\Ra (15) \text{, где } c = y_0\]
				\[(15') \qq y = ce^{\int p(x)}  + e^{\int p(x)dx} \int e^{-\int p(x)dx} g(x)dx \]
		\end{Example}

		\section{Системы дифф. уравнений}

		\begin{definition}
				Система дифф уравнений, разрешенная относительно старших производных
				\[(1) \qq \begin{cases}
					x_1^{(m_1)} = X_1(t, x_1, \dot{x_1}, ..., x_1^{(m_1-1)}, ..., x_k, \dot{x_k}, ..., x_k^{(m_k - 1)} )\\
					x_2^{(m_2)} = X_2 (...)\\
					...\\
					x_k^{(m_k)} = X_k (...)
				\end{cases}\]
				\[n = \sum_{j = 1}^k m_j \]
		\end{definition}

		\begin{Definition}
				\[\text{Реш (1): } x_1 = \varphi_1(t), ..., x_k = \varphi_k(t) \qq t \in (a,b)\]
				\[\us{j = 1, ..., k}{X_j \in C(D)} \q D \subset \R^{n + 1} \]
				Подставили и получили тождество
		\end{Definition}

		\begin{definition} [Частный случай]
					\begin{enumerate}
							\item $k = 1$\\
								\[x^{(n)} = X(t, x, \dot{x}, \ddot{x}, ..., x^{(n-1)} )  \qq (2)\]
							\item $\displaystyle \us{j = 1, ..., k}{m_j = 1}$
								\[\begin{cases}
									\dot{x_1} = X_1(t, x_1, ..., x_n)\\
									...\\
									\dot{x_n} = X_n(t, x_1, ..., x_n)

								\end{cases} \qq (3)\]
							Система в нормальной форме или нормальная система
					\end{enumerate}
					\\\\
					В (2) замена
					\[\begin{cases}
							x = x_1\\
							\dot{x} = x_2\\
							...\\
							x^{(n - 1)} = x_n
						\end{cases} \qq (4)\]

					\[\begin{cases}
						\dot{x_1} = x_2\\
						\dot{x_2} = x_3\\
						...\\
			     		\dot{x}_{n - 1} = x_n\\
						\dot{x}_n = X(t, x_1, ..., x_n)

					\end{cases} \qq (5)\]
					\[\text{в } (3)  \q x = \begin{pmatrix}
							x_1\\
							\vdots\\
							x_n
					\end{pmatrix} \qq
				    X = \begin{pmatrix}
				    	X_1\\
						\vdots\\
						X_n
				    \end{pmatrix}
				\]
				\[(3') \qq \dot{x} = X(t, x)\]
				$(3') $ - система, записанная в векторной форме
		\end{definition}

		\begin{remark}
				Будем рассматривать только системы в нормальной форме
		\end{remark}

		\\
		З. Коши
		\[\text{для } (1) : \text{ при } t = t_0 : \]
		\[\begin{cases}
				x_1 - x_{1_0}, \dot{x_1} = \dot{x}_{1_0}, ..., x_1^{(m_1 - 1)} = x_{1_0}^{(m_1 - 1)}\\
		x_2 = x_{2_0}, ..., x_2^{(m_2 - 1)} = x_{2_0}^{(m_2 - 1)}  \\
		x_k = x_{k_0}, ..., x_k^{(m_k - 1)} = x_{k_0}^{(m_k - 1)}
		\end{cases}\]
		\[\text{для } (2): \text{ при } t = t_0 \q x = x_0, \dot{x} = \dot{x}_0, ..., x^{(n - 1)} = x_0^{(n - 1)}  \]
		\[\text{для } (3): t = t_0: x_1 = x_{1_0}, x_2 = x_{2_0}, ..., x_n = x_{n_0} \]

		\begin{Remark}
				\[\text{сист } (5) \text{ и ур } (2)\]
				\[\text{реш } \begin{cases}
					x_1 = \varphi_1(t)\\
					x_2 = \varphi_2(t)\\
					...\\
					x_n = \varphi_n(t), \q t \in (a, b)
				\end{cases} \qq \text{реш } x = \varphi(t) \q t \in (a,b)\]
				Решения разные, но мы называем $(5)$ и (2) эквивалентными
				\[\varphi_1(t) = \varphi(t)\]
				\[\varphi_2(t) = \dot{\varphi}(5)\]
				\[...\]
				\[\varphi_n(t) = \varphi^{(n - 1)} (t)\]
		\end{Remark}

		\begin{definition}
				Договоримся с обозначениями
				\[a = \begin{pmatrix}
					a_1\\
					\vdots\\
					a_n
				\end{pmatrix} \text{ - вектор}\]
				\[\abs{a} = \sqrt{a_1^2 + ... + a_n^2} \text{ - норма}\]
				\[a^{(k)} = a^{\{k\}} \text{ - послед. векторов }\]
				\[a^{(k)} \us{k \to +\infty}{\to} a \rla a^{(k)}_j \us{\us{\forall j = 1, ..., n}{k \to +\infty}}{\ra } a \rla \abs{a^{(k)} - a} \us{k \to +\infty}{\to } 0 \]
				\[f(x_1, ..., x_m) = \begin{pmatrix}
					f_1(x_1, ..., x_m) \\
					...\\
					f_n(x_1, ..., x_m)
				\end{pmatrix} \text{ вектор-функция}\]
				\[\frac{\partial f}{\partial x_j} = \begin{pmatrix}
					\frac{\partial f_1}{\partial x_j} \\
					\vdots\\
					\frac{\partial f_n}{\partial x_j}
				\end{pmatrix}\]
				\[u(t) = \begin{pmatrix}
					u_1(t)\\
					\vdots\\
					u_n(t)
				\end{pmatrix} \Ra \dot{u}(t) = \begin{pmatrix}
				     \dot{u}_1(t)\\
					 \vdots\\
					 \dot{u}_n(t)
				\end{pmatrix}\]
				\[u(t) \text{ - непр на } [a, b] \Ra \int_a^b u(t) dt = \begin{pmatrix}
					\int_a^b u_1(t)dt\\
					\vdots\\
					\int_a^b u_n(t)dt
				\end{pmatrix}\]
				\[\abs{\int_a^b u(t)dt} \leq \abs{\int_a^b \abs{u(t)}dt} \text{, если } b \geq a \q \text{ здесь норма |.|}\]
				\[\sum_{k = 1}^{\infty} a^{(k)} = \begin{pmatrix}
					\sum_{k = 1} ^\infty a_1^{(k)}\\
					\vdots\\
					\sum_{k = 1}^\infty a_n^{(k)}
				\end{pmatrix}   \]
				\[a^{(k)} = \begin{pmatrix}
					a_1^{(k)}\\
					...\\
					a_n^{(k)}
				\end{pmatrix} \]
				Признак Вейерштрасса работает.
				\[\exists \text{ сх ряд } \sum^\infty_{k = 1} b_k : \abs{a^{(k)}(t) }  \leq b_k \Ra
				\sum_{k = 1}^\infty u^{(k)} (t)  \text{сх равн и абс} \q \forall t \in \Omega  \]
		\end{definition}

		\begin{Definition}
				\[(1) \qq \dot{x} = X(t, x) \qq X \in C(D), D \subset \R^{n + 1} \]
				\[x = \begin{pmatrix}
					x_1\\
					...\\
					x_n
				\end{pmatrix} \qq X = \begin{pmatrix}
					X_1\\
					...\\
					X_2
				\end{pmatrix}\]
				З.Коши (2) \q $(t_0, x_0)$
				Смысл геометрический и механический полностью совпадают с одномерным случаем\\
				геом - поле направлений\\
				мех - мгновенная скорость в точке и во времени
				\[\text{реш (1) ф-я }  x = \varphi(t) \q t \in (a,b) \text{ подст тожд в (1)}\]
		\end{Definition}

		\begin{Theorem} [Пеано]
	      \[D = \{(t, x) : \abs{t - t_0} \leq a, \ \abs{x -  x_0} \leq b\}\]
				\[X(t, x) \in C(D)\]
				\[\Ra \exists M: \ \abs{X(t, x)} \leq M \q h = \min(a, \frac{b}{M})\]
				\[\Ra \exists \text{ реш (1) } x = \varphi(t) \qq t \in [t_0 - h, t_0 + h]\]
				\[(x_0 = \varphi(t_0)) \text{ доказывается аналогично одномерному сл.}\]
		\end{Theorem}
\end{lect}
\end{document}
