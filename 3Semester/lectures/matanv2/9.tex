\documentclass[12pt, fleqn]{article}

\usepackage{../../../template/template}

 
\begin{document}

\begin{lect}{2019-10-30}
    \begin{Definition}
        \[\Omega \text{ - область в } \CC \q (\text{св., откр})\]
        \[f \text{ - гомоморфной (аналит., регулярной) в  } \Omega, \text{ если } 
        f \q \CC \text{ - дифф} \q \forall  z \in \Omega\]
        \[f'(z) \in C(\Omega) \text{ (потом узнаем, что это условите линшее)}\]
        \[f \text{ - гомом в } \Omega \rla f \in  H(\Omega)\]
        \[f \text{ - целая, если } f \in H(\CC)\]
        Формальные произв.
        \[\frac{\partial f}{\partial z}; \ \frac{\partial f}{\partial \overline{z}}\]
        \[\begin{cases}
            z = x + iy\\
            \overline{z} = x - iy
        \end{cases}\]
        \[\begin{matrix}
            x = \frac{z + \overline{z}}{2}\\
            y = \frac{z - \overline{z}}{2i}
        \end{matrix}\]
        \[\frac{\partial f}{\partial z} = \frac{\partial f}{\partial x} \cdot 
        \frac{\partial x}{\partial z} + \frac{\partial f}{\partial y} 
        \frac{\partial y}{\partial z} = \frac{1}{2}(\frac{\partial f}{\partial x} - i
        \frac{\partial f}{\partial y})\] %%%%%%TODO ПРоверить НА логичЕскую Ошибку!!
        \[\frac{\partial f}{\partial \overline{z}} 
            = \frac{1}{2}(\frac{\partial f}{\partial x} + 
        i\frac{\partial f}{\partial y})\]
    \end{Definition}

    \begin{Definition}[Усл К-Р \ в терминах формальных производных]
        \[u_x' = v_y'\]
        \[u'_y = -v_x'\]
        \[\begin{matrix}
            \frac{\partial f}{\partial x} = u'_x + iv_x'\\
            \frac{\partial f}{\partial y} = u'_y + iv'_y
        \end{matrix}\]
        \[2 \frac{\partial f}{\partial z} = \us{=0}{u'_x - v'_y} + i(\us{=0}{
        v'_x + u'_y}) = 0\]
        \[\text{Усл К-Р } \rla \frac{\partial f}{\partial \overline{z}} = 0\]
    \end{Definition}

    \begin{Definition}[Обратное отображение и якобиан]
        \[f \in H(G), \q \text{предп } z_0 \in \Omega \qq f'(z_0) \neq 0\]
        \[f = u + iv\]
        %рисунок1
        \[\text{Рассм. как отобр. } \Omega \subset \R^2\]
        \[(x, y) \os{f}{\to } (u, v)\]
        \[J_f = \begin{pmatrix}
            u'_x & u'_y\\
            v'_x & v'_y
        \end{pmatrix}\]
        \[\det(J_f) = u'_x v'_y - u'_y v'_x = (u'_x)^2 + (v'_x)^2 = \abs{f'}^2\]
        \[\det J_f = \abs{f'}^2\]
        \[\text{Если } f'(z_0) \neq 0 \Ra \det J_f(z_0) \neq 0\]
        \[\text{Можно применить теорему об обратном отобр}\]
    \end{Definition}

    \begin{Theorem}
        \[f \in H(\Omega) \q z_0 \in \Omega \q f'(z_0) \neq 0\]
        \[\text{Тогда } \exists U \subset \Omega \q U \text{ - откр. } \q z_0 \in U:\]
        \[f\big|_U \text{ - инъекция } f(U) = V \text{ - откр.}\]
        \[\text{и обр. отобр. } f^{-1}  : V \to U \text{ - гомоморф.}\]
        \[\text{причем } (f^{-1})'(f(z)) = \frac{1}{f'(z)}\]
        \[(f^{-1} )(\omega) = \frac{1}{f'(f^{-1}(\omega))}\]
        \hline
        \[\exists U, V : \q f : U \to  \us{\text{откр}}{V} \text{ - биекция из т.
        об обратном отобр.}\]
        Надо проверить диф-сть $f^{-1} $ в $g$
        %рисунок2
        \[z_1 \in U\]
        \[\omega_1 = f(z_1)\]
        \[z \in U \q \omega = f(z)\]
        \[g = f^{-1}  : \ V \to U\]
        \[\lim_{\omega \to \omega_1} \frac{g(\omega) - g(\omega_1)}{\omega - \omega_1} = 
        \lim_{\us{\Ra z \to z_1}{\omega \to \omega_1}} \frac{z - z_1}{f(z) - f(z_1)} = \]
        \[\lim_{z \to z_1} \frac{1}{\frac{f(z) - f(z_1)}{z - z_1}} = 
        \frac{1}{f'(z_1)} = g'(\omega_1) = g'(f(z_1))\]
     \end{Theorem}

     \begin{Examples}
         \[1) \q z^n \in H(\CC) \text{ целая} \qq n \in \N\]
         \[P_n(z) \text{ - мн-н - целая ф-я}\]
         \[(z^n)' = nz^{n - 1} \in H(\CC) \subset C(\CC) \]
         \[\text{Рассмотрим } n > 1 \q z \neq 0 \Ra (z^n)' \neq 0\]
         \[\text{Выделим непрерывную ветвь } \sqrt[n]{z}\]
         %рисунок3
         \[z^n = \omega\]
         \[(\sqrt[n]{z})' = \frac{1}{(z^n)'} = \frac{1}{nz^{n - 1} } = 
         \frac{1}{n\omega^{1 - \frac{1}{n}}} = \frac{1}{n}\omega^{\frac{1}{n} - 1}\]
         \[2) \q f(z) = e^z = e^x(\cos y + i\sin y)\]
         \[u'_x = e^x \cos y = v'_y\]
         \[u'_y = -e^x \sin y = -v'_x\]
         \[f'(z) = \us{u'_x}{e^x \cos y} + i \us{u'_x}{e^x \sin y }= f(z)\]
         \[f'(z) \neq 0 \q \forall  z \in \CC\]
         Рассмотрим гл. ветвь лог-ма
         \[e^z = \omega\]
         \[\ln \omega = z \qq \varphi \in (-\pi; \pi)\]
         \[(e^z)' = f'(z) = e^z\]
         \[(\ln \omega)' = \frac{1}{f'(z)} = \frac{1}{e^z} = \frac{1}{\omega} \text{ 
         ; т.к. остальные ветви отличаются на константу}\]
     \end{Examples}

     \subsection{Конформные отображения}

     \begin{Definition}
         %рисунок4
         \[\gamma(t) = x(t) + iy(t) \qq 0 \leq t \leq 1 \qq \gamma \text{ - шладкая
         кривая}\]
         \[\gamma'(t) = x'(t) + iy'(t)\]
         \[\gamma'(0) \text{ - касат. вектор к } \gamma \text{ в т. } \gamma(0)\]
         \ul{Угол между кривыми} = угол между касат. в т. пересеч.
     \end{Definition}

     \begin{Theorem}
         \[\text{Пусть } \gamma(t) \text{ - гладкая парам. кривой } \gamma\]
         \[z_0 = \gamma(0) \qq f \text{ - аналитична в окрестности } z_0\]
         \[\text{Тогда касат. к } f(\gamma(t)) \text{ в т. } f(z_))\]
         \[(f \circ \gamma)'(0) = f'(z_0) \gamma'(0)\]
     \end{Theorem}

     \begin{Proof}
         \[\lim_{t \to 0}  \frac{(f(\gamma(t))) - f(\gamma(0))}{t} = 
             \lim_{t \to 0} \underbracket{\frac{f(\gamma(t)) - 
             f(\gamma(0))}{\gamma(t) - \gamma(0)} }_{\to f'(z_0)} 
     \underbracket{     \frac{\gamma(t) - \gamma(0)}{t}}_{\to \gamma'(0)} \]
     \end{Proof}

     \begin{Consequence}
         \[\text{Пусть } f \text{ - аналит. в окр. т. } z_0\]
         \[\gamma, \widetilde{\gamma} \text{ кривые с гл. парам-ми}\]
         \[\gamma(0) =\widetilde{\gamma}(0) = z_0  \]
         \[\text{Если } f'(z_0) \neq 0 \text{, то угол (ориент.) между }
             \gamma \text{ и } 
         \widetilde{\gamma} \text{ в т. } z_0\]
         \[\text{равен углу } f(\gamma) \text{ и } \widetilde{f(\gamma)} \text{ в т. }
         f(z_0)\]
         %рисунок5
         Такие отображения называются конфорными
     \end{Consequence}

     \begin{Example}
         \[e^z; \q z^3 - \text{ конф в } \CC \setminus \{0\}\]
     \end{Example}

     \begin{Definition}[Интегралы]
         \[f : [a, b] \to  \CC \text{ - кус-непр}\]
         \[\int_a^b f(t)dt = \int_a^bu(t)dt + i\int_a^bv(t)dt\]
     \end{Definition}

     \begin{properties}
         \begin{enumerate}
             \item $\displaystyle  \abs{\int_a^bf(t)dt} \leq \int_a^b \abs{f(t)}dt$
                 \[\lambda = \frac{\int_a^bf(t)dt}{\abs{\int_a^bf(t)dt}} \text{, если } 
                 \int_a^bf(t)dt \neq 0 \text{ (иначе очев)}\]
                 \[\abs{\lambda} = 1\]
                 \[\abs{\int_a^bf(t)dt} = \lambda^{-1} \int_a^bf(t)dt = 
                 \int_a^b\lambda^{-1} f(t)dt = \real \int_a^b\lambda^{-1} f(t)dt =  \]
                 \[=\int_a^b\real\lambda^{-1}f(t)dt \leq 
                 \int_a^b \abs{\lambda^{-1}f(t) }dt = \int_a^b\abs{f(t)}dt\]
         \end{enumerate}
     \end{properties}

     \begin{Definition} [Кусочно-гл. кривые в $\CC$]
         \[\gamma : I \to \CC \q \gamma' \text{ - кус-непр} \qq \us{\text{откр}}{I}
         \subset \R\]
         \[\text{длина кривой } L(\gamma) = \int_{I} \abs{\gamma'(t)}dt\]
     \end{Definition}

     \begin{Definition}[Криволин. инт-л от ф-ии $f$]
         \[\int_\gamma f(z)dz = \int_I f(\gamma(t)) \cdot \gamma'(t)dt\]
         %рисунок6
     \end{Definition}

     \begin{properties}
         \begin{enumerate}
             \item Линейность
                 \[\int_\gamma (f + kg)(z)dz = \int_\gamma f(z)dz + k\int_\gamma g(z)dz \]
             \item Независимость от параметризации 
                 \[\widetilde{\gamma} = \gamma \circ h \qq h'(t) > 0\]
                 \[\widetilde{\gamma} : [\widetilde{a}, \widetilde{b}] \to \CC\]
                 \[h : [\widetilde{a}, \widetilde{b}] \qq \gamma: [a, b] \to \CC\]
                 \[\int_{\widetilde{a}}^{\widetilde{b}} f(
                 \widetilde{\gamma}(t)) \cdot \widetilde{\gamma}'(t)dt = 
             \int_{\widetilde{a}}^{\widetilde{b}} f(\gamma(h(t))) \cdot \gamma'(h(t)) 
         h'(t) dt = \bigg|^{h(t) = s}  \int_a^b f(\gamma(s))\gamma'(s)ds\]
            \item Изменение направления \\
                $\gamma$ - кривая с противоп. направлением
                \[\int_\gamma f(z)dz = -\int_{-\gamma} f(z)dz \]
            \item Формула Ньютона-Лейбница \qq $\gamma: [a, b] \to \CC$
                \[\int_\gamma f'(z) = \int_a^b f'(\gamma(t)) \cdot \gamma'(t)dt = 
                \int_a^b df(\gamma(t)) = \]
                \[= \int_a^b du(\gamma(t)) + i\int_a^b dv(\gamma(t)) = 
                f(\gamma(b)) - f(\gamma(a))\]
         \end{enumerate}
     \end{properties}

     \begin{Example}[1]
         \[\int_\gamma (z - a)^n dz = \int_0^{2\pi} \underbracket{(re^{it} )^n
         }_{f(\gamma(t))} \cdot \underbracket{r \cdot ie^{it} }_{\gamma'(t)} dt = \]
         \[=ir^{n + 1} \int_0^{2\pi}e^{i(n + 1)t}dt = ir^{n + 1} (\int_0^{2\pi} 
         \cos(n + 1)tdt + i\int_0^{2\pi \sin(n + 1)tdt} ) =   \]
         \[= \begin{cases}
             0,  & n \neq -1\\
             2\pi i, & n = -1
         \end{cases}\]
         \[\cos(n + 1)t = \begin{cases}
             0, & n \neq -1\\
             2\pi, & n = -1
         \end{cases}\]
         %рисунок7
         \[\gamma(t) = a + re^{it} \qq 0 \leq t \leq 2\pi \]
         \[z - a = re^{it} \]
     \end{Example}[1]

     \begin{Example}[2]
         \[hint: \q r e^{i\varphi} = r e^{-i \varphi}  \]
         \[\int_{\gamma(t) = a + e^{it} } \overline{(z - a)}^n dz =  \]
         \[=\int_0^{2\pi} r^n e^{-int} r \cdot i \cdot e^{it}dt = 
         i \cdot r^{n + 1}  \int_0^{2\pi} e^{i(1 - n)t}dt =  \]
         \[=ir^{n + 1} \left(\int_0^{2\pi} \cos(1 - n)tdt + i\int_0^{2\pi} 
         \sin(1 - n)tdt\right) = \begin{cases}
             0, & n \neq 1\\
             2\pi i r^2, & n = 1
         \end{cases}\]
     \end{Example}

     \begin{Utv}
         \[\abs{\int_\gamma f(z)dz} \leq \max_{z \in \gamma} \abs{f(z)} \cdot L(\gamma)\]
     \end{Utv}

     \begin{Proof}
         \[\gamma : [a, b] \to \CC\]
         \[\abs{\int_\gamma f(z)dz} = \abs{\int_a^b f(\gamma(t)) \gamma'(t)dt} \leq 
         \int_a^b \us{\leq \displaystyle \max_{z \in \gamma}
     \abs{f(z)}}{\abs{f(\gamma(t))} }\abs{\gamma'(t)}dt \leq \]
     \[\leq \max_{z \in \gamma} \abs{f(z)} \cdot \us{= L(\gamma)}
     {\int_a^b \abs{\gamma'(t)}dt }\]
     \end{Proof}

     \begin{Consequence}
         \[f = \sum_{j = 1}^\infty f_j \text{ - сх. равн на } \gamma \]
         Тгда этот ряд можно проинтегрировать почленно
         \[\int_\gamma f(z)dz = \sum_{j = 1}^\infty \int_\gamma f_j(z)dz\]
     \end{Consequence}

     \begin{Proof}
         \[S_N(z) = \sum_{j = 1}^N f_j(z)\]
         \[\int_\gamma f(z)dz = \int_\gamma S_n(z)dz + \int_\gamma (f - S_N)dz\]
         \[\abs{\int_\gamma(f - S_N)dz} \leq \us{\to 0 \text{ (т.к.  }S_N \rightrightarrows f)}{\max_{z \in \gamma} 
         \abs{f(z) - S_N(z)} }\cdot \us{const}{L(\gamma)} \]
         \[\int_\gamma f(z)dz = \lim_{N \to \infty}  
             \int_\gamma S_N(z)dz = \lim_{N \to \infty}  
         \int_\gamma \sum_{j = 1}^N  f_j(z)dz\]
         \[=\lim_{N \to \infty} \sum_{j = 1}^N  \int_\gamma f_j(z)dz = 
         \sum^\infty_{j = 1} \int_\gamma f_j(z)dz\]
         \[\int_\gamma \sum^\infty_{j = 1}  f_j(z)dz = \sum_{j = 1}^\infty \int_\gamma f_j(z)dz \]
     \end{Proof}

    \begin{Example}
        \[f(z) = e^{\overline{z}} \]
        \[\gamma(t) = e^{it} \qq 0 \leq t \leq 2\pi \]
        \[e^z = 1 + z + \frac{z^2}{2!} + ... + \frac{z^n}{n!} + ...\]
        \[\overline{D}(0, z) \q \text{ по пр. Вейерштрасса } \qq \sum \frac{z^n}{n!} 
        \text{ сх равн.}\]
        \[\int_{z = e^{it} } e^{\overline{z}}dz = \int_{z = e^{it} } 
        \sum_{n = 0}^\infty \frac{\overline{z}^n}{n!}dz  = 
    \sum_{n = 0}^\infty \frac{1}{n!} \int_\gamma \overline{z}^{n}dz = 2\pi i  \]
        \[\int_\gamma \overline{z}^ndz = \begin{cases}
            2 \pi i & n = 1\\
            0 & n \neq 1
        \end{cases}\]
    \end{Example}

    \begin{Lemma} [Гурса (т. Коши для \bigtriangleup)]
        \[\Omega \text{ - область } \q \Omega \subset \CC\]
        \[f \in H(\Omega \setminus \{p\})\]
        \[f \in C(\Omega)\]
        %рисунок40341 (8)
        \[\triangle ABC \subset \Omega \text{ (вместе с внутр.)}\]
        \[\triangle = \triangle ABC\]
        \[\text{Тогда } \qq \int_{\partial \triangle} f(z)dz = 0\]
    \end{Lemma}

    \begin{Proof}
        \[\RNumb{1}) \qq \text{Пусть } P \cancel{\in } \triangle\]
        %рисунок9
        \[\partial \triangle \text{ - границы треуг.}\]
        \[J = \int_{\delta \triangle} f(z)dz = \int_{AB} + \int_{BC} + \int_{CA} = 
        \sum_{j = 1}^4 \int_{\gamma_j} f(z)dz  \]
        \[\abs{J} \leq \sum_{j = 1}^4 \abs{\int_{\gamma_j} f(z)dz }  \Ra  \text{ 
        из более мелких $\triangle$ найдется хотя бы один}\]
        %рисунок10
        \[\abs{\int_{\partial \trinagle_1}  f(z)dz} \geq \frac{1}{4}\abs{J}\]
        \[\triangle \supset \triangle_1 \supset  \triangle_2 \supset ... \supset 
        \triangle_n \supset ...\]
        \[\abs{\int_{\partial \triangle_n} f(z)dz } \geq \frac{1}{4} \abs{
        \int_{\partial \triangle_{n - 1} }f(z)dz } \supset ... \supset \frac{1}{4^n} 
        \abs{J}\]
        \[\abs{J} \leq 4^n \int_{\partial \triangle n} f(z)dz \]
        \[L = L(\partial \triangle)\]
        \[L(\partial \triangle_n) = \frac{\triangle}{2^n}\] 
        \[\triangle_1 \supset \triangle_2 \supset \triangle_3 ... \supset \triangle_n 
        \supset ...\]
        \[\text{diam } \triangle_n \to 0\]
        \[\bigcap_{k = 1}^\infty \triangle_k = \{z_0\} \subset \triangle \subset \Omega \]
        \[f \text{ - диф-ма в т. } z_0 \Ra\]
        \[\lim_{z \to  z_0} \frac{f(z) - f(z_0)}{z - z_0}  = f'(z_0)\]
        \[\forall \mathcal{E} > 0 \q \exists  \delta : \abs{z - z_0} < \delta \Ra\]
        \[\abs{f(z) - f(z_0) - f'(z_0)(z - z_0)} < \mathcal{E}[z - z_0]\]
        \[\text{т.к. } \text{diam}(\triangle_n) \to  0 \Ra \exists  n_0 : \forall  n > 
        n_0\]
        \[\forall z \in \overline{\triangle_n} \qq
        \abs{f(z) - f(z_0) - f'(z_0)(z - z_0)} < \mathcal{E}\abs{z - z_0}\]
        \[\int_{\partial \triangle_n} f(z)dz = \int_{\partial \triangle_n} 
        (f(z) - f(z_0) - f'(z_0)(z - z_0))dz\]
        \[\text{т.к. } \int_{\gamma} (az + b)dz = \int_{\gamma} d(\frac{az^2}{2} + bz)  \]
        Бум! стерли! TODO
        \[\abs{\int_{\partial \triangle_n} f(z)dz} =\abs{ \int_{\partial \triangle_n} 
        (f(z) - f(z_0) - f'(z_0)(z - z_0))dz} \leq \]
        \[\leq \max_{\delta \triangle_n} \abs{f(z) - f(z_0) - f'(z_0)(z - z_0)} \cdot 
        L(\partial \triangle_n) < \mathcal{E} \cdot L(\partial \triangle_n)^2 = 
    \mathcal{E} (\frac{L}{2^n})^2\]
        \[\abs{J} \leq 4^n \abs{\int_{\partial \triangle_n} f(z) dz} < 
        4^n \mathcal{E} \frac{L^2}{4^n}  = \mathcal{E} L^2 \qq \forall \mathcal{E} > 0\]
        \[\Ra \abs{J} = 0\text{, т.е } \int_{\triangle} f(z)dz = 0 \]

        \[\RNumb{2}) \text{ если } p \in \text{ одна из вершин } \qq \text{напр } p = A\]
        %рисунок11
        \[\text{по п. } \RNumb{1} \qq \int_{\triangle BQR} f = \int_{BCQ} f = 0  \]
        \[\abs{\int_{\triangle PQA} f(z)dz} \leq M \cdot L(\triangle RQA) \]
        \[f \text{ - непр. на комп. } \Ra \abs{f} \leq M < \infty\]
        \[\forall  \mathcal{E} > 0 \q \exists  R, Q :\q  L(\triangle RQA) < \mathcal{E}\]
        %рисунок12
    \end{Proof}
\end{lect}

\end{document}
