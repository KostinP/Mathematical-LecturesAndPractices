\documentclass[main]{subfiles}

\begin{document}
\begin{lect}{2019-09-18}
		\subsection{Глобальные св-ва непрерывности}
		\begin{Theorem}[непрерывный образ компатка]
				\[f \in C(E, \R^m) \rla f: E \to \R^m \text{ - непр. в } E\]
				\[K \text{ - компакт,} \q K \subset \R^n,\q f \in C(K, \R^m)\]
				\[\text{Тогда } f(K) \text{ - компакт}\]
		\end{Theorem}

		\begin{Proof}
				рисунок 1
				\[\text{Пусть } \{U_\alpha\}_{\alpha \in A} \text{ - откр. покр } f(K)  \]
				\[f(K) \subset \bigcup_{\alpha \in A} U_\alpha \]
				\[\ra f^{-1} (U_\alpha) \text{ - откр, причем}\]
				\[K \subset \bigcup_{\alpha \in A} f^{-1} (U_\alpha)  \text{ - откр. покр. комп } \ra
				\exists f^{-1} (U_\alpha_1) ... f^{-1} (U_{\alpha_N} )\]
				\[K \subset \bigcup_{k = 1}^N f^{-1} (U_{\alpha_k} ) \ra\]
				\[f(K) \subset \bigcup_{k = 1}^N U_{\alpha_k} \text{ - выделили конечное подпокрытие} \]
				\[f(K) \text{ - компакт}\]
		\end{Proof}

		\begin{Theorem}[Вейерштрасс]
				\[K \text{ - компакт; } f \in C(K, \R^m)\]
				Тогда
				\begin{enumerate}
						\item $f$ - огр.
						\item Если m = 1, то f достигает $\sup$ и $\inf$ на K
				\end{enumerate}
		\end{Theorem}

		\begin{Proof}
				\[f: K \to \R^m \text{ - огр} \rla \exists M : \forall x \in K \q d(f(x), 0) < M\]
				\begin{enumerate}
						\item $f(k)$ - комп $\ra$ огр
						\item $\displaystyle f : K \to \R \ra  M = \sup_{x \in K} f(x) < +\infty $
							рисунок 2
							\[\forall k \in \N \q \exists x^k \in K:\]
							\[M - \frac{1}{k} < f(x^k) \leq M \ra f(x^k) \underset{k \to  \infty}{\to} M\]
							\[f(x^k) \in f(K) \text{ - компакт } \ra \text{ замнг}\]
							\[M \in f(K)\]
				\end{enumerate}
		\end{Proof}

		\begin{Theorem} [Кантор]
				\[f \in C(K, \R^m) \q K \subset \R^n \text{ - компакт } \ra f \text{ - равном. непр на } K \]
		\end{Theorem}

		\begin{Proof}
				\[f \text{ - непр } \ra \text{ непр. } \forall x \in K \q \forall \mathcal{E} > 0
				\exists \delta_x :\]
				\[\forall x' \in K \q d(x', x) < 2 \delta_x \ra d(f(x'), f(x)) < \frac{\mathcal{E}}{2}\]
				рисунок 3
				\[\{B_x(\delta_x)\}_{x \in K} \text{ - откр. покрытие } K \text{ - комп.} \]
				\[\text{выделим конечное поддпокр.}\]
				\[K \subset \bigcup_{j = 1}^N B_{x_j} (\delta_{x_j} )\]
				\[\delta = \min_{1 \leq j \leq N} \delta_{x_j} \text{ - то, что надо} \]
				\[\text{Пусть } d(\widetilde{x}, \widetilde{\widetilde{x}}) < \delta\]
				\[\widetilde{x} \in K \ra \exists x_l : \widetilde{x} \in B(x_l, \delta_{x_l}) \]
				\[d(\widetilde{\widetilde{x}}, x_l) \leq d(\widetilde{\widetilde{x}}, \widetilde{x}) +
				d(\widetilde{x}, x_l) < \delta + \delta_{x_l}  < 2 \delta_{x_l} \]
				\[\ra d(f(\widetilde{\widetilde{x}}), f(x_l)) < \frac{\mathcal{E}}{2} \text{ и }
				d(f(\widetilde{x}), f(x_l)) < \frac{\mathcal{E}}{2}\]
				\[d(f(\widetilde{x}), f(\widetilde{\widetilde{x}})) \leq
				d(f(\widetilde{x}), f(x_l)) + d(f(\widetilde{\widetilde{x}}), f(x_l)) < \mathcal{E}\]
		\end{Proof}

		\subsection{$\R^n$ как лин. пр-во}
		\begin{Definition}
				\[\text{Норма в } \R^n: \q ||\cdot||: \R^n \to [0, +\infty)\]
				Аксиомы нормы
				\begin{enumerate}
						\item $||x|| \geq 0$
						\item $||x|| = 0 \rla x = 0$
						\item $||k \cdot x|| = |k| \cdot ||x||$
						\item $||x + y|| \leq ||x|| + ||y||$
				\end{enumerate}
				Стандартная норма в $\R^n$
				\[||x|| = d(x, 0) = \sqrt{\sum_{k = 1}^n |x_k|^2 }\]
				\[||x+y|| = d(x+y, 0) = d(x, -y) \leq d(x, 0) + d(0, -y) = ||x|| + ||y||\]
				Бывают другие нормы\\
				УПР.1 пусть $||| \cdot |||$ - другая норма в $\R^n$\\
				Тогда \q$\exists c, C > 0:$
				\[c \cdot ||x|| \leq |||x||| \leq C \cdot ||x|| \q \forall x \in \R^n\]
				УПР.2 $\forall$ норма непр в $\R^n$
		\end{Definition}

		\subsection{$\R^n$ - пр-во со скал. пр-нием}
		\begin{Definition}
				\[x, y \in \R^n\]
				\[x \cdot y = (x; y) = \sum_{j = 1}^n x_j y_j \]
				\[||x||^2 = (x; x)\]
				н-во К-Б
				\[(x, y)^2 \leq ||x||^2 \cdot ||y||^2\]
		\end{Definition}

		\subsection{Линейные операторы в $\R^n$}
		\begin{Definition}
				\[LL(\R^n, \R^m) \text{ - лин. операторы}\]
				\[L \in LL (\R^n, \R^m):\]
				\[\forall x, t \in \R^n; \q \forall a, b \in \R:\]
				\[L(ax + by) = aL(x) + bL(y)\]
				\[\text{пишут } Lx \text{ вместо } L(x)\]
				\[LL(\R^n, \R^m) \text{ - лин. пр-во:}\]
				\[\text{если } A, B \in LL(\R^n, \R^m),\]
				\[\text{то } (A + B) (x) = Ax + Bx\]
				\[A + B \in LL(\R^n, \R^m)\]
				\[\forall k \in \R\]
				\[(kA)(x): k \cdot Ax\]
				\[kA \text{ - тоже лин. оператор}\]
				Кроме того $A \in LL(\R^n, \R^m), B \in LL(\R^k, \R^n)$
				\[AB = A \circ B \in LL(\R^k, \R^m)\]
				Пусть $\{e_j\}_{j = 1}^n $ - базис (ортонорм) в $\R^n$; \q $\{e^*_j\}_{j = 1}^m $ - базис в $\R^m$
				\[\text{Тогда } \forall \text{ лин. оператору соотв. }Mat(A)\]
				\[Ae_j = \sum_{k=1}^m a e^*_k  \q\q Mat(A) = \begin{pmatrix}
					a_{11} & a_{1j} & ... & a_{1n}\\
					...\\
					a_{m1} & a_{mj} & ... & a_{mn}
				\end{pmatrix}\]
				\[LL(\R^n, \R^m) \simeq Mat_\R (m \times n) \simeq \R^{mn} \]
				\[Mat(A \cdot B) = Mat(A) \cdot Mat(B) \text{ - матричное произв.}\]
				\[A \in LL(\R^n, \R^m) \q B \in LL(\R^k, \R^n)\]
		\end{Definition}

		\begin{Theorem}
				\[LL(\R^n, \R^m) \subset C(\R^n, \R^m)\]
		\end{Theorem}

		\begin{Proof}
				\[d(x, y) = ||x - y||\]
				\[A : \R^n \to  \R^m \text{ - лин. оператор}\]
				\[||Ax - Ay|| = ||\underset{A( \sum^{n}_{j = 1} (x_j - y_j) e_j )}{A(x-y)}|| =
				|| \sum^{n}_{j = 1}  (x_j - y_j) \cdot Ae_j|| \leq\]
				\[\leq \sum^{n}_{j = 1} |x_j - y_j| \cdot ||Ae_j|| \leq M \sqrt{n} ||x - y||\]
				\[M = \max_{1 \leq j \leq n} ||Ae_j|| \q\q \forall \mathcal{E} > 0 \q \exists
				\delta = \frac{\mathcal{E}}{M \sqrt{n}}\]
		\end{Proof}

		\[B_0(1) = \{x \in \R^n : ||x|| < 1\} \text{ - компакт}\]
		\[A \in LL(\R^n, \R^m) \text{ - непр на } B_0(1)\]
		\[\ra \text{ огр.}\]
		\[||Ax|| \text{ - нерп } \R^n \to  \R\]
		\[\ra \text{ достигает наиб. знач. на комп. } B_0(1)\]

		\begin{Consequence}
				\[\sup_{||x|| \leq 1}  ||Ax|| = \max_{||x|| \leq 1} ||A_x|| < \infty \]
		\end{Consequence}

		\begin{Definition}
				\[A \in LL(\R^n, \R^m)\]
				\[\text{Норма лин. оператора } A\]
				\[||A|| = \max_{|x| \leq 1} ||A_x|| \]
		\end{Definition}

		\begin{Theorem}
				\[||A|| = \max_{||x|| = 1}  ||Ax|| = \sup_{||x|| \neq 0}  \frac{||A_x||}{||x||}\]
				\[\text{т.е. } \forall x \in \R^n \q ||A_x|| \leq ||A|| \cdot ||x||\]
		\end{Theorem}

		\begin{Proof}
				\[\text{Если } A \equiv 0 \text{ - очев.} \q (||A|| = 0)\]
				\[\text{Пусть } A \not \equiv 0 \ra\]
				\[\exists x^* \in \R^n \setminus \{0\} : ||Ax^*|| \neq 0\]
				\[0 \neq \frac{||Ax^*||}{||x^*||} = || A \frac{x^*}{\underset{= y^* \in \phi_1 \subset B_0}{||x^*||}} ||\]
				\[\ra ||A|| > 0\]
				Пусть max достигается внутри ед. шара:
				\[||A|| = ||A \widetilde{x}||\]
				\[\text{где } ||\widetilde{x}|| < 1\]
				\[\text{Рассм. } \widetilde{y} = \frac{\widetilde{x}}{||x||}\]
				рисунок5?
				\[||A\widetilde{y}|| = \frac{||A\widetilde{x}||}{||\widetilde{x}||} > ||A\widetilde{x}||\]
				\[\text{т.е. } ||A\widetilde{x}|| \text{ не max!}\]
				\[\ra \max ||Ax|| \text{ в } ||x|| \leq 1 \q \text{ достиг. на сфере } ||x|| = 1\]
				\[||A|| = \max_{||x|| = 1} ||A_x|| \]
				\[||A|| = max_{||x|| = 1} ||Ax|| = \sup_{||x|| = 1} \frac{||Ax||}{||x||} \leq
				\sup_{||x|| \neq 0} \frac{||Ax||}{||x||} \]
				\[\sup_{||x|| \neq 0} \frac{||Ax||}{||x||} = \sup_{||x|| \neq 0} ||A \frac{x}{||x||}|| \leq
				\max_{||y|| = 1} ||Ay|| = ||A|| \]
		\end{Proof}

		\begin{Theorem}
				\begin{enumerate}
						\item Норма оператора действительно норма
						\item $||A \cdot B|| \leq ||A|| \cdot ||B||$
				\end{enumerate}
		\end{Theorem}

		\begin{proof}
				\begin{enumerate}
						\item проверим аксиомы нормы
							\[(1)\q ||A|| \geq 0 \text{ - очев}\]
							\[(2)\q ||A|| = 0 \rla A = 0 \text{ (начало предыдущей теоремы)}\]
							\[(3)\q ||k \cdot A|| = \max_{||x|| = 1} ||(k \cdot A)x|| =
							\max_{||x|| = 1} |k| \cdot ||Ax|| = |k| \cdot ||A|| \]
							\[(4)\q ||A + B|| = \max_{||x|| = 1} ||Ax + Bx|| \leq \max_{||x|| = 1} (||Ax|| + ||Bx||) \leq \]
							\[\leq ||A|| + ||B||\]
						\item $\displaystyle ||(AB)x|| = ||A(Bx)|| \leq ||A|| \cdot ||Bx|| \leq
							||A|| \cdot ||B|| \cdot ||x||$
							\[\sup_{||x|| \neq 0} \frac{||ABx||}{||x||} \leq ||A|| \cdot ||B||\]
							\[\sup = ||AB||\]
				\end{enumerate}
		\end{proof}

		\begin{Theorem}[оценка нормы лин. оператора]
				\[A \in LL (\R^n, \R^m) \q Mat(A) = \begin{pmatrix}
					a_{11} & ... & a_{1n}\\
					...\\
					a_{m1} & ... & a_{mn}
				\end{pmatrix}\]
				\[||A||^2 \leq \sum^n_{i = 1} \sum^m_{j = 1} |a_{ij}|^2  = ||A||^2_{HS} \text{ - норма Гильберта Шмидта} \]
				\[y = Ax = A(\sum_{j = 1}^n \cdot x_j \cdot e_j) = \sum^n_{j = 1} x_j \cdot Ae_j \]
				\[\underset{\text{k-я координата }}{y_k} = \sum^n_{j = 1}x_j(Ae_j)_k \text{ - к-я координата} \]
				\[1 \leq k \leq m\]
					\[|y_k|^2 = |\sum^n_{j = 1} x_j (Ae_j)_k|^2 \leq \sum_{j = 1}^n |x_j|^2 \cdot
				\sum^n_{j=1} \underset{= a_{jk} }{(Ae_j)_k^2} = \]
				\[= ||x||^2 \sum^n_{j = 1} |a_{kj}| \]
				\[y = \begin{pmatrix}
					y_1\\
					y_2\\
					\vdots\\
					y_m
				\end{pmatrix}
				= \begin{pmatrix}
					a_{11} & a_{12} & ... & a_{1n}\\
					a_{21} & a_{22} & ... & a_{2n}\\
					...\\
					a_{m1} & ...    & ... & a_{mn}
				\end{pmatrix}
				\begin{pmatrix}
					x_1\\
					x_2\\
					\vdots\\
					x_n
				\end{pmatrix}
				\]
				\[||y||^2 = ||Ax||^2 = \sum^m_{k = 1}|y_k|^2 \leq ||x||^2 \cdot \sum^m_{k = 1}\sum^n_{j = 1} |a_{kj}|^2\]
				\[||A|| = \sup_{||x|| \neq 0} \frac{||Ax||}{||x||} \leq \sqrt{\sum^m_{k = 1} \sum^n_{j = 1} |a_{kj}| }\]
				УПР $||A||_{HS} \leq \sqrt{n} \cdot ||A|| $
		\end{Theorem}

		\subsection{Дифференцирование}
		\begin{Definition}
				\[E \subset \R^n, \q E \text{ - откр.} \q a \in E\]
				\[f: E \to  \R^m\]
				\[f \text{ - дифф-мо в т. } a \text{, если } \exists L \in LL(\R^n, \R^m)\]
				\[f(a + h) = f(a) + Lh + \underset{\alpha(h)}{o(||h||)} \q\q ||h|| \to  0\]
				рисунок 6
				\[(h: a + h \in E)\]
				\[\alpha(h) = o(||h||) = o(h) \rla \lim_{||h|| \to 0} \frac{||\alpha(h)||}{||h||} = 0 \]
				\[f(a + h) = f(a) + Lh + o(||h||) \rla \lim_{||h|| \to 0} \frac{||f(a+h) - f(a) - Lh||}{||h||} = 0\]
				Если такой L $\exists$ то он ед.
				\[\text{Пусть } h \in \R^n : ||h|| = 1\]
				\[a + t \cdot h\]
				рисунок 7
				\[f(a + th) = f(a) + \underbrace{L(th)}_{= t \cdot Lh}  + o(th)\]
				\[||th|| \to  0\]
				\[\frac{f(a + th)f(a)}{t} = Lh + \frac{o(th)}{t} \underset{t \to 0}{\to 0}\]
				\[Lh = \lim_{t \to  0} \frac{f(a + th) - f(a)}{t} \]
				\[\forall h : ||h|| = 1 \q L \text{ опеределен однозначно } \ra \forall x \neq 0\]
				\[Lx = ||x|| \cdot L \frac{x}{||x||}\]
				\[L \text{ - дифференциал. } f \text{ в т. а}\]
				\[d_a f = L \in LL(\R^n, \R^m)\]
				\[h \in \R^n \q\q d_a f(h) \in \R^m\]
		\end{Definition}

		\begin{examples}
				\[lim_{||h|| \to  0} \frac{||f(a + h) - f(a) - Lh||}{||h||} = 0\]
					\begin{enumerate}
							\item $f = const \ra d_a f = 0$
							\item $f \in LL(\R^n, \R^m) = f(a + h) - f(a) = f(h) \ra Lh = f(h)$
								\[d_a f = f \text{(если f линеен)}\]
							\item если $f, g : \R^n \to \R^m $ - диф. в т. a, то
								\[d_a(f + g) = d_a f + d_a g\]
								\[lim_{||h|| \to 0}  \frac{||(f + g)(a + h) - (f + g)(a) - d_a f(h) - d_a g(h)||}{||h||} = \]
								\[ \leq \lim_{} \frac{|| f(a + h) = f(a) - d_a f(h)|| + || g(a + h) - g(a) - d_a g(h)||}
								{||h||}  = 0\]
							\item $d_a(kf) = kd_a f$
					\end{enumerate}
		\end{examples}

		\subsection{Производная по направлению}
		\begin{definition}
				\[\text{Пусть } ||e|| = 1, \q e \in \R^n \q\q f:E \to \R^m \q a \in E\]
				\[\frac{\partial f}{\partial e}(a) = \lim_{t \to 0} \frac{f(a + te) - f(a)}{t} \]
		\end{definition}

		\begin{theorem} [о производной по напр.]
				\[f : E \to \R^m \text{ - дифф. в т. } a\]
				\[\frac{\partial f}{\partial e} (a) = d_a f(e)\]
				рисунок 7
				\[z = f(x, y)\]
				\[f: E \to \R^1 \q\q E \subset \R^2\]
		\end{theorem}

		\begin{proof}
			\[f(a + te) - f(a) = d_a f(te) + o(te) \q ||te|| \to 0 \q\q ||te|| = |t|\]
			\[\frac{\partial f}{\partial e}(a) = \lim_{t \to 0} \frac{f(a + te) - f(a)}{t} = d_a f(e)\]
		\end{proof}

		\begin{definition}
				Частные производные $\{e_k\}_{k = 1}^n $ - базис $\R^n$
				\[f : E \to \R^m \q\q E \subset \R^n \q a \in E\]
				\[\frac{\partial f}{\partial x_k}(a) = f_{x_k}' (a) = \frac{\partial t}{\partial e_k}(a)\]
		\end{definition}

		\subsection{Матрица Якоби}

		\begin{definition}
				\[\text{Пусть } f \text{ - диф. в т. } a \in E\]
				\[\text{Временно вернмеся к обозначению}  L = d_a f\]
				\[Mat(L) \text{ - матрица Якоби}\]
				\[\begin{pmatrix}
					a_{11} & a_{12} & ... & a_{1n}\\
					...\\
					a_{m1} & a_{m2} & ... & a_{mn}
				\end{pmatrix} \q j \text{ - й столбец - координаты вектора } \]
				\[d_a f (e_j) = \frac{\partial f}{\partial e}(a) = \frac{\partial f}{\partial x_j}(a) \q\q 1 \leq j \leq n\]
				\[a_{kj}  = \frac{\partial f_k}{\partial x_j}(a)\]

				\[Mat(d_a f) = \begin{pmatrix}
					\frac{\partial f_1}{\partial x_1} & \frac{\partial f_1}{\partial x_2} & ... & \frac{\partial f_1}{\partial x_n} \\
					\frac{\partial f_2}{\partial x_1} & ...\\
				  \\
				\frac{\partial f_n}{\partial x_}
				\end{pmatrix}\]
		\end{definition}
\end{lect}
\end{document}
