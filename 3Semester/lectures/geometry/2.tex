\documentclass[main]{subfiles}

\begin{document}
	\begin{lect} {2019-09-16}
		\hsubsection{1.4}{Формула Тейлора}
		\begin{Theorem} [Ф-ма Тейлора]
			\[\vec{t} = \vec{t_0} + \vec{f'}(t_0)(t-t_0) + \frac{\vec{f''}(t_0)}{2!}(t - t_0)^2 + ...\]
			\[+ \frac{\vec{f^{(n)}}(t_0)}{n!}(t - t_0)^n + o(t - t_0)^n\]
			\[\vec{g}(t) = o(t - t_0)^n \text{, если }\]
			\[ \lim_{t \to t_0} \frac{\vec{g}(t)}{(t - t_0)^n} = \vec{0} \]
		\end{Theorem}

		\hsubsection{1.5}{Длина кривой}
		\begin{definition} [Длина кривой]
			%рисунок 1\\
			Пусть есть кривая $\vec{f}(t)$, $t \in [a,b]$

	        $a=t_0<t_1<...<t_n=b$

	        a) $\sup \sum\limits_{i=1}^n |f(t_i)-f(t_{i-1})|$

	        б) $\lim\limits_{\us{i=1..n}{\max} |t_i-t_{i-1}| \ra 0} ...$

	        -длина кривой
		\end{definition}

		\begin{utv}
			Оба определения эквивалентны
		\end{utv}

		\hsubsection{1.6}{Теорема о длине кривой}
		\begin{Theorem}
			\[S \text{ - длина кривой } \Ra S = \int_a^b |\vec{f'} (t)|dt\]
		\end{Theorem}

		\begin{definition}
				Кривая называется спрямляемой, если её длина конечна
		\end{definition}

		\begin{remark}
			Если $|\vec{f'}(t)|$ - интегр. $\Ra$ кривая спрямляемая
		\end{remark}

		\begin{Example}
			\[y = \sin \frac{1}{x} \q (0, 1]\]
			%рисунок 2
			\[y = \begin{cases}
				\sqrt{x} \sin \frac{1}{x} & x \neq 0\\
				0 						   & x = 0
			\end{cases}\]
			%рисунок 3
		\end{Example}

		\begin{proof}
		    $\bigtriangleup_i t = t_i - t_{i - 1}$, $\tau_i \in [t_{i - 1}, t_i ]$, $\bigtriangleup_i f = f(t_i) - f(t_{i - 1})$
	        \begin{multline*}
	            |\int_a^b |f'(t)| dt - \sum^n_{i = 1}(f(t_i) - f(t_{i - 1}))| \leqslant |\int_a^b |f'(t)|dt - \sum^{n}_{i=1} |f'(\tau_i)| \bigtriangleup_i t | + \\
	        	+ |\sum^{n}_{i=1} |f'(\tau_i)| \bigtriangleup t_i| - \sum^{n}_{i=1} |f(t_i) - f(t_{i - 1}) || =
	        	\RNumb{1} + \RNumb{2}
	        \end{multline*}
	        \[\RNumb{2} \leqslant \sum^{n}_{i=1} ||f'(\tau_i)|\bigtriangleup t_i - |f(t_i) - f(t_{i - 1}) || = \sum^{n}_{i=1} ||f'(\tau_i)| - |f'(\sigma_i)|| \bigtriangleup_i t \]
	        \[f'(t) \text{ - непр на } [a, b] \Ra \text{ равномерно непр. на } [a, b] \text{(т. Кантора)}\]
	        \[\forall \mathcal{E} > 0 \q \exists \delta > 0 \text{, если }|\tau_i - \sigma_i| < \delta \Ra
	        |f'(\tau_i) - |f'(\sigma_i)|| < \mathcal{E}\]
	        \[||f'(\tau_i)| - |f'(\sigma_i)|| < \mathcal{E} \text{, если } |\sigma_i - \tau_i| < \delta\]
	        \[\RNumb{2} \leqslant \sum^{n}_{i=1} \mathcal{E} \bigtriangleup_i t =
	        \mathcal{E}(b - a) \us{\mathcal{E} \to  0} \to  0\]
	        \[||f'(\tau_i)| - |f(t_i) - f(t_{i - 1}) || \leqslant ||f'(\tau_i)| - ||f(t_i)| - |f(t_{i - 1})||\]
	        \[|f(t_i)| - |f(t_{i - 1})| = |f(\sigma_i)|\bigtriangleup_i t\]
		\end{proof}

		\begin{definition}
			Параметризация $f: [a, b] \to \R^3$ называется натуральной, если $|f'(t)| = 1$
		\end{definition}

		\begin{theorem}
				Натуральная параметризация $\exists$ и ед.
		\end{theorem}

		\begin{lemma}
			Пусть $f: [a,b] \ra \R^3$, $\tau: [c,d] \ra [a,b]$ - монотонная биекция ($\tau'>0$), тогда $f \circ \tau:[c,d] \ra \R^3$
	        \\
	        Длина кривой (f) не зависит от перепараметризации ($f \circ \tau$)
		\end{lemma}

		\begin{Proof}
			\[\int_a^b |f'(t)|dt \os{?}{=} \int_c^d |(f\circ \tau)'(s) |ds\]
	        \[\int_c^d |(f\circ \tau)'(s) |ds  = \int_c^d|f'(\tau(s)) \cdot \tau'(s) |ds = \int_c^d |f'(\tau(s))| \cdot \tau'(s)ds = \int_a^b |f'(t)| dt\]
	        \[t = \tau(s)\]
		\end{Proof}

		\begin{proof} [Т]
			Существование\\
			Хотим подобрать $\tau$ : $|f'(\tau(s))| = 1$
			\[\sigma(t) = \int_a^t |f'(s)|ds\]
			\[\sigma : [a, b] \to [0, S]\]
			\[S \text{ - длина кривой}\]
			\[\sigma \text{ - возрастающая и дифф. } (\sigma'(t) = |f'(t)|)\]
			\[\sigma \text{ - биекция } \Ra \tau = \sigma^{-1} \]
			\[\int_0^t |(f \circ \tau)'(s)|ds = \int_0^t |f'(\tau(s))| \cdot \tau'(s)ds = \]
			\[ = \int_0^t |f'(\tau(s))| \frac{ds}{\sigma'(\tau(s))} =
			\int_0^t \frac{|f'(\tau(s))|}{|f'(\tau(s))|}ds = t\]
			Единственность
			\[f(t) \text{ и } g(t) \text{ - нат. параметризации}\]
			\[f, g : [0, s] \to \R^3\]
			\[f - g\]
			\[\int_0^s |(f - g)'(t)|dt = \int_0^s |f'(t) - g'(t)| dt \leq \int_0^s ||f'(t)| -|g'(t)||dt = 0\]
		\end{proof}

		\begin{examples}
				\begin{enumerate}
					\item $y = y(x)$
						\[\begin{cases}
								x = t\\
								y = y(t)
						\end{cases}\]
						\[\begin{pmatrix}
							x\\
							y
						\end{pmatrix}' = \begin{pmatrix}
							1\\
							y'(t)
						\end{pmatrix}\]
						\[s = \int_a^b \sqrt{1 + y^2(x)} dx\]
					\item \[\begin{cases}
								x = x(t)\\
								y = y(t)\\
								z = z(t)
							\end{cases}\]
							\[s = \int_a^b \sqrt{\dot{x}^2(t) + \dot{y}^2(t) + \dot{z}^2(t)}dt\]
						\item $r = r(\varphi)$
							\[\begin{cases}
									x = r(\varphi) \cos \varphi\\
									y = r(\varphi) \sin \varphi
							\end{cases}\]
							\[\begin{cases}
									x' = &r'(\varphi) \cos \varphi - r(\varphi)\sin \varphi\\
									y' = &r'(\varphi) \sin \varphi + r(\varphi)\cos \varphi
							\end{cases}\]
							\[\abs{\begin{pmatrix}
								x'\\
								y'
							\end{pmatrix}} =
							\sqrt{x'^2 + y'^2} = \sqrt{r'^2 cos'2 \varphi + r^2 sin^2 \varphi }\]
							\[= \sqrt{r'^2 + r^2}\]
							\[S = \int_{\varphi_0}^{\varphi_1} \sqrt{r'^2(\varphi) + r^2(\varphi)}d\varphi  \]
				\end{enumerate}
		\end{examples}

	\subsection{Репер Френе}
		\begin{Definition}
			\[\vec{v} = \frac{f'(t)}{|f'(t)|}\]
			\[\vec{v} = f'(t) \text{ - если парам. натуральн.}\]
			\[v \text{ - касательный вектор}\]
		\end{Definition}

		\begin{definition}
			Прямая, содерж в $\vec{v}$ наз. касательной к $\vec{f}(t)$ в точке $t_0$
            \[\vec{f}(t_0) + \vec{f}'(t_0) \cdot (t - t_0) = \vec{g}(t)\]
			\[\vec{g}(t) \text{ - ур-е касат. прямой}\]
			\[\text{Нормальная плоскость}\]
			\[f'(t_0) \cdot (\vec{h} - \vec{f}(t_0)) = 0\]
		\end{definition}

		\begin{Theorem}
			\[\delta \text{ - расстояние от }f(t) \text{ до касат. прямой}\]
			\[\Ra \lim_{t \to t_0} \frac{\delta}{|f(t) - f(t_0)|} = 0 \]
			Касательная прямая единств. с таким свойством
		\end{Theorem}
	\end{lect}
\end{document}
