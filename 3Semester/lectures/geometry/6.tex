\documentclass[main]{subfiles}

\begin{document}
	\subsection{Угол между кривыми на поверхности}
	\begin{theorem}
		Угол между кривыми
		\[\cos \alpha = \dfrac{E u_1' u_2' + F(u_1' v_2' + u_2' v_1') + G v_1' v_2'}{\sqrt{E u_1'^2 + 2 F u_1' v_1' + G v_1'^2} \sqrt{...}}\]
	\end{theorem}

	\begin{proof}
		Найдем, как вычисляется угол между кривыми
		\[\begin{cases}
		  u=u_1(t)\\
		  v=v_2(t)
		\end{cases} \qq
		\begin{cases}
		  u=u_2(t)\\
		  v=v_2(t)
		\end{cases}\]
		Нужно найти угол между $\ol{r}_t'(u_1(t), v_1(t))$ и $\ol{r}_t'(u_2(t), v_2(t))$
		\[\cos \alpha = \dfrac{\ol{r}_t'(u_1(t), v_1(t)) \cdot \ol{r}_t'(u_2(t), v_2(t))}{|\ol{r}_t'(u_1(t), v_1(t))| |\ol{r}_t'(u_2(t), v_2(t))|}\]
		\[r_t'(u_1(t), v_1(t)) = \Br{\dfrac{\d x}{\d u} \dfrac{d u_1}{dt} + \dfrac{\d x}{\d v} \dfrac{d v_1}{dt};...}\]
		\[\dfrac{d \ol{r}}{dt} (u_i(t), v_i(t)) = \dfrac{\d \ol{r}}{\d u} u_i' + \dfrac{\d \ol{r}}{\d v} v_i'\]
		\begin{multline*}
		    \dfrac{dr}{dt}(u_1(t),v_1(t)) \dfrac{dr}{dt}(u_2(t),v_2(t)) = \Br{\dfrac{\d \ol{r}}{\d u} u_1' + \dfrac{\d \ol{r}}{\d v} v_1'} \Br{\dfrac{\d \ol{r}}{\d u} u_2' + \dfrac{\d \ol{r}}{\d v} v_2'} =\\
		    = E u_1' u_2' + F(u_1' v_2'+ u_2' v_1') + G v_1' v_2'
		\end{multline*}
		\[\cos \alpha = \dfrac{E u_1' u_2' + F(u_1' v_2' + u_2' v_1') + G v_1' v_2'}{\sqrt{E u_1'^2 + 2 F u_1' v_1' + G v_1'^2} \sqrt{...}}\]
	\end{proof}

	\subsection{Внутренняя геометрия поверхности и первая форма}

	\begin{definition}
		Поверхности $\Phi_1$ и $\Phi_2$ называются \ul{изометричными}, если $\e$ параметризации $\ol{r}_1$ у $\Phi_1$ и $\ol{r}_2$ у $\Phi_2$\\
		$r_1,r_2: D \ra \R^3$ и $\forall$ кривой в D $|r_1(l)| = |r_2(l)|$
	\end{definition}

	\begin{definition}
		\ul{Внутренняя метрика} поверхности \\ $\vaphi(A,B) = \inf \{ \text{длина кривой на поверхности, соединяющей A и B} \}$
		%картинки про совпаающие штуки
	\end{definition}

	\begin{theorem}
		Если у $\Phi_1$ и $\Phi_2$ совпадают коэффициенты \RNumb{1} кв. формы, то они изометричны
	\end{theorem}

	\begin{proof}
		Уже доказали, потому что формула вычисления длины кривой одинаковая на обеих поверхностях
	\end{proof}

	\begin{remark}
		Если поверхности изометричны, то $\e D$ и параметризации\\
		$\ol{r}_1, \ol{r}_2: D \ra \R^3$, $r_i$ --- параметризация поверхности $\Phi_i$ такие что\\
		$E,F,G$ совпадают для $\ol{r}_1$ и $\ol{r}_2$
	\end{remark}

	\begin{proof}
		%картинка про изометрию
		f --- изометрия\\
		Кривая в D: $\begin{cases}
		  u = t & u'=1\\
		  v=v_0  & v'=0
		\end{cases}$
		\[l_1 = \int_a^b \sqrt{E_1 1} dt\]
		\[l_2 = \int_a^b \sqrt{E_2 1} dt\]
		т.к. $l_1 = l_2 \Ra E_1 = E_2$\\
		Аналогично $G_1 = G_2$ $\left(\begin{cases}
		  u=u_0\\
		  v=t
		\end{cases}\right)$
		\[\begin{cases}
		  u = t + u_0, & u' = 1\\
		  v = t + v_0, & v' = 1
		\end{cases}\]
		\[\int_a^b \sqrt{E_1 + 2 F_1 + G} dt = \int_a^b \sqrt{E_2 + 2F_2 + G_2} dt\]
		\[E_1 + 2F_1 + G_1 = E_2+ 2 F_2 + G_2 \Ra F_1 = F_2\]
	\end{proof}

	\begin{consequence}
		\RNumb{1} кв. форма определяет внутреннюю геометрию
		%картинка про параметризации
	\end{consequence}

	%картинка про уроки рисования от солынина

	\begin{example}
		Сфера $x^2 + y^2 + z^2 = R^2$
		\[\begin{cases}
		  x = R \cos\varphi \cos\psi\\
		  y = R \sin\varphi \cos\psi\\
		  z = R \sin\psi
		\end{cases}\]
		%рисунок сферы с подписями
		\[\ol{r} = (R \cos\varphi \cos\psi,\ R\sin\varphi \cos\psi,\ R \sin\psi)\]
		\[r'_{\varphi} = (-R \sin\varphi \cos\psi,\ R\cos\varphi \cos\psi,\ 0)\]
		\[r'_{\psi} = (R \cos\varphi \sin\psi,\ -R\sin\varphi \sin\psi,\ R \cos\psi)\]
		\[E = r_{\varphi}'^2 = R^2 \sin^2 \varphi \cos^2 \psi + R^2 \cos^2 \varphi \cos^2 \psi = R^2 \cos^2 \psi\]
		\[F = R^2 \sin\varphi \cos\varphi\cos\varphi\sin\psi - R^2 \cos\varphi \sin\varphi \cos\varphi \sin\psi + 0 = 0\]
		%рисунок про мередины и параллели перп. из-за 0
		\[G = R^2\]
		%т.к. это константа (не зависит от фи и пси) - это движение по данным мередианам и параллелям равномерное и скорость не завсит от точки перемещения
	\end{example}

	\begin{Example} [параметризация поверхности вращения]
		%картинка про вращение кривой вокруг OZ
		\[\begin{cases}
		  x = f(t) \cos\varphi\\
		  y = f(t) \sin\varphi\\
		  z = g(t)
		\end{cases}\]
	\end{Example}

	\begin{upr}
		У любой поверхности вращения $F=0$, E не зависит от $\varphi$, G тоже
	\end{upr}

	\begin{Theorem}
		\[|\ol{r}'_u \times \ol{r}'_v| = \sqrt{EG-F^2}\]
		%рисунок про геометрический смысл
	\end{Theorem}

	\begin{Proof}
		\[\ol{r}_u' \times \ol{r}_v' = (\ol{x}_u,\ \ol{y}_u,\ \ol{z}_u) \times (\ol{x}_v,\ \ol{y}_v,\ \ol{z}_v) = (y_u z_v - z_u y_v,\ z_u x_v - x_u z_v,\ x_u y_v - y_u x_v)\]
		\[|\ol{r}_u' \times \ol{r}_v'| = \sqrt{(y_u z_v - z_u y_v)^2 + (z_u x_v - x_u z_v)^2 + (x_u y_v - x_v y_u)^2} = \]
		\[= \sqrt{\us{=A}{(y_u^2 z_v^2 + z_u^2 y_v^2)} - 2 \us{=B}{(y_u z_v z_u y_v + z_u x_v z_v x_u + x_u x_v y_u y_v)}} = \sqrt{A - 2B}\]
		\[EG-F^2 = (x_u^2 + y_u^2 + z_u^2)(z_v^2 + y_v^2 + z_v^2) - (x_u x_v + y_u y_v + z_u z_v)^2 =\]
		\[= (x_u^2 x_v^2 + y_u^2 y_v^2 + z_u^2 z_v^2) + A - (x_u^2 x_v^2 + y_u^2 y_v^2 + z_u^2 z_v^2) - 2B = A - 2B\]
	\end{Proof}

	\begin{Consequence}
		\[EG-F^2 > 0\]
	\end{Consequence}
\end{document}
