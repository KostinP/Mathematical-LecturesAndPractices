\documentclass[12pt, fleqn]{article}

\usepackage{../../../template/template}

 
\begin{document}

\begin{lect} {2019-09-30 Вычисление кручения}
	\begin{Reminder}			
		\[(\vec{a}; \vec{b}; \vec{c}) = (\vec{a}; \vec{b}; \vec{c} + \alpha \vec{a})\]
		\[k = \frac{\abs{f' \times f''}}{\abs{f'}^3}\]
	\end{Reminder}

	\begin{Proof}
		\[g(s) \text{ - нат. парам.}\]
		\[g'(s) = \vec{v} \q\q \abs{\vec{v}} = 1\]
		\[g''(s) = v' = k \vec{n}\]
		\[g'''(s) = kn' = k(-k \vec{v} + \ae \vec{b}) = -k^2 \vec{v} + \ae k \vec{b}\]
		\[(g', g'', g''') = (\vec{v}; k \vec{n}; -k^2 \vec{v} + \ae k \vec{b}) = 
		(v; kn; \ae k b) = \ae k^2\]
		\[\ae = \frac{(g', g'', g''')}{k ^ 2} \text{ в нат парам.}\]
	\end{Proof}	
	\begin{Proof}
	    \[f(t) \text{ - парам } (\forall)\]
		\[S = \psi(t) = \int_a^t \abs{f'(\tau)} d\tau \q\q g(s) \text{ - нат. парам}\]
		\[\psi'(t) = \abs{f'(t)}\]
		\[g(S) = g(\psi(t)) = f(t)\]
		\[f'(t) = g'(\psi(t)) \cdot \psi'(t) = g'(s) \cdot \abs{f'(t)}\]
		\[f''(t) = g''(\psi(t))(\psi(t))^2 + g'(\psi(t))\psi''(t) = g''(s) \cdot \abs{f'(t)}^2 + 
		g'(s) \cdot \psi''(t)\]
		\[f'''(t) = g'''(\psi(t))(\psi'(t))^3 + g''(\psi(t)) \cdot 3 \psi'(t) \psi''(t) + 
		g'(\psi(t)) \cdot \psi'''(t)\]
		\[(f', f'', f''') = (\vec{f'}(s) \cdot \abs{f'(t)}; \  \vec{g}''(s) \abs{f'(t)}^2, 
		g'''(s) \cdot \abs{f'(t)} ^3) = \]
		\[ = (g', g'', g''') \cdot \abs{f'(t)}^6\]
		\[\ae = \frac{(g', g'', g''')}{k^2} = \frac{(f', f'', f''')}{\abs{f'(t)}^6} \cdot 
		\frac{\abs{f'(t)} ^6}{\abs{f' \times f''}^2} = \frac{(f', f'', f''')}{\abs{f' \times f''}^2}\]
	\end{Proof}

	\begin{Example}
		\[\begin{cases}
				x = t\\
				y = f(t)
		\end{cases}\]
		\[y = f(x) \q\q \vec{f} = (x; f(x); 0) \q\q \vec{f}'(1; f'(x); 0) \q\q f''(0; f''(x); 0)\]
		\[f'''= (0; f'''(x); 0)\]
		\[k = \frac{\abs{f''(x)}}{(1 + f'^2(x))^{\frac{3}{2}} }\]
		\[f' \times f'' = (0; 0; f''(x))\]
		\[\ae = 0\]
	\end{Example}

	\subsection{Дополнение 1: плоскости, связ. с кривыми}
рисунок 1 (вектора)
	\begin{definition}
	    Соприкас плоскость : $<\vec{v}, \vec{u}>$\\
		Нормальная плоскость кривой : $<n, b>$\\
		Спрямляющая плоскость : $<v, b>$
	\end{definition}
	
	\begin{Theorem}
		\[\vec{f}(t) = (f_1(t); f_2(t); f_3(t)) \text{ ур-е нормали плоск.}\]
		\[\vec{v} \parallel f'(t) = (f_1', f_2', f_3') \q f'_1(t_0) \cdot(x - f_1(t_0)) + 
		f_2'(t_0) \cdot (y - f_2(t_0)) + f'_3(t_0) \cdot (z - f_3(t_0)) = 0\]
		\[f' \times f'' \parallel b\] %1
		\[(f'_1, f'_2, f'_3) \times (f''_1, f''_2, f''_3) = (f'_2 f''_3 - f'_3 f''_2; 
		f'_3 f_1'' - f'_1 f''_3; f'_1 f''_2 - f'_2 f''_1)\]
		Соприкас плоск.
		\[\begin{vmatrix}
			f'_1(t_0) & f_2'(t_0) & f'_3(t_0)\\
			f_1''(t_0 & f''_2(t_0) & f''_3(t_0)\\
			x - f_1(t_0) & y - f_2(t_0) & z - f_3(t_0)
		\end{vmatrix} = 0\]
		\[(f'(t_0) \times f''(t_0)) \times f'(t_0) \parallel \vec{n}\]
		Ур-е спрям. плоск - УПР
	\end{Theorem}

	\begin{Theorem}
		рисунок 2 (кривая с точкой и плоск)
		\[\delta \text{ - расст. от } f(t) \text{ до соприкас. плоскости}\]
		Если плоскость явл. соприкас., то 
		\[\lim_{t \to t_0} \frac{\delta}{\abs{f(t) - f(t_0)}^2} = 0 \]
		Плоскость с таким соотношением ед.
	\end{Theorem}

	\begin{Proof}
		\[a) \q f(t_0) = (0, 0, 0)\]
		\[b) \q OX \parallel \vec{v}(t_0)\] % 2
		\[c) \q OY \parallel \vec{n}(t_0)\]
		\[d) \q t_0 = 0\]
		\[e) \q t \text{ - нат. параметр} \] 
		\[\text{б, в } \Ra OZ \parallel \vec{b}(t_0)\]
		
		\[f(t) = (f_1(t); f_2(t) ; f_3(t)) \Ra \delta = \abs{f_3(t)s}\]
		Соприкас $z = 0$
		\[\vec{v} \parallel f' = (f'_1, f'_2, f'_3) \parallel OX \Ra f'_2(0) = 0, \q f'_3(0) = 0 \q
		f'_1(0) \neq 0\]
		\[\vec{n} \parallel f'' = (f''_1, f_2'', f_3'') \parallel OY \Ra f''_1(0) = 0; \q f''_3(0) = 0\]%3
		\[\text{Хотим } \lim_{t \to 0} \frac{\abs{f_3(t)}}{\abs{f(t)}^2} = 0\]
		\[\lim_{t \to t_0} \frac{f_3(t)}{f_1(t)^2 + f_2(t)^2 + f_3(t)^2} = %4
		\lim_{t \to 0} \frac{f'_3(t)}{2 f_1(t) f_1'(t) + 2f_2(t)f'_2(t) + 2f_3(t) f'_3(t)}\]
		\[= \frac{1}{2} \lim_{t \to 0} \frac{f''_3(t)}{f'_1^2(t) + f_1(t) f''_1(t) + 
		f_2(t)f_2''(t) + f'_3^2(t) + f_3(t)f''_3(t)} \] %5
	\end{Proof}

	\subsection{Дополнение 2: натур. ур-я кривой}
	\begin{Theorem}
		\[g_1(s) \text{  и } g_2(s) \text{ - нат. парам. двух кривых}\]
		\[\begin{align}
			&k_1(s) & k_2(s)\\
			&\ae_1(s) & \ae_2(s)
		\end{align} \text{ - кривизны и кручения}\]
		\[\begin{align}
			\text{Если }& k_1(s)   = k_2(s)\\
						& \ae_1(s) = \ae_2(s)
		\end{align} \Ra \text{ кривые наклад. при движении пр-ва}\]
	\end{Theorem}

	\begin{Proof}
		\[v_1(s), n_1(s), b_1(s) \text{ - базис Френе \RNumb{1} кривой}\]
		\[v_2(s), n_2(s), b_2(s) \text{ - базис Френе \RNumb{2} кривой}\]
		\[\begin{align}
			\text{Считаем }& v_1(s_0) =& v_2(s_0)\\
						   & n_1(s_0) =& n_2(s_0)\\
						   & b_1(s_0) =& b_2(s_0)
		\end{align}\] %6
		\[h(s) = \vec{v}_1(s) \vec{v}_2(s) + \vec{n}_1(s) \vec{n}_2(s) + \vec{b}_1(s) \vec{b}_2(s) 
		\q h(s_0) = 3\]
		\[h'(s) = v_1' v_2 + v_1 v_2' + n_1'n_2 + n_1 n'_2 + b_1' b_2 + b_1 b'_2 = \] %7
		\[= \ul{\ul{k_1 n_1 v_2}} + \ul{k_2 v_1 n_2} + (\ul{-k_1 v_1} + \ae_1 b_1)n_2 + \ul{\ul{n_1(-k_2 v_2}} + \ae_2 b_2) - 
		\ae_1 n_1 b_2 - \ae_2 b_1 n_2 = 0\]
		\[\Ra h(s0) \equiv 3\]
		\[\Ra v_1 \equiv v_2 \q n_1 \equiv n_2 \q b_1 \equiv b_2\]
	\end{Proof}
\end{lect}

\end{document}
