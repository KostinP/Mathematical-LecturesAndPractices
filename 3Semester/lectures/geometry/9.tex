\documentclass[main]{subfiles}

\begin{document}
  \begin{lect}{2019-11-18}
      \begin{proof}[блистательная теорема Гаусса]
        M - поверхность, $X_0$ - точка M, в $X_0$ сопр
        \[E,F,G \text{ - \RNumb{1} кв. ф.}\]
        \[L,M,N \text{ - \RNumb{2} кв. ф.}\]
        ($\xi, nu$) - напр. во внутр. координатах
        \[k(\xi,n) = \frac{L\xi^2 + 2M\xi\nu + M \nu^2}{E\xi^2 + 2F\xi\nu + G\nu^2} = \frac{Nx^2 + 2Mx + L}{Gx^2 + 2Fx + E} \ra \min\]
        \[x = \frac{\nu}{\xi} (x \ra \infty)\]
        \[= \frac{\RNumb{2}(x)}{\RNumb{1}(x)}\]
        \[k'(0) = 0\]
        \[k'(x) = \frac{\RNumb{2}'(x) \RNumb{1}(x) - \RNumb{2}(x)\RNumb{1}(x)}{\RNumb{1}^2 (x)} = 0\]
        Знаменатель не равен нулю, потому что $EG-F^2 > 0$
        \[\RNumb{2}'(x) = 2Nx + 2M\]
        \[\RNumb{1}'(x) = 2Gx + 2F\]
        \[(2Nx + 2M)(Gx^2 + 2Fx + E) - (2Gx + 2F)(Nx^2 + 2Mx + L) = 0\]
        \begin{multline*}
            \cancel{NGx^3} + MGx^2 + 2NFx^2  + \cancel{2MFx} + FNx + ME-\\ - \cancel{GNx^3} - FNx^2 - 2GMx^2 - \cancel{2FMx} - GLx - FL = 0
        \end{multline*}
        \[x^2 (NF - MG) + x (EN - GL) + (ME - FL) = 0 \q |\cdot \xi^2\]
        \[\nu^2 \begin{vmatrix}
          F & G\\
          M & N
        \end{vmatrix} - \xi\nu \begin{vmatrix}
          G & E\\
          N & L
        \end{vmatrix} + \xi^2 \begin{vmatrix}
          E & F\\
          L & M
        \end{vmatrix} = 0\]
        \[\begin{pmatrix}
          \nu^2 & -\xi\nu & \xi^2\\
          E & F & G\\
          L & M & N
        \end{pmatrix}\]
      \end{proof}

      Хотим понять, что происходит с различными k:
      \[\frac{\RNumb{2}(\xi, \nu)}{\RNumb{1}(\xi, \nu)} = k \qq (k \in \R)\]
      \[x = \frac{\nu}{\xi}\]
      \[\RNumb{2}(x) = k \RNumb{1}(x)\]
      \[Nx^2 + 2 Mx + L = k(Gx^2 + 2Fx +E)\]
      \[(N - kG)x^2 + 2(M - kF)x + (L - kE) = 0\]
      Если уравнение имеет 0 корней, такое число в качестве нормальной кривизны не достигается (если k слишком велико или слишком мало)

      Откуда могут взяться два корня? Пусть в одном направлении кривизна $k_1$, в другом $k_2$. Кривизна направления с углом $\alpha$ по ф-ле Эйлера $k_1 \cos^2 \alpha + k_2 \sin^2 \alpha$\\
      У симметричного относительно OX направления такая же кривизна. Вот откуда два корня

      А в каком случае корень $x_0$ единственный?
      \[\lra \text{напр. главное и $k = k_1$ или $k_2$}\]
      \[(M-kF)^2 = (N - kG)(L-kE)\]
      Его два решения - главные кривизны\\
      <<решать мы его, конечно, не будем>>
      \[M^2 - k MF + k^2 F^2 = NL - k(GL + NE) + k^2 GE\]
      \[k^2 (GE - F^2) - k(GL + NE - 2MF) + (NL - M^2) = 0\]
      \[\Ra K = k_1 k_2 \os{\text{Виет}}{=} \frac{NL - M^2}{EG - F^2}\]
      \[H = \frac{k_1 + k_2}{2} = \frac{1}{2} \frac{GL - MF + NE}{EG - F^2}\]

      \begin{lemma}
          $\ol{a}, \ol{b}, \ol{c}, \ol{d}, \ol{e}, \ol{f}$ - вект
          \[\Ra (\ol{a}, \ol{b}, \ol{c})(\ol{d}, \ol{e}, \ol{f}) = \begin{vmatrix}
              \ol{a}\cdot\ol{d} & \ol{a}\cdot\ol{e} & \ol{a}\cdot\ol{f}\\
              \ol{b}\cdot\ol{d} & \ol{b}\cdot\ol{e} & \ol{b}\cdot\ol{f}\\
              \ol{c}\cdot\ol{d} & \ol{c}\cdot\ol{e} & \ol{c}\cdot\ol{f}
          \end{vmatrix}\]
      \end{lemma}

      \begin{Theorem}[egrerium]
          \[K = \frac{LN - M^2}{EG - F^2} \text{ выражается через $E,F,G$ и их произв.}\]
      \end{Theorem}

      \begin{Proof}[теоремы]
          \[L = \ol{f}_{uu} \ol{n}\]
          \[M = \ol{f}_{uv} \ol{n}\]
          \[N = \ol{f}_{vv} \ol{n}\]
          %рисуночек от Ладушки
          \[\ol{n} = \frac{\ol{f}_u \times \ol{f}_v}{|f_u \times f_v|} = \frac{\ol{f}_u \times \ol{f}_v}{\sqrt{EG - F^2}}\]
          \[L = \frac{(f_{uu}, f_u, f_v)}{\sqrt{EG - F^2}}\]
          \[M = \frac{(f_{uv}, f_u, f_v)}{\sqrt{EG - F^2}}\]
          \[N = \frac{(f_{vv}, f_u, f_v)}{\sqrt{EG - F^2}}\]
          \[K = \frac{1}{(EG - F^2)^2} ( (f_{uu}, f_u, f_v)(f_{vv}, f_u, f_v) - (f_{uv}, f_u, f_v)(f_{uv}, f_u, f_v) )\]
          \[= \frac{1}{EG - F^2} \begin{vmatrix}
            f_{nu}f_{vv} & f_{un}f_{n} & f_{uu}f_{v}\\
            f_{u}f_{vv} & f_{u}f_{n} & f_{u}f_{v}\\
            f_{v}f_{vv} & f_{v}f_{n} & f_{u}f_{v}
          \end{vmatrix} - \begin{vmatrix}
            f_{nv}f_{nv} & f_{uv}f_{n} & f_{uv}f_{v}\\
            f_{u}f_{nv} & f_{u}f_{n} & f_{n}f_{v}\\
            f_{v}f_{nv} & f_{v}f_{n} & f_{n}f_{v}
          \end{vmatrix} =\]

          \[F_u = (f_n f_n)_u = 2 f_n f_{uu}\]
          \[E_v = (f_n^2)_v = 2 f_n f_{nv}\]
          \[F_u = (f_n f_v)_u = f_{nu}f_v + f_n f_uv\]
          \[F_v = (f_n f_v)_v = f_n f_{vv} + f_v f_{uv}\]
          \[G_u = (f_v f_v)_u = 2 f_v f_{nv}\]
          \[G_v = (f_v f_v)_v = 2 f_{vv} f_v\]
          \[f_n f_{nv} = \frac{1}{2} E_v\]
          \[f_{nu} f_v = F_n - \frac{1}{2} E_v\]
          \[f_{vv} f_u = F_v - \frac{1}{2} G_n\]
          \[ = \frac{1}{(EG - F^2)^2} \Br{\begin{vmatrix}
              f_{nn} f_{vv} & \frac{1}{2} F_u & F_n - \frac{1}{2} F_v\\
              F_v - \frac{1}{2} G_n & E & F\\
              \frac{1}{2} G_v & F & G
            \end{vmatrix} - \begin{vmatrix}
                f_{uv}^2 & \frac{1}{2}E_v & \frac{1}{2} G_n\\
                \frac{1}{2}E_v & E & F\\
                \frac{1}{2}F_u & F & G
              \end{vmatrix}} =
          \]
          \[ = \frac{1}{(EG - F^2)^2} \Br{\begin{vmatrix}
              f_{nn} f_{vv} - f_{uv}^2 & \frac{1}{2} F_u & F_n - \frac{1}{2} F_v\\
              F_v - \frac{1}{2} G_n & E & F\\
              \frac{1}{2} G_n & F & G
            \end{vmatrix} - \begin{vmatrix}
              0 & \frac{1}{2}E_v & \frac{1}{2} G_n\\
              \frac{1}{2}E_v & E & F\\
              \frac{1}{2}G_n & F & G
            \end{vmatrix}}
          \]
          \[F_{uv} = (f_{uu} f_v)_v + (f_u f_{uv})_v = f_{uuv} f_v + \ul{f_{uu}f_{vv}} + \ul{f_{uv}f_{uv}}\]
          \[G_{uu} = 2 f_{uv}^2 + 2 f_v f_{uvu}\]
          \[E_{vv} = 2 f_{uv}^2 + 2 f_u f_{uvv}\]
          \[\Ra f_{uu} f_{vv} - f^2_{uv} = F_{uv} - \frac{1}{2}(G_{uu}+E_{vv})\]
          Можем заменить теперь в определителе и теорема будет доказана
      \end{Proof}

      \begin{proof}[леммы]
          Хотим $a \ra \ol{a} + \alpha \ol{b}$ (чтобы было ортоганально)
          \[\begin{vmatrix}
              (a + \alpha b)d & (a + \alpha b)e & (a + \alpha b)f\\
              bd & be & bf\\
              cd & ce & cf
          \end{vmatrix} \os{\text{тот же трюк с разл.}}{=} \begin{vmatrix}
              ad & ae & af\\
              bd & be & bf\\
              cd & ce & cf
          \end{vmatrix} + \ub{=0}{\begin{vmatrix}
              \alpha bd & \alpha be & \alpha bf\\
              bd & be & bf\\
              cd & ce & cf
          \end{vmatrix}}\]
          \[(a,b,c) \ra (a, b + \alpha a, c + \beta a + \gamma b)\]
          (трюк из первого семестра, $a \bot b \bot c \bot a$)\\
          Считаем, что они единичные:
          \[a = (1,0,0) \qq b = (0,1,0) \qq c = (0,0,1)\]
          \[d = (d_1, d_2, d_3) \qq e = (e_1, e_2, e_3) \qq f = (f_1, f_2, f_3)\]
          \[(d,e,f) = \begin{vmatrix}
            d_1 & e_1 & f_1\\
            d_2 & e_2 & f_2\\
            d_3 & e_3 & d_3
          \end{vmatrix}\]
      \end{proof}
  \end{lect}
\end{document}
