\documentclass[12pt, fleqn]{article}
\usepackage{../../../template/template}

%сам документ
\begin{document}
\begin{center}
  \huge Лекции по геометрии, 3 сем
  
  \Large (преподаватель Солынин А. А.)
  
  \large Записали Костин П.А., Щукин И.В.
\end{center}

Данный документ неидеальный, прошу сообщать о найденных недочетах в \href{https://vk.com/drab_existence_a}{вконтакте}
\tableofcontents
\newpage

%билеты
\section{Дифференциальная геометрия кривых и поверхностей (в $\R^3$)}
\subsection{Дифференциальная геометрия кривых}
\subsubsection{Понятие кривой}

\begin{definition}
    $f:[a,b] \rightarrow \R^3$ - вектор-функция. Образ f называется кривой, а f - параметризация этой кривой.
\end{definition}

Способы задания кривых:
\begin{enumerate}
    \item Параметрический $f:[a,b] \rightarrow \R^3$
    \item Явное задание кривой $\begin{cases} y=y(x)\\ z=z(x)
    \end{cases}$ (особенно хорошо на плоскости $y=f(x)$)
    \item Неявное задание кривой (на плоскости) $F(x,y)=0$
    \begin{example}
        Окружность: $x^2+y^2-1=0$
    \end{example}
\end{enumerate}

\begin{theorem}[о неявной функции]
$F(x,y)=0$, F - дифференцируема ($\e \frac{dF}{dx}, \frac{dF}{dy}$ - в окр. $(x_0,y_0)$). $F(x_0,y_0)=0$

Если $\frac{dF}{dy}(x_0,y_0) \neq 0 \ra \e \E>0\ \e f: (x_0-\E,x_0+E) \subset \R$ $F(x, f(x))=0$
\end{theorem}

\begin{reminder} \ \\
    $\dfrac{dF}{dx}|_{(x_0,y_0)}=\lim\limits_{x \rightarrow x_0} \frac{F(x,y_0)-F(x_0,y_0)}{x-x_0}$
\end{reminder}

\[y=f(x) \rightarrow \begin{cases} x=t\\ y=f(t) \end{cases}\q f(t)=(x(t),y(t),z(t))\]

Как задавать вектор-функцию? $f:[a,b] \rightarrow \R^3$ - вектор-функция, тогда 

$\lim\limits_{t \rightarrow t_0} f(t) = (x_0, y_0, z_0)$

$\forall \E>0\ \e \delta>0:$ если $\rho (t,t_0) < \delta$, то $\rho(f(t),(x_0,y_0))<\E$ ($\rho (t,t_0) = |t-t_0|$, $f(t)=\sqrt{(x(t)-x_0)^2+(y(t)-y_0)^2+(z(t)-z_0)^2}$)

Св-ва пределов:
\begin{enumerate}
    \item $\lim\limits_{t \rightarrow t_0} (f(t) \pm g(t)) = \lim f(t) + \lim g(t)$
    \item $\lim\limits_{t \rightarrow t_0} (f(t);g(t)) = (\lim f(t); \lim g(t))$ - скалярное умножение
    \item $\lim\limits_{t \rightarrow t_0} (f(t) x g(t)) = \lim f(t) x \lim g(t)$
\end{enumerate}

\begin{proof}
    $\lim f(t)=(\lim x(t), \lim y(t), \lim z(t))$, $f(t)=(x(t), y(t), z(t))$. Пусть $\E>0$, выберем $\delta: |x(t)-x_0|< \frac{\E}{3}$, аналогично $|y(t)-y_0|<\frac{\E}{3}$ и $|z(t)-z_0|<\frac{\E}{3}$
    
    Значит $\sqrt{(x(t)-x_0)^2+(y(t)-y_0)^2+(z(t)-z_0)^2} < \frac{\E}{\sqrt{3}}$
\end{proof}

\begin{definition}
    $f'(t_0)=\lim\limits_{t \rightarrow t_0} \frac{\overrightarrow{f}(t)-\overrightarrow{f}(t_0)}{t-t_0}$
\end{definition}

\begin{properties}{}
    \begin{enumerate}
        \item $(f(t)\pm g(t))'=f'(t)\pm y'(t)$
        \item $(c f(t))'=c f'(t)$
        \item $(f(t);g(t))'=(f'(t);g(t))+(f(t),g'(t))$
        \item $(f(t) x g(t))' = f'(t) x g(t) + f(t) x g'(t)$
        \item $(f(t),g(t),h(t))'=(f',g,h)+(f,g',h)+(f,g,h')$
    \end{enumerate}
\end{properties}

Доказывается через $f'(t)=(x'(t),y'(t),z'(t))$

Докажем векторное произведение $(f(t) x g(t))' |_{t=t_0} = \lim\limits_{t \rightarrow t_0} \frac{f(t) x g(t) - f(t_0) x g(t_0)}{t-t_0} = \lim\limits_{t \rightarrow t_0} \frac{f(t) x g(t) - f(t_0) x g(t) + f(t_0) x g(t) - f(t_0) x g(t_0)}{t-t_0} = \lim\limits_{t \rightarrow t_0} \frac{(f(t) - f(t_0)) x g(t)}{t-t_0} + \lim\limits_{t \rightarrow t_0} \frac{f(t_0) x (g(t)-g(t_0))}{t-t_0}=f'(t_0) x g(t_0) + f(t_0) x g'(t_0)$

\begin{example}
    Контрпример (т. Лагаранжа) - не всегда верна
\end{example}

Можно ли $\int\limits_a^b \overrightarrow{f}(t) dt= (\int\limits_a^b x(t) dt, \int\limits_a^b y(t) dt, \int\limits_a^b z(t) dt)$

$\overrightarrow{F}'(t)=\overrightarrow{f}(t)$

$\overrightarrow{F}(b)-\overrightarrow{F}(a)= \int\limits_a^b \overrightarrow{f}(t) dt$

$F(t)=(X(t), Y(t), Z(t))$

$f(t)=(X'(t), Y'(t), Z'(t))=(x(t),y(t),z(t))$

$\int\limits_a^b f(t) dt = (\int\limits_a^b x(t) dt,...)=(X(b)-X(a)+..._$ (по ф-ле Н-Л)

\begin{definition}
    Гладкая кривая - образ вектороднозначнойя функция
\end{definition}

\begin{definition}
    Кривая называется регулярной, если 
    
    существует производная и $f'(t) \neq \overrightarrow{0}$
\end{definition}

\begin{definition}
    Кривая называется бирегулярной, если 
    
    существует вторая производная и $f''(t) \not || f'(t)$
\end{definition}

\begin{definition}
    Параметризации $\overrightarrow{f}(t) и \overrightarrow{g}(t)$ ($f:[a,b] \rightarrow \R^3$, $g:[c,d] \rightarrow \R^3$) эквивалентны, если $\e$ биекция $\tau: [a,b] \rightarrow [c,d]:$ $\tau(a)=c,\ \tau(b)=d$, $f(t)=g(\tau(t))$
\end{definition}

\begin{definition}
    Гладкая кривая - класс эквивалентности параметризации
\end{definition}

\begin{proof}
    Докажем, что экв. параметризации - отношение эквивалентность:
    \begin{enumerate}
        \item (рефл.) $\tau=id$
        \item (симм.) $f(t)=g(\tau(t))$, $g(t)=f(\tau(t))$
        \item (тран.) $f(t)=g(b(t))$, $g(t)=h(\tau(t))$, $f(t)=h(\tau(b(t)))$
    \end{enumerate}
\end{proof}

\begin{lemma}
    $\overrightarrow{f}(t)$ - вектор-функция (регулярная), $|\overrightarrow{f}(t)|=1 \ra f'(t) \bot f(t)$
\end{lemma}

\begin{proof}
    $(f(t);f(t))=1 \ra 0=(f(t);f(t))'=2(f'(t);f(t))$. $f(t) \neq 0$ и
    
    $f'(t) \neq 0 \ra f'(t) \bot f(t)$
\end{proof}

\begin{theorem}
    $\vec{f}(t)-\vec{f}(t_0)+\vec{f}'(t_0)(t-t_0)+\dfrac{\vec{f}''(t_0)}{2!}(t-t_0)^2+...+\dfrac{\vec{f}^{(n)}}{n!}$
    
    $\vec{g}(t)=o(t-t_0)^n$, если $\lim\limits_{t \ra t_0} \dfrac{\vec{g}(t)}{(t-t_0)^n}=\vec{0}$
\end{theorem}

\subsubsection{Длина кривой}
\begin{definition}
    Пусть есть кривая $\vec{f}(t)$, $t \in [a,b]$
    
    $a=t_0<t_1<...<t_n=b$
    
    a) $\sup \sum\limits_{i=1}^n |f(t_i)-f(t_{i-1})|$
    
    б) $\lim\limits_{\us{i=1..n}{\max} |t_i-t_{i-1}| \ra 0} ...$
    
    -длина кривой
\end{definition}

\begin{utv}
	Определения а) и б) эквивалентны
\end{utv}

\begin{definition}
    Прямая называется спрямляемой, если её длина конечна
\end{definition}

\begin{definition}
    Если $|\vec{f}(t)|$ - интегр, то спрямляемая
\end{definition}

\begin{example}
	$y = \sin \frac{1}{x} \q (0, 1]$
	
	рисунок 2
	\[y = \left\{\begin{align}
		&\sqrt{x} \sin \frac{1}{x} & x \neq 0\\
		&0 						   & x = 0
	\end{align}\]
	рисунок 3
\end{example}

\begin{proof}
    $\bigtriangleup_i t = t_i - t_{i - 1}$, $\tau_i \in [t_{i - 1}, t_i ]$, $\bigtriangleup_i f = f(t_i) - f(t_{i - 1})$
    \begin{multline*}
        |\int_a^b |f'(t)| dt - \sum^n_{i = 1}(f(t_i) - f(t_{i - 1})| \leqslant |\int_a^b |f'(t)|dt - \sum^{n}_{i=1} |f'(\tau_i)| \bigtriangleup_i t | + \\
    	+ |\sum^{n}_{i=1} |f'(\tau_i)| \bigtriangleup t_i| - \sum^{n}_{i=1} |f(t_i) - f(t_{i - 1}) || =
    	\RNumb{1} + \RNumb{2}
    \end{multline*}
	\[\RNumb{2} \leqslant \sum^{n}_{i=1} ||f'(\tau_i)|\bigtriangleup t_i - |f(t_i) - f(t_{i - 1}) || = \sum^{n}_{i=1} ||f'(\tau_i)| - |f'(\sigma_i)|| \bigtriangleup_i t \]
	\[f'(t) \text{ - непр на } [a, b] \Ra \text{ равномерно непр. на } [a, b] \text{(т. Кантора)}\]
	\[\forall \mathcal{E} > 0 \q \exists \delta > 0 \text{, если }|\tau_i - \sigma_i| < \delta \Ra 
	|f'(\tau_i) - |f'(\sigma_i)|| < \mathcal{E}\]
	\[||f'(\tau_i)| - |f'(\sigma_i)|| < \mathcal{E} \text{, если } |\sigma_i - \tau_i| < \delta\]
	\[\RNumb{2} \leqslant \sum^{n}_{i=1} \mathcal{E} \bigtriangleup_i t = 
	\mathcal{E}(b - a) \us{\mathcal{E} \to  0} \to  0\]
	\[||f'(\tau_i)| - |f(t_i) - f(t_{i - 1}) || \leqslant ||f'(\tau_i)| - ||f(t_i)| - |f(t_{i - 1})||\]
	\[|f(t_i)| - |f(t_{i - 1})| = |f(\sigma_i)|\bigtriangleup_i t\]
\end{proof}

\begin{definition}
    Параметризация $f: [a,b] \ra \R^3$ называется натуральной, если $|f'(t)|=1$
\end{definition}

\begin{lemma}
    Пусть $f: [a,b] \ra \R^3$, $\tau: [c,d] \ra [a,b]$ - монотонная биекция ($\tau'>0$), тогда $f \circ \tau:[c,d] \ra \R^3$
    \\
    Длина кривой (f) не зависит от перепараметризации ($f \circ \tau$)
\end{lemma}

\begin{Proof}
    \[\int_a^b |f'(t)|dt \os{?}{=} \int_c^d |(f\circ \tau)(s) |ds\]
	\[\int_c^d |(f\circ \tau)(s) |ds  = \int_c^d|f'(\tau(s)) \cdot \tau'(s) |ds = \int_c^d |f'(\tau(s))| \cdot \tau'(s)ds = \int_a^b |f'(t)| dt\]
	\[t = \tau(s)\]
\end{Proof}

\begin{theorem}
    Натуральная параметризация $\e$ и единственная
\end{theorem}

\begin{proof}
    Существование\\
	Хотим подобрать $\tau$ : $|f'(\tau(s))| = 1$
	\[\sigma(t) = \int_a^t |f'(s)|ds\]
	\[\sigma : [a, b] \to [0, S]\]
	\[S \text{ - длина кривой}\] 
	\[\sigma \text{ - возрастающая и дифф. } (\sigma'(t) = |f'(t)|)\]
	\[\sigma \text{ - биекция } \Ra \tau = \sigma^{-1} \]
	\[\int_0^t |(f \circ t)'(s)|ds = \int_0^t |f'(\tau(s))| \cdot t'(s)ds = \]
	\[ = \int_0^t |f'(\tau(s)| \frac{ds}{\sigma'(\tau(s))} = 
	\int_0^t \frac{|f'(\tau(s))|}{|f'(\tau(s))|}ds = t\]
	Единственность
	\[f(t) \text{ и } g(t) \text{ - нат. параметризации}\]
	\[f, g : [0, s] \to \R^3\]
	\[f - g\]
	\[\int_0^s |(f \circ g)(t)|dt = \int_0^s |f'(t) - g'(t)| dt \leqslant \int_0^s ||f'(t)| -|g'(t)||dt = 0\]
\end{proof}
	
\begin{examples}
		\begin{enumerate}
			\item $y = y(x)$
				\[\left\{\begin{align}
						&x = t\\
						&y = y(t)
				\end{align}\]
				\[\begin{pmatrix}
					x\\
					y
				\end{pmatrix}' = \begin{pmatrix}
					1\\
					y'(t)
				\end{pmatrix}\]
				\[s = \int_a^b \sqrt{1 + y^2(x)} dx\]
			\item \[\displaystyle \left \{ \begin{align}
					& x = x(t)\\
					& y = y(t)\\
					& z = z(t)
					\end{align}\]
					\[s = \int_a^b \sqrt{x^2(t) + y^2(t) + z^2(t)}dt\]
			\item $r = r(\varphi)$
				\[\left\{ \begin{align}
						&x = r(\varphi) \cos \varphi\\
						&y = r(\varphi) \sin \varphi
				\end{align}\]
				\[\left\{ \begin{align}
						&x' = &r'(\varphi) \cos \varphi - r(\varphi)\sin \varphi\\
						&y' = &r'(\varphi) \sin \varphi + r(\varphi)\cos \varphi
				\end{align}\]
				\[|\begin{pmatrix}
					x'\\
					y'
				\end{pmatrix}| = 
				\sqrt{x'^2 + y'^2} = \sqrt{r'^2 cos'2 \varphi + r^2 sin^2 \varphi }\]
				\[= \sqrt{r'^2 + r^2}\]
				\[S = \int_{\varphi_0}^{\varphi_1} \sqrt{r'^2(\varphi) + r^2(\varphi)}d\varphi  \]
		\end{enumerate}
\end{examples}

\subsubsection{Ренер Френе}
\begin{Definition}
	\[\vec{v} = \frac{f'(t)}{|f'(t)|}\]
	\[\vec{v} = f'(t) \text{ - если параметризация натуральная}\]
	\[v \text{ - касательный вектор}\]
\end{Definition}

\begin{definition}
	Прямая, содерж в $\vec{v}$ наз. касательной к $\vec{f}(t)$ в точке $t_0$
	\[\vec{f(t_0)} + \vec{f}'(t_0) \cdot (t - t_0) = \vec{g}(t)\]
	\[\vec{g}(t) \text{ - ур-е касат. прямой}\]
	\[\text{Нормальная плоскость}\]
	\[f'(t_0) \cdot (\vec{h} - \vec{f}(t_0)) = 0\]
\end{definition}

\begin{Theorem}
	\[\delta \text{ - расстояние от }f(t) \text{ до касат. прямой}\]
	\[\Ra \lim_{t \to t_0} \frac{\delta}{|f(t) - f(t_0)|} = 0 \]
	Касательная прямая единств. с таким свойством
\end{Theorem}

\begin{Proof}
		
\end{Proof}
\end{document}