\documentclass[12pt, fleqn]{article}
\usepackage{../../../template/template}

%сам документ
\begin{document}
\begin{center}
  \huge Лекции по геометрии, 3 сем
  
  \Large (преподаватель Солынин А. А.)
  
  \large Записали Костин П.А., Щукин И.
\end{center}

Данный документ неидеальный, прошу сообщать о найденных недочетах в \href{https://vk.com/drab_existence_a}{вконтакте}
\tableofcontents
\newpage

%билеты
\section{Дифференциальная геометрия кривых и поверхностей (в $\R^3$)}
\subsection{Дифференциальная геометрия кривых}
\subsubsection{Понятие кривой}

\begin{definition}
    $f:[a,b] \rightarrow \R^3$ - вектор-функция. Образ f называется кривой, а f - параметризация этой кривой.
\end{definition}

Способы задания кривых:
\begin{enumerate}
    \item Параметрический $f:[a,b] \rightarrow \R^3$
    \item Явное задание кривой $\begin{cases} y=y(x)\\ z=z(x)
    \end{cases}$ (особенно хорошо на плоскости $y=f(x)$)
    \item Неявное задание кривой (на плоскости) $F(x,y)=0$
    \begin{example}
        Окружность: $x^2+y^2-1=0$
    \end{example}
\end{enumerate}

\begin{theorem}[о неявной функции]
$F(x,y)=0$, F - дифференцируема ($\e \frac{dF}{dx}, \frac{dF}{dy}$ - в окр. $(x_0,y_0)$). $F(x_0,y_0)=0$

Если $\frac{dF}{dy}(x_0,y_0) \neq 0 \ra \e \E>0\ \e f: (x_0-\E,x_0+E) \subset \R$ $F(x, f(x))=0$
\end{theorem}

\begin{reminder} \ \\
    $\dfrac{dF}{dx}|_{(x_0,y_0)}=\lim\limits_{x \rightarrow x_0} \frac{F(x,y_0)-F(x_0,y_0)}{x-x_0}$
\end{reminder}

\[y=f(x) \rightarrow \begin{cases} x=t\\ y=f(t) \end{cases}\q f(t)=(x(t),y(t),z(t))\]

Как задавать вектор-функцию? $f:[a,b] \rightarrow \R^3$ - вектор-функция, тогда 

$\lim\limits_{t \rightarrow t_0} f(t) = (x_0, y_0, z_0)$

$\forall \E>0\ \e \delta>0:$ если $\rho (t,t_0) < \delta$, то $\rho(f(t),(x_0,y_0))<\E$ ($\rho (t,t_0) = |t-t_0|$, $f(t)=\sqrt{(x(t)-x_0)^2+(y(t)-y_0)^2+(z(t)-z_0)^2}$)

Св-ва пределов:
\begin{enumerate}
    \item $\lim\limits_{t \rightarrow t_0} (f(t) \pm g(t)) = \lim f(t) + \lim g(t)$
    \item $\lim\limits_{t \rightarrow t_0} (f(t);g(t)) = (\lim f(t); \lim g(t))$ - скалярное умножение
    \item $\lim\limits_{t \rightarrow t_0} (f(t) x g(t)) = \lim f(t) x \lim g(t)$
\end{enumerate}

\begin{proof}
    $\lim f(t)=(\lim x(t), \lim y(t), \lim z(t))$, $f(t)=(x(t), y(t), z(t))$. Пусть $\E>0$, выберем $\delta: |x(t)-x_0|< \frac{\E}{3}$, аналогично $|y(t)-y_0|<\frac{\E}{3}$ и $|z(t)-z_0|<\frac{\E}{3}$
    
    Значит $\sqrt{(x(t)-x_0)^2+(y(t)-y_0)^2+(z(t)-z_0)^2} < \frac{\E}{\sqrt{3}}$
\end{proof}

\begin{definition}
    $f'(t_0)=\lim\limits_{t \rightarrow t_0} \frac{\overrightarrow{f}(t)-\overrightarrow{f}(t_0)}{t-t_0}$
\end{definition}

\begin{properties}{}
    \begin{enumerate}
        \item $(f(t)\pm g(t))'=f'(t)\pm y'(t)$
        \item $(c f(t))'=c f'(t)$
        \item $(f(t);g(t))'=(f'(t);g(t))+(f(t),g'(t))$
        \item $(f(t) x g(t))' = f'(t) x g(t) + f(t) x g'(t)$
        \item $(f(t),g(t),h(t))'=(f',g,h)+(f,g',h)+(f,g,h')$
    \end{enumerate}
\end{properties}

Доказывается через $f'(t)=(x'(t),y'(t),z'(t))$

Докажем векторное произведение $(f(t) x g(t))' |_{t=t_0} = \lim\limits_{t \rightarrow t_0} \frac{f(t) x g(t) - f(t_0) x g(t_0)}{t-t_0} = \lim\limits_{t \rightarrow t_0} \frac{f(t) x g(t) - f(t_0) x g(t) + f(t_0) x g(t) - f(t_0) x g(t_0)}{t-t_0} = \lim\limits_{t \rightarrow t_0} \frac{(f(t) - f(t_0)) x g(t)}{t-t_0} + \lim\limits_{t \rightarrow t_0} \frac{f(t_0) x (g(t)-g(t_0))}{t-t_0}=f'(t_0) x g(t_0) + f(t_0) x g'(t_0)$

\begin{example}
    Контрпример (т. Лагаранжа) - не всегда верна
\end{example}

Можно ли $\int\limits_a^b \overrightarrow{f}(t) dt= (\int\limits_a^b x(t) dt, \int\limits_a^b y(t) dt, \int\limits_a^b z(t) dt)$

$\overrightarrow{F}'(t)=\overrightarrow{f}(t)$

$\overrightarrow{F}(b)-\overrightarrow{F}(a)= \int\limits_a^b \overrightarrow{f}(t) dt$

$F(t)=(X(t), Y(t), Z(t))$

$f(t)=(X'(t), Y'(t), Z'(t))=(x(t),y(t),z(t))$

$\int\limits_a^b f(t) dt = (\int\limits_a^b x(t) dt,...)=(X(b)-X(a)+..._$ (по ф-ле Н-Л)

\begin{definition}
    Гладкая кривая - образ вектороднозначнойя функция
\end{definition}

\begin{definition}
    Кривая называется регулярной, если существует производная и $f'(t) \neq \overrightarrow{0}$
\end{definition}

\begin{definition}
    Кривая называется бирегулярной, если существует вторая производная и $f''(t) \not || f'(t)$
\end{definition}

\begin{definition}
    Параметризации $\overrightarrow{f}(t) и \overrightarrow{g}(t)$ ($f:[a,b] \rightarrow \R^3$, $g:[c,d] \rightarrow \R^3$) эквивалентны, если $\e$ биекция $\tau: [a,b] \rightarrow [c,d]:$ $\tau(a)=c,\ \tau(b)=d$, $f(t)=g(\tau(t))$
\end{definition}

\begin{definition}
    Гладкая кривая - класс эквивалентности параметризации
\end{definition}

Докажем, что экв. параметризации - отношение эквивалентность:
\begin{enumerate}
    \item (рефл.) $\tau=id$
    \item (симм.) $f(t)=g(\tau(t))$, $g(t)=f(\tau(t))$
    \item (тран.) $f(t)=g(b(t))$, $g(t)=h(\tau(t))$, $f(t)=h(\tau(b(t)))$
\end{enumerate}

\begin{lemma}
    $\overrightarrow{f}(t)$ - вектор-функция (регулярная), $|\overrightarrow{f}(t)|=1 \ra f'(t) \bot f(t)$
\end{lemma}

\begin{proof}
    $(f(t);f(t))=1 \ra 0=(f(t);f(t))'=2(f'(t);f(t))$. $f(t) \neq 0$ и
    
    $f'(t) \neq 0 \ra f'(t) \bot f(t)$
\end{proof}

\end{document}