\documentclass[main, 12pt, fleqn]{subfiles}



\begin{document}



\begin{lect} {Дифф. геометрия кривых (в $\R^3$) и поверхностей (в $\R^3$) 2019-09-09}
	\section{Дифф. геом. кривых}	
		\begin{definition}
		1) Понятие кривой
		(рис 1)
		\[f : [a, b] \to \R^3 \text{ - вектор функция}\]
		\[\text{Образ } f \text{ назыв. кривой}\]
		\[f \text{ - праметризация кривой}\]
		
		Способы задания кривых:
		\begin{enumerate}
			\item параметрический $f: [a, b] \to \R^3$
			\item явное задание
				\[\left\{ \begin{align}
						y = y(x)\\
						z = z(x)
				\end{align}\]
				(особенно хорошо на плоскости $y = f(x)$)
			\item неявное задание (на плоскости)
				\[F(x, y) = 0\]

				\[F(x, y) = 0 \text{ и } (x_0, y_0) \text{ - подх,}\]
				\[z = F(x, y) \text{ рис 2}\]
		\end{enumerate}
	\end{definition}
	
	\begin{Example}
			\[\text{Окружность: } x^2 + y^2 - 1 = 0\]
			\[y = (+-) \sqrt{1 - x^2} \text{ явное задание}\]
			рис3
	\end{Example}
	
	\addcontentsline{toc}{subsection}{Т. о неявной функции} 
	\begin{Theorem} [О неявной функции]
		\[F(x, y) = 0\]
		\[F \text{ - дифф } (\exists \frac{\partial F}{ \partial x} \text{ и } 
		\frac{\partial F}{\partial y} \text{ - непр в окр } (x_0, y_0))\]
		\[F(x_0, y_0) = 0\]
		\[\text{Если } \frac{\partial F}{\partial y}_{(x_0, y_0)}  \neq  0 \ra \exists f 
			\ \ \exists \mathcal{E} > 0:
		(x_0 - \mathcal{E}, x_0 + \mathcal{E}) \to \R\]
		\[F(x, f(x)) = 0\]
	\end{Theorem}
	
	\begin{Definition}
		\[\frac{\partial F}{\partial x}_{(x_0, y_0)} = \lim_{x \to x_0} 
		\frac{F(x, y_0) - F(x_0, y_0)}{x - x_0}\]
		\[\frac{\partial F}{\partial y} \text{ - аналогично}\]
	\end{Definition}

	\[y = f(x) \to \left\{ \begin{align}
			&x = t\\
			&y = f(t)
	\end{align}\]

	\begin{Definition}
		\[f: [a, b] \to  \R^3 \text{ - ветор. функция, то}\]
		\[f(t) = (x(t); y(t); z(t))\]
		\[ \lim_{t \to t_0} f(t) = (x_0, y_0, z_0):\]
		\[\forall \mathcal{E} > 0 \exists \delta > 0 \text{ если}\]
		\[\rho(t, t_0) < \delta \text{, то } \rho(f(t), (x_0, y_0, z_0)) < \mathcal{E}\]
		\[|t - t_0| \q\q \sqrt{((x(t) - x_0)^2 + (y(t) - y_0)^2 + (z(t) - z_0)^2}\]
	\end{Definition}
	
	\addcontentsline{toc}{subsection}{Свойства пределов} 
	\begin{Theorem} [свойства пределов]
		\[ \lim_{t \to t_0} (f(t) (+-) g(t) = \lim_{t \to t_0} f(t) (+-) \lim_{t \to t_0} g(t)\]
		\[ \lim_{t \to t_0} \underset{\text{скал}}{(f(t) \cdot g(t))} = (\lim_{t \to t_0} f(t) , \lim_{t \to t_0} g(t) )
		\text{ (,) - скалярное умножение}\]
		\[ \lim_{t \to t_0} (f(x) \times g(t)) = \lim_{t \to t_0} f(x) \times \lim_{t \to t_0} g(t) \]
	\end{Theorem}

	\begin{Proof}
		\[ \lim_{t \to t_0} f(t) = ( \lim_{t \to t_0} x(t), \lim_{t \to t_0} y(t), \lim_{t \to t_0} z(0)  )\]
		\[f(t) = (x(t), y(t), z(t))\]
		\[eps > 0 \q \text{Выбереме такое } \delta : |x(t) - x_0| < \frac{\mathcal{E}}{3}\]
		\[\text{если } |t - t_0| < \delta \ra	\begin{align}
			|y(t) - y_0| < \frac{\mathcal{E}}{3}\\
			|z(t) - z_0| < \frac{\mathcal{E}}{3}
		\end{align} \ra \sqrt{...} < \frac{\mathcal{E}}{\sqrt{3}} < \mathcal{E}\]
	\end{Proof}

	\begin{Definition}
		\[f'(t_0) = \lim_{t \to t_0} \frac{ \overline{f}(t) - \overline{f}(t_0)}{t - t_0}\]
	\end{Definition}

	\begin{theorem} [свойства]
			\begin{enumerate}
				\item \[(f(t) (+-) g(t))' = f'(t) (+-) y'(t)\]
				\item \[(c f(t))' = cf'(t)\]
				\item \[(f(t); g(t))' = (f'(t); g(t)) + (f(t); g'(t))\]
				\item \[(f(t) \times g(t))' = f'(t) \times g(t) + f(t) \times g'(t)\]
				\item \[(f(t), g(t), h(t))' = (f', g, h) + (f, g', h) + (f, g, h')\]
			\end{enumerate}

			\[f'(t) = (x'(t), y'(t), z'(t))\]
			\[f(t) = (x(t), y(t), z(t))\]

			\[(f(t) \times g(t))'|_{t = t_0} = \lim_{t \to t_0} \frac{f(t) \times g(t) - f(t_0) \times g(t_0)}{t - t_0)} = \]
			\[= \lim_{t \to t_0} \frac{f(x) \times g(x) - f(t_0) \times g(t_0) + f(t_0) \times g(t) - f(t_0) \times g(t_0}{t - t_0}\]
			\[= \lim_{t \to t_0} \frac{(f(t) - f(t_0)) \times g(t)}{t - t_0} + 
			\lim_{t \to t_0} \frac{f(t_0 \times (g(t) - g(t_0))}{t - t_0} = \]
			\[= f'(t_0) \times g(t_0) + f(t_0) \times g'(t_0)\]
	\end{theorem}

	\begin{example}
			Контрпример\\
			Т. Лагранжа  - неверна рис 4
	\end{example}
	
	\[\int_b^a \overrightarrow{f}(t) dt = (\int_a^b x(t)d(t), \int_a^b y(t)dt, \int_a^b z(t)dt) \]
	\[\overrightarrow{F}'(t) = \overrightarrow{f}(t)\]
	\[\overrightarrow{F}(b) - \overrightarrow{F}(a) = \int_a^b \overrightarrow{f}(t)dt\]
	\[F(t) = (X(t), Y(t), Z(t))\]
	\[f(t) = (X'(t), Y'(t), Z'(t)) = (x(t), y(t), z(t))\]
	\[\int_a^b f(t)dt = (\int_a^b x(t) dt, ....) = (X(b) - X(a), ....\]
	
	\addcontentsline{toc}{subsection}{Гладка кривая, регулярная кривая}
	\begin{definition}
			Гладкая кривая - диффер. векторнозначная функция и ее образ тоже\\
			Кривая называется регулярной, если $f'(t) \neq \overrightarrow{0}$\\
			(Кривая называется бирегулярной, если $\exists f''(t)$ и $f''(t) \not \parallel f'(t)$)\\
			рис 5\\
			гладкая кривая = класс эквиалентности параметризаций
	\end{definition}

	\begin{definition}
		Параметризации $\overrightarrow{f}(t) $ и $\overrightarrow{g}(t)$ эквивалентны
		\[f: [a, b] \to \R^3\]
		\[g: [c, d] \to \R^3\]
		Если $\exists$ биекция $\tau: [a, b] \to [c,d]$
		\[\tau(a) = c; \q \tau(b) = d:\]
		\[f(t) = g(\tau(t)) \q\q (\tau \text{ возрастает и гладкая})\]
	\end{definition}

	\begin{lemma}
			Эквив параметризаций - эквив.
	\end{lemma}

	\begin{proof}
			Рефл.\\
			$\tau = id$\\
			Симм.\\
			$f(t) = g(\tau(t))$\\
			$g(t) = f(\tau(t))$\\
			Транз.\\
			$f(t) = g(\sigma(t))$
			$f(t) = h(\tau(b(t))) $\\
			$\tau \circ \sigma$ - перепарам.
	\end{proof}

	\begin{Lemma}
		\[\overrightarrow{f}(t) \text{ - вектор-функция/ регуляр.}\]
		\[|\overrightarrow{f}(t)| = 1 \ra f'(t) \perp f(t)\]
	\end{Lemma}

	\begin{Proof}
		\[(f(t); f(t)) = 1\]
		\[0 = (f(t), f(t))' = 2(f'(t), f(t))\]
		\[f(t) \neq 0\]
		\[f'(t) \neq 0 \ra f'(t) \perp f(t)\]
	\end{Proof}
\end{lect}
\end{document}
