\documentclass[main]{subfiles}

\begin{document}
    %\begin{reminder}
      %рисунок1 с поверхностью
    %\end{reminder}
    \subsection{Соприкасающийся параболоид. Классификация точек поверхности}
    <<Введем нового героя>>
    \begin{Definition}
        %рисунок2 с пр-вом и точкой A
        \[A \text{ - точка на пов-ти}\]
        \begin{figure}[H]
          \includegraphics[width=3.5cm]{pics/8_2.png}
          \centering
        \end{figure}
        \[\Ra \text{ в окр. } A \text{ поверхность задается } z = f(x, y)\]
        \[x_0 = 0 \q y_0 = 0 \q z_0 = f(x_0,\ y_0) = 0\]
        Разложим $z=f(x, y)$ по ф. Тейлора
        \[z = f(x, y) = f(x_0, y_0) + f_x(x_0, y_0)x + f_y(x_0, y_0)y + \]
        \[\frac{1}{2}(f_{xx}(x_0, y_0)x^2 +
        2f_{xy}(x_0, y_0) )xy + f_{yy}(x_0, y_0)y^2) \]
        \[f_x(0, 0) = 0 \qq f_y(0, 0) = 0\]
        \[r(v, u) = \begin{cases}
          x = u\\
          y = v\\
          z = f(u, v)
        \end{cases} \qq r_u = \begin{pmatrix}
          1\\
          0\\
          f_u
        \end{pmatrix} \qq r_v = \begin{pmatrix}
          0\\
          1\\
          f_v
        \end{pmatrix}\]
        \[r_u \text{ и } r_v \text{ - лежат в кас. плоск, а это } OXY\]
        \[z = \underbracket{\frac{1}{2}(f_{xx}(0, 0)x^2 + 2f_{xy}(0, 0)xy +
          f_{yy}(0, 0)y^2) }_{\text{пов-ть 2 порядка}} +
        o(x^2 + y^2)\]
        \[z = Ax^2 + 2Bxy + Cy^2\]
        \[z = Ax^2 + Cy^2 \text{ можем поворотом привести к этому}\]
        Это может быть: \\
        \begin{matrix}
          &\text{- эллиптич. параболоид}&  A, C \text{  - одного знака}\\
          &\text{- гипербол. параболоид}&  A, C \text{ - разных знаков}\\
          &\text{- параболический цилиндр}&\q  A = 0, \q C \neq 0 \text{  или наоборот}\\
          &\text{- плоскость}& A = 0,  C = 0
        \end{matrix}
        %хахахааххахааххахаххахаххахаха
    \end{Definition}

    \begin{definition}
        A - \ul{элиптическая точкой}, если соприкас. параболоид --- элипт.\\
        А - \ul{гиперболическая точка}, если соприкас параболоид --- гиперб.\\
        А - \ul{парабол. точка}, если соприкас параб --- параб. цилинд или плоскость
    \end{definition}

    \begin{definition}
        Точка А наз. \ul{точкой округления} (омбилическая), если сопр. параб. - пар. вращения
    \end{definition}

    \begin{definition}
        Точка A - \ul{точка уплощения}, если соприкас. параб - плоскость
    \end{definition}

    \subsection{Совпадение нормальных кривизн поверхности и соприкасающегося параболоида}
    \begin{Theorem}
        \[\RNumb{1} \text{ и } \RNumb{2} \text{ формы в точке }A
        \text{ у поверхности и параболоида совпадают}\]
        %рисунок3
        \begin{figure}[H]
          \includegraphics[width=6cm]{pics/8_4.png}
          \centering
        \end{figure}

        В параметризации $\begin{cases}
          x = u\\
          y = v\\
          z = f(u, v)
        \end{cases}$

        \begin{proof} %тут какая-то фигня
            очевидно\\
            Давайте поймем, от чего зависят $E, F, G, L, M, N$?\\
            от $\ol{r}_u, \ \ol{r}_v, \ \ol{r}_{uu}, \ \ol{r}_{uv}, \ \ol{r}_{vv}   $
        \end{proof}

        \begin{consequence}
            Норм. кривизны у поверх-ти и соприкас. параб совпадают
        \end{consequence}

        \begin{definition}
            Главные кривизны $k_1$ и $k_2$
            \[\vec{a} \text{ - направление в кас. плоск}\]
            \[\ol{k}_{\vec{a}} \text{ - нормальная кривизна}\]
            \[k_{\vec{a}} \text{ - норм. кривизина в напр. } \vec{a} \]
            \[\ol{k}_{\vec{a}} = k_{\vec{a}} \overline{n}  \]
            \[k_1 = \min_{\vec{a}} k_{\vec{a}}  \qq k_2 = \max_{\vec{a}} k_{\vec{a}}   \]
        \end{definition}
    \end{Theorem}

    \begin{Definition}
        \[\vec{a}_1 \text{ и } \vec{a}_2 \text{, соотв } k_1 \text{ и } k_2 \text{ наз.
        главными направлениями}\]
    \end{Definition}

    \begin{Utv}
        \[\vec{a}_1 \perp \vec{a}_2 \text{(докажем позже)}\]
    \end{Utv}

    \begin{Definition}
        \[K = k_1 \cdot k_2 \text{ - гауссова кривизна}\]
        <<Главный герой всего нашего курса>>
    \end{Definition}

    \begin{Properties}
        \[\begin{matrix}
            K > 0 &\rla& A \text{ - эллипт типа}\\
            K < 0 &\rla& A \text{ - гиперб. типа }\\
            K = 0 &\rla& A \text{ - параб. типа}
        \end{matrix}\]
    \end{Properties}

    \begin{Utv}["Блистательная теорема Гаусса"]
        \[K \text{ - инвариант относительно изометрии пов-ти}\]
    \end{Utv}

    \begin{Definition}
        \[H = \frac{k_1 + k_2}{2} \text{ - средняя кривизна}\]
        \ul{Смысл:} В мыльных пленках (незамкн.) средняя кривизна = 0
        %рисунок4
        \begin{figure}[H]
            \includegraphics[width=5cm]{pics/8_5.png}
            \centering
        \end{figure}
    \end{Definition}

    \subsection{Теорема Эйлера}
    \begin{Theorem}[Эйлера]
        \[k_{\vec{a}} = k_1 \cos^2 \Theta + k_2 \sin^2\Theta \]
        \[\text{где } k_1, k_2 \text{ - гл. кривизны, } \Theta \text{ - угол между
        напр. } \vec{a} \text{ и } \vec{a}_1\]
    \end{Theorem}

    \begin{Proof}
        \[ z = Ax^2 + C y^2 \text{ - сопр. парабол.} \]
        \[\vec{a} = (\cos \Theta;\ \sin \Theta) \text{ - направление}\]
        %рисунок плоск с \vec{a}
        \[\begin{cases}
          x = t\cos\Theta\\
          y = t\sin\Theta\\
          z = Ax^2  + Cy^2 = t^2(A \cos^2\Theta + C \sin^2 \Theta)
        \end{cases}\]
        \[\vec{r}'(t) = \begin{cases}
            \cos \Theta\\
            \sin \Theta\\
            2t(A\cos^2\Theta + C\sin^2\Theta)
        \end{cases}\]
        \[r''(t) = \begin{cases}
            0\\
            0\\
            2(A\cos^2\Theta + C\sin^2\Theta)
        \end{cases}\]
        \[k = \frac{\abs{r'(t_0) \times r''(t_0)}}{\abs{r'(t_0)}^{3/2} }\]
        \[t_0 = 0\]
        \[r'(t_0) = \begin{pmatrix}
            \sin \Theta\\
            \cos \Theta\\
            0
        \end{pmatrix} \qq \abs{r'(t_0)} = 1\]
        \[r''(t_0) = \begin{pmatrix}
            0\\
            0\\
            2(A\cos^2 \Theta + C\sin^2 \Theta)
        \end{pmatrix}\]
        \[\abs{r''(t_0)} = 2\abs{A\cos^2\Theta + C\sin^2\Theta}\]
        \[r'' \perp r'\]
        \[\text{В данном случае } k = \abs{r''(t_0)} = 2\abs{A\cos^2 \Theta +
        C\sin^2 \Theta}\]
        \[k_{\vec{a}} = \pm k \q\text{(от сонапр. с } \vec{n}) \]
        \[k_{\vec{a}} = 2(A\cos^2 \Theta + C\sin^2 \Theta) \]
        Хотим теперь найти минимум и максимум этой штуки, но не хотим брать произв.
        \[k_{\vec{a}} = 2(A\cos^2\Theta + C(1 - \cos^2 \Theta)) = 2C + \cos^2 \Theta
        (2A - 2C)\]
        При $A = C$ \q A - точка округл.
        \[\letus A > C\]
        \[\max k_{\vec{a}} \text{ достиг при } \Theta = 0 \q(\text{или } \pi) \]
        \[k_1 = 2C + 2A - 2C = 2A\]
        \[\min k_{\vec{a}} \text{ при } \frac{\pi}{2} \q(\text{или } -\frac{\pi}{2}) \]
        \[k_2 = 2C\]
    \end{Proof}

    \begin{consequence}[1]
        Пов-ть задана ур-ем $z = f(x, y)$
        \[f(0, 0) = 0 \qq f_x(0, 0) = 0 \qq f_y(0, 0) = 0 \qq f_{xy}(0, 0) = 0 \]
        \[\Ra k_1 = f_{xx}(0, 0) \q k_2 = f_{yy} (0, 0)\]
        \[\text{ (или наоборот мы
        рассматривали только } A > C)\]
    \end{consequence}

    \begin{consequence}[2]
        Главные напр $\perp$
    \end{consequence}
\end{document}
