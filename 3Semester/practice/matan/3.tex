\documentclass[main]{subfiles}

\begin{document}
    \Date{09.09.2019}
    \subsection{Ещё больше определений}
    \begin{definition}
        \begin{enumerate}
            \item $A=\us{y \ra +\infty}{\lim\limits_{x \ra +\infty}} f(x,y)$, если

            $\forall \E>0 \ \e M>0: x>M \ y>M \Ra |f(x,y)-A| < \E$
            \item $A=\us{y \ra +\infty}{\lim\limits_{x \ra +\infty}} f(x,y)$, если

            $\forall \E>0 \ \e M>0: |x|>M \ |y|>M \Ra |f(x,y)-A| < \E$
            \item $A=\lim\limits_{P \ra \infty} f(P) \ P \in \R^2$, если

            $\forall \E>0 \ \e M>0: \rho(0, P)>M \Ra |f(x,y)-A| < \E$
        \end{enumerate}
    \end{definition}

    \begin{remark}
        Демидович по первым двум определениям
    \end{remark}

    \begin{definition}
        Для конечного предела: $A=\lim\limits_{x \ra a \  y \ra +\infty} f(x,y)$, если

        $\forall \E>0 \q \e M>0 \q \delta > 0: y>M \q |x-a| < \delta \Ra |f(x,y)-A| < \E$
    \end{definition}

    \subsection{Ещё больше примеров}

    \begin{example}
        $\us{y \ra +\infty}{\lim\limits_{x \ra +\infty}} (\dfrac{x y}{x^2+y^2})^{x^2}$
    \end{example}

    \begin{sol}
        Заметим, что $\dfrac{x y}{x^2+y^2} \leqslant \dfrac{1}{2} \Ra 2xy \leqslant x^2 + y^2 \Ra 0 \leqslant (x-y)^2\text{ для x }\neq y$
        \\
        Значит дробь стремится к 0
    \end{sol}

    \begin{example}
        $\us{y \ra 0}{\lim\limits_{x \ra 0}} (\dfrac{x y}{x^2+y^2})^{x^2}$
    \end{example}

    \begin{sol}
        При $x=y$ предел $\dfrac{1}{2}$\\
        При $x=y^2$ предел 0
    \end{sol}

    \begin{example}
        $f=\sin(\dfrac{\pi y^2}{x^2 + 3y^2})$\\
        Найти $\us{y \ra +\infty}{\lim\limits_{x \ra +\infty}} f$, $\lim\limits_{x \ra \infty} \lim\limits_{y \ra \infty} f$, $\lim\limits_{y \ra \infty} \lim\limits_{x \ra \infty} f$
    \end{example}

    \begin{sol}
        Первый не имеет предела ($x=y$, $x=\sqrt{y})$. Второй $\dfrac{\sqrt{3}}{2}$. Третий 0
    \end{sol}

    \begin{example}
        $\us{y \ra +\infty}{\lim\limits_{x \ra +\infty}} \dfrac{sin(y-x^2)}{y-x^2}$
    \end{example}

    \begin{sol}
        $z=y-x^2$, $z \ra 0 \Ra x,y \ra 0$

        $|z| \leqslant |x| + |y| \leqslant 2 \sqrt{x^2+y^2}$
    \end{sol}

    \begin{example}
        $f=\dfrac{1-\sqrt[3]{sin^4 x + cos^4 y}}{\sqrt{x^2+y^2}}$, найти $\us{y \ra 0}{\lim\limits_{x \ra 0}} f$
    \end{example}

    \begin{sol}
        $1-\sqrt[3]{t} \us{t \ra 1}{~} \dfrac{1-t}{3}$ (т.к. $1-\sqrt[3]{t}=\frac{1-t}{1+\sqrt[5]{t}+\sqrt[3]{t^2}}$)

        Значит $\us{y \ra 0}{\lim\limits_{x \ra 0}} f = \us{y \ra 0}{\lim\limits_{x \ra 0}} \frac{1}{3} \dfrac{1-(sin^4 x + cos^4 y)}{\sqrt{x^2+y^2}} = \us{y \ra 0}{\lim\limits_{x \ra 0}} \dfrac{2sin^2 y - sin^4 y - sin^4 x}{3 \sqrt{x^2+y^2}}$

        Заменим по Тейлору: $=\us{y \ra 0}{\lim\limits_{x \ra 0}} \dfrac{2y^2 + \ol{o}(y^3)-x^4 + \ol{o}(x^6)}{3 \sqrt{x^2+y^2}}$

        Попробуем оценить по модулю $|\dfrac{2y^2-x^4}{\sqrt{x^2+y^2}}|$, заметим что $y^2 \leqslant x^2 + y^2$,

        $x^4 \leqslant 2(x^2+y^2) \leqslant x^2+y^2$ (для $x^2+y^2 < 1)$,

        чтобы избавиться от $\ol{o}$ оценим так:

        $\ol{o} + y^2 \leqslant 2(x^2 + y^2)$, $\ol{o} + x^4 \leqslant 2(x^2+y^2) \leqslant x^2+y^2$

        Тогда $|\dfrac{2y^2-x^4}{\sqrt{x^2+y^2}}| \leqslant 2 \dfrac{3(x^2+y^2)}{\sqrt{x^2+y^2}} \leqslant 6 \sqrt{x^2+y^2} \ra 0$
    \end{sol}
\end{document}
