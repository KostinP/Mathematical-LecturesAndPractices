\documentclass[main]{subfiles}

\begin{document}
  \Date{28.10.2019}
  \subsection{Экстремумы}

  \begin{example}
      Найти экстремумы
      \[f(x,\ y,\ z) = x^2 + y^2 - z^2 - 4x + 6y - 2z\]
  \end{example}

  \begin{sol}
      Найдем первые производные и приравняем к 0:
      \[\begin{cases}
        f'_x = 2x - 4 = 0\\
        f'_y = 2y - 6 = 0\\
        f'_z = - 2z - 2 = 0
      \end{cases} \q \Ra x = 2 \q y = -3 \q z = -1\]
      \[\dfrac{\d^2 f}{\d x^2} = 2 \q \dfrac{\d^2 f}{\d y^2} = 2 \q \dfrac{\d^2 f}{\d z^2} = - 2\]
      \[\dfrac{\d^2 }{\d x \d y} = 0 \q \dfrac{\d^2 f}{\d y \d z} = 0 \q \dfrac{\d^2 f}{\d x \d z} = 0\]
      \[\begin{pmatrix}
        f''_{xx} & f''_{xy} & f''_{xz}\\
        f''_{xy} & f''_{yy} & f''_{yz}\\
        f''_{xz} & f''_{yz} & f''_{zz}
    \end{pmatrix} = \os{\ A_1 \ A_2 \ A_3}{\begin{pmatrix}
        2 & 0 & 0\\
        0 & 2 & 0\\
        0 & 0 & -2
      \end{pmatrix}}\]
      \[\Ra A_1 = \det \begin{pmatrix}
        2
      \end{pmatrix} = 2 \q A_2 = \det \begin{pmatrix}
        2 & 0\\
        0 & 2
      \end{pmatrix} = 4 \q A_3 = \det \begin{pmatrix}
        2 & 0 & 0\\
        0 & 2 & 0\\
        0 & 0 & -2
      \end{pmatrix} = -8\]
      \[\d^2 f = 2 (d x^2 + dy^2 - dz^2)\]
      \[d^2 f (e_1) > 0 \qq (\os{dx}{1},\ \os{dy}{1},\ \os{dz}{0})\]
      \[d^2 f (e_2) < 0 \qq (0,\ 0,\ 1)\]
      \[d^2 f = (dx + dy)^2\]
      \[f(x,\ -x)\]
  \end{sol}

  \begin{Example}
      \[f(x,\ y,\ z) = (x + 7z) e^{-(x^2 + y^2 + z^2)}\]
      \[\begin{cases}
        f'_x = e^{-(\ )} + (x + 7 z) (-2 x) e^{-(\ )} = 0\\
        f'_y = (x + 7 z) (-2y) e^{-(\ )} = 0\\
        f'_z = 7 e^{-(\ )} + (x + 7z)(-2z)e^{-(\ )} = 0
      \end{cases} \Ra x = \pm \frac{1}{10}\q y = 0\q z = \pm \frac{1}{10}\]
      Можно не дифференцировать всё, т.к. нас интересуют только слагаемые, которые мы обнуляем
      \[f''_{xx} \os{\text{в инт. точке}}{\sim} (-4x -14 z) e^{-(\ )} \q f''_{yy} \sim -2(x + 7z)e^{(\ )} \q f''_{zz} \sim (-28z-2x)e^{(\ )}\]
      \[f''_{xy} \sim 0 \q f'' _{xz} = (-14x)e^{(\ )} \q f''_{yz} = 0\]
      Матрица для точки $x = \frac{1}{10} \q y = 0 \q z = \frac{1}{10}$:
      \[\begin{pmatrix}
        -102 & 0 & -14\\
        0 & -100 & 0\\
        -14 & 0 & -198
      \end{pmatrix}\]
      \[A_1 < 0 \q A_2 > 0 \q A_3 < 0 \Ra \text{лок. max}\]
      Матрица для точки $x = -\frac{1}{10} \q y = 0 \q z = -\frac{1}{10}$:
      \[\begin{pmatrix}
        102 & 0 & 14\\
        0 & 100 & 0\\
        14 & 0 & 198
      \end{pmatrix}\]
      \[A_1,\ A_2,\ A_3 > 0 \Ra \text{лок. min}\]
  \end{Example}

  \begin{Remark}
      \[f'(x_0) = 0 \Ra (fg)'(x_0) = f g'(x_0)\]
  \end{Remark}

  \subsection{Условный экстремум}

  \begin{Theorem}
      \[f: \R^n \ra \R \qq \varphi_1,...,\varphi_k: \R^n \ra \R \q k < n\]
      Локальный экстр. f при условии $\begin{cases}
        \varphi_1(x) = 0\\
        ...\\
        \varphi_k(x) = 0
      \end{cases} \q x = (x_1,...,x_n)$
      \[\nabla \varphi_1,\ \nabla \varphi_2,\ ...,\ \nabla \varphi_k \text{ - лин. незав.}\]
      Рассмотрим вспомогательную функцию:
      \[L(x) = f(x) + \sum_{j=1}^k \lambda_k \varphi_j(x) \text{ - ф-ия Лагранжа}\]
      $\lambda_1,...,\lambda_k$ - мн-ли Лагранжа\\
      Алгоритм:
      \begin{enumerate}
          \item Ищем стац. точки L:
          \[\begin{cases}
            \dfrac{\d}{\d x_j} L(x) = 0, \q j=1...n\\
            \varphi_i(x) = 0,\q i=1...k
          \end{cases} \text{ - система из $k+n$ уравнений}\]
          $\Ra$ находим стац. точки (это точки, подозр. на экстр.)\\
          \item Нужно проверить, что в стац. точках условия $\varphi_i=0$ должны быть независимы в том смысле, что вектора $\nabla \varphi_1,..., \nabla \varphi_k$ - лин. независимы или:
          \[\rk \begin{pmatrix}
              \dfrac{\d \varphi_1}{\d x_1} & ... & \dfrac{\d \varphi_1}{\d x_n}\\
              ... & ... & ...\\
              \dfrac{\d \varphi_k}{\d x_1} & ... & \dfrac{\d \varphi_k}{\d x_n}
          \end{pmatrix}=k\]
          $k=1$ означает $\nabla \varphi_1 \neq 0$
          \item Исследуем $d^2 L$ в стац. точках\\
          $d^2 L > 0$ при усл., что $d \os{\text{усл. на $dx_1,...,dx_n$}}{\varphi_i = 0\ j = 1...k} \Ra$ усл. лок. min\\
          $d^2 L < 0$ при усл., что $d \varphi_i = 0\ j = 1...k \Ra$ усл. лок. max\\ \\
          "Пример"{} $f = \frac{x^2 - y^2}{2}$
          \[d^2 L = dx^2 - dy^2\]
          \[\varphi(x) = x + \frac{1}{2}y = 0\]
          \begin{figure}[H]
              \centering
              \includegraphics[width=5cm]{y=2x}
          \end{figure}
          \[d \varphi = dx + \frac{dy}{2} = 0\]
          \[d^2 L = \Br{\frac{dy}{2}}^2 - (dy)^2 = - \frac{3}{4} dy^2 < 0\]
      \end{enumerate}
  \end{Theorem}

  \begin{Example}
      \[\varphi_1 = x^2 + y^2 + z^2 - 1 = 0\]
      \[f(x) = x^3\]
  \end{Example}

  \begin{sol}
      Шаг 1:
      \[L(x) = x^3 + \lambda(x^2 + y^2 + z^2 - 1)\]
      \[\begin{cases}
          L'_x = 3x^2 + 2\lambda x = 0 = x(2x + 2\lambda)\\
          L'_y = 2 \lambda y = 0\\
          L'_z = 2 \lambda z = 0\\
          x^2 + y^2 + z^2 - 1 = 0
      \end{cases}\]
      \[\lambda = 0 \Ra x = 0 \q y^2 + z^2 = 1\]
      \[\lambda \neq 0 \Ra y=z=0 \qq x = 1 \q \lambda = - \frac{3}{2} \qq x = -1 \q \lambda = \frac{3}{2}\]
      Шаг 2: $(x,\ y,\ z,\ \lambda)$ - стац. точка
      \[d^2 L = (2\lambda + 6x) d^2 x + 2\lambda d y^2 + 2\lambda z^2\]
      Можем изучать при $0 = d(x^2 + y^2 + z^2 - 1) = \us{\text{фикс}}{2x} dx + \us{\text{фикс}}{2y} dy + \us{\text{фикс}}{2z} dz \us{\text{усл. на $(dx, dy, dz)$}}{=} 0$\\
      Случай 2:
      \[\lambda \neq 0 \qq dx = \dfrac{y dy + z dz}{x} = 0 \qq (y = z = 0)\]
      \[d^2 L = 2 \lambda (dy^2 + dz^2)\]
      $\lambda > 0 $ - пол. опр ($-1,\ 0,\ 0$), $\lambda < 0$ - отр. опр. ($1,\ 0,\ 0$)\\
      ($-1,\ 0,\ 0$) - лок. макс.\\
      ($1,\ 0,\ 0$) - лок. мин.\\
      Случай 1:
      \[x = 0 \q \lambda = 0 \q d^2 L = 0 \text{ - метод не работает}\]
      Но $f(x) = 0$ при $x=0$ и $y^2 + z^2 = 1 \Ra$ нет лок. мин. и лок. макс.
  \end{sol}

  \begin{Example}
      \[u = xyz \qq \begin{cases}
          x^2 + y^2 + z^2 = 1\\
          x + y + z = 0
      \end{cases}\]
      \[L(x,y,z) = xyz + \lambda_1(x^2 + y^2 + z^2) + \lambda_2(x + y + z)\]
      \[\begin{cases}
          yz + 2\lambda_1 x + \lambda_2 = 0\\
          xz + 2\lambda_1 y + \lambda_2 = 0\\
          xy + 2\lambda_1 z + \lambda_2 = 0\\
          x^2 + y^2 + z^2 = 1\\
          x + y + z = 0
      \end{cases}\]
      \[\Ra \lambda_1 = -\frac{3}{2} xyz\]
      \[\os{1+2+3}{\Ra} \lambda_2 = -\frac{1}{3}(yz + xz + xy)\]
      \[\os{4}{\Ra} (\us{=0}{x + y + z})^2 - 2(\us{=\frac{1}{2}}{xy+xz+yz})=1 \Ra \lambda_2 = \frac{1}{6}\]
      \[\begin{cases}
          (z - 2\lambda_1)(y-x) = 0\\
          (x - 2\lambda_1)(z-y) = 0\\
          (y - 2\lambda_1)(x-z) = 0
      \end{cases}\]
      \[\Ra \Br{\pm \frac{1}{\sqrt{6}},\ \pm \frac{1}{\sqrt{6}},\ \mp \frac{2}{\sqrt{6}};\ \pm \frac{1}{2\sqrt{6}},\ \pm \frac{1}{6}} \text{ и ещё ...}\]
      Следующий шаг:
      \[\rk \begin{pmatrix}
          2x & 2y & 2z\\
          1 & 1 & 1
      \end{pmatrix} = 2, \text{ кроме} x = y = z\]
      Следующий шаг:
      \[d^2 L = 2\lambda_1(dx^2 + dy^2 + dz^2) + 2x dy dz + 2y dx dz + 2z dx dy\]
      \[\text{Но нам известно:}\]
      \[\begin{cases}
          2x dx + 2y dy + 2z dz = 0\\
          dx + dy + dz = 0
      \end{cases}\]
      \[\text{Посмотрим на точку }x = y = \pm \frac{1}{\sqrt{6}}\]
      \[\Ra dz = 0 \qq dx = - dy \Ra d^2 L = (4\lambda_1 - 2z)dx^2 = \pm \sqrt{6} dx^2\]
      Ответ: $\Br{\frac{1}{\sqrt{6}},\ \frac{1}{\sqrt{6}},\ -\frac{2}{\sqrt{6}}}$ - усл. лок. мин., $\Br{-\frac{1}{\sqrt{6}},\ -\frac{1}{\sqrt{6}},\ \frac{2}{\sqrt{6}}}$ - усл. лок. макс.\\
      Остальные аналогично
  \end{Example}
\end{document}
