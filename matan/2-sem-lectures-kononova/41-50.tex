\documentclass[matan]{subfiles}

\begin{document}
  \newpage
  \section{Теорема о комплексной дифференцируемости степенного ряда. Следствие: единственность разложения в степенной ряд.}

  \begin{Theorem}[о комплексной дифференцируемости степенного ряда]
      \[C(z) = \sum_{k = 0}^\infty c_k z^k  \qq \text{ в круге } \abs{z} < R\]
      \[\text{Тогда } \exists \lim_{\substack{h \to 0 \\ h \in \CC}}
      \frac{C(z + h) - C(z)}{h} \text{ и равен } C_1(z) =
    \sum_{k = 1}^\infty c_k \cdot k z^{k - 1} \qq \forall \abs{z} < R\]
    (то есть $h \ra 0$ как угодно, т.к. $h \in \CC$)
  \end{Theorem}

  \begin{Proof}
    \[\letus \ \abs{z} + \abs{h} < R\]
    \[\frac{C(z + h) - C(z)}{h} = \sum_{k = 0}^\infty c_k \frac{(z + h)^k - z^k}{h} =
    \sum_{k = 1}^\infty c_k \frac{h}{h} \sum_{s = 0}^{k + 1} (z + h)^s z^{k - 1 - s}    \]
    *здесь когда-нибудь будет продолжение*
  \end{Proof}

  \begin{Consequence}[1]
      \[f(x) = \sum_{k = 0} ^\infty a_k x^k \qq \abs{x} < R_f\]
      \[f \in C^\infty (-R_f; \ R_f)\]
      \[f^{(n)} = \sum_{k = n}^\infty a_k \cdot k (k - 1) \cdot... \cdot(k - n + 1) x^{k - n}   \]
  \end{Consequence}

  \begin{proof}
    *здесь когда-нибудь будет док-во*
  \end{proof}

  \begin{Consequence}[2]
      \[f(x) = \sum_{k = 0}^\infty a_k \underbracket{(x - x_0)^k}_y \]
      \[\abs{x - x_0} < R_f\]
      \[\text{Тогда } a_n = \frac{1}{n!} f^{(n)}(x_0)\]
  \end{Consequence}

  \begin{proof}
    *здесь когда-нибудь будет док-во*
  \end{proof}

  \begin{Consequence}[3, единственность разложения в степ. ряд]
      \[A(x) = \sum_{k = 0}^\infty a_k(x - x_0)^k, \qq B(x) = \sum_{k = 0}^\infty b_k (x - x_0)^k  \]
      \[A(x) = B(x) : \q \abs{x - x_0} < R\]
      \[\text{Тогда } a_k = b_k \q \forall k\]
  \end{Consequence}

  \begin{Proof}
      \[a_n = \frac{1}{n!}A^{(n)}(x_0) = \frac{1}{n!} B^{(n)} (x_0) = b_n  \]
  \end{Proof}

  \begin{Utv}
      \[f(x) = \sum_{k = 0}^\infty a_k x^k \qq \abs{x} < R \]
      \[x_1, x_2 \in (-R, \ R) \q \Ra \q \int_{x_1}^{x_2} f(x)dx = \sum_{k = 0} ^\infty \frac{a_k}{k + 1} (x_2^{k + 1} - x_1^{k + 1})  \]
  \end{Utv}

  \newpage
  \section{Ряд Тейлора. Примеры ($e^x,\sin x,\ln(1 + x), e^{-\frac{1}{x^2}}$).}

  *здесь когда-нибудь будет полный билет*

  \begin{Definition}
  	\[f \in C^{\infty} (U_{x_0}) \qq U_{x_0} \text{ - окр } x_0 \]
  	\[\text{Ряд } \sum^\infty_{n = 0} \frac{f^{(n)}(x_0)}{n!}(x - x_0)^n \text{ назыв. Рядом Тейлора ф-и в т } x_0\]
  \end{Definition}

  \begin{examples}
  	\begin{enumerate}
  		\item $\displaystyle e^x = \sum_{k = 0}^\infty \frac{x_k}{k!}$
  		\item $\displaystyle \cos x = \sum_{k = 0}^\infty (-1)^k \frac{x^{2k} }{(2k)!}$
  		\item $\displaystyle \sin x = \sum_{k = 0}^\infty (-1)^k \frac{x^{2k + 1} }{(2k + 1)!}$
  		\item $\displaystyle \ln (1 + x) = \sum_{k = 1}^\infty (-1)^k \frac{x^k}{k}$
  	\end{enumerate}
  \end{examples}

  \begin{TTheorem}[формула Стирлинга]
    \[n! \sim \sqrt{2\pi n} \abs{\frac{n}{e}}^n \q (n\ra \infty)\]
    \[\sqrt{2\pi n} \abs{\frac{n}{e}}^n < n! < \sqrt{2\pi n} \abs{\frac{n}{e}}^n e^{\frac{1}{12n}}\]
  \end{TTheorem}

  \begin{proof}
    *здесь никогда не будет док-ва*
  \end{proof}

  \newpage
  \section{Биномиальный ряд $(1 + x)^\upalpha$}

  \begin{Definition}
  	\[(1 + x)^\alpha \q\q \alpha \in \R\]
  	Запишем (формально) ряд Тейлора для $(1 + x)^\alpha$ в т. $x_0 = 0$
  	\[\frac{f^{(k)} (0)}{k!} = \frac{\alpha(\alpha - 1) \cdot ... \cdot (\alpha - k + 1)}{k!} =
  	C_{\alpha}^k \]
  	Найдем интервал сходимость $\displaystyle \sum_{k = 0}^\infty c_{\alpha}^k z^k \q z \in \CC$ (по Даламберу)
  	\[\lim_{k \to \infty} \abs{\frac{c_{\alpha}^{k + 1} z^{k + 1}}{c_{\alpha}^k z^k }} =
  	\lim_{k \to \infty}  \]
    *здесь когда-нибудь будет продолжение*
    \[(1+x)^{\alpha} = \sum_{k=0}^{\infty} C_{\alpha}^k x^k,\q x \in (-1,\ 1),\q C_{\alpha}^k = \frac{\alpha(\alpha - 1)...(\alpha - k + 1)}{k!}\]
  \end{Definition}

  \newpage
  \section{Признак Абеля-Дирихле для равномерной сходимости функциональных рядов (доказательство одного).}

  \begin{Theorem}
      \[\sum_{k = 0}^\infty a_k(t)b_k(t) \qq \begin{matrix}
          a_k : E \to \CC\\
          b_k : E \to \R\\
          E \subset \CC
      \end{matrix} \]
      \[b_k(t) - \text{ монот по } k \q \forall t\]
      \[\text{т.е}\q b_{k + 1}(t) \leq b_k(t) \q \forall t (\text{ или наоборот})  \]
      %\[\text{Абель}\]
      Абель
      \begin{enumerate}
          \item $ \displaystyle \sum_{k = 0}^\infty a_k $ - сход р/м на $E$
          \item $\abs{b_k(t)} \leq M \qq \forall  k, \q \forall t \in E$
      \end{enumerate}
      %\[\text{Дирихле}\]
      Дирихле
      \begin{enumerate}
          \item $\displaystyle \abs{\sum_{k = 0}^N a_k(t) } \leq M \q \forall N, \forall t \in E$
          \item $b_k(t) \rightrightarrows 0$
      \end{enumerate}
      Тогда $\displaystyle \sum_0^{\infty} a_k(t)b_k(t)$ - сход равномерно на $E$
  \end{Theorem}

  \begin{proof}
      *здесь когда-нибудь будет док-во*
  \end{proof}

  \begin{lemma}
    Если $B_k$ - монотонна, то
    \[\sum_{k=m}^n a_k B_k \leq 4 \us{m \leq k \leq n + 1}{\max}(|A_k|) \max (|B_m|,\ |B_{n+1}|)\]
  \end{lemma}

  \begin{proof}
      *здесь когда-нибудь будет док-во*
  \end{proof}

  \begin{proof}[продолжение док-ва теоремы]
      *здесь когда-нибудь будет док-во*
  \end{proof}

  \newpage
  \section{Теорема Абеля. Сумма ряда $\sum\limits_{n=1}^\infty \frac{(-1)^{n-1}}{n}$.}

  \begin{Theorem}
      \[hint: \q z \in [0, w] \rla z = t \cdot w \q 0 \leq t \leq 1\]
      \[\sum_{k = 0}^\infty c_k z^k \qq c_k \in \CC \]
      \[\text{Пусть } \sum c_k z^k \text{ сход при } z = w \in \CC\]
      \[\text{Тогда } \sum_{k = 0}^\infty c_k z^k \text{  - сход р-но на } [0, w] \]
      \[\Ra f(z) = \sum_{k = 0}^\infty c_k z^k \in C[0, w] \]
  \end{Theorem}

  \begin{Proof}
      \[f(t, w) = \sum_{k = 0}^\infty c_k t^k w^k \qq t \in [0, 1] \]
      \[\sum c_k w^k \text{ - сход (равн по t, т.к. не зависит от t)}\]
      \[t^k \text{ - убывает}\]
      \[\abs{t^k} \leq 1 \qq \forall t \in [0, 1] \qq \forall k \in \N\]
      \[\Ra \text{ по пр. Абеля-Дирихле ряд сход. равномерно}\]
  \end{Proof}

  \begin{Example}
      \[\ln(1 + x) = \sum_{k = 1}^\infty \frac{(-1)^{k + 1}x^k }{k} \qq \forall x: \
      -1 < x < 1\]
      \[\text{при } x = 1 \qq \sum_{k = 1}^\infty \frac{(-1)^{k + 1} }{k} \text{
      - гармонич. знакочеред, он сход, т.о. }\]
      \[ \sum_{1}^\infty \frac{(-1)^k}{k}x^k \text{  - сх. при $x = 1 \Ra$ по т. Абеля }\]
      \[f(x) = \sum_1^\infty \frac{(-1)^{k + 1}x^k }{k} \in C[0, 1]\]
      В частности $\displaystyle \lim_{x \to 1-} f(x) = f(1) $
      \[\text{если } x \in (0, 1) \text{, то } f(x) = \sum_1^\infty
      \frac{(-1)^{k - 1}x^k }{k} = \ln(1 + x)\]
      \[\lim_{x \to 1-} \ln(1 + x) = \ln 2 \]
      \[1 - \frac{1}{2} + \frac{1}{3} - ... = \ln 2\]
  \end{Example}

  \newpage
  \section{Интеграл комплекснозначной функции. Скалярное произведение и норма в пространстве $C(\CC / \R)$, в пространстве $R([a; b])$. Ортогональность. Пример: $e_k(x) = e^{2 \pi i k x}$.}

  \begin{Definition}
      \[f : [a, b] \to \CC\]
      \[f(x) = u(x) + iv(x)\]
      \[u(x) = \real f(X)\]
      \[v(x) = \im f(x)\]
      \[f \text{ - инт. по Риману } \q f \in R_\CC [a, b], \text{ если } u, v \in R[a, b]\]
      \[\int_a^bf(t)dt := \int_a^b u(t)dt + i\int_a^b v(t)dt\]
  \end{Definition}

  \begin{properties}
      \begin{enumerate}
          \item $\displaystyle \int_a^b (f + g) = \int_a ^b f + \int_a^b g$
          \item $\displaystyle \int_a^b f = \int_a^c f + \int_c^b f$
          \item $\displaystyle \int_a^b kf = k\int_a^b f \q (k \in \CC)$
          \item $\displaystyle \int_a^b \overline{f} = \int_a^b u - iv = \overline{\int_a^b f}$
              (комплексное сопряжение)
          \item $\displaystyle F' = f$
              \[\int_a^b f = F(b) - F(a)\]
          \item $\displaystyle \abs{\int_a^b f} \leq \int_a^b \abs{f}$
      \end{enumerate}
  \end{properties}

  \begin{proof}[6]
    *здесь когда-нибудь будет док-во*
  \end{proof}

  \begin{Definition}[периодич. функции]
      \[f(x + t) = f(x) \qq \forall x\]
      \[\text{Будем считать, что } T = 1 \text{ (иначе используем сжатие/растяжение)}\]
      Периодич. функции с пер. 1 образуют линейное пр-во
      \[f, g \text{ - период. } T = 1\]
      \[\Ra f + k \cdot g \text{ - тоже период } T = 1\]
      \[\text{Если } f \text{ - периодична с периодом } T = 1 \text{, то}\]
      \[\int_0^1 f = \int_c^{c + 1} f \qq \forall c \in \R \]
      \[0 < c < 1\]
      \[\int_0^1 f = \int_0^c f + \int_c^1 f = \int_0^c f(t + 1)dt + \int_c^1 f =
      \int_1^{c + 1} f(s)ds + \int_c^1 f \]
      *здесь когда-нибудь будет корректное опредление (или хотя бы полное)*
  \end{Definition}

  \begin{definition}
      Рассмотрим пр-во функций с пер $T = 1$ и $\in R_\CC [0, 1] \rla R_\CC[0, 1]$\\
      Введем на этом пр-ве структуру евклидова пр-ва
      \[<f, g> = \int_0^1 f \cdot \overline{g} \text{ - скал. произведение}\]
  \end{definition}

  \begin{Definition}[асимптотическое определение ск. произведения]
      \[<...> : X \times X \to \CC\]
      \begin{enumerate}
          \item $\forall x, y \in X$ \q $<x, y> = \overline{<y, x>}$
          \item $\forall x_1, x_2 \in X$ \q $\forall y \in X$
              \[<x_1  + x_2, y> \ = \ <x_1, y> + <x_2, y>\]
          \item $\forall k \in \CC \q \forall x, y \in X$
              \[<kx, y> \ = \ k <x, y>\]
              \[<x, ky> \ = \ \overline{k} <x, y>\]
          \item $<x, x> \ \geq 0$ причем $<x, x> \ = 0 \ \rla \ x = 0$\\
              Но для $f \in R_\CC [0, 1] $ необязательно из $ \q <f, f> \ = \  0$ следует, что $f = 0$
      \end{enumerate}
  \end{Definition}

  \begin{Definition} [Норма в лин. пр-ве X со скал. произв.]
      \[\Abs{x} = \sqrt{<x, x>} \text{ - норма}\]
      \[\Abs{f} = \sqrt{\int_0^1 \abs{f}^2}\]
      \begin{enumerate}
          \item $\Abs{x} \geq 0$
              \[\Abs{x} = 0 \rla x = 0\]
          \item $\forall k \q \Abs{kx} = \Abs{k} \cdot \Abs{x}$
      \end{enumerate}
  \end{Definition}

  \begin{example}
    *здесь когда-нибудь будет пример*
  \end{example}

  \begin{definition}
    Будем считать $f \sim g$ ($f, g \in R_{\CC}[0,1]$), если $<f-g,f-g> = 0$
  \end{definition}

  \begin{Definition}
      \[f \perp g \q (f \text{ ортогонально } g) \q \rla \q <f, g> = 0 \]
  \end{Definition}


  \begin{Example}
      \[e_n = e^{2\pi i n x} \qq x \in [0, 1]\]
      \begin{enumerate}
        \item $\Abs{e_n} = 1 \qq \forall n \in \Z$
        \[\Abs{e_n}^2 = \int_0^1 e_n \overline{e_n} = \int_0^1 e^{2\pi i n x } \cdot e^{-2\pi i nx} = 1  \]
        \[\overline{e^{i\varphi}} = \cos \varphi + \overline{i \sin \varphi} = \cos \varphi - i\sin \varphi = e^{-i \varphi} \]
        \item $<e_n, e_m> \ = \int_0^1 e^{2\pi i nx} \cdot e^{-2\pi i m x} = \int_0^1 e^{2\pi i x (n - m)} = \begin{cases}
            0, & n \neq m\\
            1, & n = m
        \end{cases}$
        \[= \delta_{nm} \text{ - с. Крон.}\]
        т.о. \q $e_n \perp e_m \qq \forall n \neq m$
      \end{enumerate}
  \end{Example}

  \newpage
  \section{Свойства скалярного произведения и нормы (теорема Пифагора, неравенство Коши-Буняковского-Шварца, неравенство треугольника).}

  \begin{properties}
      \begin{enumerate}
          \item $\Abs{f + g}^2 = \Abs{f}^2 + \underbracket{<f, g> + <g, f>}_{2\real <f, g>}  + \Abs{g}^2 = $
              \[ = \Abs{f}^2 + 2\real <f, g> + \Abs{g}^2\]
          \item По т. Пифагора, если $f \perp g \ \Ra$
              \[\Abs{f + g}^2 = \Abs{f}^2 + \Abs{g}^2\]
          \item $\Abs{f + g}^2 + \Abs{f - g}^2 = 2(\Abs{f}^2 + \Abs{g}^2)$
          \item нер-во КБШ
              \[\abs{<f, g>} \leq \Abs{f} \cdot \Abs{g}\]
          \item Н-во треугольника
              \[\Abs{f + g} \leq \Abs{f} + \Abs{g}\]
      \end{enumerate}
  \end{properties}

  \begin{Proof} [КБШ]
      \[(*)\abs{<f, g>} = \abs{\int f \overline{g}} \leq \int \abs{f} \abs{g}\]
      \[0 \leq \int(\abs{f} + \lambda \abs{g})^2 =
      \underbracket{ \Abs{f}^2 + 2\lambda\int \abs{f}\abs{g} + \lambda^2 \Abs{g}^2}_{\text{кв. трехчлен отн } \lambda}
      \qq \forall  \lambda \in \R\]
      \[D \leq 0\]
      \[\frac{D}{4} = (\int \abs{f}\abs{g})^2 - \Abs{f}^2 \Abs{g}^2 \leq 0 \Ra \int \abs{f}\abs{g} \leq \Abs{f} \Abs{g} \q (**)\]
      \[(*) \text{ и } (**) \Ra \abs{ <f, g>} \leq \Abs{f} \cdot \Abs{g}\]

  \end{Proof}

  \begin{Proof}[Нер-во треуг-ка]
      \[\Abs{f + g}^2 = \Abs{f}^2 + 2\real <f, g> + \Abs{g}^2 \leq
      \Abs{f}^2 + 2\abs{<f, g>} + \Abs{g}^2 \os{\text{КБШ}}{\leq }\]
      \[\leq \Abs{f}^2 + 2\Abs{f} \cdot \Abs{g} + \Abs{g} = (\Abs{f} + \Abs{g})^2 \Ra \Abs{f + g} \leq \Abs{f} + \Abs{g}\]
  \end{Proof}

  *здесь когда-нибудь будет исправлено на определение*???
  \begin{Theorem}[Аксиомы нормы]
      \[X - \text{лин. пр-во} \qq \Abs{...} : X \to [0, +\infty)\]
      \begin{enumerate}
          \item $\Abs{x} = 0 \q\rla\q x = 0$
          \item $\Abs{kx} = \Abs{k} \cdot \Abs{x} \qq \forall k \in \CC, \q \forall x \in X$
          \item $\forall x, y \in X$ \qq $\Abs{x + y} \leq \Abs{x} + \Abs{y}$
      \end{enumerate}
  \end{Theorem}

  \newpage
  \section{Коэффициенты Фурье функции по ортогональной системе $e_k$. Ряд Фурье. Пример: тригонометрический полином.}

  \begin{Definition}
      Тригонометрическим многочленом степени $N$ назовем:
      \[T_n = \sum_{k = -N}^N c_k e_k(x) = \sum_{k = -N}^N c_k e^{2\pi i kx} = \sum_{k = -N}^N c_k (\cos
      (2\pi kx) + i\sin(2\pi kx))   \]
      Как найти $c_k$, если известен $T_n(x)$ ?
      \[T_n = \sum_{k = -N}^N c_k e_k \q \bigg| \cdot <..., e_m>\]
      \[<T_n, e_m>  \ = c_m \cdot \us{=1}{<e_m, e_m>} \q (\text{т.к.} <e_k, e_m> = \delta_{km} )\]
      \[c_m = <T_N, e_m> = \int_0^1 T_N \overline{e}_m\]

      \[\letus f, g \text{ - тригоном. полиномы, коэфф. в разложении по } e_k \text{ будем обозначать } \hat{f}(k) \in \CC\]
      \[\text{т.е. } f = \sum_{k = -N}^N \hat{f}(k)e_k, \q \hat{f}(k) = <f, e_k> \]
      \[g = \sum_{k = -N}^N \hat{g}(k)e_k, \qq \hat{g}(k) = <g, e_k> \]
      \[<f, g> = <(\sum_{k = -N}^N \hat{f}(k)e_k ), (\sum_{j=-N}^N \hat{g}(j)e_j )> = \]
      \[= \sum_{k, j =-N}^N \hat{f}(k)\ol{\hat{g}}(j) <\us{=\delta_{kj} }{e_k, e_j}> = \sum_{k = -N}^N \hat{f}(k) \overline{\hat{g}}(k) \]
      \[\Abs{f}^2 = \sum_{k = -N}^N \abs{\hat{f}(k)}^2 \qq \hat{f}(k) = <f, e_k> \]
  \end{Definition}

  \begin{Definition}
      \[\hat{f}(k) = <f, e_k> = \int_0^1 f \cdot \overline{e}_k \text{ - коэфф. Фурье
      функции } f\]
      \[\text{по ортог. системе функций } \{e_k\}_{k \in \Z} \]
  \end{Definition}

  \begin{Definition}
      \[\text{Ряд Фурье функции } f : \q \sum_{k = -\infty}^\infty \hat{f}(k)e_k(x) \]
  \end{Definition}

  \begin{example}
    *здесь когда-нибудь будет пример*
  \end{example}
  \newpage
  \section{Свойства коэффициентов Фурье (коэффициенты Фурье сдвига, производной).}

  \begin{properties}
      \begin{enumerate}
          \item $\displaystyle f_a(t) = f(t + a) \q \Ra \hat{f}_a(k) =
              \int_0^1 f(t + a)e ^{- 2\pi i kt}dt = $
              \[= \int_a^{1 + a} f(x) \cdot e^{-2\pi i k (x - a)} dx =
              \int_0^1 f(x)\cdot e^{-2\pi i k x} \cdot e^{e\pi i k a} = e^{2\pi i k a } \hat{f}(k)\]
          \item Пусть $f \in C^{1} (\R / \Z) $
              \[\hat{f}'(k) = \int_0^1 f'(t) \cdot e^{-2\pi i kt} dt =  \]
              Интегрируем по частям
              \[= \underbracket{ f(t)e^{-2\pi i kt}}_{= 0 \text{ т.к. } T = 1}
                  \bigg|_0^1  + 2\pi i k
              \underbracket{\int_0^1 f(t)
          e^{-2\pi i k t}dt}_{\hat{f}(k)}  \]
          \[\hat{f'}(k) = 2\pi ik \hat{f}(k)\]
          \item Коэф. Фурье фещ. функции
              \[f \in R[-\frac{1}{2}, \frac{1}{2}]\]
              \[\hat{f}(k) = \int_0^1 f(t) \cdot e^{-2 \pi ikt}dt \]
              \[\hat{f}(-k) = \int_0^1 f(t) \cdot e^{2 \pi k t}dt \]
              \[\Ra \hat{f}(k) = \overline{\hat{f}(-k)}\]
          \item Коэфф. Ферье четной функции
              \[f \text{ - четная}\]
              \[\hat{f}(k) = \int_{-k}^k f(t)e^{-2\pi ikt}dt = \int_{-k}^k
                  \underbracket{f \cdot \cos 2\pi k t}_{\text{четная}}
              - i \int_{-k}^k \underbracket{f \cdot \sin 2\pi kt}_{\text{нечетная } = 0} = \]
              \[= \int_{-k}^k f \cos 2\pi kt = \hat{f}(-k) \text{ поскольку четная} \]
      \end{enumerate}
      Если $f$ - вещ и четная $\Ra \q \hat{f}(k) = \hat{f}(-k) = \overline{\hat{f}(k)}$
      \[\Ra \hat{f}(k) \in \R\]
  \end{properties}


  \newpage
  \section{Неравенство Бесселя. Лемма Римана-Лебега (light).}

  \begin{Definition} [Неравенство Бесселя]
      \[\sum_{k = -\infty}^{+\infty} \abs{\hat{f}(k)}^2 \leq \int_0^1 \abs{f}^2 =
      \Abs{f}^2\]
  \end{Definition}

  \begin{Lemma}
      \[f \in R[0, 1], \q S_N(f) = \sum_{k = -N}^N \hat{f}(k)e_k \]
      \[f - S_N(f) \perp S_N(f)\]
  \end{Lemma}

  \begin{Proof}
      \[<f, S_N(f)> = \int_0^1 f(t) \cdot \overline{\sum_{k = -N}^N \hat{f}(k)
      \cdot e_k(t)dt} = \sum_{k = -N}^N \overline{\hat{f}(k)} \int_0^1
      f(t) \cdot \overline{e_k}dt = \sum_{k = -N}^N \abs{\hat{f}(k)}^2 =\]
      \[= <S_N(f), S_N(f)>\ = \ \Abs{S_N(f)}^2\]

      \[<f, S_N(f)> \q  = \q  <S_N(f), S_N(f)>\]
      \[\rla <f - S_N(f), S_N(f)> \q  = 0\]
  \end{Proof}

  \begin{consequence}
      т.к. $f - S_N(f) \perp S_N(f)$, то $\Abs{f}^2 = \Abs{f - S_N(f)}^2 + \Abs{S_N(f)}^2$
  \end{consequence}

  \begin{Proof}[нер-ва Бесселя]
      \[\sum_{k = -N}^N \abs{\hat{f}(k)^2} = \Abs{S_N(f)}^2 \leq \Abs{f}^2 =
      \int_0^1 \abs{f}^2 \]
      \[\text{предельный переход в нер-ве} \q (N \to \infty)\]
      \[\sum_{k = -\infty}^{+\infty} \abs{\hat{f}(k)}^2 \leq \int_0^1 \abs{f}^2\]
  \end{Proof}

  \begin{Consequence}[Лемма Римана-Лебега]
      \[f \in R_{\CC}[0, 1] \Ra \hat{f}(k) \to 0 \q k \to  + \infty \q k \to -\infty\]
  \end{Consequence}

  \begin{proof}
      Необходимо усл. сх-ти ряда и нер-во Бесселя.\\
      В нер-ве Бесселя ряд возрастает и ограничен сверху, значит он сходится.
      \[\Ra \hat{f}(k) \to 0\]
  \end{proof}
\end{document}
