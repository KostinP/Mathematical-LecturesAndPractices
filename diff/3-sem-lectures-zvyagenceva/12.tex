\documentclass[main]{subfiles}

\begin{document}
\begin{lect}{2019-11-21}
    \section{Линейное однородное уравнение}
    \begin{Definition}
        \[L(x) = \sum_{j = 0}^n p_j(t) x^{(n - j)} \qq p_0(t) \equiv 1, \q x^{(0)} = x   \]
        \[(1) \q L(x) = 0 \q \text{ л.о.у} \qq p_j(t) \in C(a, b)\]
    \end{Definition}

    \begin{Definition}
        \[\varphi_1(t), ..., \varphi_n(t) \text{ - реш } (1), \q t \in (a, b) \Ra \text{ лин. незав}\]
        \[\Ra \varphi_1(t), ..., \varphi_n(t) \text{ - фунд. сист. решений } (1) \text{ на } (a, b)\]
    \end{Definition}

    \begin{Theorem}
        \[\exists \text{ ф.с.р.}\]
    \end{Theorem}

    \begin{Proof}
        \[A = \{a_{ij} \}_{i, j = 1}^n : \q \det A \neq 0 \]
        \[\RNumb{1} \text{ З.К. } t = t_0 \q x(t_0) = a_{11},
        \q \dot{x}(t_0) - a_{21}, ..., x^{(n - 1)}(t_0) = a_{n1}    \]
        \[\RNumb{2} \text{ З.К. } t = t_0 \q x(t_0) = a_{12}, \q \dot{x}(t_0) = a_{22}, ...,
        x^{(n - 1)}(t_0) = a_{n2}    \]
        ...
        \[n \text{ - а я З.К. } t = t_0 \q x(t_0) = a_{1n}, \q \dot{x}(t_0) = a_{2n}, ...,
        x^{(n - 1)}(t_0) = a_{nn}    \]
        \[t_0 \in (a, b)\]
        \[\varphi_j(t) \text{ - реш } j \text{ - ой З.К. } \qq W(t_0) = \det A \neq 0 \Ra
        \varphi_1(t), ..., \varphi_n(t) \text{ - ЛНЗ}\]
    \end{Proof}

    \begin{Definition}
        \[x(t) = C_1 \varphi_1(t) + ... + C_n \varphi_n(t) \q \text{, где } \varphi_1(t), ..., \varphi_n(t)
        \text{ - ф.с.р } (1) \q t \in (a, b)\]
        \[C_1, ..., C_n \text{ - произв. конст.}\]
        \[x(t) = C_1 \varphi_1(t) + ... + C_n \varphi_n(t) \qq (2) \qq  \text{ - общее решение (1)}\]
    \end{Definition}

    \begin{theorem}[2]
        \begin{enumerate}
            \item $\forall C_1, ..., C_n \qq (2) $ - реш $(1)$
            \item $x = \xi(t)$ - реш $(1) \Ra \exists \overline{C}_1, ..., \overline{C}_n:$
                \[\xi(t) = \overline{C}_1\varphi_1(t) + ... + \overline{C}_n \varphi_n(t)\]
        \end{enumerate}
    \end{theorem}

    \begin{proof}
        \begin{enumerate}
            \item доказано (осн. хар. св-во)
            \item $x = \xi(t)$ - реш $(1), \q t\in (a, b)$
        \end{enumerate}
        \[t_0 \in (a, b)\]
        \[(3) \q\begin{cases}
            C_1 \varphi_1(t_0) + C_2 \varphi_2(t_0) + ... + C_n\varphi_n(t_0) = \xi(t_0)\\
            C_1 \dot{\varphi}_1(t_0) + C_2 \dot{\varphi}_2 + ... + C_n \dot{\varphi}(t_0) = \dot{\xi}(t_0)\\
            ...\\
            C_1 \varphi_1^{(n - 1)} (t_0) + C_2 \varphi_2^{(n - 1)} (t_0) + ... + C_n\varphi_n^{(n - 1)} (t_0) =
            \xi^{(n - 1)} (t_0)
        \end{cases}\]
        Определитель $(3) \q W(t_0) \neq 0 \Ra \exists ! \text{ реш } (3)\q \overline{C}_1, ..., \overline{C}_n$
        \[x(t) = \overline{C}_1 \varphi_1(t) + ... + \overline{C}_n \varphi_n(t) \text{ - реш } (1) \qq
        t \in (a, b)\]
        \[x(t_0) = \xi (t_0)\]
        \[\dot{x}(t_0) = \dot{\xi}(t_0)\]
        \[...\]
        \[x^{(n - 1)}(t_0) = \xi^{(n - 1)}(t)  \]
        \[\Ra x(t) \text{ и } \xi(t) \text{ решают одну З.К.}\]
        \[\Ra x(t) \equiv \xi(t) \text{ на } (a, b)\]
    \end{proof}

    \begin{remark}
        Мы не умеем строить ф.с.р для уравнений с переменными коэф.
    \end{remark}

    \section{Линейное неоднородное уравнение}

    \[(1) \qq L(x) = g(t) \text{ - л.н.у}\]
    \[\qq\qq (q(t) \neq 0, \q q(t) \in C(a, b))\]
    \[(2) \qq L(x) = 0 \text{ - соотв. л.о.у}\]

    \begin{Theorem}
        \[t \in (a, b)\]
        \[\begin{matrix}
            x = \varphi(t) \text{ - реш (2)}\\
            x = \psi(t) \text{ - реш (1)}
        \end{matrix} \bigg| \Ra (\varphi(t) + \psi(t)) \text{ - реш } (1)  \]
    \end{Theorem}

    \begin{Proof}
        \[L(\varphi + \psi) = \us{= 0 \text{ т.к. реш (2)}}{ L(\varphi)} + \us{= q(t) \text{т.к реш}(1)}{L(\psi)}
            = q(t)\]
        \[\Ra (\varphi + \psi) \text{ - реш (1)},\q t \in (a, b)\]
    \end{Proof}

    \begin{Definition}
        \[x(t) = C_1 \varphi_1(t) + ... + C_n \varphi_n(t) + \psi(t) \qq (3)\]
        \[\text{ где } \varphi_1(t), ..., \varphi_n(t) \text{ - ф.с.р } (1) \q t \in (a, b)\]
        \[C_1, ..., C_n \text{ - произв. конст.}\]
        \[x(t) = C_1 \varphi_1(t) + ... + C_n
        \varphi_n(t) + \psi(t) \qq (3) \qq  \text{ - общее решение (1) (л.н.у)}\]
    \end{Definition}

    \begin{theorem}[2]
        \begin{enumerate}
            \item $\forall C_1, ..., C_n \q (3)$ дает реш (1)
            \item $\forall $ реш $(1) \q x = \xi(t) \qq \exists \overline{C}_1, ..., \overline{C_n}:$
                \[\xi(t) = \overline{C}_1 \varphi_1(t) + ... + \overline{C}_n\varphi_n(t) + \psi(t)\]
        \end{enumerate}
    \end{theorem}

    \begin{Proof}
        \[1) \q \text{ - доказано  (осн. хар. св-во + Т1)}\]
        \[2) \q t_0 \in (a, b)\]
        \[(4) \qq \begin{cases}
            C_1 \varphi_1(t_0) + ... + C_n\varphi_n(t_0) + \psi(t_0) = \xi(t_0)\\
        C_1 \dot{\varphi}_1(t_0) + ... + C_n \dot{\varphi}_n(t_0) - \dot{\psi}(t_0) = \dot{\xi}(t_0)\\
        ...\\
        C_1 \varphi_1^{(n - 1)} (t_0) + ... + C_n\varphi_n^{(n - 1)} (t_0) + \psi^{(n - 1)} (t_0) =
        \xi^{(n - 1)} (t_0)
        \end{cases}\]
        \[\text{ опред (4)} \q W(t_0) \neq 0 \Ra \exists  ! \text{ реш } (4)) \]
        \[x(t) = \overline{C}_1 \varphi_1(t) + ... + \overline{C}_n \varphi_n(t) + \psi(t) \text{ - реш } (1) \qq
        t \in (a, b)\]
        \[ x(t) \text{ и } \xi(t) \text{ решают одну З.К. (из (4))}\]
        \[\Ra x(t) \equiv \xi(t) \text{ на } (a, b)\]
    \end{Proof}

    \section{Метод Лагранжа, вариация произвольных постоянных}
    \[L(x) = x^{(n)}  + p_1(t) x^{(n - 1)} + ... + p_{n - 1}(t) \dot{x} + p_n (t)x   \]
    \[p_j(t), \ q(t) \in C(a, b) \qq j = 1, ..., n\]
    \[(1) \q L(x) = q(t) \text{ - л.н.у}\]
    \[(2) \q L(x) = 0 \text{ - л.о.у}\]
    \[\varphi_1(t), ..., \varphi_n(t) \text{ - ф.с.р (2)}\]
    \[x(t) = C_1 \varphi_1(t) + ... + C_n \varphi_n(t) \text{ - общ. реш (2)}\]
    \[\qq( C_1, ..., C_n \text{ - произв. константы})\]
    \[(3) \qq \psi(t) = \sum_{j = 1}^n u_j(t)\varphi_j(t) \text{ - в таком виде ищем реш (1)} \]
    \[\dot{\psi}(t) = \sum_{j = 1}^n u_j(t)\dot{\varphi}_j(t) + \underbrace{\sum_{j = 1}^n
    \dot{u}_j(t)\varphi_j(t)}_{ = 0 \text{ усл 1}}   \]
    \[\ddot{\psi}(t) = \sum_{j = 1}^n u_j(t)\ddot{\varphi}_j(t) + \underbrace{\sum_{j = 1}^n
    \dot{u}_j(t)\dot{\varphi}_j(t)}_{ = 0 \text{ усл 2}}   \]
    \[...\]
    \[\psi^{(n - 1)} (t) = \sum_{j = 1}^n u_j(t) \varphi_j^{(n - 1)}(t)  +
    \underbrace{ \sum_{j = 1}^n \dot{u}_j(t) \varphi_j^{n - 2}(t)}_{ = 0 \text{ усл (n - 1)}} \]
    \[\psi^{(n)} (t) = \sum_{j = 1}^n u_j(t) \varphi_j^{(n)}(t)  +
    \underbrace{ \sum_{j = 1}^n \dot{u}_j(t) \varphi_j^{n - 1}(t)}_{ = 0 \text{ усл n}} \]
    \[L(\psi(t)) = \sum_{m = 0}^n p_m(t) \psi^{(n - m)}(t)  = \]
    \[(p_0 \equiv 1, \q \psi^{(0)}(t) \equiv \psi(t) )\]
    \[= \sum_{m = 0}^n p_m(t) \cdot \left( \sum_{j = 1}^n u_j(t)\varphi_j(t) \right)^{(n - m)} =
    \sum_{m = 0}^n p_m(t) \left(\sum_{j = 1}^n u_j(t) \varphi_j^{(n - m)}  \right) = \]
    \[= \sum_{j = 1}^n \dot{u}_j (t) \varphi_j^{(n - 1)}(t) =   \]
    \[\sum_{j = 1}^n u_j(t) \left(\us{L(\varphi_j) = 0}{\sum_{m = 0}^n p_m(t) \varphi_j^{(n - m)}(t)  }\right)  +
    \sum_{j = 1}^n \dot{u}_j (t) \varphi_j^{(n - 1)}(t) = q(t)  \q \text{ (усл n)} \]
    \[\begin{cases}
        \sum_{j = 1}^n \dot{u}_j(t)\varphi(t) = 0\\
        \sum_{j = 1}^n \dot{u}_j(t)\dot{\varphi}(t) = 0 \\
         ...\\
         \sum_{j = 1}^n \dot{u}_j(t)\varphi^{(n - 2)} (t) = 0 \\
         \sum_{j = 1}^n \dot{u}_j(t)\varphi^{(n - 1)} (t) = q(t) \\
    \end{cases}\]
    (4) - система в вариациях
    \[t \in (a, b) \text{ фикс}\]
    \[\text{опр-ль } (4) \q W(t)\]
    \[\varphi_1, ..., \varphi_n \text{ - ф.с.р}\]
    \[\Ra W(t) \neq 0\]
    \[\exists ! \text{ реш } (4)\]
    \[\dot{u}_1(t), ..., \dot{u}_n(t) \text{ - непр на } (a, b)\]
    коэф. непр. зависят от $t$, решение тоже
    \[\Ra \exists \text{ первообразные } \q u_j(t) = \int \dot{u}_j (t)dt\]
    \[\psi(t) \text{ дает решение } (1)\]

    \section{Линейные однор. ур. с пост. коэф}
    \[L(x) = x^{(n)} = a_1 x^{(n - 1)} + ... + a_{n - 1} \dot{x} + a_n x   \]
    \[(1) \q L(x) = 0 \text{ - л.о.у. с пост. коэф. (л. автономное о.у. \q л.а.о.у)}\]
    (автономные - системы, куда явно не входит $t$)
    \[x = e^{\lambda t} \text{ - реш } (1) \rla L(e^{\lambda t} ) = 0 \]
    \[(2) \q L(e^{\lambda t} ) = e^{\lambda t} \cdot \us{\text{мн-н}}{P(\lambda)}, \text{ где } \]
    \begin{Definition}
        \[P(\lambda) = \lambda^n + a_1 \lambda^{n - 1} + ... + a_{n - 1}\lambda + a_n  \]
        характеристический мн-н $(1)$, его корни $\lambda_j$ - характ. числа
        \[e^{\lambda t} \text{ - реш }(1) \rla \us{\text{характ. ур-е}}{P(\lambda) = 0} \qq (3) \]
    \end{Definition}

    \subsection{Построение ф.с.у}
    \[1) \q \lambda_1, ..., \lambda_n \text{ - вещ., различные корни }(3)\]
    \[\Ra e^{\lambda_1 t}, ..., e^{\lambda_n t} \text{ - ф.с.р.} (1)  \]
    УПР доказать лнз
\end{lect}
\end{document}
