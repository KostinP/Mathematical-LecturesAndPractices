\documentclass[main]{subfiles}

\begin{document}
\begin{lect}
    \section{Введение}
    \subsection{Литература}
    \begin{enumerate}
        \item Учебник Бибиков "Обыкновенные дифферинциальные уравнения"
        \item Филиппов - задачи
        \item "Методы интегрирования"
        \item Каддинктон Ливенгсон "Обыкновенные дифференциальные уравнения"
        \item Яругии
    \end{enumerate}

    \subsection{Введение}
    $F(x,y,y',y'',...,y^{(n)}) = 0$\\
    $x$ - неизвестная переменная\\
    $y = y(x)$ - неизвестная функция, ...

    \begin{definition}
    Порядок уравнения - порядок старшей производной
    \end{definition}
    \\
    Кроме того, $x = \frac{dx}{dt}$, $x^{(k)} = \frac{d^k x}{dt^k}$

    \subsection{Применение}
    \begin{enumerate}
        \item механика
        \item электротехника
        \item физика: $\dot{Q} = k Q$, $Q = Q_0 e^{kt}$
        \item упр. движением
        \item биология, экология
    \end{enumerate}

    \begin{example}[из биологии]
        x - хищни, y - жертва
       \[\begin{cases}
           \dot{x} = -ax+cxy\\
           \dot{y} = by-dxy
       \end{cases} \qq a,b,c,d > 0,\ x,y>0\]
    \end{example}

    \section{Дифферинциальные уравнения первого порядка}
    \subsection{Введение}
    \[(1)\q \dot{x} = X(t,x)\]
    \[X(t,x) \in C(G),\text{ G - обл, }G \subset \R^2\]
    Но чаще будем $\in C(D)$ $D \subset \R^2$

    \begin{definition}
        Решение (1) - функция $x=\upvarphi(t)$, $t \in <a,b>:$
        \[\dot{\upvarphi}(t) \equiv X(t,\upvarphi(t))\text{ на <a,b>}\]
    \end{definition}

    \begin{enumerate}
      \item $\forall t \in <a,b>$ $(t, \upvarphi(t)) \in D$
      \item $\upvarphi(t)$ - дифф на $<a,b>$
      \item $\upvarphi(t)$ - непр. дифф. (X- непр на D)
    \end{enumerate}

    \begin{definition}
    (2) Задача Коши - задача нахождения решения (1) $x=\upvarphi(t):\ \upvarphi(t_0)=x_0$ $((t_0, x_0) \in D)$
    \end{definition}
    \\
    Геометрический смысл уравнения первого порядка - уравнение 1 задаёт поле направлений на множестве G

    \begin{definition}
        График решения называется интегральной кривой
    \end{definition}
    \\
    В каждой точке задано направление, которое совпадает с касательной в этой точке к интегральной кривой
    \[\dot{\upvarphi}(t) |_{t=t_0} = X(t_0, x_0)\]
    \subsection{Метод изоклин}
    \begin{definition}
        Изоклина - это кривая, на которой поле направлений постоянно
    \end{definition}
    \\
    Уравнение изоклин $X(t,x) = c$, где $c = \const$

    \begin{example}
        \[\dot{x} = -\frac{t}{x}\q (x = \upvarphi(t))\]
        \[-\frac{t}{x} = \tg \alpha\]
        \[x = -\frac{1}{c} t$, $c \neq 0\]
        \[c = 1 \q (\alpha=\frac{\pi}{4})\q x = -t \text{ - уравнение изоклин}\]
        \[c = -1\q (\alpha=-\frac{\pi}{4}) \q x = t\]
        Решение задачи Коши (1, 1) - это $x = \sqrt{2-t^2}$\\
        Решение задачи Коши (1,-1) - это $x = -\sqrt{2-t^2}$
    \end{example}

    \subsection{Теорема Пеано}
    (1) $\dot{x} = X(t,x)$, $X \in C(D)$\\
    $D=\{(t,x):|t-t_0| \leqslant ... \leqslant |x-x_0| \leqslant b \}$\\
    (2) $(t_0,x_0)$\\
    По теореме Вейерштрасса $\exists M:\ |X(t,x)| \leqslant M\ \forall(t,x) \in D$\\
    $h=min(a,\frac{b}{M})$

    \begin{theorem2}[Пеано]
        $\exists$ реш. задачи К. (1), (2) $x=\upvarphi(t)$ опр-е на $[t_0-h,\ t_0+h]$ - отрезок Пеано
    \end{theorem2}

    \begin{Definition}
        \[\{\upvarphi_k(t)\}_{k=1}^\infty$, $t \in [c,d]\]
        \begin{enumerate}
            \item $\upvarphi_k(t)$ - равномерно ограничена на $[c,d]$, если $\exists N:\ |\upvarphi_k(t)| \leqslant N$ $\forall k \in \mathds{N}$, $\forall t \in [c,d]$
            \item $\upvarphi_k(t)$ - равностепенно непр на [c,d],  если $\forall \E > 0$ $\exists \delta > 0:$ $\forall t_1, t_2 \in [c,d]$ $|t_1-t_2| < \delta$ $\ra$ $|\upvarphi_k(t_1)-\upvarphi_k(t_2)| < \E$ $\forall k \in \mathds{N}$
        \end{enumerate}
    \end{Definition}

    \begin{lemma}[](Арцелло - Асколи)]
        $\upvarphi_k(t)$, $k\in \mathds{N}$, равномерно огр. и равностепенно непр на $[c,d]$ $\ra$ $\exists$ подпосл $\upvarphi_k_j(t):$ $\upvarphi_k_j(t) \overset{[c,d]}{\underset{j \ra + \infty}{\rightrightarrows}} \upvarphi(t)$
    \end{lemma}
\end{lect}
\end{document}
