\documentclass[main]{subfiles}

\begin{document}
    \section{Неопределенные интегралы}
    \subsection{Самое важное из теории про интегралы}

    \begin{utv}
        Если $f \in R[a,b]$, то $f$ - ограничена на $[a,b]$.
    \end{utv}

    \begin{Theorem}[первая теорема о среднем]
        \[f,g \in R[a,b],\ g \geqslant 0,\ m \leqslant f \leqslant M\] \[\forall x \in [a,b] \Rightarrow \e \upmu \in [m,M]: \int\limits^a_b f g = \upmu \int\limits^a_b g\]
    \end{Theorem}

    \begin{consequence}
        Eсли $f \in C[a,b],\ g \in R[a,b],\ g \geqslant 0 \Rightarrow \e \xi \in [a,b]: \int\limits^a_b f g = f(\xi) \int\limits^a_b g$
    \end{consequence}

    \begin{Theorem}[Формула Валлиса]
        \[\lim\limits_{n \rightarrow \infty} \frac{2*2*4*4*...*(2n)(2n)}{1*3*3*5*5...(2n-1)(2n+1)} = \dfrac{\pi}{2}$ (или $\lim\limits_{n \rightarrow \infty} \frac{1}{n} (\frac{(2n)!!}{(2n-1)!!})^2 = \pi)\]
    \end{Theorem}

    \begin{Theorem}[формула Тейлора с остаточным членом в интегральной форме]
        \begin{multline*}
            $$f \in C^{n+1} ([a,b]) \Rightarrow f(b)=\sum\limits_{k=0}^n \frac{f^{(k)}(a)}{k!} (b-a)^k + R_n (b,a), \\ \text{ где }R_n(b,a)=\frac{1}{n!} \int\limits_a^b f^{(n+1)}(t) (b-t)^n dt$$
        \end{multline*}
    \end{Theorem}

    \begin{Theorem}[Бонне или вторая теорема о среднем]
        \begin{multline*}
            $$f \in C[a,b],\ g\in C^1[a,b], g - \text{монотонна} \\
            \Rightarrow \e \xi \in [a,b]: \int\limits_a^b f g = g(a) \int\limits_a^\xi f  + g(b) \int\limits_\xi^b f$$
        \end{multline*}
    \end{Theorem}

    \begin{Theorem}[замена переменной]
        \[\upvarphi \subset C^1 [\upalpha,\upbeta],\ f \in C(\upvarphi([\upalpha,\upbeta])),\text{ тогда } \int\limits_{\upvarphi(\upalpha)}^{\upvarphi(\upbeta)} f = \int\limits_{\upalpha}^{\upbeta} (f \circ \upvarphi) \upvarphi'\]
    \end{Theorem}

    \begin{Theorem}[замена переменной]
        \begin{multline*}
            $$f \in R[a,b],\ \upvarphi \in C^1 [\upalpha, \upbeta],\ \upvarphi \text{ - строго возрастает}, \\
            \upvarphi(\upalpha) = a,\q \upvarphi(\upbeta) = b,
            \text{ тогда } \int\limits_a^b f = \int\limits_\upalpha^\upbeta (f \circ \upvarphi) \upvarphi'$$
        \end{multline*}
    \end{Theorem}

    \begin{Theorem} [критерий Больцано-Коши для несобственных интегралов]
        \[f: [a, \upomega) \rightarrow \R,\q -\infty < a < \upomega \leqslant +\infty,\q f \in R[a,b]\q \forall b \in (a, +\infty),\text{ тогда:}\]
        \[\int\limits_a^\upomega f\text{ - сх }\lra \q \forall \E > 0\ \e B \in (a, \upomega): \forall b_1,b_2 \in (B, \upomega)\ |\int\limits_{b_1}^{b_2}| < \E\]
    \end{Theorem}

    \begin{Property} [интегрирование по частям]
        \begin{multline*}
            $$\text{Пусть } f,g \in C^1 [a, \upomega),\q \e \lim\limits_{x \rightarrow \upomega_-} f(x) g(x) \in \R, \text{ тогда:}\\
            \int\limits_a^\upomega f' g\text{ и } \int\limits_a^\upomega f g'\text{ - сх или расх одновременно, причем }\\
            \int\limits_a^\upomega f g' = f g |_a^\upomega - \int\limits_a^\upomega f' g (f g |_a^\upomega =  \lim\limits_{x \rightarrow \upomega_-} (f(x) g(x) - f(a) g(a))$$
        \end{multline*}
    \end{Property}

    \begin{Property} [замена переменной]
        \begin{multline*}
            $$\text{Если }\int\limits_a^\upomega f \text{ - сх},\q \upvarphi: [\upalpha, \upupsilon) \rightarrow [a, \upomega),\q \upvarphi \in C^1 [\upalpha, \upupsilon),\q \upvarphi\text{ - монот.},\\
            \upvarphi(\upalpha)=a,\q \lim\limits_{t \rightarrow \upupsilon} \upvarphi(t) = \upomega,\text{ тогда }\int\limits_a^\upomega f = \int\limits_\upalpha^\upupsilon (f \circ \upvarphi) \upvarphi'$$
        \end{multline*}
    \end{Property}

    \begin{Theorem}[\RNumb{1} признак сравнения]
        \[f,g: [a, \upomega) \rightarrow \R,\q f,g \geqslant 0,\q f,g \in R[a,b],\q b \in (a, \upomega),\]
        \[0 \leqslant f(x) \leqslant g(x)\q \forall x \in [a, \upomega)\]
        Тогда $\int\limits_a^\infty g$ - сх $\Rightarrow$ $\int\limits_a^\upomega f$ - сх ($\int\limits_a^\upomega f$ - расх $\Rightarrow$ $\int\limits_a^\infty g$ - расх)
    \end{Theorem}

    \begin{Theorem}[\RNumb{2} признак сравнения]
        \[f, g: [a, \upomega) \rightarrow (0, +\infty)$, $f,g \in R[a,b]$ $\forall b \in (a, \upomega)\]
        Тогда если $\e \lim\limits_{x \rightarrow \upomega_-} \dfrac{f(x)}{g(x)} \in (0, +\infty)$, то $\int\limits_a^\upomega f$ и $\int\limits_a^\upomega g$ - сх или расх одновременно
    \end{Theorem}

    \begin{definition}
        $f: [a, \upomega) \rightarrow \R$, $f \in R[a,b]$ $\forall b \in (a, \upomega)$

        $\int\limits_a^\upomega f$ - сх абсолютно $\lra$ $\int\limits_a^\upomega |f|$ - сх

        $\int\limits_a^\upomega f$ - сх условно $\lra$ $\int\limits_a^\upomega f$ - сх, $\int\limits_a^\upomega |f|$ - расх
    \end{definition}

    \begin{utv}
        $\int\limits_a^\upomega f$ - сх абсолютно $\Rightarrow$ сходится
    \end{utv}

    \begin{Theorem} [признак Абеля-Дирихле]
        \[f,g: [a, \upomega) \rightarrow \R,\q f \in C[a,\upomega),\q g \in C^1 [a,\upomega),\text{ g - монотонна.}\]
        Тогда если выполнено одно из условий:
        \[\text{(A) }\int_a^\upomega \text{f - сх, g - огр}\]
        \[\text{(Д) }F(x) := \int\limits_a^x \text{f - огр, }g(x) \underset{x \rightarrow \upomega_-}{\rightarrow} 0\]
        Тогда $\int\limits_a^\upomega f g$ - сх
    \end{Theorem}

    \newpage
    \subsection{Простейшие интегралы}

    \begin{definition}
        $f: <a,b> \ra \R$, $F(x)$ - первообразная, если $F'(x) - f(x)\q \forall x \in <a,b>$
        \[\int f(x) = F(x) + C\]
    \end{definition}

    \begin{utv}
        \begin{enumerate}
          \item $\int x^{\alpha} dx = \dfrac{x^{\alpha + 1}}{\alpha + 1} + C,\q \alpha \neq 1,\q \alpha \in \R$
          \item $\int \dfrac{dx}{x} = \ln|x| + C$ ($\ln x + C_1$ при $x>0$, $\ln|x| + C_2$ при $x<0$)
          \item $\int \dfrac{dx}{x^2 - a^2} = \dfrac{1}{2a} \ln \abs{\dfrac{x-a}{x+1}} + C,\q a \neq 0$
          \item $\int \dfrac{dx}{x^2 + a^2} = \dfrac{1}{a} \arctg \Br{\dfrac{x}{a}} + C,\q a \neq 0$
          \item $\int \dfrac{dx}{\sqrt{a^2 - x^2}} = \arcsin \Br{\dfrac{x}{a}} + C,\q a > 0 \q a > |x|$
          \item $\int \dfrac{dx}{\sqrt{x^2 \pm a^2}} = \ln|x + \sqrt{x^2 \pm a^2}| + C$
          \item $\int a^x dx = \dfrac{a^x}{\ln a} + C,\q a>0,\q a \neq 1$
          \item $\int \dfrac{dx}{\cos^2 x} = \tg x + C$
          \item $\int \dfrac{dx}{\sin^2 x} = -\ctg x + C$
          \item $\int \cos x dx = \sin x + C$
          \item $\int \sin x dx = -\cos x + C$
          \item $\int \tg x dx = -\ln|\cos x| + C$
          \item $\int \ctg x dx = \ln|\sin x| + C$
          \item $\int \dfrac{dx}{\sin x} = \ln|\tg \frac{x}{2}| + C$
          \item $\int \dfrac{dx}{\cos x} = \ln|\tg(\frac{x}{2} + \frac{\pi}{4})| + C$
          \item $\int \sh x dx  = \ch x + C$
          \item $\int \ch x dx = \sh x + C$
          \item $\int \dfrac{1}{\ch^2 x} dx = \th x + C$
          \item $\int \dfrac{1}{\sh^2 x} dx = -\cth x + C$
          %\item $\int \dfrac{x dx}{a^2 \pm x^2} = \pm \frac{1}{2} \ln |a^2 \pm x^2| + C$
          %\item $\int \dfrac{x dx} = \pm \sqrt{a^2 \pm x^2} + C \q a>0$
          \item $\int \sqrt{a^2 - x^2} dx = \frac{x}{2} \sqrt{a^2 - x^2} + \frac{a^2}{2} \arcsin \frac{x}{a} + C, \q a>0$
          \item $\int \sqrt{x^2 \pm a^2} dx = \frac{x}{2} \sqrt{x^2 \pm a^2} \pm \frac{a^2}{x} \ln|x + \sqrt{x^2 \pm a^2}| + C,\q a>0$
        \end{enumerate}
    \end{utv}

    Тригонометрические интегралы: \href{https://ru.wikipedia.org/wiki/%D0%A1%D0%BF%D0%B8%D1%81%D0%BE%D0%BA_%D0%B8%D0%BD%D1%82%D0%B5%D0%B3%D1%80%D0%B0%D0%BB%D0%BE%D0%B2_%D0%BE%D1%82_%D1%82%D1%80%D0%B8%D0%B3%D0%BE%D0%BD%D0%BE%D0%BC%D0%B5%D1%82%D1%80%D0%B8%D1%87%D0%B5%D1%81%D0%BA%D0%B8%D1%85_%D1%84%D1%83%D0%BD%D0%BA%D1%86%D0%B8%D0%B9#%D0%98%D0%BD%D1%82%D0%B5%D0%B3%D1%80%D0%B0%D0%BB%D1%8B%2C_%D1%81%D0%BE%D0%B4%D0%B5%D1%80%D0%B6%D0%B0%D1%89%D0%B8%D0%B5_%D1%82%D0%BE%D0%BB%D1%8C%D0%BA%D0%BE_%D0%BA%D0%BE%D1%81%D0%B8%D0%BD%D1%83%D1%81}{википедия}

    Экспоненциальные интегралы: \href{https://ru.wikipedia.org/wiki/%D0%A1%D0%BF%D0%B8%D1%81%D0%BE%D0%BA_%D0%B8%D0%BD%D1%82%D0%B5%D0%B3%D1%80%D0%B0%D0%BB%D0%BE%D0%B2_%D0%BE%D1%82_%D1%8D%D0%BA%D1%81%D0%BF%D0%BE%D0%BD%D0%B5%D0%BD%D1%86%D0%B8%D0%B0%D0%BB%D1%8C%D0%BD%D1%8B%D1%85_%D1%84%D1%83%D0%BD%D0%BA%D1%86%D0%B8%D0%B9}{википедия}

    Логарифмические интегралы: \href{https://ru.wikipedia.org/wiki/%D0%A1%D0%BF%D0%B8%D1%81%D0%BE%D0%BA_%D0%B8%D0%BD%D1%82%D0%B5%D0%B3%D1%80%D0%B0%D0%BB%D0%BE%D0%B2_%D0%BE%D1%82_%D0%BB%D0%BE%D0%B3%D0%B0%D1%80%D0%B8%D1%84%D0%BC%D0%B8%D1%87%D0%B5%D1%81%D0%BA%D0%B8%D1%85_%D1%84%D1%83%D0%BD%D0%BA%D1%86%D0%B8%D0%B9}{википедия}

    Иррациональные интегралы: \href{https://ru.wikipedia.org/wiki/%D0%A1%D0%BF%D0%B8%D1%81%D0%BE%D0%BA_%D0%B8%D0%BD%D1%82%D0%B5%D0%B3%D1%80%D0%B0%D0%BB%D0%BE%D0%B2_%D0%BE%D1%82_%D0%B8%D1%80%D1%80%D0%B0%D1%86%D0%B8%D0%BE%D0%BD%D0%B0%D0%BB%D1%8C%D0%BD%D1%8B%D1%85_%D1%84%D1%83%D0%BD%D0%BA%D1%86%D0%B8%D0%B9}{википедия}

    Обратные тригонометрические интегралы: \href{https://ru.wikipedia.org/wiki/%D0%A1%D0%BF%D0%B8%D1%81%D0%BE%D0%BA_%D0%B8%D0%BD%D1%82%D0%B5%D0%B3%D1%80%D0%B0%D0%BB%D0%BE%D0%B2_%D0%BE%D1%82_%D0%BE%D0%B1%D1%80%D0%B0%D1%82%D0%BD%D1%8B%D1%85_%D1%82%D1%80%D0%B8%D0%B3%D0%BE%D0%BD%D0%BE%D0%BC%D0%B5%D1%82%D1%80%D0%B8%D1%87%D0%B5%D1%81%D0%BA%D0%B8%D1%85_%D1%84%D1%83%D0%BD%D0%BA%D1%86%D0%B8%D0%B9}{википедия}

    Гиперболические интегралы: \href{https://ru.wikipedia.org/wiki/%D0%A1%D0%BF%D0%B8%D1%81%D0%BE%D0%BA_%D0%B8%D0%BD%D1%82%D0%B5%D0%B3%D1%80%D0%B0%D0%BB%D0%BE%D0%B2_%D0%BE%D1%82_%D0%B3%D0%B8%D0%BF%D0%B5%D1%80%D0%B1%D0%BE%D0%BB%D0%B8%D1%87%D0%B5%D1%81%D0%BA%D0%B8%D1%85_%D1%84%D1%83%D0%BD%D0%BA%D1%86%D0%B8%D0%B9}{википедия}

    Ещё больше интегралов: \href{http://eqworld.ipmnet.ru/ru/auxiliary/aux-integrals.htm}{многоссылок}

    Решение интегралов: \href{https://ru.numberempire.com/integralcalculator.php}{NumberEmpire} и \href{https://www.wolframalpha.com/calculators/integral-calculator/}{WolframAlpha}

    \newpage
    \subsection{Замена переменной}

    \begin{Property} [замена переменной]
        \[f:<a,b> \ra \R \q \varphi: <c,d> \ra <a,b> \q \e \varphi'(t),\q t \in <c,d>\]
        \[\Ra G(t)+C = \int f(\varphi(t))\varphi'(t) dt = \int f(x) dx = F(x) + C = F(\varphi(t)) + C\]
        Но обычно делают так:
        \[F(x)+C = \us{x=\varphi(t) \text{ - биекция}}{\int f(x) dx} = \int  f(\varphi(t)) \varphi'(t) dt = G(t) + C = G(\varphi^{-1}(x)) + C\]
    \end{Property}

    \begin{Remark}
        \[\int f(x) dx = F(x) + C \RA \int f(ax + b) dx = \frac{1}{a} F(ax + b) + C \q (a \neq 0)\]
    \end{Remark}

    \begin{Example}[раскрывай скобки, заменяй страшное]
        \[\int \dfrac{(\sqrt{x} + 1)^3}{x \sqrt x} dx = \left[\begin{matrix}
            \sqrt{x} = t\\
            \frac{dx}{2 \sqrt{x}} = dt
        \end{matrix}\right] = 2 \int \dfrac{(t+1)^3}{t^2} dt = 2 \int(t + 3 + \frac{3}{t} + \frac{1}{t^2}) dt =\]
        \[= t^2 + 6t + 6\ln t - \frac{2}{t} + C = x + 6\sqrt{x} + 3 \ln x - \frac{2}{\sqrt{x}} + C\]
    \end{Example}

    \begin{Example}[заменяй страшное]
        \[\int \frac{e^{\arcsin x} + x + 1}{\sqrt{1 - x^2}} dx = \left[\begin{matrix}
            t = \arcsin x\\
            dt = 0 \frac{dx}{\sqrt{1-x^2}}
        \end{matrix}\right] = \int(e^t + \sin t + 1) dt =\]
        \[= e^t - \cos x + t + C = e^{\arcsin x} - \us{= \sqrt{1 - x^2}}{\cos (\arcsin x)} + \arcsin x + C\]
    \end{Example}

    \begin{Example}[хотим x вверх]
        \[\int \frac{dx}{x \sqrt{1 + x^2}} = \left[\begin{matrix}
            t = \frac{1}{x}\\
            dt = -\frac{dx}{x^2}\\
            x > 0
        \end{matrix}\right] = - \int \frac{dt}{\sqrt{1 + t^2}} = -\ln(\frac{1}{x} + \sqrt{1 + \frac{1}{x^2}}) + C\]
        \[x \leq 0: \q \ln(-\frac{1}{x} + \sqrt{1 + \frac{1}{x^2}}) + C\]
    \end{Example}

    \begin{Example}[хотим полный квадрат]
        \[\int \frac{dx}{\sqrt{2x^2 - x + 2}} = \frac{1}{\sqrt{2}} \int \frac{dx}{\sqrt{(x-\frac{1}{4})^2 + \frac{15}{18}}} = \ln|x-\frac{1}{4}+\sqrt{2x^2-x+2}|+C\]
    \end{Example}

    \begin{Example}[хотим производную знаменателя в числитель]
        \[\int \frac{x}{\sqrt{5+x-x^2}} dx = \int \frac{-\frac{1}{2}(1-2x)+\frac{1}{2}}{\sqrt{5+x-x^2}} = -\frac{1}{2} \int \frac{1-2x}{\sqrt{5+x-x^2}} + \int \frac{dx}{5+x-x^2} = \]
        \[=-\sqrt{5+x-x^2} + \frac{1}{2} \int \frac{dx}{\sqrt{\frac{21}{4} - (x-\frac{1}{2})^2}} =-\sqrt{5+x-x^2} + \frac{1}{2} \arcsin \frac{x-\frac{1}{2}}{\sqrt{\frac{21}{4}}} + C\]
    \end{Example}

    \begin{Example}[выражаем через себя]
        \[I_n = \int x^n e^{-x} dx = \left[\begin{matrix}
            u = x^n & du = n x^{n-1}\\
            dv = e^{-x} dx & v = -e^{-x}
        \end{matrix}\right] = -x^n e^{-x} + n I_{n-1}\]
        \[I_n = -x^n e^{-x} + n I_{n-1} = -x^n e^{-x} + n (-x^{n-1} e^{-x} + (n-1) I_{n-2}) =\]
        \[= -e^x (x^n + n x^{n-1} + n(n-1) x^{n-2} + ...+ n!)\]
    \end{Example}

    \begin{Example}
        \[\int \cos^n x dx = \left[\begin{matrix}
            u = \cos^{n-1} x & du = (n-1)\sin x \cos^{n-2} x\\
            dv = \cos x dx & v = \sin x
        \end{matrix}\right] = \]
        \[= cos^{n-1} x \sin x + (n-1) \int \cos^{n-2} x (1-\cos^2 x) dx = cos^{n-1} x \sin x + (n-1) (J_{n-2} - J_n)\]
        \[J_1 = \sin x \qq J_2= \frac{\cos x \sin x _ x}{2} = \frac{x}{2} + \frac{1}{4} \sin 2x\]
        \[J_n = \frac{1}{n} (\cos^{n-1} \sin x + \frac{n-1}{n-2} \cos^{n-3} x \sin x + \frac{n-1}{n-2} \frac{n-3}{n-4} \cos^{n-5} x \sin x + ...)\]
    \end{Example}

    \begin{Example}
        \[J_n = \int \sin^n x = ... = \frac{-\cos x \sin^{n-1} x + (n-1)J_{n-2}}{n}\]
    \end{Example}

    \begin{Example}
        \[L_n = \int \ln^n x dx = ... = x \ln^n x - n J_{n-1}\]
        \[L_n = x (\ln^n x - n \ln^{n-1} x + n(n-1) \ln^{n-2} x - ... + (-1)^{n+1} n! \ln x  +(-1)^n n!)\]
    \end{Example}

    \newpage
    \subsection{Интегрирование по частям}
    \begin{Example}[заменяй страшное]
        \[\int x^{\alpha} \ln x dx = \left[\begin{matrix}
            u = \ln x & du = \frac{1}{x} dx\\
            dv = x^{\alpha} dx & v = \frac{x^{\alpha + 1}}{\alpha + 1}
        \end{matrix}\right] = \frac{x^{\alpha+1}\ln x}{\alpha+1} - \frac{x^{\alpha +1}}{(\alpha_1)^2} + C\]
    \end{Example}

    \begin{Example}[замена страшного]
        \[\int \arcsin^2 x dx = \left[\begin{matrix}
            u = \arcsin^2 x & du = \frac{2 \arcsin x}{\sqrt{1-x^2}}\\
            dv = dx & v = x
        \end{matrix}\right] = x \arcsin^2 x - 2 \int \frac{x \arcsin x}{\sqrt{1-x^2}} dx =\]
        \[\left[\begin{matrix}
            \arcsin x = u & du = \frac{1}{\sqrt{1-x^2}}\\
            dv = \frac{x}{\sqrt{1-x^2}} & v = \frac{\sqrt{1-x^2}}{2}
        \end{matrix}\right] = x \arcsin^2 x + 2 \arcsin x \sqrt{1-x^2} - x + C\]
    \end{Example}

    \begin{Example}[повторяем интеграл с новым коэф.]
        \[\int e^x \sin x dx = \left[\begin{matrix}
            u = e^x & du = e^x\\
            dv = \sin x dx & v = -\cos x
        \end{matrix}\right] = -e^x \cos x + \int e^x \cos x dx =\]
        \[= \left[\begin{matrix}
            u = e^x & du = e^x\\
            dv = \cos x dx & v = \sin x
        \end{matrix}\right] = e^x \cos x + e^x \sin x = - \int e^x \sin x dx\]
        \[\Ra \int e^x \sin x dx = \frac{e^x \sin x - e^x \cos x}{2}\]
    \end{Example}

    \begin{Example}[сводим к прошлому]
        \[\int e^x \cos x dx = \left[\begin{matrix}
            u = x & du = dx\\
            dv = \cos x dx & v = \sin x
        \end{matrix}\right] = x - \int \sin x dx =x \sin x + \cos x dx\]
    \end{Example}

    \begin{Example}[можно обобщать метод]
        \[e^{ax} \cos (bx) dx = \left[\begin{matrix}
            u = e^{ax} & du = a e^{ax}\\
            dv = \cos (bx) dx & v = \frac{1}{b} \sin (bx)
        \end{matrix}\right] = e^{ax} \frac{1}{b} \sin (bx) - \frac{a}{b} \int e^{ax} \sin x dx =\]
        \[= \left[\begin{matrix}
            u = e^{ax} & du = a e^{ax}\\
            dv = \sin (bx) dx & v = -\frac{1}{b} \cos (bx)
        \end{matrix}\right] = ... = \frac{e^{ax}}{a^2 + b^2} (b\sin(bx) + a\cos{bx}) + C\]
    \end{Example}

    \begin{Example}[замечаем производные]
        \[\frac{dx}{x \ln x \ln (\ln x)} = \int \frac{d(\ln x)}{\ln x \ln(\ln x)} = \frac{d(\ln(\ln x))}{\ln(\ln x)} = \ln|\ln(\ln x)| + C\]
    \end{Example}

    \begin{Example}
        \[\int \frac{\sin x \cos x}{\sqrt{a^2 \sin x + b^2 \cos x}}\]
        \begin{enumerate}
          \item $|a| = |b| \neq 0$
          \[= \frac{1}{|a|} \int \sin x \cos x = \frac{1}{2|a|} \sin^2 x + C\]
          \item $|a| \neq b$
          \[= \frac{1}{2} \int \frac{d(\sin^2 x)}{\sqrt{(a^2 - b^2) \sin x + b^2}} = \frac{\sqrt{(a^2 - b^2) \sin^2 x + b^2}}{a^2 - b^2} + C\]
        \end{enumerate}
    \end{Example}

    \newpage
    \subsection{Тригонометрические подстановки $a \geq 0$}

    \begin{Example}[видим ограничение - пытаемся использовать]
        \[\us{\qq |x|<1}{\int \frac{dx}{(\sqrt{1 - x^2})^3}} = \left[\begin{matrix}
            x = \sin t & -\frac{\pi}{2} < t < \frac{\pi}{2}\\
            dx = \cos t\ dt & (1-x^2)^{\frac{3}{2}} = \cos^3 t
        \end{matrix}\right] = \int \frac{dt}{\cos^3 t} = ...\]
    \end{Example}

    \begin{Example}[вспоминаем триг. функции]
        \[\us{\qq x>\sqrt{2}}{\int \frac{x^2 dx}{(\sqrt{x^2 - 2})}} = \left[\begin{matrix}
            x = \frac{\sqrt{2}}{\cos t} & 0 < t < \frac{\pi}{2}\\
            dx = \sqrt{2} \sec t \tg t\ dt & \frac{x^2}{\sqrt{x-2}} = \frac{2 \sec^2 t}{\sqrt{2} \tg t}
        \end{matrix}\right] = 2 \int \sec^3 t\ dt = ...\]
    \end{Example}

    \begin{Example}[придумываем, как связать с тригонометрией]
        \[\us{-a \leq x \leq a}{\int \sqrt{a^2 - x^2} dx} = \left[\begin{matrix}
            x = a\sin t & -\frac{\pi}{2} < t < \frac{\pi}{2}\\
            dx = a\cos t dt & \sqrt{a^2 - x^2} = a \cos x
        \end{matrix}\right] = a^2 \int \cos^2 t = ...\]
    \end{Example}

    \begin{Example}[фокус]
        \[\us{-\infty < x < +\infty}{\int \frac{dx}{\sqrt{x^2 + a^2}}} = \left[\begin{matrix}
            x = a\arctg t & -\frac{\pi}{2} < t < \frac{\pi}{2}\\
            dx = a\sec^2 t dt & (x^2 + a^2)^3 = a^3 \sec^6 x
        \end{matrix}\right] = \int \frac{a \sec^2 t}{a^3 \sec^6 t} =\]
        \[= \frac{1}{\sqrt{a}} \int \cos t\ dt = \frac{x}{\sqrt{a} \sqrt{a^2 + x^2}} + C\]
    \end{Example}

    \begin{Example}[решаем школьные задачи]
        \[\us{0 \leq x < 2a}{\int x \sqrt{\frac{x}{2a-x}}} = \left[\begin{matrix}
            x = 2a \sin^2 t & \\
            dx = 4a \sin t \cos t\ dt & x \sqrt{\frac{x}{2a - x}} = \frac{2a\sin^3 t}{\cos t}
        \end{matrix}\right] = 8a^2 \int \sin^4 t\ dt\]
    \end{Example}

    \begin{Example}
        \[\us{a < x < b}{\int \frac{dx}{\sqrt{(x-a)(b-x)}}} =\]
        \[= \left[\begin{matrix}
            x = (b-a)\sin^2 t + a & 0 < t < \frac{\pi}{2}\\
            dx = 2(b-a)\sin t \cos t\ dt & \sqrt{(x-a)(x-b)} = (b-a) \sin t \cos t
        \end{matrix}\right]\]
    \end{Example}

    \newpage
    \subsection{Полезные простые интегралы}

    \begin{Example}
        \[\int \frac{dx}{a+bx} \os{ab > 0}{=} \sign a \frac{1}{\sqrt{|b|}} \int \frac{d(\sqrt{|b|}x)}{(\sqrt{|a|})^2 + (\sqrt{|b|} x)^2} + C\]
    \end{Example}

    \begin{Example}
        \[\int \frac{dx}{\sqrt{a+bx^2}} \os{b>0}{=} \frac{1}{\sqrt{b}} \ln|x\sqrt{b}+\sqrt{a+b x^2}| + C\]
    \end{Example}

    \begin{remark}
        Если $y = ax^2 + bx + c,\q (a \neq 0)$, то:
        \[\int \frac{dx}{\sqrt{ax^2 + bx + c}} = \frac{1}{sqrt{a}} \ln \abs{\frac{y'}{2} + \sqrt{ay}} + C,\q a > 0\]
        \[\int \frac{dx}{\sqrt{ax^2 + bx + c}} = \frac{1}{\sqrt{-a}} \arcsin \frac{-y'}{\sqrt{b^2 - 4ac}},\q a < 0\]
    \end{remark}

    \newpage
    \subsection{Интегрирование рациональных функций}
    \begin{Utv}
        \[R(x) = \frac{P(x)}{Q(x)},\q P,Q \in \R[x] \text{ - рац. ф-ия}\]
        Если $\deg P \geq \deg Q \RA \dfrac{P}{Q} = \dfrac{P_1}{Q} + P_2 \q \deg P_1 < \deg Q$
    \end{Utv}
    Пусть $\deg P < \deg Q$
    \begin{definition}[простейшие над $\R$]
        \begin{enumerate}
          \item $\dfrac{a}{(x - c)^n},\q a,c \in \R,\q n \in \N$
          \item $\dfrac{bx + c}{(x^2 + px + q)^n},\q b,c,p,q \in \R,\q p^2 - 4q < 0 \q n \in \N$
        \end{enumerate}
    \end{definition}

    \begin{utv}[основаная теорема алгебры]
        Если $P(x)$ - мн-н, тогда он представим в виде произв. мн-ов 1-ой и 2-ой степени (неразрешимых в $\R$)
        \[P(x) = A \prod_{k=1}^n (x - a_k) \cdot \prod_{j=1}^l (x^2 + p_j x + q_j), \q \deg P = n + l \cdot 2,\q p_j^2 - 4q_j < 0\]
    \end{utv}

    \begin{Utv}[метод неопр. коэф.]
        \[\frac{P(x)}{Q(x)} = R(x) + \sum_{k=1}^n \sum_{j=1}^{m_k} \frac{A_{n_j}}{(x - a_k)^j} + \sum_{k=1}^l \sum_{j=1}^{s_k} \frac{M_{k_j} x + N_{k_j}}{(x^2 + p_k x + q_k)^j},\]
        где $m_k$ - кр-ть корня мн-на, $s_k$ - кр-ть корня мн-на $(x^2 - p_k x + q_k)$
    \end{Utv}

    \begin{Utv}[Лайфхак, $Q(c) \neq 0$]
        \[\letus f(x) = \frac{P(x)}{(x - c)^m Q(x)} = \frac{a_m}{(x-c)^m} + r(x) \ |\ (x-c)^m\]
        \[\frac{P(x)}{Q(x)} = a_m + (x - c)^m r(x) \RA a_m = \frac{P(c)}{Q(c)}\]
        \[f(x) - \frac{a_m}{(x - c)^m} = \frac{P_1(x)}{(x - c) Q_1(x)} \text{ и т.д.}\]
        \[f(x) = \frac{a_m}{(x - c)^m} + ... + \frac{a_1}{(x - c)} + r(x) \ |\ (x - c)^m\]
        \[g(x) = \frac{P(x)}{Q(x)} = a_m + a_{m-1} (x - c) + ... + a_1 (x - c)^{m-1} + r(x) (x-c)^m\]
        (как ф-ла Ткйлора $k=0,1,...,m-1$)\\
        \[a_{m-k} = \frac{g^{(k)}(c)}{k!} \q \text{(секрет)}\]
    \end{Utv}

    \begin{Utv}
        \[\int \frac{dx}{(x - c)^m} = \begin{cases}
            \ln|x - c| + C, & n = 1\\
            -\frac{1}{(x-1)(x-c)^{n-1}} + C, & n > 1
        \end{cases}\]
    \end{Utv}

    \begin{Utv}
        \[\int \frac{Mx + N}{(x^2 + px + q)^n}dx = \frac{M}{2} \ub{I_n}{\int \frac{2(x + \frac{p}{2})}{(x + \frac{p}{2})^2 + q - (\frac{p^2}{4})^n}} + (N - \frac{M_p}{2}) \ub{J_n}{\int \frac{dx}{(x^2 + px + q)^n}}\]
        \[I_n = \frac{M}{2} \int \frac{d(x^2 + px + q)}{(x^2 + px + q)^n} = \begin{cases}
            \ln(x^2 + px + q), & n = 1\\
            -\frac{1}{(n-1)(x^2 + px + q)^{n-1}}, & n>1
        \end{cases} + C\]
        \[J_1 = \frac{1}{a} \arctg \frac{x}{a} + C,\q n=1\]
        \[J_n = \int \frac{dx}{(x^2 + a^2)^n} = \frac{1}{2(n-1)a^2} \Br{\frac{x}{(x^2 + a^2)^{n-1}} + (2n - 3) J_{n-1}},\q n > 1\]
    \end{Utv}

    \begin{Example}
        \[\int \frac{x^3 + 5x + 7}{(x + 2)(x + 1)x(x - 1)(x - 2)} dx\]
        \[a_1 = \frac{-8 - 10 + 7}{(-1)(-2)(-3)(-4)} = -\frac{11}{24} \q a_2 = -\frac{1}{6}\q a_3 = \frac{7}{4}\q a_4 = \frac{13}{321}\q a_5 = \frac{25}{4}\]
        \[= - \frac{11}{24} \ln|x + 2| - \frac{1}{6} \ln|x + 1| + \frac{7}{4} \ln|x| + \frac{13}{321}\ln|x - 1| + \frac{25}{4} \ln|x - 2| + C\]
    \end{Example}

    \begin{Utv}
        \[\deg P = n \text{. Будто ф-ла Тейлора в т. $x = a$}\]
        \[\int \frac{P(x)}{(x - a)^{n+1}} dx = \sum_{k=0}^n \frac{P^{(k)}(a)}{k!} \int \frac{1}{(x-a)^{n+1-k}} dx =\]
        \[= - \sum_{k=0}^{n-1} \frac{P^{(k)}(a)}{k!} \frac{1}{(x-a)^{n-k} (n-k)} + \frac{P^{(n)}(a)}{n!} \ln|x - a| + C\]
    \end{Utv}

    \newpage
    \subsection{Метод Остроградского}
    \begin{Alg}
        \[Q = A \prod_{k=1}^n (x-q_k)^{m_k} \prod_{k=1}^l (x^2 + p_k x + q_k)^{s_k}\]
        \[Q_1 = A \prod_{k=1}^n (x - q_k)^{m_k - 1} \prod_{k=1}^l (x^2 + p_k x + q_k)^{s_k - 1}\]
        \[W = \prod_{k=1}^n (x - q_k) \prod_{k=1}^l (x^2 + p_k x + q_k)\]
        \[Q = Q_1 W\]
        \[\int \frac{P(x)}{Q(x)} = \us{\text{алг. часть}}{\frac{P_1(x)}{Q_1(x)}} + \us{\text{трансц. часть}}{\int \frac{R(x)}{W(x)}} \q (\deg P < \deg Q)\]
        $P_1$ $\os{(R)}{-}$ неопр. коэф. $\deg P_1 < \deg Q_1$
        \[\frac{P}{Q} = \frac{P_1'}{Q_1} - \frac{P_1 Q_1'}{Q_1^2} + \frac{R}{W} \RA P = P_1' W - \frac{P_1 Q_1' W}{Q_1} + R Q_1\]
    \end{Alg}

    \begin{Example}
        \[\int \frac{dx}{(1 - x^3)^2} = \frac{a_0 + a_1 x + a_2 x^2}{1 - x^3} + \int \frac{b_0 + b_1 x + b_2 x^2}{1 - x^3} dx\]
        \[1 = (a_1 + 2a_1 x)(1 - x^3) - (a_0 + a_1 x + a_2 x) (-3 x^2) + (b_0 + b_1 x + b_2 x^2)(1 - x^3)\]
        \[\begin{matrix}
            x^3: & 0 = -a_1 + 3a_1 - b_0\\
            x^2: & 0 = 3a_0 - a_2 + b_2\\
            x: & 0 = 2a_2 + b_1\\
            1: & 1 = a_1 + b_0\\
            x^4 & -2a_2 + 3a_2 - b_1\\
            x^5 & 0 = - b_2\\
        \end{matrix} \RA \begin{matrix}
            a_0 = - \frac{2}{9}\\
            a_1 = \frac{1}{3}\\
            a_2 = -\frac{2}{3}\\
            b_0 = \frac{2}{3}\\
            b_1 = -\frac{2}{3}\\
            b_2 = 0
        \end{matrix}\]
    \end{Example}

    \begin{utv}
        При каких условиях рац. ф-ия?
        \[\int \frac{\alpha x^2 + 2 \beta x + \gamma}{(a x^2 + 2 bx + c)^2} dx\]
        \begin{enumerate}
            \item $a \neq 0,\q b^2 - ac = 0 \RA ax^2 + bx + c = a(x - x_0)^2$
            \[\frac{\alpha x^2 + 2 \beta x + \gamma}{(ax^2 + 2bx + c)^2} = \frac{\alpha(x - x_0)^2 + 2 \alpha x_0 (x - x_0) + \alpha x_0^2 + 2 \beta (x - x_0) + 2 \beta x_0 + \gamma}{a^2 (x - x_0)^4} =\]
            \[= \frac{\alpha}{a^2 (x - x_0)^2} + \frac{2 \alpha x_0}{a^2 (x - x_0)^3} + \frac{\alpha x_0^2 + 2 \beta x_0 + \gamma}{a^2 (x - x_0)^4}\]
            \item $a \neq 0,\q b^2 - ac \neq 0$
            \[\frac{\alpha x^2 + 2 \beta x + \gamma}{(ax^2 + 2bx + c)^2} = \Br{\frac{Ax + B}{a x^2 + 2 \beta x + c}} + \frac{Cx + D}{ax^2 + 2bx + c}\]
            \[\Ra \alpha x^2 + 2 \beta x + \gamma = A(ax^2 + 2 \beta x + c) - (2ax + 2b)(Ax + B) + (Cx + D)(ax^2 + 2bx + c)\]
            \[D = \frac{2b\beta - a\gamma - c\alpha}{2(b^2 - ac)} \qq a\gamma + c\alpha = 2b\beta \q D = 0\]
            \item $a = 0,\q b \neq 0$
            \[\frac{\alpha x^2 + 2 \beta x + \gamma}{(ax^2 + 2bx + c)^2} = \frac{\alpha(x + \frac{c}{2b})^2 - \frac{ac}{b}(x + \frac{c}{2b}) + \frac{ac^2}{4b^2} + 2\beta(x + \frac{c}{2b}) - \frac{\beta c}{b} + \gamma}{4b^2 (x + \frac{c}{2b})^2} =\]
            \[= \frac{\alpha}{4b^2} + \frac{2\beta - \frac{\alpha c}{b}}{4b^2 (x + \frac{c}{2b})} + \frac{\frac{\alpha c^2}{4b^2} - \frac{\beta c}{b} + \gamma}{4b^2 (x + \frac{c}{2b})^2}\]
            \[2\beta - \frac{ac}{b} = 0\qq ac = 2b\beta \qq \alpha \gamma + c \alpha = 2 b \beta\]
            \item $a = b = 0,\q c \neq 0$
            \[\frac{\alpha x^2 + 2 \beta x + \gamma}{(ax^2 + 2bx + c)^2} = \frac{\alpha x^2 + 2 \beta x + \gamma}{c^2} \qq \alpha \gamma + c \alpha = 2 b \beta\]
        \end{enumerate}
    \end{utv}

    \begin{Example}
        \[\int \frac{x^{2n - 1}}{x^n _ 1} dx \os{n \neq 0}{=} \left[\begin{matrix}
            t = x^n\\
            dt = n x^{n-1} dx
        \end{matrix}\right] = \frac{1}{n} \int \frac{t}{t + 1} dt = \frac{1}{n} \int (1 - \frac{1}{t + 1}) dt = \]
        \[ = \frac{1}{n} (x^n - \ln|x^n + 1|) + C\]
        \[\os{n = 0}{=} \int \frac{x^{-1}}{1 + 1} = \frac{1}{2} \ln|x| + C\]
    \end{Example}

    \begin{Example}
        \[I = \int \frac{dx}{(x + a)^m (x + b)^k} \os{a \neq b}{=}
        \frac{1}{(b - a)^{m + n - 1}} \int \frac{(1 - t)^{m + n - 2}}{t^m} dt\]
        \[t = \frac{x + a}{a + b} \qq 1 - t
        = \frac{b - a}{x + b} \qq x + b = \frac{b - a}{1 - t}\]
        \[dt = \frac{b - a}{(x + b)^2} dx
        = \frac{(1 - t)^2}{b - a} dx
        \RA dx = \frac{b - a}{(1 - t)^2} dt\]
        \[x + a = t (x + b) = \frac{t(b - a)}{1 - t}\]
        \[= \frac{1}{(b - a)^{m + n - 1}}
        \Br{\sum_{k = 0}^{m + n - 2} C_{m + n - 2}^k t^k (-1)^{m + n - 2 - k}}{t^m} dt =\]
        \[= \frac{1}{(b-a)^{m + n - 1}}
        \Br{\sum_{\os{k = 0}{k \neq m - 1}}^{m + n - 2}
        \int \frac{C^k_{m + n - 2} t^k (-1)^{m + n - 2 - k}}{t^k} dt
        + (-1)^{m-1} C_{m+n-2}^{m-1} \ln|t|} =\]
        \[\sum_{\os{k = 0}{k \neq m - 1}}^{m + n - 2}
        \int \frac{C^k_{m + n - 2} t^k (-1)^{m + n - 2 - k}}{t^k} dt =\]
        \[= \sum_{k = 0}^{m-2} (-1)^{m + n - k - 1} \frac{C_{m + n - 2}^k}{t^{m - k - 1}} +
        \sum_{k=m}^{m + n - 2} C_{m + n - 2}^k (-1)^{m + n -2 - k} \int t^{k-n} dt = \]
        \[= \sum_{k=0}^{m-2} (-1)^{m + n - k - 1} \frac{C_{m - k - 2}^k}{t^{m - k - 1}} +
        \sum_{k = m}^{m + n - 2} \frac{t^{k - m + 1}}{k - m + 2} C_{m + n - 2}^k (-1)^{m + n - 2} + C\]
        \[\Ra I =
        \frac{1}{(b - a)^{m + n - 1}}
        \Big(\sum_{k=0}^{m-2} (-1)^{m + n - k - 1} \frac{C_{m - k - 2}^k}{t^{m - k - 1}} +\]
        \[+ \sum_{k = m}^{m + n - 2} \frac{t^{k - m + 1}}{k - m + 2} C_{m + n - 2}^k (-1)^{m + n - 2}
        + (-1)^{m-1} C_{m + n - 2}^{m-1} \ln|t|\Big) + C\]
        \[I \os{a = b}{=} \frac{1}{1 - m - n} (x + a)^{1 - m - n} + C\]
    \end{Example}

    \newpage
    \subsection{Интегрирование тригонометрических функций}
    \begin{alg}
        Всегда срабатывает:
        \[\int R(\sin x,\ \cos x) dx = \int R \Br{\frac{2t}{1 + t^2},\ \frac{1 - t^2}{1 + t^2}} \frac{2 dt}{1 + t^2},\]
        где $t = \tg \frac{x}{2}$, т.к. $x = \arctg t$\\
        Бывает срабатывает и рекомендуется:
        \begin{enumerate}
            \item $R(-\sin x,\ -\cos x) = R(\sin x,\ \cos x) \q t = \tg x \q -\frac{\pi}{2} < x < \frac{\pi}{2}$
            \item $R(-\sin x,\ \cos x) = - R(\sin x,\ \cos x) \q t = \cos x \q 0 < x < \pi$
            \item $R(\sin x,\ -\cos x) = R(\sin x,\ \cos x) \q t = \sin x \q -\frac{\pi}{2} < x < \frac{\pi}{2}$
        \end{enumerate}
        Аналогично имеет место:
        \[\int R(\sh x,\ \ch x) dx = \int R\Br{\frac{2t}{1 - t^2},\ \frac{1 + t^2}{1 - t^2}} \frac{2 dt}{1 - t^2},\q t = \th \frac{x}{2}\]
    \end{alg}

    \begin{Example}
        \[I_{\alpha,\beta} = \int \sin^{\alpha} x \cos^{\beta} x\ dx = \left[\begin{matrix}
            u = \cos^{\beta - 1} x & du = (\beta - 1) \cos^{\beta - 2} x \sin x\ dx\\
            dv = \sin^{\alpha} x \cos x\ dx & v = \frac{\sin^{\alpha + 1}x}{\alpha + 1}
        \end{matrix}\right] =\]
        \[= \frac{\sin^{\alpha + 1}x \cos^{\beta - 1} x}{\alpha + 1} \frac{\beta - 1}{\alpha + 1}(I_{\alpha,\beta -2} - I_{\alpha,\beta})\]
        \[I_{\alpha,\beta} = ... = - \frac{\cos^{\beta + 1}x \sin^{\alpha - 1} x}{\beta + 1} + \frac{\alpha - 1}{\beta + 1}(I_{\alpha - 2,\beta} - I_{\alpha,\beta})\]
    \end{Example}

    \begin{Example}[хитрые приемчики из школы]
        \[I_n = \int \frac{dx}{(a + b\cos x)^n} = \us{G_1}{\int \frac{\sin^2 x}{(a + b\cos x)^n}} + \us{G_2}{\int \frac{\cos^2 x}{(a + b\cos x)^n}}\]
        \[G_1 = \left[\begin{matrix}
            u = \sin x & du = \cos x\ dx\\
            dv = \frac{\sin x}{(a + b\cos x)^n}dx & v = -\frac{1}{(n-1)b} \frac{1}{(a + b\cos x)^{n-1}}dx
        \end{matrix}\right] =\]
        \[= \frac{-\sin x}{(n-1) b (a + b\cos x)^{n-1}} + \frac{1}{b(n-1)} \int \frac{\cos x}{(a + b\cos x)^{n-1}} dx =\]
        /* $\cos x = \frac{1}{b}(b \cos x + a) - \frac{a}{b}$ */
        \[= \frac{-\sin x}{(n-1) b (a + b\cos x)^{n-1}} + \frac{1}{b^2 (n-1)} I_{n-2} - \frac{a}{b^2(n-1)} I_{n-1}\]
        /* $\cos^2 x = \frac{1}{b^2}(a + b\cos x)^2 - \frac{a^2}{b^2} - \frac{2a}{b} \cos x =$\\
        $\frac{1}{b^2} (a + b \cos x)^2 - \frac{2a}{b}\Br{\frac{1}{b}(a + b\cos x) - \frac{a}{b}} - \frac{a^2}{b^2}$ */
        \[G_2 = \int \frac{\cos^2 x}{(a + b \cos x)^n} = \frac{1}{b^2} I_{n-2} - \frac{2a}{b^2} I_{n-1} + \frac{a^2}{b^2} I_n\]
        \[\Ra I_n = \frac{1}{(b^2 - a^2)} \Br{\frac{b \sin x}{(n-1)(a + b\cos x)^{n-1}} + \frac{a(2n-3)}{n-1} I_{n-1} + \frac{n-2}{n-1} I_{n+2}} + C\]
    \end{Example}

    \begin{Consequence}
        \[K_n = \int \frac{dx}{\cos^n x} = \frac{\sin x}{(n-1) \cos^{n-1} x} + \frac{n-2}{n-1} K_{n-2}\]
        \[S_n = \int \frac{dx}{\sin^n x} = S_{n-2} - \frac{\cos x}{(n-1)\sin^{n-1}x} - \frac{1}{n-1} S_{n-2}\]
    \end{Consequence}

    \begin{Example}[больше! больше!]
        \[\int \frac{dx}{\sin(x + a) \sin (x + b)} \os{a \neq b}{=} \frac{1}{\sin(a-b)} \Br{ \int \frac{\cos(x + b)}{\sin (x + b)} - \int \frac{\cos(x + a)}{\sin x + a}} = \]
        $\sin(a - b) = \sin((a + x) - (x - b)) = \sin(a + x) \cos(x - b) - \sin(x - b) \cos(a + x)$
        \[= \frac{1}{\sin (a - b)} \ln \abs{\frac{\sin(x + b)}{\sin(x + a)}} + C\]
    \end{Example}

    \begin{Example}[привет, школа!]
        \[\int \frac{dx}{a \sin x + b \cos x}  = \frac{1}{\sqrt{a^2 + b^2}} \int \frac{dx}{\sin(x + \varphi)} = \frac{\ln|\tg (\frac{x + \varphi}{2})|}{\sqrt{a^2 + b^2}} + C\]
        \[\cos \varphi = \frac{a}{\sqrt{a^2 + b^2}} \qq \sin y = \frac{b}{\sqrt{a^2 + b^2}}\]
    \end{Example}

    \begin{Example}
        \[I_n = \int \frac{dx}{(a \sin x + b \cos x)^n} = \sqrt{a^2 + b^2} \us{\text{умеем}}{\int \frac{dx}{\sin^n (x + \varphi)}}\]
        \[... \RA I_n = \frac{\frac{b}{a^2 + b^2} (\sin x - \cos x)}{(a\sin x + b\cos x)^{n-1}} - (n-2) I_n \frac{n-2}{a^2 + b^2} I_{n-2}\]
    \end{Example}

    \begin{Example}[хитрый способ]
        \[\int \frac{a_1 \sin x + b_1 \cos x}{a\sin x + b\cos x}\]
        Что делать? Можно предположить, что интеграл имеет определенный вид и найти коэффициенты
        \[\int \frac{a_1 \sin x + b_1 \cos x}{a\sin x + b\cos x} = Ax + B \ln|a\sin x + b\cos x| + C\]
        \[a_1 \sin x + b_1 \cos x = A(a\sin x + b\cos x) + B \us{= a\cos x - b\sin x}{d(a\sin x + b\cos x)}\]
        \[A = \frac{a a_1 + b b_1}{a^2 + b^2} \qq B = \frac{a b_1 - a_1 b}{a^2 + b^2} \qq (a^2 + b^2 \neq 0)\]
        Аналогично берутся:
        \[\int \frac{a_1 \sin x + b_1 \cos x}{(a \sin x + b \cos x)^2} dx \text{ и } \int \frac{a_1 \sin x + b_1 \cos x + c_1}{a \sin x + b \cos x + c} dx\]
    \end{Example}

    \begin{Example}[любимый приемчик]
        \[\int \frac{\cos^{n-1} \frac{x + a}{2}}{\sin^{n+1} \frac{x - a}{2}} = \left[\begin{matrix}
            t = \frac{\cos \frac{x + a}{2}}{\sin \frac{x - a}{2}}\\
            dt = \frac{-\frac{1}{2} \cos a}{\sin^2 \frac{x-a}{2}} dx
        \end{matrix}\right] = - \frac{2}{\cos a} \int t^{n-1} dt = - \frac{2}{n\cos a} \Br{\frac{\cos \frac{x+a}{2}}{\sin \frac{x-a}{2}}}^n + C\]
    \end{Example}

    \begin{remark}
        Теперь понятно, почему работало подобное:
        \begin{enumerate}
          \item $R(t,\ \sqrt{1 + t^2}) \q t = \tg u\q - \frac{\pi}{2} < u < \frac{\pi}{2}$
          \item $R(t,\ \sqrt{t^2 - 1}) \q t = \us{\Br{\frac{1}{\cos u}}}{\frac{1}{\sin u}} \q 0 < u < \frac{\pi}{2}$
          \item $R(t,\ \sqrt{1 - t^2}) \q t = \us{\cos u}{\sin u} \q \us{0 < u < \pi}{-\frac{\pi}{2} < u < \frac{\pi}{2}}$
        \end{enumerate}
    \end{remark}

    \newpage
    \subsection{Дифференциальный бином}
    \begin{Task}
        \[I = \int x^m (a + b x^n)^p dx \qq a,b \in \R,\qq m,n,p \in \Q\]
        При каких $m,n,p\q \forall a,b$ $I$ - эл-ая ф-ия?
    \end{Task}

    \begin{utv}
        \begin{enumerate}
          \item $p \in \Z,\q k$ - общий знам. $m$ и $n$
          \[t = x^{\frac{1}{k}} \RA I = \int R(t) dt\]
          \item $\frac{m+1}{n} \in \Z \q p = \frac{r}{l},\q l \in \N$
          \[t = (a + bx^n)^{\frac{1}{l}}\]
          \item $\frac{m + 1}{n} + p \in \Z$
          \[t = \Br{\frac{a}{x^n} + b}^{\frac{1}{l}},\q p = \frac{r}{l},\q l \in \N\]
        \end{enumerate}
    \end{utv}

    \begin{Example}
        \[\int \frac{dx}{x \sqrt[3]{1 + x^5}} = \int x^{-1} (1 + x^5)^{-\frac{1}{3}} = \left[\begin{matrix}
            \text{\RNumb{2} случ.} & t = \sqrt[3]{1 + x^5}\\
            x = \sqrt[5]{t^3 - 1} & dx = \frac{3}{5} t^2 (t^3 - 1)^{\frac{4}{5}}
        \end{matrix}\right] = \]
        \[= \frac{3}{5} \int \frac{t}{t^3 - 1} = ...\]
    \end{Example}

    \newpage
    \subsection{Подстановки Эйлера}
    \begin{Utv}[\RNumb{1}]
        \[y = ax + bx + c,\text{ рассмотрим } R(x, \sqrt{y})\]
        \[\int \frac{P_n(x)}{\sqrt{y}} dx = P_{n-1}(x) \sqrt{y} + \lambda \int \frac{dx}{\sqrt{y}}\]
        \[\Ra \frac{P_n}{\sqrt{y}} = P_{n-1}'(x) \sqrt{y} + P_{n-1} \frac{(2ax + b)}{2 \sqrt{y}} + \frac{\lambda}{\sqrt{y}}\]
    \end{Utv}

    \begin{Utv}[подстановки эйлера]
        \[\sqrt{ax^2 + bx + c} = \pm \sqrt{a} x + z,\q a > 0\]
        \[\sqrt{ax^2 + bx + c} = xz + \sqrt{c},\q c > 0\]
        \[\sqrt{a(x - x_1)(x - x_2)} = z(x = x_i)\]
        \[\int \frac{P_n(x)}{Q_n(x)\sqrt{y}}, \text{ можно разложить сперва }\frac{P(x)}{Q(x)} \text{ на простые дроби}\]
        \[\int R(x, \Br{\frac{\alpha x + \beta}{\gamma x + \delta}}^{r_1},...,\Br{\frac{\alpha x + \beta}{\gamma x + \delta}}^{r_n}) dx \qq r_i = \frac{p_i}{q_i},\q p_i,q_i \in \Z\]
        Если $m = \text{НОК}(q_1,...,q_n)$, то $t = \Br{\frac{\alpha x + \beta}{\gamma x + \delta}}^m$ сводится к $\int R'(t) dt$
    \end{Utv}

    \newpage
    \subsection{Ещё про иррациональные интегралы}
    \begin{Utv}[\RNumb{2}]
        \[\int \frac{dx}{(x-x_0)^n \sqrt{y}} = \left[\begin{matrix}
            t = \frac{1}{x - x_0} & dt = \frac{-dx}{(x - x_0)^2}\\
            x = x_0 + \frac{1}{t} & dx = - \frac{dt}{t^2}
        \end{matrix}\right] = - \int \frac{t^{n-2} dt}{\sqrt{\frac{c_1}{t^2} + \frac{b_1}{t} + a_1}} = - \sign t \us{\text{\RNumb{1} тип}}{\int \frac{t^{n-1} dt}{\sqrt{a_1 t^2 + b_1 t + c_1}}}\]
    \end{Utv}

    \begin{remark}
        Ещё способ $R(x,\ \sqrt{a x^2 + bx + c})$
        \[\frac{R_1(x)}{\sqrt{y}} + R_2(x) \qq (\sqrt{ax^2 + bx + c})^m = (ax^2 + bx + c)^{\left[\frac{m}{2} + 1\right]} \sqrt{y}^{m(\mod 2) - 1}\]
        \[\frac{P_1 + P_2 \sqrt{y}}{P_3 + P_4 \sqrt{y}} \os{\text{сопряж.}}{=} \frac{(P_1 + P_2 \sqrt{y})(P_3 - P_4 \sqrt{y})}{P_3^2 - P_4^2 (\sqrt{y})^2} \ra \w{P_1} + P_2 \sqrt{y} + \frac{P_3}{\sqrt{y}}\]
    \end{remark}

    \begin{Utv}[\RNumb{3}]
        \[\int \frac{x\ dx}{(x^2 +q)^m \sqrt{ax^2 + c}} \q q > 0\]
        \[t = \sqrt{ax^2 + c} \qq dt = \frac{ax}{\sqrt{ax^2 + c}} dx\]
    \end{Utv}

    \begin{Utv}[\RNumb{4}]
        \[\int \frac{x\ dx}{(x^2 +q)^m \sqrt{ax^2 + c}} \q q \os{?}{>} 0\]
        \[t = (\sqrt{ax^2 +c})' = \frac{ax}{\sqrt{ax^2 + c}} \text{ - замена Абеля}\]
    \end{Utv}

    \begin{Utv}[\RNumb{5}]
        \[\int \frac{x + d}{(x^2 + px + q)^{\frac{2m + 1}{2}}} \q x+ d = \frac{1}{2}(2x + p) + (d - \frac{p}{2})\]
        \[t = (\sqrt{x^2 + px + q})'\]
    \end{Utv}

    \begin{Utv}
        \[\int \frac{x + d}{(x^2 + px + q)^m \sqrt{ax^2 + bx + c}}dx\]
        Если $p = \frac{b}{a}$, то $t = x + \frac{p}{2} \ \ra\ \RNumb{2}, \RNumb{4}$\\
        Если $p \neq \frac{b}{a}$, то $x = \frac{\alpha t + \beta}{t + 1}$ найти $\alpha$ и $\beta$ так, чтобы в обоих мн-аъ коэф. $t$ был равен 0$\ \ra\ \RNumb{3},\RNumb{4}$
    \end{Utv}

    \begin{Remark}[*]
        \[\int R(x,\ \sqrt{ax + b},\ \sqrt{cx + d}) dx =
        \left[\begin{matrix}
            t = \sqrt{ax + b} & x = \frac{t^2 - b}{a}\\
            dx = \frac{2t}{a} dt & \sqrt{cx + d} = \sqrt{\frac{c}{a} t^2 + d - \frac{bc}{a}}
        \end{matrix}\right] =\]
        \[=\int R(\frac{t^2 - b}{a},\ t,\ \sqrt{\frac{a}{c}t^2 + d - \frac{bc}{a}} \frac{2t}{a})dt = \int R^*(t,\ \sqrt{...})\]
    \end{Remark}

    \begin{Example}
        \[J_n = \int \frac{x^n}{\sqrt{ax^2 + bx + c}} = \frac{1}{2a} \int \frac{x^{n-1}(2ax + b)}{\sqrt{ax^2 + bx + c}}dx - \frac{b}{2a} J_{n-1} =\]
        /* $x = \frac{1}{2a}(2ax + b) - \frac{b}{2a}$ */
        \[= \left[\begin{matrix}
            u = x^{n-1} & v = 2 \sqrt{ax^2 + bx + c}\\
            du = (n-1) x^{n-2} & dv = \frac{2ac + b}{\sqrt{ax^2 + bx + c}}dx
        \end{matrix}\right] =\]
        \[= \frac{1}{2a} (2 x^{n-1} \sqrt{y} - 2(n-1) \int x^{n-2} \sqrt{y} dx) - \frac{b}{2a} J_n =\]
        \[= \frac{1}{2a} (2x^{n-1}\sqrt{y} - 2(n-1) \us{= a J_n + b J_{n-1} + c J_{n-2}}{\int \frac{x^{n-2} y}{\sqrt{y}} dx}) - \frac{b}{2a} J_n\]
        \[\Ra J_n = \frac{1}{na} \Br{x^{n-1} \sqrt{y} - b (2n - 1) J_{n-1} - c (n-1) J_{n-2}}\]
    \end{Example}

    \newpage
    \subsection{Трансцендентные функции и прикольности}
    \begin{Upr}
        \[\frac{1}{1 + x^{2n}} = \frac{1}{b} \sum_{k=1}^n \frac{1 - x \cos \frac{2k-1}{2n}\pi}{x^2 - 2 x \cos \frac{2k-1}{2n}\pi + 1}\]
    \end{Upr}

    \begin{Upr}
        \[\int \Br{\frac{\sin \frac{x - a}{2}}{\sin \frac{x + a}{2}}}^n dx = 2 _{n-1} \cos a - I_{n -2} + \frac{2\sin a}{n-1} \Br{\frac{\sin \frac{x - a}{2}}{\sin \frac{x + a}{2}}}^{n-1}\]
    \end{Upr}

    \begin{Upr}
        \[P_n(x) e^{ax} dx = e^{ax} (\frac{P(x)}{a} - \frac{P'(x)}{a^2} + ... + (-1)^n \frac{P^{(n)}(x)}{a^{n+1}}) + C\]
    \end{Upr}

    \begin{Upr}
        \[A(x) = P(x) - \frac{P^{(2)}(x)}{a^2} + \frac{P^{(4)}(x)}{a^4} - ... \qq B(x) = P'(x) - \frac{P^{(3)}(x)}{a^2} + ...\]
        \[\int P(x) \cos ax\ dx = \frac{\sin ax}{a} A(x) + \frac{\cos ax}{a^2} B(x) + C\]
        \[\int P(x) \sin ax \ dx = - \frac{\cos ax}{a} A(x) + \frac{\sin ax}{a^2} B(x) + C\]
    \end{Upr}

    \begin{example}
        $a_1,...,a_n$ - соизмеримы $(a_i = k_i a)$
        \[\int R(e^{a_1 x},...,e^{a_n x}) dx = \frac{1}{a} \int R(t^{k_1},...,t^{k_n}) \frac{dt}{t} = \int R^*(t) dt\]
        \[t = e^{ax} \qq dt = at dx\]
    \end{example}

    \begin{task}
        Д-ть, что интеграл $\int R(x) e^{ax} dx$, где $R$ - рац. ф-ия, знаменатель которой имеет лишь действительные корни, выражается через элементарные функции и трансцендентную функцию $\int \frac{e^{ax}}{x} dx = li(e^{ax}) + C$, где $li(x) = \int \frac{dx}{\ln x}$
    \end{task}

    \begin{Sol}
        \[R(x) = P(x) + \sum_k \sum_{i = 1}^{m_k} \frac{A_{k_i}}{(x - x_k)^i},\]
        $m_k$ - кратность корня $x_i$
        \[\int R(x) e^{ax} = \int P(x) e^{ax} dx + \sum_k \sum_{i=1}^{m_k} A_{k_i} \int \frac{e^{ax}}{(x - x_k)^i} dx\]
        \[\int \frac{e^{ax}}{(x - x_k)^i} dx = \int \frac{e^{a(t + x_k)}}{t^i} dt = \frac{e^{x_k}}{1 - i} \int e^{at} d\Br{\frac{1}{t^{i-1}}} =\]
        \[= \frac{e^{a x_k}}{1 - i} e^{at} \frac{1}{t^{i-1}} - \frac{a e^{a x_k}}{1 - i} \int \frac{e^{at}}{t^{i-1}}dt\]
        понижаем до первой степени t:
        \[\int \frac{e^{ax}}{(x - x_k)^i} dx = e^{ax_k} li(e^{at})\]
    \end{Sol}

    \begin{task}
        В каком случае интеграл $\int P\Br{\frac{1}{x}} e^x dx$, где $P\Br{\frac{1}{x}} = a_0 + \frac{a_1}{x} + ... + \frac{a_n}{x}$ и $a_i = \const$ представляет собой элементарную функцию?
    \end{task}

    \begin{sol}
        Возьмём по частям:
        \[\int P\Br{\frac{1}{x}} e^x dx = a_0 e^x + a_1 li(e^x) - \frac{a_2}{x} e^x + a_2 li(e^x) - \]
        \[- \frac{a_3}{2x^2} - \frac{a_3}{2x} + \frac{a_3}{2} li(e^x) - ... - \frac{a_n}{(n-1)x^{n-1}} - \frac{a_n}{(n-1)(n-2) x^{n-2}} - ... - \frac{a_n}{(n-1)! x} + \frac{a_n}{(n-1)!} li(e^x)\]
        Значит когда:
        \[\sum_{i = 1}^n \frac{a_i}{(i - 1)!} = 0\]
    \end{sol}

    \begin{Example}
        \[\int |x| dx = (\sgn x) \int x \ dx = \sgn x \frac{1}{2} x^2 + C = \frac{x|x|}{2} + C\]
    \end{Example}

    \begin{Example}
        \[\int x f''(x) dx = \int x d(f'(x)) = x f'(x) - \int f'(x) dx = x f'(x) - f(x) + C\]
    \end{Example}

    \begin{Example}
        \[f(x) - ? \qq f'(\sin^2 x) = \cos^2 x\]
        \[\Ra f'(\sin^2 x) = 1 - \sin^2 x \RA f'(x) = 1 - x\]
        \[\Ra f(x) = \int f'(x) dx = x - \frac{x^2}{2} + C\]
    \end{Example}
\end{document}
