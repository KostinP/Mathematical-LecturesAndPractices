\documentclass[discrete.tex]{subfiles}

\begin{document}
  \section{Избыточное кодирование. Коды Хэмминга}
  Хорошая статья на \href{https://habr.com/ru/post/140611/}{habr}

  \begin{sol}[по Григорьевой]
    Пусть наше изначальное сообщение 0110011\\ \ \\
    Вставим на позиции степени двойки $x_i$:\\
    \[y = \os{1}{x_1} \q \os{2}{x_1} \q \os{3}{0} \q \os{4}{x_3} \q \os{5}{1} \q \os{6}{1} \q \os{7}{0} \q \os{8}{x_4} \q \os{9}{0} \q  \os{10}{1} \q  \os{11}{1}\]
    Матрица A (1,2,...11 преобразовали в двоичный код):
    \[A = \left(\begin{tabular}{ccccccccccc}
      1 & 0 & 1 & 0 & 1 & 0 & 1 & 0 & 1 & 0 & 1\\
      0 & 1 & 1 & 0 & 0 & 1 & 1 & 0 & 0 & 1 & 1\\
      0 & 0 & 0 & 1 & 1 & 1 & 1 & 0 & 0 & 0 & 0\\
      0 & 0 & 0 & 0 & 0 & 0 & 0 & 1 & 1 & 1 & 1
    \end{tabular}\right)\]
    Умножим матрицу A на вектор y:
    \[A y = B = \begin{pmatrix}
      x_1 + 2\\
      x_2 + 3\\
      x_3 + 2\\
      x_4 + 2
    \end{pmatrix}\]
    Выберем x так, чтобы сумма была четной:
    \[x_1 = 0 \qq x_2 = 1 \qq x_3 = 0 \qq x_4 = 0\]
    \[01001100011\]
    Теперь допустим ошибку в n и попробуем её исправить:
    \[n = \fbox{1}110011 \Ra y = 01\fbox{1}01100011\]
    Умножим y снова на матрицу A:
    \[B^* = \begin{pmatrix}
      3\\5\\2\\2
    \end{pmatrix} \os{\mod 2}{=} \begin{pmatrix}
      1\\1\\0\\0
    \end{pmatrix}\]
    Это третий столбец, значит ошибка в первом бите
  \end{sol}
\end{document}
