\documentclass[discrete.tex]{subfiles}

\begin{document}

\section{Теорема об энтропии}
\begin{theorem}
  Единственная функция, удовлетворяющая 6-ти св-ам энтропии - это функция $H(\{p_1,...,p_m\}) = \sum_{i=1}^m p_i \log_2 \frac{1}{p_i}$
\end{theorem}

\begin{proof}
  Обозначим $y_i=2^{-s_i}$, так что $s_i=\log_2 \frac{1}{y_i}$

  В этих обозначениях условие 1 преобразуется к виду:
  \[\sum_{i \in 1:m} y_i \leqslant 1, \qq (@)\]
  а целевая функция к виду:
  \[T(p,y) = \sum_{i \in 1:m} p_i \log_2 \frac{1}{y_i}\]
  Условие 2 перейдет в условие $0 < y_i < 1$

  Второе неравенство из этой пары следует только из полученного ограничения на сумму переменных $y_i$, так что останется только условие положительности.

  Целевая функция T состоит из отдельных слагаемых, соответствующих отдельным переменным. Каждое слагаемое убывает с ростом аргумента. Если бы неравенство  $(@)$ выполнялось как строгое, то увеличение любой из переменных уменьшило бы значение целевой функции. Поэтому в точке минимума условие 1 выполняется как равенство. Выразим переменную $y_m$ через остальные:
  \[y_m = 1 - \sum_{i \in 1 : m-1} y_i\]
  Подставим её в целевую функцию и приравняем к нулю производные целевой функции $T(p,y)$ по всем оставшимся переменным:
  \[\dfrac{\d T}{\d y_i} = -p_i \cdot \log_2 e \cdot \frac{1}{y_i} + p_m \cdot \log_2 e \cdot \frac{1}{y_m}\]
  Откуда
  \[\frac{p_1}{y_1} = \frac{p_2}{y_2} = ... = \frac{p_m}{y_m}\]
  Так как обе суммы ($p_i$ и $y_i$) равны 1, то $y_i=p_i$. Это единственная точка, в которой возможен экстремум функции $T(p,y)$, и следовательно, $\min_y T(p,y) = H(p)$
\end{proof}
\end{document}
