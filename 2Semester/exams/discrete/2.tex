\documentclass[discrete.tex]{subfiles}

\begin{document}
  \section{Вектора из нулей и единиц}

  Пусть мн-во B состоит из двух элементов которые отождествляются с 0 и 1, т.е. $B = 0:1$

  Произведение m экзмемпляров такого мн-ва обозначим за\\
  $B^m=(0:1)^m$, состоит из $2^m$ эл-ов
  \begin{definition}
    Вектор из нулей и единиц - упорядоченный набор из фиксированного числа нулей и единиц, т.е. эл-т мн-ва $B^m$
  \end{definition}
  Упорядоченный набор из чисел оычно называется вектором, m - размерностью вектора, каждый отдельный элемент набора - компонента вектора
  \begin{remark}
    Модели, в которых используются наборы из 0 и 1:
    \begin{enumerate}
      \item Геометрическая интерпретация

      Точкой в m-мерном пространстве является m-мерный вектор, каждая его компонента - одна из декартовых координат точки. Набор из 0 и 1, рассматриваемый как точка в пространстве, определяет вершину куба, построенного на ортах (единичных отрезках) координатных вероятностей
      %картинка
      \item Логичнская интерпретация

      Операции над векторами выполняются покомпонентно, т.е. независимо над соотв. компонентами векторов-операндов
      \begin{Example}
        \[\begin{matrix}
          x && 0 & 0 & 0 & 1 & 1\\
          y && 1 & 1 & 1 & 0 & 1\\
          x \wedge y && 0 & 0 & 0 & 0 & 1\\
          x \vee y && 1 & 1 & 1 & 1 & 1\\
          x \equiv y && 0 & 0 & 0 & 0 & 1\\
          x \not \equiv y && 1 & 1 & 1 & 1 & 0
        \end{matrix}\]
      \end{Example}
      \item Двоичное представление (натуральные числа)

      Число представляется в виде суммы степеней 2
      \item Состояние памяти компьютера
      \item Сообщение, передаваемое по каналу связи
      \item Можно задавать подмножества мн-ва $1:n$
    \end{enumerate}
  \end{remark}
\end{document}
