\documentclass[discrete.tex]{subfiles}

\begin{document}

\section{Условные вероятности и формула Байеса}

\begin{definition}[Условная вероятность]
    Пусть A - событие, $P(A) > 0$, тогда:
    \[P(B \ |\ A) = \frac{P(B \cap A)}{P(A)}\]
    Вероятность события $B$, если произошло $A$\\
    (все исходы, при которых произошли B и A на исходы с A)
\end{definition}

\begin{remark}
  События независимы, если $P(B \ |\ A) = P(B)$ (очевидно)
\end{remark}

\begin{Reminder}[Формула полной вероятности]
    \[S_1, ..., S_n \qq S_i \cap S_j = \varnothing \]
    \[P(A) = \sum_{i = 1}^n P(A \cap S_i) = \sum_{i = 1} ^n P(A \ | \ S_i) \cdot P(S_i) \]
\end{Reminder}

\begin{Theorem}[Формула Байеса]
  \[P(S_i \ | \ A) = \frac{P(A \ | \ S_i) P(S_i)}{
  \displaystyle \sum_{i = 1}^n P(A \ | \ S_i) \cdot P(S_i) } \qq \text{по Григорьевой}\]
  \[P(A \ | \ B) = \frac{P(B\ |\ A)P(A)}{P(B)} \qq \text{по Интернету}\]
\end{Theorem}

\begin{proof}
  Подставим $P(B \cup A) = P(B\ |\ A) P(A)$ в формулу полной вероятности:
  \[P(B) = \sum P(B\ |\ A_i) P(A_i)\]
  \[P(B \cup A) = P(B\ |\ A) P(A) = P(A\ |\ B) P(B)\]
  \[\Ra P(A_i\ |\ B) = \frac{P(B\ |\ A_i) P(A_i)}{\sum_j P(B\ |\ A_j) P(A_j)}\]
\end{proof}



\end{document}
