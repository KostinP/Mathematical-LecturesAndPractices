\documentclass[discrete.tex]{subfiles}

\begin{document}
  \section{Условные вероятности и формула Байеса}

  \begin{definition}[Условная вероятность]
      Пусть A - событие, $P(A) > 0$, тогда:
      \[P(B \ |\ A) = \frac{P(B \cap A)}{P(A)}\]
      Вероятность события $B$, если произошло $A$\\
      (все исходы, при которых произошли B и A на исходы с A)
  \end{definition}

  \begin{remark}
    События независимы, если $P(B \ |\ A) = P(B)$ (очевидно)
  \end{remark}

  \begin{Reminder}[Формула полной вероятности]
      \[B_1, ..., B_n \qq B_i \cap B_j = \varnothing \]
      \[P(A) = \sum_{i = 1}^n P(A \cap B_i) = \sum_{i = 1} ^n P(A \ | \ B_i) \cdot P(B_i) \]
  \end{Reminder}

  \begin{example}
    Старая линия завода выпускает в 2 раза меньше продукции, чем новая ($2P(S)=P(N)$). А доля брака у нее в 4 раза больше($P(Br|S) = 4 P(Br|N)$). Что можно сказать о доле брака ($P(Br)$) в продукции?
  \end{example}

  \begin{proof}
    По ф-ле полной вероятности:
    \begin{multline*}
      $P(Br) = P(Br|S) P(S) + P(Br|N) P(N) =\\
       4P(Br|N)P(S)+2P(Br|N)P(S) = 2P(Br|N)$
    \end{multline*}
  \end{proof}

  \begin{Theorem}[Формула Байеса]
    \[P(B_i \ | \ A) = \frac{P(A \ | \ B_i) P(B_i)}{
    \displaystyle \sum_{i = 1}^n P(A \ | \ B_i) \cdot P(B_i) } \qq \text{по Григорьевой}\]
    \[P(A \ | \ B) = \frac{P(B\ |\ A)P(A)}{P(B)} \qq \text{по Интернету}\]
  \end{Theorem}

  \begin{proof}
    Подставим $P(B \cup A) = P(B\ |\ A) P(A)$ в формулу полной вероятности:
    \[P(B) = \sum_i P(B\ |\ A_i) P(A_i)\]
    \[P(B \cup A) = P(B\ |\ A) P(A) = P(A\ |\ B) P(B)\]
    \[\Ra P(A_i\ |\ B) = \frac{P(B\ |\ A_i) P(A_i)}{\sum_j P(B\ |\ A_j) P(A_j)}\]
  \end{proof}

  \begin{example}
    Пусть у нас есть две колоды: 36 и 52 карты. Выбираем с равной вероятностью одну из колод. Достаем из нее карту и хотим угадать, какая это из колод. Пусть $T\diamondsuit$
    \[P(B_{36} | T\diamondsuit) =
    \frac{P(T\diamondsuit | B_{36}) P(B_{36})}{P(T\diamondsuit | B_{36}) P(B_{36}) + P(T\diamondsuit | B_{52}) (B_{52})} = \frac{\frac{1}{36} \frac{1}{2}}{\frac{1}{36}\frac{1}{2} + \frac{1}{52}\frac{1}{2}} = \frac{52}{88}\]
  \end{example}
\end{document}
