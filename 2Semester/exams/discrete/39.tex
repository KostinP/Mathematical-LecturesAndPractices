\documentclass[discrete.tex]{subfiles}

\begin{document}
  \section{Связность. Компоненты связности и сильной связности}

  \begin{definition}
    Граф называется связным, если любые его две вершины соединены цепью
  \end{definition}

  \begin{definition}
    Компонента связности - $\max$ связный подграф
  \end{definition}

  \begin{definition}
    Пусть задан граф $<M,N,T>$. Определим на мн-ве $M$ отношение $C$, относя к нему те пары вершин $(x,y) \in M \times M$, которые могут быть соединены цепью и все пары $(x,x)$. Это отношение, очевидно, рефлексивно и симметрично
  \end{definition}

  \begin{lemma}
    Отношение $C$ - транзитивно, т.е. из того что цепями могут быть соединены все вершины пары $(x,y)$ и вершины пары $(y,z)$, следует, что цепью модно соединить и вершину x с вершиной z
  \end{lemma}

  \begin{proof}
    Рассмотрим цепь $x=x_0,...,x_m=y$, соединеняющую $x$ и $y$ и перенумерованную от $x$ к $y$, и цепь $y=y_0,...,y_n=z$, соединеняющую $y$ и $z$ и перенумерованную от $y$ к $z$. Пусть $k$ - наименьший номер вершины из первой цепи, содержащейся во второй цепи (такая, очевидно, существует, так как $x_m=y$ является вершиной из первой цепи, содержащейся во второй цепи). Пусть $l$ - номер этой вершины второй цепи. Тогда последовательность вершин
    \[v_0=x_0,...,v_k=x_k=y_l,\q v_{k+1}=y_{l+1},...,v_{k+n-l}=y_n=z\]
    "снабженная"{} соответствующей последовательностью дуг, и образует искомую цепь
  \end{proof}

  \begin{consequence}
    C задает отношение эквивалентности, а значит задает разбиение мн-ва $M$ на классы эквивалентности
  \end{consequence}

  \begin{definition}
    Граф называется сильно связанным, если любые две различные его вершины $i_1$ и $i_2$ можно соединить путем с началом $i_1$ и концом $i_2$
  \end{definition}

  \begin{remark}
    В любом подграфе можно однозначно выделить сильно связанные подграфы, которые называются его компонентами сильной связности
  \end{remark}

  \begin{definition}
    Аналогично введем отношение
    \[R(i_1,i_2) = \text{"существует путь из $i_2$ в $i_2$"}\]
  \end{definition}

  \begin{remark}
    Будем считать, что каждая вершина соеденина сама с собой. Тогда это отношение рефлексивно. Кроме того, оно транзитивно, так как если взять путь из $i_1$ в $i_2$ и путь из $i_2$ в $i_3$ дают путь из $i_1$ в $i_3$ (цепи так просто не сцеплялись). Обратное отношение $R^{-1}$ также транзитивно $S=R \cup R^{-1}$ и транзитивно и симметрично. Таким образом, мы снова имеем отношение эквивалентности S. Классы эквивалентности в этом отношении и является компонентами сильной связности.
  \end{remark}
\end{document}
