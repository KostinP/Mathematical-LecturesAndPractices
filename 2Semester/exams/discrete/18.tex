\documentclass[discrete.tex]{subfiles}

\begin{document}

\section{Двоичный поиск и неравенство Крафта}
ИСПРАВИТЬ ЭТО. В РОМАНОВСКОМ ЛУЧШЕ.
\begin{theorem}
  Для того чтобы набор из целых чисел от 1 до m мог быть набором длин путей в схеме с m исходами необходимо и достаточно, чтобы:
  \[\sum_{i \in 1:m} 2^{-S_i} \leqslant 1,\q \text{$S_i$ - числа из набора}\]
\end{theorem}

\begin{proof}
  ($\Ra$)

  Рассмотрим поисковую схему - двойное дерево T с m листьями, K - вершина, находящаяся на расстоянии t от корня, $a_k=2^{-t}$, $r_0$ - корень $\Ra a_{r_0} = 1$

  Докажем, что $a_{r_0} \geqslant \sum_{k \in F} a_k$, где F - мн-во листьев

  Для каждого нелиста $k \in M \setminus F \Ra a_k \geqslant \sum_{r \in \next(k)} a_r$, где $\next(k)$ - мн-во прямых потомков k
  \[\sum_{k \im M \setminus F} \geqslant \sum a\]
\end{proof}
\end{document}
