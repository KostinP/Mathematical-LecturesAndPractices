\documentclass[discrete.tex]{subfiles}

\begin{document}

\section{Два алгоритма перебора перестановок. Нумерация перестановок}
\[|P_k| = k! = |T_k|\]
$P_k$ - мн-во всех перестановок\\
$T_k$ - произведение k любых таких множеств $M_i$, каждый из которых представляет собой мн-во чисел от 0 до $i-1$
\[T_k = \{0\} \times \{0,1\} \times ... \times \{0,1,...,k-1\}\]
Построим взаимно однозначное соответствие между $P_k$ и $T_k$. Возьмем перестановку ($t_1,...,t_k$) следующим образом: для любого $i \in 1:k$ найдем число значений, меньше $r_i$ среди $r_{i+1},...,r_k$ - это число мы и примем в качестве $t_i$\\
В соответсвии с таким определеничем чисел $t_i$ в мн-ве $T_k$ будет соответственно ??? значения $m_i$ не возраст., а убывающая до единицы

\begin{Example}[4, 8, 1, 5, 7, 2, 3, 6]
\[\begin{matrix}
  i & 1 & 2 & 3 & 4 & 5 & 6 & 7 & 8\\
  r_i & 4 & 8 & 1 & 5 & 7 & 2 & 3 & 6\\
  t_i & 3 & 6 & 0 & 2 & 3 & 0 & 0 & 0\\
  m_i & 8 & 7 & 6 & 5 & 4 & 3 & 2 & 1
\end{matrix}\]
По ($t_1,...,t_k$) легко восстановить исходную перестановку. Для этого меняя i от 1 до k нужно нужно проверить мн-во значений $S_i$, которые могут быть в перестановке на i месте. Для $i=1$ $S_1 = 1:8$, $t_1 = 3 \Ra r_1 = 4$, далее $S_2 = 1:3 \cap 5:8$, $t_2 = 6 \Ra r_2 = 8$. Если использовать это отображение при переборе, то перестановки будут перебираться в лексикографическом порядке
\end{Example}

\begin{definition}
  ($r_1,...,r_k$) предшествует ($R_1,...,R_k$), если начала перестановок совпадают до индекса d, а дальше $r_d < R_d$\\
\end{definition}

\begin{utv}
  Из этого перестановки перебираются в лексикографическом порядке, можно вывести правило получения следующего:
  \begin{enumerate}
    \item В ($r_1,...,r_k$) найти наибольший суффикс ($r_t,...,r_k$), в котором $r_t > ... > r_k$ ($r_{i-1} < r_t$) %произошел тролленг
    \item Выбрать ($r_t,...,r_k$) элемент следующий по велечине после $r_{t-1}$, поставить после в возр. порядке
    \[\begin{tabular}{c|cccc|cccc}
      \num & \multicolumn{4}{c}{t_k} & \multicolumn{4}{c}{p_k}\\
      \hline
      0 &  0 & 0 & 0 & 0 &  1 & 2 & 3 & 4\\
      1 &  0 & 0 & 1 & 0 &  1 & 2 & 4 & 3\\
      2 &  0 & 1 & 0 & 0 &  1 & 3 & 2 & 4\\
      3 &  0 & 1 & 1 & 0 &  1 & 3 & 4 & 2\\
      4 &  0 & 2 & 0 & 0 &  1 & 4 & 2 & 3\\
      5 &  0 & 2 & 1 & 0 &  1 & 4 & 3 & 2
    \end{tabular}\]
    Ещё один алгоритм
  \end{enumerate}
\end{utv}

\end{document}
