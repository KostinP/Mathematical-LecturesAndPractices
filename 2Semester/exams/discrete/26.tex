\documentclass[discrete.tex]{subfiles}
 
\begin{document}

\section{Сжатие информации по методу Зива-Лемпеля}

\begin{definition} [Алгоритм]
    X - Входная фраза (некая строка, которую мы строим)\\
    Точкой обозначена конкатенация
    \begin{enumerate}
        \item Занести все возможные символы в словарь. (им всем будет присвоен код)
        \item Считаем один символ из сообщения и добавим его в X
        \item Считаем символ Y, если это символ конца сообщения, то вернем код X (он уже
            лежит в cловаре), иначе:
            \begin{enumerate}
                \item Если X.Y уже есть в словаре, то присвоим X = X.Y, перейдем к шагу 2
                \item Иначе вернем код для X, добавим X.Y в словарь и присвоим входной
                    фразе значение Y \q X = Y, перейдем к шагу 2
            \end{enumerate}
    \end{enumerate}
\end{definition}

\begin{Example}
    \[abrakadabra\]
    \[a - 1 \q b - 2 \q r - 3 \q k - 4 \q d - 5\]
    \[ab - 6 \q br - 7 \q ra - 8 \q ak - 9 \q ka - 10\]
    \[ad - 11 \q ab - 12 \q bra - 13\]
    \[\text{Output: } \us{a}{1} \ \us{b}{2} \ \us{r}{3} \ 1 \ 4 \ 1 \ 5 \ 1 \ 7 \ 1\]
\end{Example}

\end{document}
