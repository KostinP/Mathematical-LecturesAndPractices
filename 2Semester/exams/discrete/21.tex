\documentclass[discrete.tex]{subfiles}

\begin{document}
  \section{Операции над строками переменной длины}

  \begin{definition}
    A - конечное мн-во, называется алфовитом. Его элементы буквами. Произвольная конечная последовательность букв называется строкой. Кол-во букв в этой последовательности - длина строки
  \end{definition}

  \begin{enumerate}
    \item Нахождение длины строки
    \item Выделение подстроки. Операция выделяет подстроку с заданным числом букв, начиная с заданного места
    \begin{definition}
      Начальная подстрока строки - префикс. Конечная - суффикс. Голова строки ($\head$) - префикс из одной буквы. Остальная часть - хвост ($\tail$)
    \end{definition}
    \item Конкатенация строк (сцепка)
    \item Обращение строк (переворот)
    \item Сравнение строк по предшествованию, обычно используется лексикографическое сравнение.

    Пусть буквы из A можно сравнивать

    Тогда для строк a и b результат лексикографического сравнения равен
    \begin{enumerate}
      \item $a=b$, если a и b - пусты
      \item $a<b$, если a пустая, а b - нет
      \item $a>b$, если b пустая, а a - нет
      \item $a<b$, если $\head a < \head b$
      \item $a>b$, если $\head a > \head b$
      \item Результат сравнения $\tail a$ и $\tail b$, если $\head a = \head b$
    \end{enumerate}
    \item Поиск образца в строке
    \item Подстановка (вместо заданного образца нужно вписать другой)
    \item Преобразование

    S - строка, $A' = \{ a\ |\ a \in S\}$, \q $\varphi: A' \Ra V$, \q $\varphi(B)$ - последовательность элементов из B
    \item Фильтрация (все символы делятся на подмн-ва A и B). Если $s_i \in A$, то остается в строке, если $\in B$ - удаляется
    \item Слияние

    Пусть на алфавите задан порядок. $s_1, s_2$ - строки из A. На каждом шаге к результату приписывается $m = \min (\head s_1, \head s_2$) и удаляется из старого списка
    \item Поиск максимального совпадения
  \end{enumerate}

  \begin{definition}
    Функция $f(s)$, заданная на множестве всех строк $S$ называется аддитивной, если $\e \varphi: A \ra \R$:
    \[f(s) = \varphi(\head s) + f(\tail s),\]
    когда строка непустая и $f(s) = 0$ для пустой строки
  \end{definition}

  \begin{example}
    Длина строки - аддитивная функция
  \end{example}

  \begin{definition}
    Функция $f(s)$, заданная на множестве всех строк $S$ называется мультипликативной, если $\e \psi: A \ra \R$:
    \[f(s) = \psi(\head s) \cdot f(\tail s),\]
    когда строка непустая и $f(s) = 1$ для пустой строки
  \end{definition}

  \begin{example}
    Вероятность совмещения независимыз в совокупности событий, заданных списком s
  \end{example}

  \begin{definition}
    Функция $f(s)$, заданная на множестве всех строк $S$ называется мультипликативной, если $\e \varphi, \psi: A \ra \R$:
    \[f(s) = \varphi(\head s) + \psi(\head s) \cdot f(\tail s),\]
    когда строка непустая и $f(s)$ как-то определена для пустой строки
  \end{definition}

  \begin{example}
    Схема Горнера: Полином представляется в виде
    \[P(x;\{a_0,...,a_n\}) = a_0 + x \cdot P(x; \{a_1,...,a_n\})\]
    $a_0$ - голова $\{a_1,...,a_n\}$ - хвост
  \end{example}

  \begin{definition}
    Функция $f: S \ra D$, где D - некоторое множество, называется марковской, если $\e \theta: A \times D \ra D$:
    \[f(s) = \theta(\head s, f(\tail s))\]
    и задано начальное значение для пустой строки $f(s) = r_0 \in D$
  \end{definition}

  \begin{remark}
    Все предыдущие функции являются марковскими
  \end{remark}
\end{document}
