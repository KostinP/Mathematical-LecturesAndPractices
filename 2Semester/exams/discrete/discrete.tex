\documentclass[12pt, fleqn]{article}

\usepackage{../../../template/template}
\usepackage{../../../template/fortickets}

\begin{document}
%\tableofcontents

\section{Некоторые определения из теории множеств. Прямое произведение, разбиение множеств. Мощность объединения}

\begin{definition}
  Пустое множество ($\varnothing$) - мно-во, которому $\nin$ ни один элемент
\end{definition}

\begin{definition}
  Число элементов мн-ва A - мощность $|A|$
\end{definition}

\begin{definition}
  Множество чисел от k до l обозначается $k:l$
\end{definition}

\begin{definition}
  Мн-во A - подмн-во мн-ва B ($A \subset B$), если каждый элемент из A принадлежит B
\end{definition}

\begin{definition}
  C - объединение A и B ($A \cap B$), если оно состоит из всех элементов A и B ($C = \{x | x \in A \text{ и } x \in B\}$)
\end{definition}

\begin{definition}
  $\us{i=1}{\os{n}{\cap}} A_i,\q \us{i=1}{\os{n}{\cup}} A_i$ - объединение и пересечение конечного числа мн-в
  \[(\us{i \in I}{\cap} A_i,\q \us{i \in I}{\cup} A_i) \text{ - аналогично}\]
\end{definition}

\begin{definition}
  Если пересечение мн-в пусто, то они называются дизъюнктивными
\end{definition}

\begin{definition}
  Мн-во C называется разностью мн-в A и B ($C = A \setminus B$), если оно состоит из всех эл-в, принадлежащих А и не принадлежащих B
\end{definition}

\begin{definition}
  $A \triangle B = A \setminus B \cap B \setminus A$ - симметрическая разность
\end{definition}

\begin{definition}
  Мн-во упорядоченных пар ($i,j$), где $i \in A$, $j \in B$ называется прямым произведением мн-в A и B
  \[A \times B = \{(i,j)| i \in A,\q j \in B\}\]
\end{definition}

\begin{remark}
  Мощность прямого произведения $|A \times B| = |A| \cdot |B|$. Аналогично произведение $\forall$ конечного числа множеств
\end{remark}

\begin{definition}
  Пусть $A_1,...,A_k$ - ненулевые и попарно дизъюнктивные, $M = A_1 \cap ... \cap A_k$ и мн-во $\{A_1,...,A_k\}$ называется разбиением M\\
  (если они попарно не дизъюнктивные, то это покрытие)
\end{definition}

\begin{definition}
  Разбиение A мн-ва M называется измельчением B, если $\forall A_i \in A$ содержится в некотором $B_i \in B$
\end{definition}

\begin{definition}
  Пусть A, B - размельчения мн-ва M, разбиение C называется произведением A и B, если оно является из измельчением, причем самым крупным $C = A \cdot B$
\end{definition}

\begin{theorem}
  Произведение двух разбиений существует
\end{theorem}

\begin{proof}
  Предъявим разбиение, которое будет пересечением $A = \{A_1,...,A_k\}$ и $B = \{B_1,...,B_l\}$, точнее $D_{ij} = A_i \cup B_j,\q i \leqslant k,\q j \leqslant l$ и $\mathcal{P} = \cup D_{ij}$ (т.е. без пустых строк). Покажем, что тогда оно самое крупное.

  Пусть $\e F = \{F_1,...,F_t\}$ - измельчение A и B, тогда $\forall F_k\q \e A_{i_k},\ B_{i_k}: F_k \subdet A_{i_k},\ B_{i_k} \Ra F_k \subset (A_{i_k} \cup B_{i_k}) = D_{i_k j_k} \Ra$ мельче F
\end{proof}


\section{Вектора из нулей и единиц}


\section{Алгоритм перебора 0-1 векторов. Коды Грея}


\section{Перебор элементов прямого произведения множеств}


\section{Размещения, сочетания, перестановки без повторений}


\section{Размещения, сочетания, перестановки с повторениями}


\section{Два алгоритма перебора перестановок. Нумерация перестановок}


\section{Задача о минимуме скалярного произведения}


\section{Числа Фибоначчи. Теорема о представлении}


\section{Перебор сочетаний. Нумерация сочетаний}


\section{Бином Ньютона и его комбинаторное использование}


\section{Свойства биномиальных коэффициентов}


\section{Основные определения теории вероятностей}


\section{Условные вероятности и формула Байеса}


\section{Математическое ожидание и дисперсия случайной величины}


\section{Схема Бернулли}


\section{Случайные числа. Схема Уолкера}


\section{Двоичный поиск и неравенство Крафта}


\section{Энтропия. 2 леммы}


\section{Теорема об энтропии}


\section{Операции над строками переменной длины}


\section{Поиск образца в строке (Карпа-Рабина, Бойера-Мура)}


\section{Суффиксное дерево}


\section{Задача о максимальном совпадении двух строк}


\section{Код Шеннона-Фано. Алгоритм Хаффмена. 3 леммы}


\section{Сжатие информации по методу Зива-Лемпеля}


\section{Метод Барроуза-Уилера}


\section{Избыточное кодирование. Коды Хэмминга}


\section{Шифрование с открытым ключом}


\section{Сортировки (5 методов)}


\section{Информационный поиск и организация информации}


\section{Хеширование}


\section{АВЛ-деревья}


\section{B-деревья}


\section{Биноминальные кучи}


\section{Основные определения теории графов}


\section{Построение транзитивного замыкания}


\section{Обходы графа в ширину и глубину. Топологическая сортировка}


\section{Связность. Компоненты связности и сильной связности}


\section{Алгоритм поиска контура и построение диаграммы порядка}


\section{Теорема о связном подграфе}


\section{Деревья. Теорема о шести эквивалентных определениях дерева}


\section{Задача о кратчайшем остовном дереве. Алгоритм Прима}


\section{Алгоритм Краскала}


\section{Задача о кратчайшем пути. Алгоритм Дейкстры}


\section{Алгоритм Левита}


\section{Задача о кратчайшем дереве путей}


\section{Сетевой график и критические пути. Нахождение резервов работ}


\section{Задача о максимальном паросочетании в графе. Алгоритм построения}


\section{Теорема Кенига}


\section{Алгоритм построения контролирующего множества}


\section{Задача о назначениях. Венгерский метод}


\section{Задача коммивояжера. Метод ветвей и границ}


\section{Метод динамического программирования. Задача линейного раскроя}


\section{Приближенные методы решения дискретных задач. Жадные алгоритмы}


\section{Алгоритмы с гарантированной оценкой точности. Алгоритм Эйлера}


\section{Жадные алгоритмы. Задача о системе различных представителей}


\section{Приближенные методы решения дискретных задач}


\section{Конечные автоматы}


\section{Числа Фибоначчи. Производящие функции}


\section{Числа Каталана}


\section{?Алгоритм Кристофидеса (возможно будет)}

\end{document}
