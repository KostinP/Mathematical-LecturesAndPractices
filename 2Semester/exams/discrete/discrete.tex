\documentclass[12pt, fleqn]{article}

\usepackage{../../../template/template}
\usepackage{../../../template/fortickets}

\begin{document}
%\tableofcontents

\section{Некоторые определения из теории множеств. Прямое произведение, разбиение множеств. Мощность объединения}


\section{Вектора из нулей и единиц}


\section{Алгоритм перебора 0-1 векторов. Коды Грея}


\section{Перебор элементов прямого произведения множеств}


\section{Размещения, сочетания, перестановки без повторений}


\section{Размещения, сочетания, перестановки с повторениями}


\section{Два алгоритма перебора перестановок. Нумерация перестановок}


\section{Задача о минимуме скалярного произведения}


\section{Числа Фибоначчи. Теорема о представлении}


\section{Перебор сочетаний. Нумерация сочетаний}


\section{Бином Ньютона и его комбинаторное использование}


\section{Свойства биномиальных коэффициентов}


\section{Основные определения теории вероятностей}


\section{Условные вероятности и формула Байеса}


\section{Математическое ожидание и дисперсия случайной величины}


\section{Схема Бернулли}


\section{Случайные числа. Схема Уолкера}


\section{Двоичный поиск и неравенство Крафта}


\section{Энтропия. 2 леммы}


\section{Теорема об энтропии}


\section{Операции над строками переменной длины}


\section{Поиск образца в строке (Карпа-Рабина, Бойера-Мура)}


\section{Суффиксное дерево}


\section{Задача о максимальном совпадении двух строк}


\section{Код Шеннона-Фано. Алгоритм Хаффмена. 3 леммы}


\section{Сжатие информации по методу Зива-Лемпеля}


\section{Метод Барроуза-Уилера}


\section{Избыточное кодирование. Коды Хэмминга}


\section{Шифрование с открытым ключом}


\section{Сортировки (5 методов)}


\section{Информационный поиск и организация информации}


\section{Хеширование}


\section{АВЛ-деревья}


\section{B-деревья}


\section{Биноминальные кучи}


\section{Основные определения теории графов}


\section{Построение транзитивного замыкания}


\section{Обходы графа в ширину и глубину. Топологическая сортировка}


\section{Связность. Компоненты связности и сильной связности}


\section{Алгоритм поиска контура и построение диаграммы порядка}


\section{Теорема о связном подграфе}


\section{Деревья. Теорема о шести эквивалентных определениях дерева}


\section{Задача о кратчайшем остовном дереве. Алгоритм Прима}


\section{Алгоритм Краскала}


\section{Задача о кратчайшем пути. Алгоритм Дейкстры}


\section{Алгоритм Левита}


\section{Задача о кратчайшем дереве путей}


\section{Сетевой график и критические пути. Нахождение резервов работ}


\section{Задача о максимальном паросочетании в графе. Алгоритм построения}


\section{Теорема Кенига}


\section{Алгоритм построения контролирующего множества}


\section{Задача о назначениях. Венгерский метод}


\section{Задача коммивояжера. Метод ветвей и границ}


\section{Метод динамического программирования. Задача линейного раскроя}


\section{Приближенные методы решения дискретных задач. Жадные алгоритмы}


\section{Алгоритмы с гарантированной оценкой точности. Алгоритм Эйлера}


\section{Жадные алгоритмы. Задача о системе различных представителей}


\section{Приближенные методы решения дискретных задач}


\section{Конечные автоматы}


\section{Числа Фибоначчи. Производящие функции}


\section{Числа Каталана}


\section{?Алгоритм Кристофидеса (возможно будет)}

\end{document}
