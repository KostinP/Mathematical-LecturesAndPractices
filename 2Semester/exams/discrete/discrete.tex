\documentclass[12pt, fleqn]{article}

\usepackage{../../../template/template}
\usepackage{../../../template/fortickets}
\usepackage{../../../template/KillContents}

\begin{document}
  \subfile{preamble.tex}

  %\section{Некоторые определения из теории множеств. Прямое произведение, разбиение множеств. Мощность объединения}
  \subfile{1.tex}

  %\section{Вектора из нулей и единиц}
  \subfile{2.tex}

  %\section{Алгоритм перебора 0-1 векторов. Коды Грея}
  \subfile{3.tex}

  %\section{Перебор элементов прямого произведения множеств}
  \subfile{4.tex}

  %\section{Размещения, сочетания, перестановки без повторений}
  \subfile{5.tex}

  %\section{Размещения, сочетания, перестановки с повторениями}
  \subfile{6.tex}

  %\section{Два алгоритма перебора перестановок. Нумерация перестановок}
  \subfile{7.tex}

  %\section{Задача о минимуме скалярного произведения}
  \subfile{8.tex}

  %\section{Числа Фибоначчи. Теорема о представлении}
  \subfile{9.tex}

  %\section{Перебор сочетаний. Нумерация сочетаний}
  \subfile{10.tex}

  %\section{Бином Ньютона и его комбинаторное использование}
  \subfile{11.tex}

  %\section{Свойства биномиальных коэффициентов}
  \subfile{12.tex}

  %\section{Основные определения теории вероятностей}
  \subfile{13.tex}

  %\section{Условные вероятности и формула Байеса}
  \subfile{14.tex}

  %\section{Математическое ожидание и дисперсия случайной величины}
  \subfile{15.tex}

  %\section{Схема Бернулли}
  \subfile{16.tex}

  %\section{Случайные числа. Схема Уолкера}
  \subfile{17.tex}

  %\section{Двоичный поиск и неравенство Крафта}
  \subfile{18.tex}

  %\section{Энтропия. 2 леммы}
  \subfile{19.tex}

  %\section{Теорема об энтропии}
  \subfile{20.tex}

  %\section{Операции над строками переменной длины}
  \subfile{21.tex}

  %\section{Поиск образца в строке (Карпа-Рабина, Бойера-Мура)}
  \subfile{22.tex}

  %\section{Суффиксное дерево}
  \subfile{23.tex}

  %\section{Задача о максимальном совпадении двух строк}
  \subfile{24.tex}

  %\section{Код Шеннона-Фано. Алгоритм Хаффмена. 3 леммы}
  \subfile{25.tex}

  %\section{Сжатие информации по методу Зива-Лемпеля}
  \subfile{26.tex}

  %\section{Метод Барроуза-Уилера}
  \subfile{27.tex}

  %\section{Избыточное кодирование. Коды Хэмминга}
  \subfile{28.tex}

  %\section{Шифрование с открытым ключом}
  \subfile{29.tex}

  %\section{Сортировки (5 методов)}
  \subfile{30.tex}

  %\section{Информационный поиск и организация информации}
  \subfile{31.tex}

  %\section{Хеширование}
  \subfile{32.tex}

  %\section{АВЛ-деревья}
  \subfile{33.tex}

  %\section{B-деревья}
  \subfile{34.tex}

  %\section{Биноминальные кучи}
  \subfile{35.tex}

  %\section{Основные определения теории графов}
  \subfile{36.tex}

  %\section{Построение транзитивного замыкания}
  \subfile{37.tex}

  %\section{Обходы графа в ширину и глубину. Топологическая сортировка}
  \subfile{38.tex}

  %\section{Связность. Компоненты связности и сильной связности}
  \subfile{39.tex}

  %\section{Алгоритм поиска контура и построение диаграммы порядка}
  \subfile{40.tex}

  %\section{Теорема о связном подграфе}
  \subfile{41.tex}

  %\section{Деревья. Теорема о шести эквивалентных определениях дерева}
  \subfile{42.tex}

  %\section{Задача о кратчайшем остовном дереве. Алгоритм Прима}
  \subfile{43.tex}

  %\section{Алгоритм Краскала}
  \subfile{44.tex}

  %\section{Задача о кратчайшем пути. Алгоритм Дейкстры}
  \subfile{45.tex}

  %\section{Алгоритм Левита}
  \subfile{46.tex}

  %\section{Задача о кратчайшем дереве путей}
  \subfile{47.tex}

  %\section{Сетевой график и критические пути. Нахождение резервов работ}
  \subfile{48.tex}

  %\section{Задача о максимальном паросочетании в графе. Алгоритм построения}
  \subfile{49.tex}

  %\section{Теорема Кенига}
  \subfile{50.tex}

  %\section{Алгоритм построения контролирующего множества}
  \subfile{51.tex}

  %\section{Задача о назначениях. Венгерский метод}
  \subfile{52.tex}

  %\section{Задача коммивояжера. Метод ветвей и границ}
  \subfile{53.tex}

  %\section{Метод динамического программирования. Задача линейного раскроя}
  \subfile{54.tex}

  %\section{Приближенные методы решения дискретных задач. Жадные алгоритмы}
  \subfile{55.tex}

  %\section{Алгоритмы с гарантированной оценкой точности. Алгоритм Эйлера}
  \subfile{56.tex}

  %\section{Жадные алгоритмы. Задача о системе различных представителей}
  \subfile{57.tex}

  %\section{Приближенные методы решения дискретных задач}
  \subfile{58.tex}

  %\section{Конечные автоматы}
  \subfile{59.tex}

  %\section{Числа Фибоначчи. Производящие функции}
  \subfile{60.tex}

  \section{Числа Каталана}
  \subfile{61.tex}

  \section{?Алгоритм Кристофидеса (возможно будет)}
\end{document}
