\documentclass[12pt, fleqn]{article}

\usepackage{../../../template/template}
\usepackage{../../../template/fortickets}
%Для бинома Ньютона
\def\x{\hspace{3ex}}    %BETWEEN TWO 1-DIGIT NUMBERS
\def\y{\hspace{2.45ex}}  %BETWEEN 1 AND 2 DIGIT NUMBERS
\def\z{\hspace{1.9ex}}    %BETWEEN TWO 2-DIGIT NUMBERS
\stackMath

\begin{document}
%\tableofcontents

\section{Некоторые определения из теории множеств. Прямое произведение, разбиение множеств. Мощность объединения}
\begin{definition}
  Пустое множество ($\varnothing$) - мно-во, которому $\nin$ ни один элемент
\end{definition}

\begin{definition}
  Число элементов мн-ва A - мощность $|A|$
\end{definition}

\begin{definition}
  Множество чисел от k до l обозначается $k:l$
\end{definition}

\begin{definition}
  Мн-во A - подмн-во мн-ва B ($A \subset B$), если каждый элемент из A принадлежит B
  %картинка
\end{definition}

\begin{definition}
  C - объединение A и B ($A \cap B$), если оно состоит из всех элементов A и B ($C = \{x | x \in A \text{ и } x \in B\}$)
  %картинка
\end{definition}

\begin{definition}
  $\us{i=1}{\os{n}{\cap}} A_i,\q \us{i=1}{\os{n}{\cup}} A_i$ - объединение и пересечение конечного числа мн-в
  \[(\us{i \in I}{\cap} A_i,\q \us{i \in I}{\cup} A_i) \text{ - аналогично}\]
\end{definition}

\begin{definition}
  Если пересечение мн-в пусто, то они называются дизъюнктивными
  %картинка
\end{definition}

\begin{definition}
  Мн-во C называется разностью мн-в A и B ($C = A \setminus B$), если оно состоит из всех эл-в, принадлежащих А и не принадлежащих B
  %картинка
\end{definition}

\begin{definition}
  $A \triangle B = A \setminus B \cap B \setminus A$ - симметрическая разность
  %картинка
\end{definition}

\begin{definition}
  Мн-во упорядоченных пар ($i,j$), где $i \in A$, $j \in B$ называется прямым произведением мн-в A и B
  \[A \times B = \{(i,j)| i \in A,\q j \in B\}\]
\end{definition}

\begin{remark}
  Мощность прямого произведения $|A \times B| = |A| \cdot |B|$. Аналогично произведение $\forall$ конечного числа множеств
\end{remark}

\begin{definition}
  Пусть $A_1,...,A_k$ - ненулевые и попарно дизъюнктивные, $M = A_1 \cap ... \cap A_k$ и мн-во $\{A_1,...,A_k\}$ называется разбиением M\\
  (если они попарно не дизъюнктивные, то это покрытие)
  %картинка
\end{definition}

\begin{definition}
  Разбиение A мн-ва M называется измельчением B, если $\forall A_i \in A$ содержится в некотором $B_i \in B$
  %картинка
\end{definition}


\begin{definition}
  Пусть A, B - размельчения мн-ва M, разбиение C называется произведением A и B, если оно является из измельчением, причем самым крупным $C = A \cdot B$
  %картинка
\end{definition}

\begin{theorem}
  Произведение двух разбиений существует
\end{theorem}

\begin{proof}
  Предъявим разбиение, которое будет пересечением $A = \{A_1,...,A_k\}$ и $B = \{B_1,...,B_l\}$, точнее $D_{ij} = A_i \cup B_j,\q i \leqslant k,\q j \leqslant l$ и $\mathcal{P} = \cup D_{ij}$ (т.е. без пустых строк). Покажем, что тогда оно самое крупное.

  Пусть $\e F = \{F_1,...,F_t\}$ - измельчение A и B, тогда $\forall F_k\q \e A_{i_k},\ B_{i_k}: F_k \subdet A_{i_k},\ B_{i_k} \Ra F_k \subset (A_{i_k} \cup B_{i_k}) = D_{i_k j_k} \Ra$ мельче F
\end{proof}


\section{Вектора из нулей и единиц}
Пусть мн-во B состоит из двух элементов которые отождествляются с 0 и 1, т.е. $B = 0:1$

Произведение m экзмемпляров такого мн-ва обозначим за $B^m=(0:1)^m$, состоит из $2^m$ эл-ов
\begin{definition}
  Вектор из нулей и единиц - упорядоченный набор из фиксированного числа нулей и единиц, т.е. эл-т мн-ва $B^m$
\end{definition}
Упорядоченный набор из чисел оычно называется вектором, m - размерностью вектора, каждый отдельный элемент набора - компонента вектора
\begin{remark}
  Модели, в которых используются наборы из 0 и 1:
  \begin{enumerate}
    \item Геометрическая интерпретация

    Точкой в m-мерном пространстве является m-мерный вектор, каждая его компонента - одна из декартовых координат точки. Набор из 0 и 1, рассматриваемый как точка в пространстве, определяет вершину куба, построенного на ортах (единичных отрезках) координатных вероятностей
    %картинка
    \item Логичнская интерпретация

    Операции над векторами выполняются покомпонентно, т.е. независимо над соотв. компонентами векторов-операндов
    \begin{Example}
      \[\begin{matrix}
        x && 0 & 0 & 0 & 1 & 1\\
        y && 1 & 1 & 1 & 0 & 1\\
        x \wedge y && 0 & 0 & 0 & 0 & 1\\
        x \vee y && 1 & 1 & 1 & 1 & 1\\
        x \equiv y && 0 & 0 & 0 & 0 & 1\\
        x \not \equiv y && 1 & 1 & 1 & 1 & 0
      \end{matrix}\]
    \end{Example}
    \item Двоичное представление (натуральные числа)

    Число представляется в виде суммы степеней 2
    \item Состояние памяти компьютера
    \item Сообщение, передаваемое по каналу связи
    \item Можно задавать подмножества мн-ва $1:n$
  \end{enumerate}
\end{remark}

\section{Алгоритм перебора 0-1 векторов. Коды Грея}
\begin{definition}
  Код Грея — такое упорядочение k-ичных (обычно двоичных) векторов, что соседние вектора отличаются только в одном разряде
\end{definition}

\begin{alg}
  it - номер итерации, $k_{it}$ - номер обновляемой компоненты
  \[\begin{tabular}{c|c}
    x_4 \q x_3 \q x_2 \q x_1 & it \q k_{it}\\
    \hline
    0 \q 0 \q 0 \q 0 & 0 \q 1\\
    0 \q 0 \q \ul{0 \q 1} & 1 \q 2\\
    0 \q 0 \q 1 \q 1 & 2 \q 1\\
    0 \q \ul{0 \q 1 \q 0} & 3 \q 3\\
    0 \q 1 \q 1 \q 0 & 4 \q 1\\
    0 \q 1 \q \ul{1 \q 1} & 5 \q 2\\
    0 \q 1 \q 0 \q 1 & 6 \q 1\\
    \ul{0 \q 1 \q 0 \q 0} & 7 \q 4\\
    ... & ...
  \end{tabular}\]
  Суть алгоритма: зафиксируем нулевое значение у m-й компоненты и переберем все наборы длины $m-1$ для ост. компонент. Перебрав их меняем значение m-й компоненты на 1 и перебинаем набор длины $m-1$ в обратном порядке
\end{alg}

\begin{remark}
  Явная формула для проверки $G_i = i \oplus (\lfloor i/2 \rfloor)$
\end{remark}

\section{Перебор элементов прямого произведения множеств}
{\bf ВНИМАНИЕ! ВЫ ВСТУПАЕТЕ НА ЗЕМЛЮ ТУПОГО ПЕРЕПИСЫВАНИЯ ИЗ ТУПОГО ПЕРЕПИСЫВАНИЯ!!!}
\[M(1:k) = M_1 \times M_2 \times ... \times M_k\]
\[|M_1 \times M_2 \times ... \times M_k| = \prod_{i \in 1:k} m_i, \text{ где }m_i = |M_i|\]
Пусть каждое $M_i$ состоит из целых чисел от 0 до $m_i - 1$, тогда каждый элемент $M(1:k)$ - последовательность неотрицательных чисел $r_1,...,r_k$, причем $r_i < m_i$
\[\num(r_1,...,r_k) = \sum^k_{i=0} r_i \cdot (\prod_{j=1}^{i-1} m_j) = r_1 + r_2 m_1 + ... + r_k m_1 \cdot ... \cdot m_{k-1}\]

\section{Размещения, сочетания, перестановки без повторений}
\begin{definition}
  Перестановка из n без повторений - упорядоченный набор из n неповторяющихся элементов, каждый из которых берется из диапазона $1:n$
  \[|P_k| = n!\]
\end{definition}

\begin{definition}
  Размещение - упорядоченный набор из k неповторяющихся элементов из диапазона $1:n$
  \[A_n^k = \dfrac{n!}{(n-k)!} = n (n-1)(n-k+1)\]
\end{definition}

\begin{definition}
  Сочетание - набор из k неповторяющихся элементов из диапазона $1:n$ (порядок не важен)
  \[|C_n^k| = \dfrac{n!}{(n-k)! k!}\]
\end{definition}

\section{Размещения, сочетания, перестановки с повторениями}
Перестановки с повторениями:\\
Последовательность длины n, составленных из k разных символов, i-ый из которых повторяется $n_i$ раз ($n_1 + n_2 + ... + n_k = n$)
\begin{example}[aabc]
  Перестановки: $abac,\ baac,\ aabc,\ aacb,\ abca,\ baca,\ acba,\ acab,\ bcaa,\ cbaa,\ caba,\ caab$\\
  Число перестановок с повторениями длины из k разных элементов взятых соответственно по $n_1,...,n_k$ раз каждый обозначается
  \[P(n_1,n_2,...,n_k) = \dfrac{n!}{n_1! n_2! ... n_k!}\]
\end{example}

\section{Два алгоритма перебора перестановок. Нумерация перестановок}
\[|P_k| = k! = |T_k|\]
$P_k$ - мн-во всех перестановок\\
$T_k$ - произведение k любых таких множеств $M_i$, каждый из которых представляет собой мн-во чисел от 0 до $i-1$
\[T_k = \{0\} \times \{0,1\} \times ... \times \{0,1,...,k-1\}\]
Построим взаимно однозначное соответствие между $P_k$ и $T_k$. Возьмем перестановку ($t_1,...,t_k$) следующим образом: для любого $i \in 1:k$ найдем число значений, меньше $r_i$ среди $r_{i+1},...,r_k$ - это число мы и примем в качестве $t_i$\\
В соответсвии с таким определеничем чисел $t_i$ в мн-ве $T_k$ будет соответственно ??? значения $m_i$ не возраст., а убывающая до единицы

\begin{Example}[4, 8, 1, 5, 7, 2, 3, 6]
\[\begin{matrix}
  i & 1 & 2 & 3 & 4 & 5 & 6 & 7 & 8\\
  r_i & 4 & 8 & 1 & 5 & 7 & 2 & 3 & 6\\
  t_i & 3 & 6 & 0 & 2 & 3 & 0 & 0 & 0\\
  m_i & 8 & 7 & 6 & 5 & 4 & 3 & 2 & 1
\end{matrix}\]
По ($t_1,...,t_k$) легко восстановить исходную перестановку. Для этого меняя i от 1 до k нужно нужно проверить мн-во значений $S_i$, которые могут быть в перестановке на i месте. Для $i=1$ $S_1 = 1:8$, $t_1 = 3 \Ra r_1 = 4$, далее $S_2 = 1:3 \cap 5:8$, $t_2 = 6 \Ra r_2 = 8$. Если использовать это отображение при переборе, то перестановки будут перебираться в лексикографическом порядке
\end{Example}

\begin{definition}
  ($r_1,...,r_k$) предшествует ($R_1,...,R_k$), если начала перестановок совпадают до индекса d, а дальше $r_d < R_d$\\
\end{definition}

\begin{utv}
  Из этого перестановки перебираются в лексикографическом порядке, можно вывести правило получения следующего:
  \begin{enumerate}
    \item В ($r_1,...,r_k$) найти наибольший суффикс ($r_t,...,r_k$), в котором $r_t > ... > r_k$ ($r_{i-1} < r_t$) %произошел тролленг
    \item Выбрать ($r_t,...,r_k$) элемент следующий по велечине после $r_{t-1}$, поставить после в возр. порядке
    \[\begin{tabular}{c|cccc|cccc}
      \num & \multicolumn{4}{c}{t_k} & \multicolumn{4}{c}{p_k}\\
      \hline
      0 &  0 & 0 & 0 & 0 &  1 & 2 & 3 & 4\\
      1 &  0 & 0 & 1 & 0 &  1 & 2 & 4 & 3\\
      2 &  0 & 1 & 0 & 0 &  1 & 3 & 2 & 4\\
      3 &  0 & 1 & 1 & 0 &  1 & 3 & 4 & 2\\
      4 &  0 & 2 & 0 & 0 &  1 & 4 & 2 & 3\\
      5 &  0 & 2 & 1 & 0 &  1 & 4 & 3 & 2
    \end{tabular}\]
    Ещё один алгоритм
  \end{enumerate}
\end{utv}
\section{Задача о минимуме скалярного произведения}
Пусть заданы числа $x_1,...,x_m$ и $y_1,...,y_m$. Составим пары ($x,y$), включив каждое $x_i$ и $y_i$ ровно в одну пару. Затем перемножим числа каждой пары и сложим полученное произведение. Требуется найти $\min$ такое разбиение чисел на пары S
\begin{Theorem}
  \[\ol{x} = (x_1,...,x_n) \qq x_1 \geqslant x_2 ... \geqslant x_n\]
  \[\ol{y} = (y_1,...,y_n) \qq y_1 \leqslant y_2 ... \leqslant y_n\]
  \[S = \sum_{i=1}^n x_i y_i \ra \min\]
\end{Theorem}

\begin{proof}
  Покажем, что если найдутся пары чисел ($x_i,\ y_i$) и ($x_j,\ y_j$): $x_i < x_j$, $y_i < y_j$, то S можно уменьшить, заменив парами ($x_i,\ y_j$) и ($x_j,\ y_i$)\\
  Действительно,
\end{proof}

\section{Числа Фибоначчи. Теорема о представлении}
\begin{definition}
  Последовательность чисел Фибоначчи F:
  \[F_0 = 0,\q F_1=1,\q F_n = F_{n-1} + F_{n-2},\ n>1\]
\end{definition}

\begin{Utv}
  \[\varphi_n = \dfrac{F_{n+1}}{F_n} \text{ - сходится}\]
\end{Utv}

\begin{Consequence}
  \[\varphi_n = \dfrac{F_{n+1}}{F_n} = \dfrac{F_{n-1} + F_n}{F_n} = 1 + \dfrac{1}{\varphi}\]
  \[\Ra \varphi = \dfrac{\sqrt{5}+1}{2}\]
\end{Consequence}

\begin{lemma}
  При $n > 1$ выполнено $\varphi^{n+2} = v^{n+1} + v^n$
\end{lemma}

\begin{Proof}

\end{Proof}

\begin{lemma}
  При $k > 2$ выполнено:
  \[F_{2k} = F_{2k-1} + F_{2k-3} + ... + F_1\]
  \[F_{2k+1} = 1 + F_{2k} + F_{2k-2} + ... + F_0\]
\end{lemma}

\begin{proof}[по индукции]
  ($k=3$):
  \[F_6 = 8 = 5 + 2 + 1\]
  \[F_7 = 13 = 1 + 8 + 3 + 1 + 0\]
  ($k \ra k+1$):
  \[F_{2(k+1)} = F_{2k+2} = F{2k+1} + F_{2k} = F_{2k+1} + F_{2k-1} + ... + F_{1} = ???\]
\end{proof}

\begin{theorem}
  Любое натуральное число можно однозначно представить в виде суммы чисел Фибоначчи
  \[s = F_{i_0} + F_{i_1} + ... + F_{i+r}, \text{ где } i_{k-1} + 1 < i_k,\q k \in 1:r \q i_0 = 0\]
\end{theorem}

\begin{proof}
  Существование:

  Пусть $j(s)$ - номер масимального числа Фиббоначи, не превосходящего s. Положим $s'=s-F_j(s)$. Из определения $j(s)$ следует, что $s'<F_{j(s)-1}$, иначе число Фиббоначи не было бы максимальным. Теперь мы получим искобое представление для s как представление s', дополненное слагаемым $F_{j(s)}$\\ \\
  Единственность:

  Пусть есть ещё одно представление $s=F_{j_0}+...+F_{j_q}$. Н.У.О. считаем, что $j_q < j(s)$. Если мы заменим $F_{j_q}$ на $F_{j(q)-1}$, то правая часть разве что лишь увеличится. Аналогично заменим с возможным увеличением предпоследнее слагаемое на $F_{j(s)-3}$. ???
\end{proof}

\section{Перебор сочетаний. Нумерация сочетаний}
Состояние вычислительного процесса. Массив ($x_1,...,x_m$) номеров, включенных в сочетание. Начальное состояние: принять $x_i=i \q \forall i \in 1 : m$. Стандартный шаг: просматривать компоненты вектора $x$, начиная с $x_m$ и искать первую компоненту, которую можно увеличить (нельзя $x_m = n,\ x_{m-1} = n-1$ и т.д.). Если такой нет, то закончить процесс. В противном случае пусть k - наибольшее число, для которого $x_k < n - m + k$, тогда увеличиьть x на единицу, а для всех следующиъ за k-ый продолжаем, но ряд от значения $x_k$, т.е. $x_i=x_k+(i-k)$
\[\begin{tabular}{c|ccccc|c}
  \num & \multicolumn{5}{c}{\text{Сочетание}} & k\\
  \hline
  1 &  1 & 2 & 3 & 4 & 5 &  5\\
  2 &  1 & 2 & 3 & 4 & 6 &  5\\
  3 &  1 & 2 & 3 & 4 & 7 &  5\\
  4 &  1 & 2 & 3 & 5 & 6 &  4\\
  5 &  1 & 2 & 3 & 5 & 7 &  5
\end{tabular}\]
Удобно использовать вектора из 0 и 1, чтобы перенумеровать. С каждым сочетанием из n по m можно связать вектор из n нулей и единицЮ в крирпрм единиц ровно m - числа, входящие в данное сочетание просто задают номера этих единиц
\[\num (b[1:n],\ m) = \begin{cases}
  \num(b[1:n],\ n) & b[n] = 0\\
  l^{m}_{n-1} + \num(b[1:n],\ m-1) & b[n] = 1
\end{cases}\]

\begin{Example}
  \begin{multline*}
    $\num((0,\ 1,\ 0,\ 1,\ 0,\ 0,\ 1),\ 3) =\\
    \qq  = C^3_6 + \num((0,\ 1,\ 0,\ 1,\ 0,\ 0),\ 2) = C^3_6 + C^2_3 + \num((0,\ 1,\ 0),\ 1) =\\
    \qq \qq \qq = C^3_6 + C^2_3 + \num((0,\ 1),\ 1) = C^3_6 + C^2_3 + C^1_1 + \num((0),\ 0) = 24$
  \end{multline*}
\end{Example}

\section{Бином Ньютона и его комбинаторное использование}
Треугольник Паскаля (в узлах $C^k_n$):\\
\Longstack[l]{
n=0\\
n=1\\
n=2\\
n=3\\
n=4\\
n=5\\
n=6\qquad\ \\
}
\Longstack{
1\\
1\x 1\\
1\x 2\x 1\\
1\x 3\x 3\x 1\\
1\x 4\x 6\x 4\x 1\\
1\x 5\y 10\z 10\y 5\x 1\\
1\x 6\y 15\z 20\z 15\y 6\x 1\\
\overline{0\x 1\x 2\x 3\x 4\x 5\x 6}
}

\begin{definition}
  Бином Ньютона: $(a+b)^n = \sum_{k=0}^n C^k_n a^k b^{n-k}$
\end{definition}

\begin{Lemma}
  \[C^{k-1}_{n-1} + C^k_{n-1} = C^k_n\]
\end{Lemma}

\begin{Proof}[по индукции]
  \begin{multline*}
    $(a+b)^n = a (a+b)^{n-1} + b (a+b)^{n-1} = \\
    = \sum_{k=0}^{n-1} C^k_{n-1} a^{k+1} b^{(n-1)-k} +
    \sum_{k=0}^{n-1} C^k_{n-1} a^k b^{1+(n-1)-k} = \\
    = \sum^n_{k=1} C_{n-1}^{k-1} a^k b^{n-k} +
    \sum_{k=0}^{n-1} C^k_{n-1} a^k b^{n-k} =
    \sum_{k=0}^{n} (C^{k-1}_{n-1} + C^k_{n-1})a^k b^{n-k}$
  \end{multline*}
\end{Proof}

\begin{consequence}
  $a = 1,\q b = 1$:
  \[\sum_{k=0}^n C_n^k = 2^n\]
  $a = 1,\q b = -1$:
  \[\sum_{k=0}^n C_n^k (-1)^k = 0\]
  \[\text{(благодаря $C_n^k = C_n^{n-k}$)}\]
  $a = 1,\q b = i$:
  \[\sum_{k=0}^n C_n^k i^k = (1+i)^n = (\sqrt{2} \cdot \cos \frac{\pi}{4} + i \sin \frac{\pi}{4})^n = 2^{\frac{n}{2}} \cdot e^{i n \frac{\pi}{4}}\]
\end{consequence}

\section{Свойства биномиальных коэффициентов}
\begin{enumerate}
  \item $C_n^k = C_n^{n-k}$
  \item $C^{k-1}_{n-1} + C^k_{n-1} = C^k_n$
  \item $C^m_n C^k_m = C^k_n C^{m-k}_{n-k}$
\end{enumerate}

\section{Основные определения теории вероятностей}


%\section{Условные вероятности и формула Байеса}
\subfile{14.tex}

%\section{Математическое ожидание и дисперсия случайной величины}
\subfile{15.tex}

%\section{Схема Бернулли}
\subfile{16.tex}

%\section{Случайные числа. Схема Уолкера}
\subfile{17.tex}

\section{Двоичный поиск и неравенство Крафта}


\section{Энтропия. 2 леммы}


\section{Теорема об энтропии}


\section{Операции над строками переменной длины}


\section{Поиск образца в строке (Карпа-Рабина, Бойера-Мура)}


\section{Суффиксное дерево}


\section{Задача о максимальном совпадении двух строк}


\section{Код Шеннона-Фано. Алгоритм Хаффмена. 3 леммы}


%\section{Сжатие информации по методу Зива-Лемпеля}
\subfile{26.tex}

%\section{Метод Барроуза-Уилера}
\subfile{27.tex}

\section{Избыточное кодирование. Коды Хэмминга}


\section{Шифрование с открытым ключом}


\section{Сортировки (5 методов)}


\section{Информационный поиск и организация информации}


\section{Хеширование}


%\section{АВЛ-деревья}
\subfile{33.tex}

\section{B-деревья}


%\section{Биноминальные кучи}
\subfile{35.tex}

\section{Основные определения теории графов}


\section{Построение транзитивного замыкания}


\section{Обходы графа в ширину и глубину. Топологическая сортировка}


\section{Связность. Компоненты связности и сильной связности}


\section{Алгоритм поиска контура и построение диаграммы порядка}


\section{Теорема о связном подграфе}


\section{Деревья. Теорема о шести эквивалентных определениях дерева}


\section{Задача о кратчайшем остовном дереве. Алгоритм Прима}


\section{Алгоритм Краскала}


\section{Задача о кратчайшем пути. Алгоритм Дейкстры}


\section{Алгоритм Левита}


\section{Задача о кратчайшем дереве путей}


\section{Сетевой график и критические пути. Нахождение резервов работ}


\section{Задача о максимальном паросочетании в графе. Алгоритм построения}


\section{Теорема Кенига}


\section{Алгоритм построения контролирующего множества}


\section{Задача о назначениях. Венгерский метод}


\section{Задача коммивояжера. Метод ветвей и границ}


\section{Метод динамического программирования. Задача линейного раскроя}


\section{Приближенные методы решения дискретных задач. Жадные алгоритмы}


\section{Алгоритмы с гарантированной оценкой точности. Алгоритм Эйлера}


\section{Жадные алгоритмы. Задача о системе различных представителей}


\section{Приближенные методы решения дискретных задач}


\section{Конечные автоматы}


\section{Числа Фибоначчи. Производящие функции}


\section{Числа Каталана}


\section{?Алгоритм Кристофидеса (возможно будет)}

\end{document}
