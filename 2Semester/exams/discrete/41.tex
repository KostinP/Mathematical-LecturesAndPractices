\documentclass[discrete.tex]{subfiles}

\begin{document}

\section{Теорема о связном подграфе}

\begin{theorem}
    В связном графе $G=<M, N, T>$ можно выделить связный подграф $T=<M, N', T>$, где $|N'| = |M|-1 = k$ и можно пронумеровать вершины из $M$ числами $0:k$, а ребра $1:k$ т.ч. для любой дуги $u \in N'$:
    \[\num(u)=\max\{\num(\Beg u),\num(\End u)\}\]
\end{theorem}

\begin{proof}
  Используем индукцию. Пудем строить частичный подграф графа $<M',N',T>$, расширяя его мн-во вершин и мн-во дуг до тех пор, пока само мн-во $M'$ не совпадет с самим $M$. При этом на каждом шаге построения число вершин будет на 1 больше числа дуг, частичный подграф будет связным, и на его вершинах и дугах будет определена требуемая нумерация.

  Выберем произвольную вершину $i_0 \in M$ и примем первоначально $M' = \{i_0\}$, $N' = \varnothing$, $\num (i_0) = 0$. То есть в начале $k=0$

  Предположим, что уже построен частичный подграф $<M',N',T>$, обладающий перечисленными св-ми и такой, что $M \subset M' \neq \varnothing$

  Выберем какую-либо вершину $\ol{i}$, не входящую ещё в $M'$. Так как граф $<M,N,T>$ связен, существует цепь, соединяющая $i_0$ и $\ol{i}$, В этой цепи начальная вершина принадлежит $M'$, а конечная $\ol{i}$ не принадлежит, и значит, в цепи найдутся две последовательно две нумерованные вершины $i'$ и $i''$, соединенные дукой $u'$, такие что $i' \in M'$, $i'' \neq M$. Добавим теперь вершину $i''$ в мн-во $M'$, а дугу $u'$ в мн-во $N'$, увеличив число $j$ на 1 и приписав вершине и дуге это новое значение $k$ в качестве их номера в нумерации. Новые $M'$ и $N'$ обладают всеми перечисленными св-ми: дуг на 1 меньше чем вершин, граф связен, а така как все старые вершины были соединены между собой цепью, по предположению, $i''$ имеет соединение с $i'$, а через нее со всеми остальными вершинами. Нумерация определена, а на ней выполнено соотношение, в том числе и для новой дуги, номер которой совпадает с номером $i''$ и больше номера $i'$, который был введен раньше

  Этот процесс можно продолжать, пока не множество $M'$ не совпадет с $M$
\end{proof}

\begin{consquence}
  Если $|N| < |M| - 1$, граф не может быть связным
\end{consquence}

\begin{consquence}
  Если $|N| > |M| - 1$, граф содержит циклы
\end{consquence}

\end{document}
