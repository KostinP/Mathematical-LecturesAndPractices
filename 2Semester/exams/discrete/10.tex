\documentclass[discrete.tex]{subfiles}

\begin{document}
  \section{Перебор сочетаний. Нумерация сочетаний}
  
  Состояние вычислительного процесса. Массив ($x_1,...,x_m$) номеров, включенных в сочетание. Начальное состояние: принять $x_i=i \q \forall i \in 1 : m$. Стандартный шаг: просматривать компоненты вектора $x$, начиная с $x_m$ и искать первую компоненту, которую можно увеличить (нельзя $x_m = n,\ x_{m-1} = n-1$ и т.д.). Если такой нет, то закончить процесс. В противном случае пусть k - наибольшее число, для которого $x_k < n - m + k$, тогда увеличиьть x на единицу, а для всех следующиъ за k-ый продолжаем, но ряд от значения $x_k$, т.е. $x_i=x_k+(i-k)$
  \[\begin{tabular}{c|ccccc|c}
    \num & \multicolumn{5}{c}{\text{Сочетание}} & k\\
    \hline
    1 &  1 & 2 & 3 & 4 & 5 &  5\\
    2 &  1 & 2 & 3 & 4 & 6 &  5\\
    3 &  1 & 2 & 3 & 4 & 7 &  5\\
    4 &  1 & 2 & 3 & 5 & 6 &  4\\
    5 &  1 & 2 & 3 & 5 & 7 &  5
  \end{tabular}\]
  Удобно использовать вектора из 0 и 1, чтобы перенумеровать. С каждым сочетанием из n по m можно связать вектор из n нулей и единиц, в криром единиц ровно m - числа, входящие в данное сочетание просто задают номера этих единиц
  \[\num (b[1:n],\ m) = \begin{cases}
    \num(b[1:n],\ n) & b[n] = 0\\
    l^{m}_{n-1} + \num(b[1:n],\ m-1) & b[n] = 1
  \end{cases}\]

  \begin{Example}
    \begin{multline*}
      $\num((0,\ 1,\ 0,\ 1,\ 0,\ 0,\ 1),\ 3) =\\
      \qq  = C^3_6 + \num((0,\ 1,\ 0,\ 1,\ 0,\ 0),\ 2) = C^3_6 + C^2_3 + \num((0,\ 1,\ 0),\ 1) =\\
      \qq \qq \qq = C^3_6 + C^2_3 + \num((0,\ 1),\ 1) = C^3_6 + C^2_3 + C^1_1 + \num((0),\ 0) = 24$
    \end{multline*}
  \end{Example}
\end{document}
