\documentclass[discrete.tex]{subfiles}

\begin{document}

\section{Размещения, сочетания, перестановки без повторений}
\begin{definition}
  Перестановка из n без повторений - упорядоченный набор из n неповторяющихся элементов, каждый из которых берется из диапазона $1:n$
  \[|P_k| = n!\]
\end{definition}

\begin{definition}
  Размещение - упорядоченный набор из k неповторяющихся элементов из диапазона $1:n$
  \[A_n^k = \dfrac{n!}{(n-k)!} = n (n-1)(n-k+1)\]
\end{definition}

\begin{definition}
  Сочетание - набор из k неповторяющихся элементов из диапазона $1:n$ (порядок не важен)
  \[|C_n^k| = \dfrac{n!}{(n-k)! k!}\]
\end{definition}

\end{document}
