\documentclass[discrete.tex]{subfiles}

\begin{document}

\section{Поиск образца в строке (Карпа-Рабина, Бойера-Мура)}
ИСПРАВИТЬ ЭТО. ДОПОЛНИТЬ.
\begin{definition}
  Пусть заданы две строки: t (text) и p (pattern). Говорят, что образец входит точно в текст с позиции j, если $t[j:j+m-1] = p[1:m]$
\end{definition}

Наивный метод: ходим по строке, ищем первый символ, затем второй... Сложность $m \cdot n$

\begin{definition}
  Пусть задана числовая последовательность $p_1,...,p_n$, определим скользящую сумму как $s_i=p_i+p_{i+1}+...+p_{r+i}$, где r - некоторое фиксированное число
\end{definition}

\begin{remark}
  Нетрудно заметить, что $s_{k+1} = s_k - p_k + p_{k+r}$
\end{remark}

ДОПИСАТЬ
%Тогда если захешировать текст текст при $r = o.Length$, то можно будет быстро искать слова длины r. Если проделать то же самое для $r=1:t.Length$, получим поиск любой строки за $o(m+n)$, но при трате памяти $o(n^2)$
%Аналогично можно сделать меньше повторений скользящих сумм, если для фиксированного x рассмотреть $\sigma_k(x) = \sum_{i=0}^{r-1} p_{k+i} x^{r-i-1} = p_k x^{r-1} + p_{k+1} x^{r-2} + ... + p_{k+r-1}$, можно получить рекурсивно $\sigma_{k+1}(x) = \sigma_k(x) - p_k x^r + p_{k+r}$

\begin{alg}[метод Карпа-Раббина]
  $o(m+n)$, но по памяти $o(n^2)$ че блять, Романовский, соси хуй
\end{alg}

\begin{alg}[метод Бойера-Мура]
  Сравнение начинается с последнего символа образца, который совмещается с началом. Если совпал $\Ra$ нашли. Если нет, пользуемся эвристиками:
  \begin{enumerate}
    \item Эвристика стоп-символа.
    \begin{enumerate}
      \item Если не совпало и текущего символа нет в строке, то следующую проверку можно сдвинуть на длину образа\\
      $\begin{matrix}
        \text{строка:} & . & . & . & . & . & . & \fbox{\text{П}} & . \q . \q . \q . \q . \q . \q .\\
        \text{шаблон:} & \text{К} & \text{О} & \text{Л} & \text{О} & \text{К} & \text{О} & \fbox{\text{Л}} & & \\
        \text{next step:} & & & & & & & & \ \text{К} \ \ \text{О}\ \ \text{Л}\ \ \text{О}\ \ \text{К}\ \ \text{О}\ \ \text{Л}
      \end{matrix}$
      \item Если такой символ есть, сдвигаемся до последнего вхождения в образце\\
        $\begin{matrix}
        \text{строка:} & . & . & . & . & . & . & \fbox{\text{К}} & . & .\\
        \text{шаблон:} & \text{К} & \text{О} & \text{Л} & \text{О} & \fbox{\text{К}} & \text{О} & \text{Л}\\
        \text{next step:} & & & \text{К} & \text{О} & \text{Л} & \text{О} & \fbox{\text{К}} & \text{О} & \text{Л}
        \end{matrix}$
    \end{enumerate}
    \item Правило хорошего окончания\\
    $\begin{matrix}
      \q &\text{строка:} & ... & \text{К} & \text{КОЛ} & ...\\
      &\text{шаблон:} & \text{КОЛ} & \text{О} & \text{КОЛ}\\
      &\text{next step:} & & & \text{КОЛ} & \text{ОЛОКОЛ}
    \end{matrix}$\\
    Пример для суффиксов и сдвигов: $\clock - 1,\ \text{Л} - 4,\ \text{ОЛ} - 4,\ \text{КОЛ} - 4$
  \end{enumerate}
\end{alg}

\begin{aalg}[Кнута-Морриса-Пратта]

\end{aalg}
\end{enumerate}
\end{document}
