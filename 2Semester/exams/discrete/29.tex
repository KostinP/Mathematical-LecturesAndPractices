\documentclass[discrete.tex]{subfiles}

\begin{document}
  \section{Шифрование с открытым ключом}
  В системах с шифрованием с открытым ключом используется два ключа: один для шифрования, другой для дешифровки

  У каждой стороны есть два ключа: публичный и секретный, который знает только его владелец. Для того чтобы A послал зашифрованное сообщение стороне B необходимо использовать публичный ключ B, потому что лишь B обладает секретным ключом для дешифровки этого сообщения

  Примеры:
  \begin{enumerate}
    \item "Рюкзак"{} Меркла-Хеллмана
    \item El Camol
    \item RSA - один из самых популярных
  \end{enumerate}

  \begin{alg}[RSA]
    n - большое целое число, пр-ние двух простых сомножителей p и q
    \[\varphi = p \cdot q - p - q + 1\]
  \end{alg}

  \begin{Lemma}[1]
    \[a^{\varphi} = 1 (\mod n) \qq \text{a взаимно просто с n} \]
  \end{Lemma}

  \begin{proof}
    $\Phi(n)$ - мн-во всех чисел, не превосходящих n и вз. простых с ним
    \[|\Phi(n)| = \varphi\]
    Если $ba \os{\mod n}{=} ca$, то $c-b \devide n$. Таким образом
    \[\prod_{b \in \Phi(n)} b = \prod_{b \in \Phi(n)} (ba) = a^{\varphi} \prod_{b \in \Phi(n)} b \mod n\]
  \end{proof}

  \begin{lemma}[2]
    $\forall$e вз. простого с $\varphi$ $\e! d \in 1:n$, для которого $e d = 1 (\mod \varphi)$
  \end{lemma}

  \begin{proof}
    Если $\gcd(e,\varphi) = 1$, то $\e d,k$: $de + k\varphi = 1$ (по алгоритму Евклида). Верно $\forall$ вз. простых $e$ и $\varphi$. Однозначность определения d получается док-ом от противного
  \end{proof}

  \begin{alg}[RSA]
    Получатель шифра выбирает $e$, вз. простое с $\varphi$ и вычисляет "обратное" для него d. n и e сообщаются отправителям, это открытая часть системы шифрования

    Отправленный текст перестраивается так, чтобы он состоял из отдельных чисел в диапазоне от 1 до n. Каждое кодируемое число x возводится в степень e по $\mod n$. Результат y передается получателю. Он имея число d, вычисляет $y^d$ по $\mod n$. Т.к. $y^d = (x^e)^d = x^{ed}$, то если $x \in \Phi(n) \Ra x^{de} = x$
  \end{alg}

  \begin{Example}
    \[n=1093709 = \os{p}{997} \cdot \os{q}{1097}\]
    \[\varphi = 1091616\]
    \[e = 397 \qq d = 145777\]
    n и e сообщаются отправителям\\

    Кодируем текст:
    \[x^397 = x^{110001101_2} = x^{256} x^{128} x^{8} x^{4} x^1 = y\]

    Отправляем:
    \[\text{Получатель вычисляет $y^d$ по $\mod n = x$}\]
  \end{Example}

\end{document}
