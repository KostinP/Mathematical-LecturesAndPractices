\documentclass[discrete.tex]{subfiles}

%Для бинома Ньютона
\def\x{\hspace{3ex}}    %BETWEEN TWO 1-DIGIT NUMBERS
\def\y{\hspace{2.45ex}}  %BETWEEN 1 AND 2 DIGIT NUMBERS
\def\z{\hspace{1.9ex}}    %BETWEEN TWO 2-DIGIT NUMBERS
\stackMath

\begin{document}
  \section{Бином Ньютона и его комбинаторное использование}
  
  Треугольник Паскаля (в узлах $C^k_n$):\\
  \Longstack[l]{
  n=0\\
  n=1\\
  n=2\\
  n=3\\
  n=4\\
  n=5\\
  n=6\qquad\ \\
  }
  \Longstack{
  1\\
  1\x 1\\
  1\x 2\x 1\\
  1\x 3\x 3\x 1\\
  1\x 4\x 6\x 4\x 1\\
  1\x 5\y 10\z 10\y 5\x 1\\
  1\x 6\y 15\z 20\z 15\y 6\x 1\\
  \overline{0\x 1\x 2\x 3\x 4\x 5\x 6}
  }

  \begin{definition}
    Бином Ньютона: $(a+b)^n = \sum_{k=0}^n C^k_n a^k b^{n-k}$
  \end{definition}

  \begin{Lemma}
    \[C^{k-1}_{n-1} + C^k_{n-1} = C^k_n\]
  \end{Lemma}

  \begin{Proof}[по индукции]
    \begin{multline*}
      $(a+b)^n = a (a+b)^{n-1} + b (a+b)^{n-1} = \\
      = \sum_{k=0}^{n-1} C^k_{n-1} a^{k+1} b^{(n-1)-k} +
      \sum_{k=0}^{n-1} C^k_{n-1} a^k b^{1+(n-1)-k} = \\
      = \sum^n_{k=1} C_{n-1}^{k-1} a^k b^{n-k} +
      \sum_{k=0}^{n-1} C^k_{n-1} a^k b^{n-k} =
      \sum_{k=0}^{n} (C^{k-1}_{n-1} + C^k_{n-1})a^k b^{n-k}$
    \end{multline*}
  \end{Proof}

  \begin{consequence}
    $a = 1,\q b = 1$:
    \[\sum_{k=0}^n C_n^k = 2^n\]
    $a = 1,\q b = -1$:
    \[\sum_{k=0}^n C_n^k (-1)^k = 0\]
    \[\text{(благодаря $C_n^k = C_n^{n-k}$)}\]
    $a = 1,\q b = i$:
    \[\sum_{k=0}^n C_n^k i^k = (1+i)^n = (\sqrt{2} \cdot( \cos \frac{\pi}{4} + i \sin \frac{\pi}{4}))^n = 2^{\frac{n}{2}} \cdot e^{i n \frac{\pi}{4}}\]
  \end{consequence}
\end{document}
