\documentclass[discrete.tex]{subfiles}

\begin{document}
  \section{Информационный поиск и организация информации}

  \begin{definition}
    База данных - совокупность равноправный записей, каждая из которых имеет информационную и идентифицирующие части

    Идентеф. часть = ключ записи (уникален для каждого)
  \end{definition}

  Поиск ключа:
  \begin{enumerate}
    \item Дихтомия (деление поисковой области пополам)\\
    Если ключ $\in \Z$, рассматриваем обл. его значений ($a:b$) выьирается промежуточное значение $d$

    Рассмотрим $a:(d-1)$ $d:b$. Смотрим в какой области искомый ключ и т.д. продолжается до тех пор, пока область не $\varnothing$ (нет ключа) или не стала обозримой (одна запись или небольшой список)

    Можно рассматривать как дерево

    $\sim j \log_2 N$ (при N записях)

    Критерии качества:
    \begin{enumerate}
      \item Затраты на поиск в худшем случае
      \item Мат. ожидание, затрач. на поиск
    \end{enumerate}
  \end{enumerate}

\end{document}
