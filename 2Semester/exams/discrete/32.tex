\documentclass[discrete.tex]{subfiles}

\begin{document}
  \section{Хеширование}
  Самый удобный метод. инф. поиска - массив (лучше всего одномерный)

  элемент $\ra$ индекс

  Элемент с индексом j разыскивается выискиванием его адреса $A_j = A_0 + \Delta A \cdot j$\\
  $A_0$ - адрес элемента с нулевым индексом, $\Delta A$ - шаг массива (расстояние между послед. записями)

  Если диапазон индексов велик, то использовать уже трудно и затратно

  Рассмотрим $\varphi: k \ra I$, где k - мн-во ключей, I - приемл. диапазон индексов. Функция распределяет по индексам ключи почти равномерно, устраняя зависимости.

  \begin{example}
    Для строк - сопоставим символу кода из ASCI $d = \{d_i\}$
    \[A(d) = \sum_i r^2 d_i \text{ - полиноминальная взв. сумма}\]
    \[r \in \Z \qq (\r,I) \text{ - ВП}\]
    Производим вычисления в остатках по $\mod I$
  \end{example}

  \begin{task}
    На задано некоторое количество точке $S=\{s_i\},\ i \im 1:m$, и требуется найти те пары точек $i,j \in 1:m$, расстояние между которыми меньше заданого порогового значения d. Попарное сравнение точек имеет сложность $O(m^2)$.

    Для ускорения полезным оказывается алгоритм П.Элайеса. Найдем прямоугольник, содержащий все точки, и разобьем его на достаточно мелкие одинаковые квадраты. Каждый такой квадрат будет адресовываться двуся целыми числами $i_x$ и $i_y$ - порядковыми номерами по ости X и Y. Каждая точка попадает в какой-то квадрат и теперь для близких точек достаточно сравнивать между собой точки одного квадрата и точки близких квадратов.

    Таких квадратов может оказаться слишклм много, но многие их них могут оказаться пустыми. Чтобы не хранить это все, мы введем асссоциативный массив - специальную конструкцию, которая внешне будет выглядеть "почти как массив"{}. При её создании нам и понадобится хеширование.

    Каждому квадрату, т.к. определяющей его паре чисел ($i_x,i_y$), сопоставим индекс $\psi(i_x,i_y) = (a i_x + b i_y) \mod D$, где коэффициенты a и b выбраны взаимно простыми между собой и с D - размером хеш-массива.
  \end{task}
\end{document}
