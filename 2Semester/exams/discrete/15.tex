\documentclass[discrete.tex]{subfiles}

\begin{document}
  \section{Математическое ожидание и дисперсия случайной величины}

  \begin{definition}
    Пусть $\Omega = \{\omega\}$ - множество элементарных событий (произвольное непустое множество, элементы которого - элементарные события), $p_{\omega}$ - вероятность элементарного события $\omega$, тогда функция f: $\Omega \ra \R$ называется случайной величиной.
  \end{definition}

  \begin{definition}
    Случайные величины $\xi$ и $\eta$ называют незавсисимы, если независимы события $\{\omega \ |\ \xi(\omega) = a\}$ и $\{\omega \ |\ \eta(\omega) = b\}$ для любых значений a и b
  \end{definition}

  \begin{example}
    Пусть мы подбрасываем монеты с Васей по очереди. $\Omega = \{\text{ор, ре}\}$. Пусть мы выигрываем соответственно $10,\ -33$, а Вася $10,\ -30$.

    Тогда $\xi(\text{ор.})=5$, $\xi(\text{ре.})=-10$, а у Васи $\eta(\text{ор.})=10$, $\xi(\text{ре.})=-30$. Мы знаем, что $\xi,\ \eta$ - независмы, то есть:
    \[P(\xi=a,\ \eta=b) = P(\xi=a) \cdot P(\eta=b) \q \forall a,b\]
    Где $a,b$ - 5, 10, -33, -30,..., а P - вероятность события
  \end{example}

  \begin{Definition}
      \[E(\xi) = \sum_{\omega \in \Omega} \xi(\omega) \rho_{\omega} = \sum_a a \cdot P(\xi = a) \ \text{ назыается мат. ожиданием случайной
      величины } \xi\]
  \end{Definition}

  \begin{properties}
      \begin{enumerate}
          \item \q Если $P(\xi = a) = 1 \Ra E(\xi) = a$
          \item \q $\xi, \eta \text{ - случ. вел. }  \q E(\xi + \eta) =
              E(\xi) + E(\eta)$ \qq (Линейность)
          \item \q Если $\xi = c \eta$, где $c = const$ \q $E(\xi) = cE(\eta)$
          \item \q $E(\xi \cdot \eta) = E(\xi) \cdot E(\eta) \qq$ если $
              \xi $ и $\eta$ нез.
      \end{enumerate}
  \end{properties}

  \begin{Proof}
      \[\text{Линейность } \qq E(\xi + \eta) = \sum_{\omega \in \Omega} (\xi(\omega) +
      \eta(\omega))\rho_{\omega} = \sum_{\omega \in \Omega} \xi(\omega) \rho_{\omega} + \]
      \[+ \sum_{\omega \in \Omega} \eta(\omega) \rho_{\omega} = E(\xi) + E(\eta)\]
  \end{Proof}

  \begin{definition}
      Мат. ожидание квадрата отклонения случайной величины от ее мат. ожидания называется
      дисперсией этой случайно величины
      \[D(\xi) = E(\xi - E(\xi))^2\]
      Дисперсия характеризует разброс случайной величины вокруг ее мат. ожидания
      \[D(\xi) = E(\xi - E(\xi))^2 = E(\xi ^ 2) - E(2 \xi E(\xi)) +
      E(E^2(\xi)) = \]
      \[= E(\xi^2) - 2E(\xi) E(E(\xi)) + E^2(\xi) =
      E(\xi^2) - E^2(\xi)\]
  \end{definition}

  \begin{properties} [Дисперсии]
      \begin{enumerate}
          \item Дисперсия неотрицательна. Если $P(\xi = a) = 1$, то $D(\xi) = 0$
          \item $\xi = \eta + c, \qq c = const, \qq \Ra D(\xi) = D(\eta)$
          \item $\xi = c\eta, \qq c = const,  \qq \Ra D(\xi) = c^2D(\eta)$
          \item $\xi $ и $\eta$ нез. с.в.  $\Ra D(\xi + \eta) = D(\xi) +
              D(\eta)$
      \end{enumerate}
  \end{properties}

  \begin{Proof}
      \begin{enumerate}
        \item (очевидно)
        \item $xi = \eta + c$
        \[E(\xi) = \sum_{\omega \in \Omega} \xi(\omega) \rho_{\omega} =
        \sum_{\omega \in \Omega} (\eta(\omega) + c) \rho_{\omega} =  \sum_{\omega \in \Omega}\eta(\omega)\rho_{\omega} +
            \sum_{\omega \in \Omega}c \rho_{\omega} =\]
        \[= E(\eta) + c\]
        \[D(\xi) = E(\xi - E(\xi))^2 = E(\eta + c - E(\eta) - c)^2 = D(\eta)\]
        \item $D(\xi) = E(c\eta - E(c\eta))^2 = E(c\eta - cE(\eta))^2 =
        c^2E(\eta - E(\eta))^2 = c^2 D(\eta)$
        \item $D(\xi + \eta) = E(\xi + \eta - E(\xi + \eta))^2 =$
        \[= E(\xi - E(\xi) + \eta - E(\eta))^2 = E(\xi - E(\xi))^2 +
        2E((\xi - E(\xi)(\eta - E(\eta))))) + \]
        \[+ E(\eta - E(\eta))^2 = D(\xi) + D(\eta) + 2(E(\xi \eta - \xi
        E(\eta) - \eta E(\xi) + E(\xi)E(\eta))) =\]
        \[=  D(\xi) + D(\eta) + 2(E(\xi\eta) - E(\xi)E(\eta)) -
        E(\eta)E(\xi) + E(\xi)E(\eta)) = D(\xi) + D(\eta)\]
      \end{enumerate}
  \end{Proof}

  \begin{definition}
    Корень из дисперсии называется средним квадратичным отклонением
  \end{definition}

  \begin{Definition}
    \[\rho(\xi,\ \eta) \eqdef \frac{m(\xi,\ \eta)}{\sqrt{D(\xi) D(\eta)}} = \frac{E(\xi \eta) - E(\xi) E(\eta)}{\sqrt{D(\xi) D(\eta)}}\]
  \end{Definition}
\end{document}
