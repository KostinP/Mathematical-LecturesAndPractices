\documentclass[discrete.tex]{subfiles}

\begin{document}
  \section{Размещения, сочетания, перестановки с повторениями}
  
  \begin{definition}
    Перестановка - последовательность длины n, составленная из k разных символов, i-ый из которых повторяется $n_i$ раз ($n_1 + n_2 + ... + n_k = n$)
    \[P(n_1,n_2,...,n_k) = \frac{n!}{n_1! \cdot ... \cdot n_k!}\]
  \end{definition}

  \begin{example}[Перестановки с повторениями, aabc]
    Перестановки:
    \[abac,\ baac,\ aabc,\ aacb,\ abca,\ baca,\ acba,\ acab,\ bcaa,\ cbaa,\ caba,\ caab\]
  \end{example}

  \begin{definition}[размещения с повторениями]
    Аналогичное определение
    \[\ol{A}_n^k = n^k\]
  \end{definition}

  \begin{definition}[сочетания с повторениями]
    Аналогичное определение
    \[|\ol{C}_n^k| = C^k_{n+k-1} = C^{n-1}_{n+k-1}\]
  \end{definition}
\end{document}
