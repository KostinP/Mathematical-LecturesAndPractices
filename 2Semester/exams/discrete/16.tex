\documentclass[discrete.tex]{subfiles}
 
\begin{document}

\section{Схема Бернулли}

\begin{task}
    Стрелок делает 5 выстрелов, какова вероятность того, что он попадет не меньше 4 раз?
    \[p \text{ - вероятность попасть в мишень}\]
    \[P_5(A) = P_5(4) + P_5(5) = P_5(4) + p^5\]
    \[P_5(A) = C_5^4 \cdot p^4 (1- p) + p^5\]
    hint: мы выбираем, когда стрелок промахнется из всех выстрелов
\end{task}

\begin{Definition}
    \[P_n(m) = C_n^m p^m (1 - p)^{n - m}  \q \text{формула Бернулли}\]
    \[n \text{ - число попыток } \q m \text{ - удачных событий}\]
\end{Definition}

\end{document}
