\documentclass[discrete.tex]{subfiles}

\begin{document}
  \section{Схема Бернулли}

  \begin{task}
      Стрелок делает 5 выстрелов, какова вероятность того, что он попадет не меньше 4 раз?
      \[p \text{ - вероятность попасть в мишень}\]
      \[P_5(A) = P_5(4) + P_5(5) = P_5(4) + p^5\]
      \[P_5(A) = C_5^4 \cdot p^4 (1- p) + p^5\]
      hint: мы выбираем, когда стрелок промахнется из всех выстрелов
  \end{task}

  \begin{Definition}
      \[P_n(m) = C_n^m p^m (1 - p)^{n - m}  \q \text{формула Бернулли}\]
      \[n \text{ - число попыток } \q m \text{ - удачных событий}\]
  \end{Definition}

  \begin{proof}
    Рассмотрим последовательность независмых, случайных велечин $\delta_1,...,\delta_n,...$, каждая из которых принимает зва значения: 1 с вероятностью $p$ и 0 с вероятнсотью $q=1-p$. Такая вероятностная схема называется схемой Бернулли.

    Случайная величина $\xi_n$, полученная при сложении n таких случайных велечин $\delta_i$ имеет распределение, называемое биноминальным распределением. Чтобы $\xi_n = k$, нужно чтобы ровно k из случайных велечин $\delta_1,...,\delta_n$ принимали значение 1, а остальные должны равняться нулю. Вероятность этого события при фиксированных местах единиц и нулей равна $p^k q^{n-k}$, и если учесть все возможные $C^k_n$ расположений этих мест, то получим $P(\xi_n = k) = C^k_n p^k q^{n-k}$ - k-ый член биноминального распределения
  \end{proof}

  \begin{Remark}
    \[E(\xi_n) = np, \qq D(\xi_n) = npq\]
  \end{Remark}
  *это когда-нибудь будет дополнено*
\end{document}
