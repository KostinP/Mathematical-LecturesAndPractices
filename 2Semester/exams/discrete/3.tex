\documentclass[discrete.tex]{subfiles}

\begin{document}

\section{Алгоритм перебора 0-1 векторов. Коды Грея}
\begin{definition}
  Код Грея — такое упорядочение k-ичных (обычно двоичных) векторов, что соседние вектора отличаются только в одном разряде
\end{definition}

\begin{alg}
  it - номер итерации, $k_{it}$ - номер обновляемой компоненты
  \[\begin{tabular}{c|c}
    x_4 \q x_3 \q x_2 \q x_1 & it \q k_{it}\\
    \hline
    0 \q 0 \q 0 \q 0 & 0 \q 1\\
    0 \q 0 \q \ul{0 \q 1} & 1 \q 2\\
    0 \q 0 \q 1 \q 1 & 2 \q 1\\
    0 \q \ul{0 \q 1 \q 0} & 3 \q 3\\
    0 \q 1 \q 1 \q 0 & 4 \q 1\\
    0 \q 1 \q \ul{1 \q 1} & 5 \q 2\\
    0 \q 1 \q 0 \q 1 & 6 \q 1\\
    \ul{0 \q 1 \q 0 \q 0} & 7 \q 4\\
    ... & ...
  \end{tabular}\]
  Суть алгоритма: зафиксируем нулевое значение у m-й компоненты и переберем все наборы длины $m-1$ для ост. компонент. Перебрав их меняем значение m-й компоненты на 1 и перебинаем набор длины $m-1$ в обратном порядке
\end{alg}

\begin{rremark}
  Явная формула для проверки $G_i = i \oplus (\lfloor i/2 \rfloor)$
\end{rremark}

\end{document}
