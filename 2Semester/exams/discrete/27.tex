\documentclass[discrete.tex]{subfiles}

\begin{document}
  \section{Метод Барроуза-Уилера}

  \begin{alg}
      \begin{enumerate}
          \item Составляется таблица всех циклических сдвигов строки.
          \item Производится лексикографическая сортировка строк таблицы.
          \item В качестве выходной строки выбирается последний столбец таблицы преобразования и номер строки, совпадающей с исходной.
      \end{enumerate}
  \end{alg}

  \begin{Example}
      \[
      \begin{tabular}{c|c|c|c}
          \text{Вход} & \text{ц. сдвиги} & \text{сортировка} & \text{Выход}\\ \hline
          & \ul{abacaba} & aabacab & \\ \hline
          & bacabaa & abaabac &\\ \hline
          & acabaab & \ul{abacaba} & \\ \hline
          abacaba & cabaaba & acabaab & bcabaaa, 3\\ \hline
                  & abaabac & baabaca &\\\hline
                  & baabaca & bacabaa &\\\hline
                  & aabacab & cabaaba &
      \end{tabular}
      \]
      BWT("abacaba") = ("bcabaaa"{}, 3)
  \end{Example}

  \begin{alg} [обратного преобразования]
      Пусть нам дали BWT(S) = (A, x)\\ Тогда выпишем в столбик A, отсортируем, слева допишем
      A, снова отсортируем, так n раз,  где n - длина строки A.
      После последней сортировки мы получим, что строка с номером x - S
  \end{alg}
\end{document}
