\documentclass[discrete.tex]{subfiles}

\begin{document}
  \section{Задача о минимуме скалярного произведения}
  
  Пусть заданы числа $x_1,...,x_m$ и $y_1,...,y_m$. Составим пары ($x,y$), включив каждое $x_i$ и $y_i$ ровно в одну пару. Затем перемножим числа каждой пары и сложим полученное произведение. Требуется найти $\min$ такое разбиение чисел на пары S
  \begin{Theorem}
    \[\ol{x} = (x_1,...,x_n) \qq x_1 \geqslant x_2 ... \geqslant x_n\]
    \[\ol{y} = (y_1,...,y_n) \qq y_1 \leqslant y_2 ... \leqslant y_n\]
    \[S = \sum_{i=1}^n x_i y_i \ra \min\]
  \end{Theorem}

  \begin{proof}
    Покажем, что если найдутся пары чисел ($x_i,\ y_i$) и ($x_j,\ y_j$): $x_i < x_j$, $y_i < y_j$, то S можно уменьшить, заменив парами ($x_i,\ y_j$) и ($x_j,\ y_i$)

    Действительно, так как $(x_j-x_i)(y_j-y_i)>0$, то, раскрывая скобки, получим после переноса $x_i y_i + x_j y_j > x_i y_j + x_j y_i$

    Поскольку число возможных расположений равно m!, т.е. конечное число, то начиная с любого расположения за конечное число шагов мы закончим процесс улучшений на расположении, которое дальше улучшить невозможно. На нем и достигается минимум
  \end{proof}
\end{document}
