\documentclass[discrete.tex]{subfiles}

\begin{document}

\section{Основные определения теории вероятностей}
ИСПРАВИТЬ ЭТО. В РОМАНОВСКОМ ЛУЧШЕ, ЧЕМ В КОНСПЕКТЕ ДЕВУШКИ
\begin{definition}
  A - событие, $p(A) \subset [0;1]$ - характеристика события, p - вероятность события A
\end{definition}

\begin{definition}
  S - мн-во элементарных событий (мн-во исходов), если его элементы равноправны
\end{definition}

\begin{definition}
  Пусть A - событие, $A \subset S$, тогда:
  \[\frac{\abs{A}}{\abs{S}} = p(A)\]
  Событие с вероятностью 1 называется достоверным, событие с вероятностью 0 - невозможным
\end{definition}

\begin{definition}[совмещение событий]
  Событие, которое составлено из всех элементарных событий (исходов), входящих и в A, и в B, называется совмещением A и B ($A \cup B$)
  \[P(A \cup B) \leqslant P(A),\ P(B)\]
\end{definition}

\begin{definition}
  Событие, состоящее из всех эл. событий, вхлдящих или в A, или в B, называется объединением A и B ($A \cap B$)
\end{definition}

\begin{definition}
  A, B - события, $P(A \cup B) = 0 \Ra$ A, B - несовместные события
  \[P(A) + P(B) = P(A \cap B)\]
\end{definition}

\begin{definition}
  События A и B называются независимыми, если $P(A \cup B) = P(A) \cdot P(B)$
\end{definition}

\begin{definition}
  Разбиение мн-ва S на несовместные события $S_1,...,S_n$ называется полной системой событий
  \[P(A) = \sum_{i=1}^m P(A \cup S_i) \text{ - ф-ла полной вероятности}\]
\end{definition}

\begin{definition}
  $A_1,...,A_k$ - независимы в совокупности, если:
  \[\forall I \subset 1:k \q P(\cup_{i \in I}A_i) = \prod_{i \in I} P(A_i)\]

\end{definition}
\end{document}
