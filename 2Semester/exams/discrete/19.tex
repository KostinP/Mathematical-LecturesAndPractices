\documentclass[discrete.tex]{subfiles}

\begin{document}
  \section{Энтропия. 2 леммы}

  \begin{definition}
    Энтропия случайной схемы - мера содержания в этой схеме неопределенности. $\rho$ - вероятностная схема с m исходами, вероятности которых равны $p_1,...,p_m$
    \[H(\{p_1,...,p_m\}) = \sum_{i=1}^m p_i \log_2 \frac{1}{p_i}\]
  \end{definition}

  \begin{properties}
    \begin{enumerate}
      \item Энтропия непрерывно зависит от вероятности при фиксированном m
      \item При перестановке в наборе $\{ p_1,...,p_m\}$ энтропия не меняется
      \item Нужно ввести шкалу для измерения неопределенности. Примем за 1 схему с двумя равновероятными исходами
      \[(H(\{\frac{1}{2},\frac{1}{2}\}) = 1\]
      \item При фиксированном m наибольшей неопределенностью обладает схема, в которой все события равновероятны
      \[(H(\{\frac{1}{m},...,\frac{1}{m}\}) = h(m))\]
      \item $h(m)$ возрастает с ростом m
      \item Рассмотрим схему $P_m$ с m исходами и вероятностями $\{p_1,...,p_m\}$ и схему $R_k$ с $\{r_1,...,r_k\}$. Образуем комбинированную схему с $m+k-1$ исходами следующим образом: выбирается случайным образом один из исходов схемы $P_m$ и если произошел m-й исход, выбирается случайно один из исходов схемы $R_k$, а остальные $m-1$ исходов схемы $P_m$ считаются окончательно. В комбинированной схеме PR получаем исходы:
      \[1,2,...,m-1,(m,1),(m,2),...,(m,k)\]
      с вероятностями:
      \[p_1,p_2,...,p_{m-1},p_m q_1,...,p_m q_k\]
      Легко видеть, что $H(PR)=H(P_m)+p_m H(R_k)$
    \end{enumerate}
  \end{properties}

  \begin{theorem}
    Единственная функция на множестве всех вероятностных схем, удовлетворяющая условиям 1-6 - это функция H
  \end{theorem}

  \begin{Lemma}[1]
    \[g(m) = \log_2 m\]
  \end{Lemma}

  \begin{proof}
    $Q_k,\ Q_l$ - две схемы с равновероятными исходами
    \[Q_{kl} \qq (1,1),\ (1,2),\ ...,\ (k,l)\]
    \[g(kl) = g(k) + g(l)\]
    \[g(m^k) = k g(m)\]
    \[g(2^k) = 2 g(k)\]
    \[S = [\log_2 m^k]\]
    \[g(2^s) \leqslant g(m^k) \leqslant g(2^{s+1}) \Ra s \leqslant k g(m) \leqslant s+1\]
    \[\Ra 0 \leqslant g(m) - \frac{\lfloor k \log_2 m\rfloor}{k} = g(m) - \log_2 m + \frac{\{k \log_2 m\}}{k} \leqslant \frac{1}{k}\]
    При $k \ra \infty \q g(m) = \log_2 m$
  \end{proof}

  \begin{lemma_2}
    Если набор веростностей $\{p_1,...,p_m$ состоит из рациональных чисел, то $G(\{p_1,...,p_m\}) = H(\{p_1,...,p_m\})$
  \end{lemma_2}

  \begin{proof}
    *здесь когда-нибудь будет док-во*
  \end{proof}

  \begin{proof}[теоремы]
    *здесь когда-нибудь будет док-во*
  \end{proof}
\end{document}
