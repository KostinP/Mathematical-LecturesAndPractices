\documentclass[discrete.tex]{subfiles}

\begin{document}
  \section{Числа Фибоначчи. Теорема о представлении}
  
  \begin{definition}
    Последовательность чисел Фибоначчи F:
    \[F_0 = 0,\q F_1=1,\q F_n = F_{n-1} + F_{n-2},\ n>1\]
  \end{definition}

  \begin{Utv}
    \[\varphi_n = \dfrac{F_{n+1}}{F_n} \text{ - сходится}\]
  \end{Utv}

  \begin{Consequence}
    \[\varphi_n = \dfrac{F_{n+1}}{F_n} = \dfrac{F_{n-1} + F_n}{F_n} = 1 + \dfrac{1}{\varphi}\]
    \[\Ra \varphi = \dfrac{\sqrt{5}+1}{2}\]
  \end{Consequence}

  \begin{lemma}
    При $n > 1$ выполнено $\varphi^{n+2} = \varphi^{n+1} + \varphi^n$
  \end{lemma}

  \begin{Proof}
    *здесь когда-нибудь будет док-во*
  \end{Proof}

  \begin{lemma}
    При $k > 2$ выполнено:
    \[F_{2k} = F_{2k-1} + F_{2k-3} + ... + F_1\]
    \[F_{2k+1} = 1 + F_{2k} + F_{2k-2} + ... + F_0\]
  \end{lemma}

  \begin{proof}[по индукции]
    ($k=3$):
    \[F_6 = 8 = 5 + 2 + 1\]
    \[F_7 = 13 = 1 + 8 + 3 + 1 + 0\]
    ($k \ra k+1$):
    \[F_{2(k+1)} = F_{2k+2} = F{2k+1} + F_{2k} = F_{2k+1} + F_{2k-1} + ... + F_{1} = ???\]
  \end{proof}

  \begin{theorem}
    Любое натуральное число можно однозначно представить в виде суммы чисел Фибоначчи
    \[s = F_{i_0} + F_{i_1} + ... + F_{i+r}, \text{ где } i_{k-1} + 1 < i_k,\q k \in 1:r \q i_0 = 0\]
  \end{theorem}

  \begin{proof}
    Существование:

    Пусть $j(s)$ - номер масимального числа Фиббоначи, не превосходящего s. Положим $s'=s-F_j(s)$. Из определения $j(s)$ следует, что $s'<F_{j(s)-1}$, иначе число Фиббоначи не было бы максимальным. Теперь мы получим искобое представление для s как представление s', дополненное слагаемым $F_{j(s)}$\\ \\
    Единственность:

    Пусть есть ещё одно представление $s=F_{j_0}+...+F_{j_q}$. Н.У.О. считаем, что $j_q < j(s)$. Если мы заменим $F_{j_q}$ на $F_{j(q)-1}$, то правая часть разве что лишь увеличится. Аналогично заменим с возможным увеличением предпоследнее слагаемое на $F_{j(s)-3}$. ???
  \end{proof}
\end{document}
