\documentclass[matan, 12pt, fleqn]{subfiles}

\begin{document}

\section{Степенной ряд (в $\CC$). Радиус сходимости. Формула Коши-Адамара.}

\begin{Definition}
    \[\text{Будем рассматривать }\]
	\[\sum_{k = 0}^\infty c_k z^k \q\q c_k, z \in  \CC \]
\end{Definition}

\begin{Definition}
	\[x = \real z \]
	\[y = \im z\]
	\[\abs{z} = \sqrt{x^2 + y^2}\]
	\[x = \abs{z} \cos{\varphi}\]
	\[y = \abs{z} \sin{\varphi}\]
	\[\abs{z_1 - z_2} = \sqrt{(x_1 - x_2)^2 + (y_1 - y_2)^2}\]
	\[C_n = a_n + i b_n \q n \in \N\]
	\[\lim_{n \to \infty} c_n = c \text{, если } \forall \mathcal{E} > 0 \q \exists N : \forall n > N
    \q \abs{c_n - c} < \mathcal{E}\]
\end{Definition}

\begin{Utv}
		\[c_n \us{n \to \infty}{\to } c \rla \begin{align}
			&a_n \to a\\
			&b_n \to b
		\end{align} \q (n \to \infty)\]
		\[c_n = a_n + i b_n\]
		\[c = a + ib\]
		\[a, b \in \R\]
		\[a_n, b_n \in \R\]
\end{Utv}

\begin{definition}
	Радиусом сх-ти степ. ряда $\sum c_n z^n$ назыв $R \in [0, +\infty]$ такое, что $(z \neq 0)$
	\[\forall z : \abs{z} < R \text{ - ряд. сх}\]
	\[\forall z : \abs{z} > R \text{ - ряд расх.}\]
\end{definition}

\begin{examples}

		\begin{enumerate}
			\item $\displaystyle \sum_{k = 0}^\infty k! z^k \text{ по пр. Даламб \q расх }
				\forall z \neq 0 \q R = 0$
				\[\lim_{k \to +\infty} \frac{\abs{(k + 1)! z^{k + 1}}}{\abs{k!z^k}} = \infty, \q z \neq 0\]
			\item $\displaystyle \sum_{k = 0}^\infty \frac{z^k}{k!} \text{ - сх. } \forall z \in \CC$
			\item $\displaystyle \sum_{n = 1}^\infty \frac{z^n}{n} \q\q $
				\[\begin{align}
					&z^* = -1& : \q &\sum \frac{(-1)^n}{n} \text{ - сх } \Ra \text{ сх. равн. }
					\forall \abs{z} \leq d < 1\\
					&z_0 = 1& : \q &\sum \frac{1}{n} \text{ - расх } \Ra \forall \abs{z} > 1
				\end{align}\]
		\end{enumerate}
\end{examples}

\begin{Theorem} [ф-ма Коши-Адамара]
		\[\sum_{k = 0}^\infty c_k z^k \q R \text{ - рад. сх-ти} \]
		\[\frac{1}{R} = \overline{\lim_{k \to \infty} } \sqrt[k]{\abs{c_k}}\]
\end{Theorem}
\newpage
\section{Теорема о комплексной дифференцируемости степенного ряда. Следствие: единственность разложения в степенной ряд.}


\newpage
\section{Ряд Тейлора. Примеры ($e^x,\sin x,\ln(1 + x), e^{-\frac{1}{x^2}}$).}

\begin{Definition}
	\[f \in C^{\infty} (U_{x_0}) \qq U_{x_0} \text{ - окр } x_0 \]
	\[\text{Ряд } \sum^\infty_{n = 0} \frac{f^{(n)}(x_0)}{n!}(x - x_0)^n \text{ назыв. Рядом Тейлора ф-и в т } x_0\]
\end{Definition}

\begin{examples}
	\begin{enumerate}
		\item $\displaystyle e^x = \sum_{k = 0}^\infty \frac{x_k}{k!}$
		\item $\displaystyle \cos x = \sum_{k = 0}^\infty (-1)^k \frac{x^{2k} }{(2k)!}$
		\item $\displaystyle \sin x = \sum_{k = 0}^\infty (-1)^k \frac{x^{2k + 1} }{(2k + 1)!}$
		\item $\displaystyle \ln (1 + x) = \sum_{k = 1}^\infty (-1)^k \frac{x^k}{k}$
	\end{enumerate}
\end{examples}


\newpage
\section{Биномиальный ряд $(1 + x)^\upalpha$}

\begin{Definition}
	\[(1 + x)^\alpha \q\q \alpha \in \R\]
	Запишем (формально) ряд Тейлора для $(1 + x)^\alpha$ в т. $x_0 = 0$
	\[\frac{f^{(k)} (0)}{k!} = \frac{\alpha(\alpha - 1) \cdot ... \cdot (\alpha - k + 1)}{k!} =
	C_{\alpha}^k \]
	Найдем интервал сходимость $\displaystyle \sum_{k = 0}^\infty c_{\alpha}^k z^k \q z \in \CC$ (по Даламберу)
	\[\lim_{k \to \infty} \abs{\frac{c_{\alpha}^{k + 1} z^{k + 1}}{c_{\alpha}^k z^k }} =
	\lim_{k \to \infty}  \]
\end{Definition}

\newpage
\section{Признак Абеля-Дирихле для равномерной сходимости функциональных рядов (доказательство одного).}

\begin{Definition}
    \[\sum_{k = 0}^\infty a_k(t)b_k(t) \qq \begin{matrix}
        a_k : E \to \CC\\
        b_k : E \to \R\\
        E \subset \CC
    \end{matrix} \]
    \[b_k(t) - \text{ монот по } k \q \forall t\]
    \[\text{т.е}\q b_{k + 1}(t) \leq b_k(t) \q \forall t (\text{ или наоборот})  \]
    %\[\text{Абель}\]
    Абель
    \begin{enumerate}
        \item $ \displaystyle \sum_{k = 0}^\infty a_k $ - сход р/м на $E$
        \item $\abs{b_k(t)} \leq M \qq \forall  k, \q \forall t \in E$
    \end{enumerate}
    %\[\text{Дирихле}\]
    Дирихле
    \begin{enumerate}
        \item $\displaystyle \abs{\sum_{k = 0}^N a_k(t) } \leq M \q \forall N, \forall t \in E$
        \item $b_k(t) \rightrightarrows 0$
    \end{enumerate}
    Тогда $\displaystyle \sum_0^{\infty} a_k(t)b_k(t)$ - сход равномерно на $E$
\end{Definition}

\begin{Proof}
    Дописать
\end{Proof}

\newpage
\section{Теорема Абеля. Сумма ряда $\sum\limits_{n=1}^\infty \frac{(-1)^{n-1}}{n}$.}

\begin{Definition}
    \[hint: \q z \in [0, w] \rla z = t \cdot w \q 0 \leq t \leq 1\]
    \[\sum_{k = 0}^\infty c_k z^k \qq c_k \in \CC \]
    \[\text{Пусть } \sum c_k z^k \text{ сход при } z = w \in \CC\]
    \[\text{Тогда } \sum_{k = 0}^\infty c_k z^k \text{  - сход р-но на } [0, w] \]
    \[\Ra f(z) = \sum_{k = 0}^\infty c_k z^k \in C[0, w] \]
\end{Definition}

\begin{Proof}
    \[f(t, w) = \sum_{k = 0}^\infty c_k t^k w^k \qq t \in [0, 1] \]
    \[\sum c_k w^k \text{ - сход (равн по t, т.к. не зависит от t)}\]
    \[t^k \text{ - убывает}\]
    \[\abs{t^k} \leq 1 \qq \forall t \in [0, 1] \qq \forall k \in \N\]
    \[\Ra \text{ по пр. Абеля-Дирихле ряд сход. равномерно}\]
\end{Proof}

\begin{Example}
    \[\ln(1 + x) = \sum_{k = 1}^\infty \frac{(-1)^{k + 1}x^k }{k} \qq \forall x: \
    -1 < x < 1\]
    \[\text{при } x = 1 \qq \sum_{k = 1}^\infty \frac{(-1)^{k + 1} }{k} \text{
    - гармонич. знакочеред, он сход, т.о. }\]
    \[ \sum_{1}^\infty \frac{(-1)^k}{k}x^k \text{  - сх. при $x = 1 \Ra$ по т. Абеля }\]
    \[f(x) = \sum_1^\infty \frac{(-1)^{k + 1}x^k }{k} \in C[0, 1]\]
    В частности $\displaystyle \lim_{x \to 1-} f(x) = f(1) $
    \[\text{если } x \in (0, 1) \text{, то } f(x) = \sum_1^\infty
    \frac{(-1)^{k - 1}x^k }{k} = \ln(1 + x)\]
    \[\lim_{x \to 1-} \ln(1 + x) = \ln 2 \]
    \[1 - \frac{1}{2} + \frac{1}{3} - ... = \ln 2\]
\end{Example}

\newpage
\section{Интеграл комплекснозначной функции. Скалярное произведение и норма в пространстве $C(\CC / \R)$, в пространстве $R([a; b])$. Ортогональность. Пример: $e_k(x) = e^{2 \pi i k x}$.}

\begin{Definition}
    \[f : [a, b] \to \CC\]
    \[f(x) = u(x) + iv(x)\]
    \[u(x) = \real f(X)\]
    \[v(x) = \im f(x)\]
    \[f \text{ - инт. по Риману } \q f \in R_\CC [a, b], \text{ если } \]
    \[u, v \in R[a, b]\]
    \[\int_a^bf(t)dt = \int_a^b u(t)dt + i\int_a^b v(t)dt\]
\end{Definition}

\begin{properties}
    \begin{enumerate}
        \item $\displaystyle \int_a^b (f + g) = \int_a ^b f + \int_a^b g$
        \item $\displaystyle \int_a^b f = \int_a^c f + \int_c^b f$
        \item $\displaystyle \int_a^b kf = k\int_a^b f \q (k \in \CC)$
        \item $\displaystyle \int_a^b \overline{f} = \int_a^b u - iv = \overline{\int_a^b f}$
            (комплексное сопряжение)
        \item $\displaystyle F' = f$
            \[\int_a^b f = F(b) - F(a)\]
        \item $\displaystyle \abs{\int_a^b f} \leq \int_a^b \abs{f}$
    \end{enumerate}
\end{properties}

\begin{Definition}[Периодич. функции]
    \[f(x + t) = f(x) \qq \forall x\]
    \[\text{Будем считать, что } T = 1\]
    Периодич. функции с пер. 1 образуют линейное пр-во
    \[f, g \text{ - период. } T = 1\]
    \[\Ra f + k \cdot g \text{ - тоже период.} T = 1\]
    \[\text{Если } f \text{ - периодич. } T = 1 \text{, то}\]
    \[\int_0^1 f = \int_c^{c + 1} f \qq \forall c \in \R \]
    \[0 < c < 1\]
    \[\int_0^1 f = \int_0^c f + \int_c^1 f = \int_0^c f(t + 1)dt + \int_c^1 f =
    \int_1^{c + 1} f(s)ds + \int_c^1 f \]
\end{Definition}

\begin{definition}
    Рассмотрим пр-во функций с пер $T = 1$ и $\in R_\CC [0, 1] \rla R_\CC[0, 1]$
    Введем на этом пр-ве структуру евклидова пр-ва
    \[<f, g> = \int_0^1 f \cdot \overline{g} \text{ - скал. произведение}\]
\end{definition}

\begin{Definition} [Норма в лин. пр-ве X со скал. произв.]
    \[\Abs{x} = \sqrt{<x, x>}\]
    \[\Abs{f} = \sqrt{\int_0^1 \abs{f}^2}\]
    \begin{enumerate}
        \item $\Abs{x} \geq 0$
            \[\Abs{x} = 0 \rla x = 0\]
        \item $\forall k \q \Abs{kx} = \Abs{k} \cdot \Abs{x}$
    \end{enumerate}
\end{Definition}

\begin{Definition}
    \[f \perp g \q (f \text{ ортогонально } g) \q \rla \q <f, g> = 0 \]
\end{Definition}


\begin{Example}
    \[a) e_n = e^{2\pi i n x} \qq x \in [0, 1] \qq \Abs{e_n} = 1 \qq \forall n \in \Z\]
    \[\Abs{e_n}^2 = \int_0^1 e_n \overline{e_n} = \int_0^1 e^{2\pi i n x } \cdot e^{-2\pi i nx} = 1  \]
    \[\overline{e^{i\varphi}} = \cos \varphi + \overline{i \sin \varphi} = \cos \varphi - i\sin \varphi = e^{-i \varphi} \]
    \[b) <e_n, e_m> \ = \int_0^1 e^{2\pi i nx} \cdot e^{-2\pi i m x} = \int_0^1 e^{2\pi i x (n - m)} = \begin{cases}
        0, & n \neq m\\
        1, & n = m
    \end{cases} = \delta_{nm} \text{ - с. Крон.} \]
    т.о. \q $e_n \perp e_m \qq \forall n \neq m$
\end{Example}

\newpage
\section{Свойства скалярного произведения и нормы (теорема Пифагора, неравенство Коши-Буняковского-Шварца, неравенство треугольника).}

\begin{Properties}
    \[<...> : X \times X \to \CC\]
    \begin{enumerate}
        \item $\forall x, y \in X$ \q $<x, y> = \overline{<y, x>}$
        \item $\forall x_1, x_2 \in X$ \q $\forall y \in X$
            \[<x_1  + x_2, y> \ = \ <x_1, y> + <x_2, y>\]
        \item $\forall k \in \CC \q \forall x, y \in X$
            \[<kx, y> \ = \ k <x, y>\]
            \[<x, ky> \ = \ \overline{k} <x, y>\]
        \item $<x, x> \ \geq 0$ причем $<x, x> \ = 0 \ \rla \ x = 0$\\
            Но для $f \in R_\CC [0, 1] $ необязательно из $ \q <f, f> \ = \  0$ следует, что $f = 0$
    \end{enumerate}
\end{Properties}

\begin{properties}
    \begin{enumerate}
        \item $\Abs{f + g}^2 = \Abs{f}^2 + \underbracket{<f, g> + <g, f>}_{2\real <f, g>}  + \Abs{g}^2 = $
            \[ = \Abs{f}^2 + 2\real <f, g> + \Abs{g}^2\]
        \item По т. Пифагора, если $f \perp g \ \Ra$
            \[\Abs{f + g}^2 = \Abs{f}^2 + \Abs{g}^2\]
        \item $\Abs{f + g}^2 + \Abs{f - g}^2 = 2(\Abs{f}^2 + \Abs{g}^2)$
        \item нер-во КБШ
            \[\abs{<f, g>} \leq \Abs{f} \cdot \Abs{g}\]
        \item Н-во треугольника
            \[\Abs{f + g} \leq \Abs{f} + \Abs{g}\]
    \end{enumerate}
\end{properties}

\begin{Proof} [КБШ]
    \[(*)\abs{<f, g>} = \abs{\int f \overline{g}} \leq \int \abs{f} \abs{g}\]
    \[0 \leq \int(\abs{f} + \lambda \abs{g})^2 =
    \underbracket{ \Abs{f}^2 + 2\lambda\int \abs{f}\abs{g} + \lambda^2 \Abs{g}^2}_{\text{кв. трехчлен отн } \lambda}
    \qq \forall  \lambda \in \R\]
    \[D \leq 0\]
    \[\frac{D}{4} = (\int \abs{f}\abs{g})^2 - \Abs{f}^2 \Abs{g}^2 \leq 0 \Ra \int \abs{f}\abs{g} \leq \Abs{f} \Abs{g} \q (**)\]
    \[(*) \text{ и } (**) \Ra \abs{ <f, g>} \leq \Abs{f} \cdot \Abs{g}\]

\end{Proof}

\begin{Proof}[Нер-во треуг-ка]
    \[\Abs{f + g}^2 = \Abs{f}^2 + 2\real <f, g> + \Abs{g}^2 \leq
    \Abs{f}^2 + 2\abs{<f, g>} + \Abs{g}^2 \os{\text{КБШ}}{\leq }\]
    \[\leq \Abs{f}^2 + 2\Abs{f} \cdot \Abs{g} + \Abs{g} = (\Abs{f} + \Abs{g})^2 \Ra \Abs{f + g} \leq \Abs{f} + \Abs{g}\]
\end{Proof}

\begin{Theorem}[Аксиомы нормы]
    \[X - \text{лин. пр-во} \qq \Abs{...} : X \to [0, +\infty)\]
    \begin{enumerate}
        \item $\Abs{x} = 0 \q\rla\q x = 0$
        \item $\Abs{kx} = \Abs{k} \cdot \Abs{x} \qq \forall k \in \CC, \q \forall x \in X$
        \item $\forall x, y \in X$ \qq $\Abs{x + y} \leq \Abs{x} + \Abs{y}$
    \end{enumerate}
\end{Theorem}

\newpage
\section{Коэффициенты Фурье функции по ортогональной системе $e_k$. Ряд Фурье. Пример: тригонометрический полином.}

\begin{Definition}
    Тригонометрическим многочленом степени $N$ назовем:
    \[T_n = \sum_{k = -N}^N c_k e_k(x) = \sum_{k = -N}^N c_k e^{2\pi i kx} = \sum_{k = -N}^N c_k (\cos
    (2\pi kx) + i\sin(2\pi kx))   \]
    Как найти $c_k$, если известен $T_n(x)$ ?
    \[T_n = \sum_{k = -N}^N c_k e_k \q \bigg| \cdot <..., e_m>\]
    \[<T_n, e_m>  \ = c_m \cdot \us{=1}{<e_m, e_m>} \q (\text{т.к.} <e_k, e_m> = \delta_{km} )\]
    \[c_m = <T_N, e_m> = \int_0^1 T_N \overline{e}_m\]

    \[\letus f, g \text{ - тригоном. полиномы, коэфф. в разложении по } e_k \text{ будем обозначать } \hat{f}(k) \in \CC\]
    \[\text{т.е. } f = \sum_{k = -N}^N \hat{f}(k)e_k, \q \hat{f}(k) = <f, e_k> \]
    \[g = \sum_{k = -N}^N \hat{g}(k)e_k, \qq \hat{g}(k) = <g, e_k> \]
    \[<f, g> = <(\sum_{k = -N}^N \hat{f}(k)e_k ), (\sum_{j=-N}^N \hat{g}(j)e_j )> = \]
    \[= \sum_{k, j =-N}^N \hat{f}(k)\ol{\hat{g}}(j) <\us{=\delta_{kj} }{e_k, e_j}> = \sum_{k = -N}^N \hat{f}(k) \overline{\hat{g}}(k) \]
    \[\Abs{f}^2 = \sum_{k = -N}^N \abs{\hat{f}(k)}^2 \qq \hat{f}(k) = <f, e_k> \]
\end{Definition}

\begin{Definition}
    \[\hat{f}(k) = <f, e_k> = \int_0^1 f \cdot \overline{e}_k \text{ - коэфф. Фурье
    функции } f\]
    \[\text{по ортог. системе функций } \{e_k\}_{k \in \Z} \]
\end{Definition}

\begin{Definition}
    \[\text{Ряд Фурье функции } f : \q \sum_{k = -\infty}^\infty \hat{f}(k)e_k(x) \]
\end{Definition}
\newpage
\section{Свойства коэффициентов Фурье (коэффициенты Фурье сдвига, производной).}

\begin{properties}
    \begin{enumerate}
        \item $\displaystyle f_a(t) = f(t + a) \q \Ra \hat{f}_a(k) =
            \int_0^1 f(t + a)e ^{- 2\pi i kt}dt = $
            \[= \int_a^{1 + a} f(x) \cdot e^{-2\pi i k (x - a)} dx = 
            \int_0^1 f(x)\cdot e^{-2\pi i k x} \cdot e^{e\pi i k a} = \]
            \[= e^{2\pi i k a } \hat{f}(k) \]
        \item Пусть $f \in C^{1} (\R / \Z) $
            \[\hat{f}'(k) = \int_0^1 f'(t) \cdot e^{-2\pi i kt} dt =  \]
            Интегрируем по частям
            \[= \underbracket{ f(t)e^{-2\pi i kt}}_{= 0 \text{ т.к. } T = 1}
                \bigg|_0^1  + 2\pi i k 
            \underbracket{\int_0^1 f(t) 
        e^{-2\pi i k t}dt}_{\hat{f}(k)}  \]
        \[\hat{f'}(k) = 2\pi ik \hat{f}(k)\]
        \item Коэф. Фурье фещ. функции
            \[f \in R[-\frac{1}{2}, \frac{1}{2}]\]
            \[\hat{f}(k) = \int_0^1 f(t) \cdot e^{-2 \pi ikt}dt \]
            \[\hat{f}(-k) = \int_0^1 f(t) \cdot e^{2 \pi k t}dt \]
            \[\Ra \hat{f}(k) = \overline{\hat{f}(-k)}\]
        \item Коэфф. Ферье четной функции
            \[f \text{ - четная}\]
            \[\hat{f}(k) = \int_{-k}^k f(t)e^{-2\pi ikt}dt = \int_{-k}^k 
                \underbracket{f \cdot \cos 2\pi k t}_{\text{четная}} 
            - i \int_{-k}^k \underbracket{f \cdot \sin 2\pi kt}_{\text{нечетная } = 0} = \]
            \[= \int_{-k}^k f \cos 2\pi kt = \hat{f}(-k) \text{ поскольку четная} \]
    \end{enumerate}
    Если $f$ - вещ и четная $\Ra \q \hat{f}(k) = \hat{f}(-k) = \overline{\hat{f}(k)}$
    \[\Ra \hat{f}(k) \in \R\]
\end{properties}


\newpage
\section{Неравенство Бесселя. Лемма Римана-Лебега (light).}

\begin{Definition} [Неравенство Бесселя]
    \[\sum_{k = -\infty}^{+\infty} \abs{\hat{f}(k)}^2 \leq \int_0^1 \abs{f}^2 = 
    \Abs{f}^2\]
\end{Definition}

\begin{Lemma}
    \[f \in R[0, 1], \q S_N(f) = \sum_{k = -N}^N \hat{f}(k)e_k \]
    \[f - S_N(f) \perp S_N(f)\]
\end{Lemma}

\begin{Proof}
    \[<f, S_N(f)> = \int_0^1 f(t) \cdot \overline{\sum_{k = -N}^N \hat{f}(k)
    \cdot e_k(t)dt} = \sum_{k = -N}^N \overline{\hat{f}(k)} \int_0^1 
    f(t) \cdot \overline{e_k}dt = \sum_{k = -N}^N \abs{\hat{f}(k)}^2 =\]
    \[= <S_N(f), S_N(f)>\ = \ \Abs{S_N(f)}^2\]

    \[<f, S_N(f)> \q  = \q  <S_N(f), S_N(f)>\]
    \[\rla <f - S_N(f), S_N(f)> \q  = 0\]
\end{Proof}

\begin{consequence}
    т.к. $f - S_N(f) \perp S_N(f)$, то $\Abs{f}^2 = \Abs{f - S_N(f)}^2 + \Abs{S_N(f)}^2$
\end{consequence}

\begin{Proof}[нер-ва Бесселя]
    \[\sum_{k = -N}^N \abs{\hat{f}(k)^2} = \Abs{S_N(f)}^2 \leq \Abs{f}^2 = 
    \int_0^1 \abs{f}^2 \]
    \[\text{предельный переход в нер-ве} \q (N \to \infty)\]
    \[\sum_{k = -\infty}^{+\infty} \abs{\hat{f}(k)}^2 \leq \int_0^1 \abs{f}^2\]
\end{Proof}

\begin{Consequence}[Лемма Римана-Лебега]
    \[f \in R_{\CC}[0, 1] \Ra \hat{f}(k) \to 0 \q k \to  + \infty \q k \to -\infty\]
\end{Consequence}

\begin{proof}
    Необходимо усл. сх-ти ряда и нер-во Бесселя.\\
    В нер-ве Бесселя ряд возрастает и ограничен сверху, значит он сходится.
    \[\Ra \hat{f}(k) \to 0\]
\end{proof}

\newpage
\section{Вычисление интеграла Дирихле $\int\limits_0^\infty \frac{\sin x}{x}$.}


\newpage
\section{Ядра Дирихле, их свойства. Выражение частичных сумм ряда Фурье через ядра Дирихле.}

    \[S_N(f) = \sum_{k = -N}^N \hat{f}(k)e_k(x) = \sum_{k = -N}^N \int_0^1 
    f(t)e^{-2 \pi ikt} dt \cdot e^{2\pi ikt} = \sum_{k = -N}^N \int_0^1 f(t) 
    e^{2\pi ik(x - t)}dt  = \]
    \[= \int_0^1 f(t) \underbracket{\sum_{k = -N}^N e^{2\pi ik(x - t)}}_
    {\text{Ядро Дирихле}}   \]

\begin{Definition}[ядро Дирихле]
    \[D_N(y) = \sum_{k = -N}^N e^{2\pi ky} \os{\text{геом. прог.}}{=} 
    e^{-2\pi iNy} \frac{1 - e^{2\pi i(2N + 1)y} }{1 - e^{2\pi i y} } = \]
    \[= e^{-2\pi iNy} \frac{1 -e^{2\pi i (2N + 1)y} }{1 -e^{2\pi i y} } \cdot 
    \frac{1 - e^{-2\pi iy} }{1 - e^{-2\pi i y} } = 
    \frac{e^{-2\pi iNy} + e^{2\pi iNy} - e^{-2\pi i (N+1)y} - e^{2\pi i(N + 1)y}    }
    {1 + 1 - e^{2\pi iy} - e^{-2\pi iy} } = \]
    \[ = \frac{ 2\cos 2\pi Ny - 2\cos 2\pi (N + 1)y}{2 - 2\cos 2\pi y} =\]
    Через разность косинусов
    \[= \frac{2\sin \pi (2N + 1)y \sin \pi y}{2 \sin^2 \pi y} = 
    \frac{\sin \pi(2N + 1)y}{\sin \pi y}\]
    \[D_n(y) = \begin{cases}
        \displaystyle \frac{\sin \pi(2N + 1)y}{\sin \pi y}, & y \neq 0\\
        2N + 1, & y = 0
    \end{cases}\]
\end{Definition}

\begin{properties}
    \begin{enumerate}
        \item $D_N(-y) = D_N(y)$ четная
        \item $D_N \in C[- \frac{1}{2}; \frac{1}{2}]$
        \item $\displaystyle D_n = \sum_{j = -N}^N e_j(y) $
            \[\hat{D}_N(k) =\ <D_n,\  e_k>\]
            \[\hat{D}_n(k) = \begin{cases}
                1, & \abs{k} \leq N\\
                0, & \abs{k} > N
            \end{cases}\]
            \[\Ra \hat{D}_N(0) = \ <D_N, \ e_0> = \int_{-\frac{1}{2}}^{\frac{1}{2}} 
            D_N(t)dt = 1\]
    \end{enumerate}
    Таким образом, част. суммы р. Фурье выражаются через ядро Дирихле.
    \[S_N f(x) = \int_0^1 f(t) \cdot D_N(x - t)dt\]
\end{properties}

\newpage
\section{Свертка. Простейшие свойства. Свертка с тригонометрическими и алгебраическими полиномами.}

\begin{Definition} [Свертка функций]
    \[f, g \in R\left[-\frac{1}{2}, \frac{1}{2}\right]\]
    \[(f * g)(x) = \int_{-\frac{1}{2}}^{\frac{1}{2}} f(x)g(x - t)dt \q \text{ - свертка } 
    f \text{ и } g\]
    \[\text{т.о } S_n = f * D_N\]
\end{Definition}

\begin{properties}
    \begin{enumerate}
        \item $f * g = g * f$ коммутативность
            \[\int_{-\frac{1}{2}}^{\frac{1}{2}} f(t)g(x - t) =
            \left[x - t = s\right] = - \int_{x + \frac{1}{2}}^{x - \frac{1}{2}} 
            f(x - s)g(s)ds = \]
            \[=\int_{-\frac{1}{2}}^{\frac{1}{2}} g(s)f(x - s)ds = g * f  \]
        \item $f * (g_1 + g_2) = f * g_1 + f * g_2$
        \item $f * (kg) = k(f * g)$
        \item $f \in R[-\frac{1}{2}, \frac{1}{2}], \q T_N$ - тригонометр. полином
            степ $\leq N$\\
            Тогда $f * T_n$ - тригоном. полином степ $\leq N$
            \[f * T_N = \int_{-\frac{1}{2}}^{\frac{1}{2}} f(t)T_N(x - t)dt = 
            \int_{-\frac{1}{2}} ^{\frac{1}{2}} f(t)\sum_{k=-N}^N c_k 
            e^{2\pi i k (x -t)}dt  = \]
            \[=  \sum_{k = -N}^N c_k \int_{-\frac{1}{2}}^{\frac{1}{2}} f(t) 
            e^{-e\pi i kt}dt \cdot e^{2\pi ikx} \text{ - триг. полином степ. }N\]
        \item $f \in R[-\frac{1}{2}, \frac{1}{2}]$
            \[P_N \text{ - алг. полином степ } \leq N\]
            \[P_N(x) = \sum_{k = 0}^N a_k \cdot x^k \]
            \[f * P_N = \int_{-\frac{1}{2}}^{\frac{1}{2}}f(t) \sum_{k = 0}^N 
            a_k (x - t)^k  \]
    \end{enumerate}
\end{properties}
\newpage
\section{Принцип локализации Римана.}

\[S_N f(x) = \int_{-\frac{1}{2}}^{\frac{1}{2}} f(x - t)D_N(t)dt \]

\begin{Lemma}
    \[f \in R[-\frac{1}{2}, \frac{1}{2}]\]    
    \[\forall \delta > 0 \q \lim_{N \to \infty} \int f(x - t)D_N(t)dt = 0 \]
    \[\delta \leq \abs{t} \leq \frac{1}{2}\]
\end{Lemma}

\newpage
\section{Теорема о поточечной сходимости ряда Фурье для локально-Гельдеровой функции.}


\newpage
\section{Ядра Фейера, их свойства. Связь с $\upsigma_N(f)$.}


\newpage
\section{Аппроксимативная единица. Определение, примеры. Теорема о равномерной сходимости свертки с аппроксимативной единицей.}


\newpage
\section{Теорема Фейера. Теорема Вейерштрасса.}


\newpage
\section{Среднеквадратичное приближение функций, интегрируемых по Риману, тригонометрическими полиномами.}


\newpage
\section{Равенство Парсеваля.}

\end{document}
