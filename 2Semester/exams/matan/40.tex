\documentclass[matan, 12pt, fleqn]{subfiles}

\begin{document}

\section{Степенной ряд (в $\CC$). Радиус сходимости. Формула Коши-Адамара.}

\begin{Definition}
    \[\text{Будем рассматривать }\]
	\[\sum_{k = 0}^\infty c_k z^k \q\q c_k, z \in  \CC \]
\end{Definition}

\begin{Definition}
	\[x = \real z \]
	\[y = \im z\]
	\[\abs{z} = \sqrt{x^2 + y^2}\]
	\[x = \abs{z} \cos{\varphi}\]
	\[y = \abs{z} \sin{\varphi}\]
	\[\abs{z_1 - z_2} = \sqrt{(x_1 - x_2)^2 + (y_1 - y_2)^2}\]
	\[C_n = a_n + i b_n \q n \in \N\]
	\[\lim_{n \to \infty} c_n = c \text{, если } \forall \mathcal{E} > 0 \q \exists N : \forall n > N 
	\q \abs{c_n - c} < \mathcal{E}\]
\end{Definition}

\begin{Utv}
		\[c_n \us{n \to \infty}{\to } c \rla \begin{align}
			&a_n \to a\\
			&b_n \to b
		\end{align} \q (n \to \infty)\]
		\[c_n = a_n + i b_n\]
		\[c = a + ib\]
		\[a, b \in \R\]
		\[a_n, b_n \in \R\]
\end{Utv}

\begin{definition}
	Радиусом сх-ти степ. ряда $\sum c_n z^n$ назыв $R \in [0, +\infty]$ такое, что $(z \neq 0)$
	\[\forall z : \abs{z} < R \text{ - ряд. сх}\]
	\[\forall z : \abs{z} > R \text{ - ряд расх.}\]
\end{definition}

\begin{examples}

		\begin{enumerate}
			\item $\displaystyle \sum_{k = 0}^\infty k! z^k \text{ по пр. Даламб \q расх } 
				\forall z \neq 0 \q R = 0$
				\[\lim_{k \to +\infty} \frac{\abs{(k + 1)! z^{k + 1}}}{\abs{k!z^k}} = \infty, \q z \neq 0\]
			\item $\displaystyle \sum_{k = 0}^\infty \frac{z^k}{k!} \text{ - сх. } \forall z \in \CC$
			\item $\displaystyle \sum_{n = 1}^\infty \frac{z^n}{n} \q\q $
				\[\begin{align}
					&z^* = -1& : \q &\sum \frac{(-1)^n}{n} \text{ - сх } \Ra \text{ сх. равн. } 
					\forall \abs{z} \leq d < 1\\
					&z_0 = 1& : \q &\sum \frac{1}{n} \text{ - расх } \Ra \forall \abs{z} > 1
				\end{align}\]
		\end{enumerate}
\end{examples}

\begin{Theorem} [ф-ма Коши-Адамара]
		\[\sum_{k = 0}^\infty c_k z^k \q R \text{ - рад. сх-ти} \]
		\[\frac{1}{R} = \overline{\lim_{k \to \infty} } \sqrt[k]{\abs{c_k}}\]
\end{Theorem}
\newpage
\section{Теорема о комплексной дифференцируемости степенного ряда. Следствие: единственность разложения в степенной ряд.}


\newpage
\section{Ряд Тейлора. Примеры ($e^x,\sin x,\ln(1 + x), e^{-\frac{1}{x^2}}$).}

\begin{Definition}
	\[f \in C^{\infty} (U_{x_0}) \qq U_{x_0} \text{ - окр } x_0 \]
	\[\text{Ряд } \sum^\infty_{n = 0} \frac{f^{(n)}(x_0)}{n!}(x - x_0)^n \text{ назыв. Рядом Тейлора ф-и в т } x_0\]
\end{Definition}

\begin{examples}
	\begin{enumerate}
		\item $\displaystyle e^x = \sum_{k = 0}^\infty \frac{x_k}{k!}$
		\item $\displaystyle \cos x = \sum_{k = 0}^\infty (-1)^k \frac{x^{2k} }{(2k)!}$
		\item $\displaystyle \sin x = \sum_{k = 0}^\infty (-1)^k \frac{x^{2k + 1} }{(2k + 1)!}$
		\item $\displaystyle \ln (1 + x) = \sum_{k = 1}^\infty (-1)^k \frac{x^k}{k}$
	\end{enumerate}
\end{examples}


\newpage
\section{Биномиальный ряд $(1 + x)^\upalpha$}

\begin{Definition}
	\[(1 + x)^\alpha \q\q \alpha \in \R\]
	Запишем (формально) ряд Тейлора для $(1 + x)^\alpha$ в т. $x_0 = 0$
	\[\frac{f^{(k)} (0)}{k!} = \frac{\alpha(\alpha - 1) \cdot ... \cdot (\alpha - k + 1)}{k!} = 
	C_{\alpha}^k \]
	Найдем интервал сходимость $\displaystyle \sum_{k = 0}^\infty c_{\alpha}^k z^k \q z \in \CC$ (по Даламберу)
	\[\lim_{k \to \infty} \abs{\frac{c_{\alpha}^{k + 1} z^{k + 1}}{c_{\alpha}^k z^k }} = 
	\lim_{k \to \infty}  \]
\end{Definition}

\newpage
\section{Признак Абеля-Дирихле для равномерной сходимости функциональных рядов (доказательство одного).}


\newpage
\section{Теорема Абеля. Сумма ряда $\sum\limits_{n=1}^\infty \frac{(-1)^{n-1}}{n}$.}


\newpage
\section{Интеграл комплекснозначной функции. Скалярное произведение и норма в пространстве $C(\CC \setminus \R)$, в пространстве $R([a; b])$. Ортогональность. Пример: $e_k(x) = e^{2 \pi i k x}$.}


\newpage
\section{Свойства скалярного произведения и нормы (теорема Пифагора, неравенство Коши-Буняковского-Шварца, неравенство треугольника).}


\newpage
\section{Коэффициенты Фурье функции по ортогональной системе $e_k$. Ряд Фурье. Пример: тригонометрический полином.}


\newpage
\section{Свойства коэффициентов Фурье (коэффициенты Фурье сдвига, производной).}


\newpage
\section{Неравенство Бесселя. Лемма Римана-Лебега (light).}


\newpage
\section{Вычисление интеграла Дирихле $\int\limits_0^\infty \frac{\sin x}{x}$.}


\newpage
\section{Ядра Дирихле, их свойства. Выражение частичных сумм ряда Фурье через ядра Дирихле.}


\newpage
\section{Свертка. Простейшие свойства. Свертка с тригонометрическими и алгебраическими полиномами.}


\newpage
\section{Принцип локализации Римана.}


\newpage
\section{Теорема о поточечной сходимости ряда Фурье для локально-Гельдеровой функции.}


\newpage
\section{Ядра Фейера, их свойства. Связь с $\upsigma_N(f)$.}


\newpage
\section{Аппроксимативная единица. Определение, примеры. Теорема о равномерной сходимости свертки с аппроксимативной единицей.}


\newpage
\section{Теорема Фейера. Теорема Вейерштрасса.}


\newpage
\section{Среднеквадратичное приближение функций, интегрируемых по Риману, тригонометрическими полиномами.}


\newpage
\section{Равенство Парсеваля.}

\end{document}

