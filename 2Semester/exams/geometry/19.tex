\documentclass[geometry.tex]{subfiles}

\begin{document}
  \section{Линейная связность}

  \begin{definition}
      Линейно связное пр-во - топ. пр-во, в котором любые две точки можно соединить непр. кривой\\
      $(X, \Omega)$ - лин. св., если $\exists f:$\\
      $f: [0, 1] \rightarrow X \text{(путь в X)} \mid f(0) = x \text{(нач. пути)}; \
      f(1) = y \text{(кон. пути)},\\  \forall x, y \in X$
  \end{definition}

  \begin{theorem}
      X - топ. пр-во\\
      X - лин.св. $\ra$ X - св.
  \end{theorem}

  \begin{theorem}
      A, B - лин. св.  $\q A \cap B \neq \varnothing \ra A \cup B$ - лин.св.
  \end{theorem}

  \begin{theorem}
      X, Y - топ. пр-во; $\q f: X \rightarrow Y$ - непр.\\
      X - лин. св. $\ra$ f(x) - лин. св.
  \end{theorem}
\end{document}
