\documentclass[geometry.tex]{subfiles}

\begin{document}
  \section{Линейная связность}

  \begin{definition}
      X - топ. пр-во, $f: [0,1] \ra X$ - непр.\\
      f называется путем в X
      \[f(0) \text{ - начало пути}\]
      \[f(1) \text{ - конец пути}\]
  \end{definition}

  \begin{definition}
      X называется лин. связным, если $\forall$две точки X можно соединить путём
  \end{definition}

  \begin{remark}
      Начало и конец пути меняются: $g(t) := f(1-t)$
  \end{remark}

  \begin{example}
      *здесь когда-нибудь будет пример*
  \end{example}

  \begin{theorem}
      X - топ. пр-во\\
      X - лин. св. $\Ra$ X - св.
  \end{theorem}

  \begin{proof}
      *здесь когда-нибудь будет док-во*
  \end{proof}

  \begin{example}
      *здесь когда-нибудь будет пример*
  \end{example}

  \begin{definition}
      Компоненты лин. связности - max лин. св. мн-ва
  \end{definition}

  \begin{remark}
      *Компоненты лин. связности не всегда замкнуты*
  \end{remark}

  \begin{ttheorem}
      A, B - лин. св.  $\q A \cap B \neq \varnothing \ra A \cup B$ - лин.св.
  \end{ttheorem}

  \begin{ttheorem}
      X, Y - топ. пр-во; $\q f: X \rightarrow Y$ - непр.\\
      X - лин. св. $\ra$ f(x) - лин. св.
  \end{ttheorem}
\end{document}
