\documentclass[geometry.tex]{subfiles}

\begin{document}
  \section{Топологические пространства. Примеры.}

  \begin{definition}
      X - мн-во\\
      $\Omega \subset 2^X = \{A \subset X\}$ - мн-во подмн. X\\
      $(X, \Omega)$ - назыв. тополог. пр-вом, если\\
      \begin{enumerate}
          \item $\forall \{U_i\}_{i \in I} \in \Omega \Rightarrow \us{i \in I}{\cup} U_i \in \Omega$
          \item $U_1, U_2, ..., U_n \Rightarrow U_1 \cap U_2 \cap ... \cap U_n \in \Omega$
          \item $\varnothing; \  X \in \Omega$\\\\
          $\Omega$ - тополог. на X\\
          $U \in \Omega$ - назыв. открытым мн-вом
      \end{enumerate}
  \end{definition}

  \begin{definition}
      $(X, \Omega)$ - топ. пр-во; $F \subset X$ \\
      F - назыв. замк., если $X \setminus F \in \Omega$
  \end{definition}

  \begin{theorem}
      \begin{enumerate}
          \item $\us{i \in I}{\cap} F_i \text{- замк, если } F_i - \text{замк}$
          \item $F_1 \cup F_2$ - замк ($F_1, F_2$ - замк.)
          \item $\varnothing, X$ - замк.
      \end{enumerate}
  \end{theorem}

  \begin{examples}
      \begin{enumerate}
          \item $(X, \rho)$ - топ. пр-во
          \item дискр. пр-во: $\Omega = 2^X$
          \item антидискр. пр-во: $\Omega = \{\varnothing, X\}$

      \begin{definition}
          $(X, \Omega)$ - метризуемо, если $\exists$ метрика $\rho: X \times X \rightarrow \R_X$\\
          $\Omega = $ мн-во откр. подмн. в $\rho$\\
          Антидискр. - не метризуемо, если |X| > 1
      \end{definition}
          \item Стрелка\\
                $X = \R  \ $ или $\   \R_+ = \{x \geq 0\}$\\
                $\Omega = \{(a, +\infty)\} \cup \{\varnothing\} \cup \{X\}$
          \item Связное двоеточие\\
                $X = \{a, b\}$\\
                $\Omega = \{\varnothing, X, \{a\}\}$
          \item Топология конечных дополнений (Зариского)\\
                X - беск. мн-во\\
                Замкнутые конечные мн-ва и X \\
                $\Omega = \{A \  | \  X \setminus A \text{ конечно}\}$
      \end{enumerate}
  \end{examples}
\end{document}
