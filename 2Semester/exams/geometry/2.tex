\documentclass[geometry.tex]{subfiles}

\begin{document}
  \section{Открытые и замкнутые множества. Свойства}

  \begin{definition}
      $B(x_0, \mathcal{E}) = \{x \in X \ | \ \rho(x, x_0) < \mathcal{E}\}$\\
      Называется открытым шаром с центром в $x_0$ и радиусом $\mathcal{E}$\\
      $\mathcal{E}$ - окр.  $x_0$
  \end{definition}

  \begin{definition}
      $U \subset X \quad U$ - откр., если:
      \[\forall x \in U \quad \exists \mathcal{E}\text{: } B(x, \mathcal{E}) \subset U\]
  \end{definition}

  \begin{definition}
      $Z \subset X \quad Z -$ замкн., если $X \setminus Z$ - откр. мн-во
  \end{definition}

  \begin{theorem}[св-ва откр. мн-в]
      \begin{enumerate}
          \item $\{ U_\alpha \}_{\alpha \in A}$ - семейство откр. мн-в
                 \[\Rightarrow \bigcup_{\alpha \in A}U_\alpha - \text{откр.}\]
          \item $U_1,...,U_n$ - откр.(конеч. число) \[\Rightarrow \bigcap_{i = 1}^n U_i - \text{откр.}\]
          \item $\varnothing, X - $ откр.
      \end{enumerate}
  \end{theorem}
  \begin{proof}
      \begin{enumerate}
          \item $\forall x \in \bigcup\limits_{\alpha \in A} U_\alpha \Rightarrow \exists \alpha_0\text{: } x \in U_{\alpha_0}$
                 \[U_{\alpha_0} - \text{откр.}\Rightarrow \exists \mathcal{E}\text{: } B(x, \mathcal{E}) \subset U_{\alpha_0}\]
                 \[B(x, \mathcal{E}) \subset \bigcup_{\alpha \in A} U_\alpha \Rightarrow
                 \bigcup_{\alpha \in A} U_\alpha - \text{откр.}\]
          \item $\forall x \in \bigcap\limits_{i = 1}^n U_i \Rightarrow \forall i \q x \in U_i$
                \[\exists \mathcal{E}_i\text{: } B(x, \mathcal{E}_i) \subset U_i\]
                \[\mathcal{E} = \min_{i = 1,..., n}\{\mathcal{E}_i\} \quad B(x, \mathcal{E}) \subset B(x, \mathcal{E}_i) \subset U_i\]
                \[B(x, \mathcal{E}) \;\subset\; \bigcap\limits_{i=1}^n U_i\; \Rightarrow\; \bigcap\limits_{i = 1} ^ n U_i - \text{откр}\]
      \end{enumerate}
  \end{proof}

  \begin{Example}
    \[U_i = \left(- \frac{1}{i}, \frac{1}{i}\right)\]
    \[\bigcap_{i = 1}^\infty U_i = \{0\} \text{ - объясняет, почему должно быть конечное число в пересеч.} \]
  \end{Example}

  \begin{lemma}
      $B(x_0, r) - $ откр.\\
      $\forall$ метр. пр-ва $X \quad \forall x_0 \q \forall r > 0$
  \end{lemma}
  \begin{proof}
      $x \in B(x_0, r)\\
      \rho(x_0, x) = d < r\\
      \mathcal{E}=\frac{r-d}{2}\\
      B(x, \mathcal{E}) \subset B(x_0, r) ?\\
      \text{Здесь очень внимательно надо смотреть на предположение}\\
      x_1 \text{ лежит в предполагаемой области за пределами шарика} B(x_0, r) \\
      \sqsupset \exists x_1 \in B(x, \mathcal{E}) \setminus B(x_0, r) \\
      \rho(x_1, x) < \mathcal{E} = r - d\\
      \rho(x_0, x) = d\\
      \rho(x_1, x_0) \geq r \\
      \rho(x_1, x_0) \geq  \rho(x_1, x) + \rho(x, x_0) \\
      \rho(x_1, x_0) \geq r \quad \text{и} \quad \rho(x_1, x) + \rho(x, x_0) < r
      $\\
      противореч. нер-ву $\triangle$
  \end{proof}
  \begin{theorem}[св-ва замк. мн-в]
      \begin{enumerate}
          \item $\{F_i\}_{i \in A} - $ замкн. $$\Rightarrow \bigcap_{i \in A} F_i - \text{замк.}$$
          \item $F_1, ..., F_n - $ замк. $$\Rightarrow \bigcup_{i = 1}^n F_i - \text{замк.}$$
          \item $\varnothing$ и $X$ замк.
      \end{enumerate}
          $F_i = X \setminus U_i, \quad U_i$ - откр.\\
          $\bigcap F_i = \bigcap (X \setminus U_i) = X \setminus \bigcup U_i$
  \end{theorem}
\end{document}
