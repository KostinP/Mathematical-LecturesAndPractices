\documentclass[geometry.tex]{subfiles}

\begin{document}
  \section{Инициальная топология. Топология произведения как инициальная.}

  \begin{definition}
      $\forall i \in I \q f_i: X \rightarrow Y_i$ - все непр\\
      $(Y_i, \Omega_i)$ - топ. пр-во\\
      %$\{f_{i1}^{-1}(U_1) \cap f_{i2}^{-1}(U_2) \cap ... \cap f_{ik}^{-1}(U_k) \  | \ U_j \in \Omega_{i j}; \q j = 1, ..., k; \q k \in \N \}$
      \[\{f^{-1}_{i1} (U_1) \cap f^{-1}_{i2}(U_2) \cap ... \cap f^{-1}_{ik}(U_k) | \
      \us{j = 1, ..., k \in \N}{U_j \in \Omega_{ij}}\} \text{ - база нек. топологии} \]
     % - база нек. топологии\\
      $\Omega_X$ - соотв. топология (инициальная топология)
  \end{definition}

  \begin{definition}
      $\{f_i^{-1}(U)\}$ - предбаза топологии\\
      Предбаза топологии Y - семейство открытых подмн-в топ. пр-ва X такое,
      что совокупность всех
      множеств, являющихся пересечением конечного числа элементов Y, образует базу 
      пространства $X$
  \end{definition}

  \begin{theorem}
      Топология произведения совпадает с инициальной
  \end{theorem}

  \begin{definition}
      \[\prod_{i \in I} x_i = \{f: I \rightarrow \bigcup_{i \in I} x_i \ | \ f(i) \in X_i \}\]\\
      \[p_k : \prod_{i \in I} x_i \rightarrow X_k \q k \in I\]\\
      \[p_k(f) = f(k) \ra  \text{если } x_i \text{- топ.} \ra \prod_{i \in I} x_i - \text{топ.}\]
  \end{definition}
\end{document}
