\documentclass[geometry.tex]{subfiles}

\begin{document}
  \section{База топологии. Критерий базы.}

  \begin{definition}
      X - топ. пр-во; $\quad A \subset X$ \\
      $Int A = \cup U, \quad U \in \Omega \quad U \subset A$\\
      $Cl A = \cap F, \quad F - $ замк. $F \supset A$ \\
      $\partial A = Cl A \setminus Int A$
  \end{definition}

  \begin{definition}
      $x_0 \in X$\\
      окр. $x_0$ назыв. $\forall U \in \Omega: x_0 \in U$
  \end{definition}

  \begin{definition}
      $x_0$ назыв. внутр. т. A, если $\exists U_{x_0} \subset A$\\
      $x_0$ назыв. внеш. т. A, если $\exists U_{x_0} \cap A = \varnothing$\\
      $x_0$ назыв. граничной, если $\forall U_{x_0} \quad (U_{x_0} \not \subset A)$ и $(U_{x_0} \cap A \neq \varnothing)$
  \end{definition}

  \begin{definition}
      $(X, \Omega)$ - топ. пр-во\\
      $\mathcal{B} \subset \Omega \quad \mathcal{B}$ назыв. базой топологии, если\\
      \[\forall U \in \Omega \quad \exists \{V_i\} \in \mathcal{B}: \quad U = \bigcup_{i \in I} V_i\]
  \end{definition}

  \begin{example}
      $X = \R^n$ или другое метр. пр-во\\
      $\B = \{B(x_0, \mathcal{E}) \; | \; x_0 \in X, \mathcal{E} > 0\}$ - база топологии\\
      $\forall U$ - откр. $\quad \forall x_0 \in U \quad \exists \mathcal{E}: B(x_0, \mathcal{E}) \subset U$\\
      \[\bigcup_{x_0 \in U} B(x_0, \mathcal{E}) = U\]
  \end{example}

  \begin{theorem}[Критерий базы]
      X - мн-во $\B$ - нек. совокупность подмн-в X\\
      $\B$ - база $\Omega \Leftrightarrow$ \begin{enumerate}
          \item \[\bigcup_{U_i \in \B} U_i = X\]
          \item $\forall U, V \in \B \quad \forall x \in U \cap V \quad \exists W \in \B : x \in W; W \subset U\cap V$
      \end{enumerate}
  \end{theorem}

  \begin{proof}
      $\ra$ очев\\
      \[\la \Omega = \{\bigcup_{i \in I} U_i | \quad U_i \in \B\}\]\\
      \begin{enumerate}
          \item \[\bigcup_{j \in J}(\bigcup_{i \in I_j} U_i) = \bigcup_{i, j} U_i \]
          \item \[(\bigcup_j U_j) \cap (\bigcup_i U_i)  =  \bigcup_{i, j} (U_i \cap U_j) =
          \bigcup_{i, j} (\bigcup_{x \in U_i \cap U_j} W_x)\]\\
          \[x \in W_x \subset U_i \cap U_j\]
          \[\bigcup_{x \in U_i \cap U_j} W_x = U_i \cap U_j \q W_x \in \B\]
          \item \[\varnothing = \bigcup_{i \in \varnothing} U_i \q X = \bigcup_{U_i \in \B} U_i\]
      \end{enumerate}
  \end{proof}

  \begin{theorem} [База окр. точки]
          X - мн-во $\forall x \in X \q \exists \B_x \subset 2^x$
          \begin{enumerate}
              \item $x \in U \q \forall U \in \B_x$
              \item $U, V \in \B_x \Ra \forall t \in U \cap V \q\exists W \in \B_x: \q t \in  W \subset U \cap V$
              \item $y \in U \q(U \in \B_x) \Ra \exists V \in \B_y: \q V \subset U$
              \setItemnumber{0}
              \item \[\B_x \neq \varnothing \Ra \bigcup_{x \in X} \B_x - \text{база нек. топологии}\]
            Топология, порожденная системой окр.
          \end{enumerate}
  \end{theorem}
\end{document}
