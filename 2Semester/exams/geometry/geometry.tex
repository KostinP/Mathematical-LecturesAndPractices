\documentclass[12pt, fleqn]{article}
\usepackage[english, russian]{babel} %Russian lang
\usepackage[utf8]{inputenc}
\usepackage{amsfonts, amsmath, amssymb} %Math symbols
\usepackage{ntheorem}
\usepackage[paperwidth=170mm,paperheight=240mm,centering,height=186mm,width=120mm]{geometry}

\usepackage{pstricks-add}
\usepackage{auto-pst-pdf}
\usepackage{pst-pdf}


\geometry{left=1.5cm}
\geometry{right=1.5cm}
\geometry{top=1.5cm}
\geometry{bottom=15mm}

\newenvironment{question}[1]{\hspace*{-4em} #1}{\newpage}
% \newcommand{\definition}{\paragraph{\underline{\bfseries Опр}}}
\newcommand{\R}{\mathbb{R}}
\newcommand{\N}{\mathbb{N}}
\newcommand{\B}{\mathcal{B}}
\newcommand{\ra}{\Rightarrow}
\newcommand{\la}{\Leftarrow}
\newcommand{\rla}{\Leftrightarrow}
\newcommand{\lra}{\Leftrightarrow}
\newcommand{\q}{\quad}
\newcommand\setItemnumber[1]{\setcounter{enumi}{\numexpr#1-1\relax}}

\theoremstyle{nonumbermarginbreak}
\newtheorem{theorem}{\hspace*{-2em}\underline{\bfseries Теор}}[section]
\newtheorem{definition}{\hspace*{-2em}\underline{\bfseries Опр}}[section]
\newtheorem{examples}{\hspace*{-2em}\underline{\bfseries Примеры}}[section]
\newtheorem{example}{\hspace*{-2em}\underline{\bfseries Пример}}[section]
\newtheorem{remark}{\hspace*{-2em}\underline{\bfseries Замеч}}[section]
\newtheorem{proof}{\hspace*{-2em}\underline{\bfseries Док-во}}[section]
\newtheorem{lemma}{\hspace*{-2em}\underline{\bfseries Лемма}}[section]
\newtheorem{hypothesis}{\hspace*{-2em}\underline{\bfseries Предп}}[section]
\newtheorem{consequence}{\hspace*{-2em}\underline{\bfseries След}}[section]


\begin{document}

    \begin{question}{1. Метрические пространства. Примеры.}
        \begin{definition} 
            X - мн-во (X $\neq$ $\varnothing$) \\
            $\rho: X \times X \rightarrow \R$ (метрика)\\
            Пара $(X, \rho)$ назыв. метр. пр-вом, если:\\
            \begin{enumerate} 
                \item $\rho(x, y) \geq 0$ и $\rho(x, y) = 0 \Leftrightarrow x = y$ 
                \item $\rho(x, y) = \rho(y, x)$
                \item нер-во $\bigtriangleup$ \\ $\rho(x, z) \leq \rho(x, y) + \rho(y, z)$
            \end{enumerate}
        \end{definition}

        \begin{examples}
            \begin{enumerate}
                \item $\R, \R^2, \R^3$ со станд. $\rho$
                \item На $\R^2$\\
                      $\rho_1((x_1, y_1), (x_2, y_2)) = |x_1 - x_2| + |y_1 - y_2|$\\
                      $\rho_\infty = max\left\{|x_1 - x_2|, |y_1 - y_2|\right\}$\\
                      $\rho_p = (|x_1 - x_2|^p + |y_1 - y_2|^p)^{\frac{1}{p}}$
                      $\rho_2$ - Евклидова метрика
                \item X - мн-во\\
                    \[\rho(a, b) =  \begin{cases}
                            0, &a = b\\
                            1, &a \neq b
                                \end{cases}- \text{Дискретная метрика}\]
            \end{enumerate}
        \end{examples}

    \end{question}
    
    \begin{question}{2. Открытые и замкнутые множества. Свойства}
        \begin{definition} 
            $B(x_0, \epsilon) = \{x \in X \; | \; \rho(x, x_0) < \mathcal{E}\}$\\
            Открытый шар с центром в $x_0$ и радиусом $\mathcal{E}$\\
            $\mathcal{E}$ - откр. в $x_0$
        \end{definition}
        
        \begin{definition} 
            $U \subset X \quad U -$откр, если\\
            $\forall x \in U \quad \exists \mathcal{E}: B(x, \mathcal{E}) \subset U$
        \end{definition}

        \begin{definition} 
            $Z \subset X \quad Z -$ замкн., если $X \setminus Z$ - откр. мн-во
        \end{definition}

        \begin{theorem}[св-ва откр. мн-в] 
            \begin{enumerate}
                \item $\{ U_\alpha \}_{\alpha \in A}$ - семейство откр. мн-в 
                       $$\Rightarrow \bigcup_{\alpha \in A}U_\alpha - \text{откр.}$$
                \item $U_1...U_n$ - откр.(конеч. число) \[\Rightarrow \bigcap_{i = 1}^n U_i - \text{откр.}\] 
                \item $\varnothing, X - $ откр.
            \end{enumerate}
        \end{theorem}
        \begin{proof} 
            \begin{enumerate}
                \item \[\forall x \in \bigcup_{\alpha \in A} U_\alpha \Rightarrow \exists \alpha_0: x \in U_{\alpha_0}\]\\
                       \[U_{\alpha_0} - \text{откр.}\Rightarrow \exists \mathcal{E} : B(x, \mathcal{E}) \subset U_{\alpha_0}\]\\
                       \[B(x, \mathcal{E}) \subset \bigcup_{\alpha \in A} U_\alpha \Rightarrow 
                       \bigcup_{\alpha \in A} U_\alpha - \text{откр.}\]
                \item $$\forall x \in \bigcap_{i = 1}^n U_i \Rightarrow \forall i \; x \in U_i$$\\
                      $$\exists \mathcal{E}_i : B(x, \mathcal{E}_i) \subset U_i$$\\
                      $$\mathcal{E} = \min_{i = 1,..., n}\{\mathcal{E}_i\} \quad B(x, \mathcal{E}) \subset B(x, \mathcal{E}_i) \subset U_i$$
                      $$B(x, \mathcal{E}) \;\subset\; \bigcap_{i=1}^n U_i\; \Rightarrow\; \bigcap_{i = 1} ^ n U_i - \text{откр}$$
            \end{enumerate}
        \end{proof}
        \begin{lemma} 
            $B(x_0, r) - $ откр.\\
            $\forall$ метр. пр-ва $X \quad \forall x_0 \; \forall r > 0$
        \end{lemma}
        \begin{proof} 
            $x \in B(x_0, r)\\
            \rho(x_0, x) = d < r\\
            \mathcal{E}=\frac{r-d}{2}\\
            B(x, \mathcal{E}) \subset B(x_0, r) ?\\
            \sqsupset \exists x_1 \in B(x, \mathcal{E}) \setminus B(x_0, r) \\
            \rho(x_1, x) < \mathcal{E} = r - d\\
            \rho(x_0, x) = d\\
            \rho(x_1, x_0) \geq r \\
            \rho(x_1, x_0) \geq  \rho(x_1, x) + \rho(x, x_0) \\
            \rho(x_1, x_0) \geq r \quad \text{и} \quad \rho(x_1, x) + \rho(x, x_0) < r
            $\\
            противореч. нер-ву $\triangle$

            
            %LaTeX with PSTricks extensions
%%Creator: inkscape 0.91
%%Please note this file requires PSTricks extensions
\psset{xunit=.5pt,yunit=.5pt,runit=.5pt}
\begin{pspicture}(744.09448819,1052.36220472)
{
\newrgbcolor{curcolor}{1 1 1}
\pscustom[linestyle=none,fillstyle=solid,fillcolor=curcolor]
{
\newpath
\moveto(239.6186142,881.1134712)
\curveto(239.6186142,818.13553174)(187.99906661,767.08182448)(124.32312775,767.08182448)
\curveto(60.64718888,767.08182448)(9.0276413,818.13553174)(9.0276413,881.1134712)
\curveto(9.0276413,944.09141067)(60.64718888,995.14511793)(124.32312775,995.14511793)
\curveto(187.99906661,995.14511793)(239.6186142,944.09141067)(239.6186142,881.1134712)
\closepath
}
}
{
\newrgbcolor{curcolor}{0 0 0}
\pscustom[linewidth=2.34745908,linecolor=curcolor]
{
\newpath
\moveto(239.6186142,881.1134712)
\curveto(239.6186142,818.13553174)(187.99906661,767.08182448)(124.32312775,767.08182448)
\curveto(60.64718888,767.08182448)(9.0276413,818.13553174)(9.0276413,881.1134712)
\curveto(9.0276413,944.09141067)(60.64718888,995.14511793)(124.32312775,995.14511793)
\curveto(187.99906661,995.14511793)(239.6186142,944.09141067)(239.6186142,881.1134712)
\closepath
}
}
{
\newrgbcolor{curcolor}{1 1 1}
\pscustom[linestyle=none,fillstyle=solid,fillcolor=curcolor]
{
\newpath
\moveto(116.97471619,924.46495741)
\curveto(116.97471619,909.23457261)(104.06300282,896.88790625)(88.13559723,896.88790625)
\curveto(72.20819164,896.88790625)(59.29647827,909.23457261)(59.29647827,924.46495741)
\curveto(59.29647827,939.69534221)(72.20819164,952.04200857)(88.13559723,952.04200857)
\curveto(104.06300282,952.04200857)(116.97471619,939.69534221)(116.97471619,924.46495741)
\closepath
}
}
{
\newrgbcolor{curcolor}{0 0 0}
\pscustom[linewidth=2.6828804,linecolor=curcolor]
{
\newpath
\moveto(116.97471619,924.46495741)
\curveto(116.97471619,909.23457261)(104.06300282,896.88790625)(88.13559723,896.88790625)
\curveto(72.20819164,896.88790625)(59.29647827,909.23457261)(59.29647827,924.46495741)
\curveto(59.29647827,939.69534221)(72.20819164,952.04200857)(88.13559723,952.04200857)
\curveto(104.06300282,952.04200857)(116.97471619,939.69534221)(116.97471619,924.46495741)
\closepath
}
}
{
\newrgbcolor{curcolor}{0 0 0}
\pscustom[linewidth=2.29102707,linecolor=curcolor]
{
\newpath
\moveto(87.737936,922.78140472)
\lineto(129.89045,863.01529472)
}
}
{
\newrgbcolor{curcolor}{0 0 0}
\pscustom[linewidth=2.29102707,linecolor=curcolor]
{
\newpath
\moveto(129.89045,863.01529472)
\lineto(222.94411,941.30048472)
}
}
{
\newrgbcolor{curcolor}{0 0 0}
\pscustom[linestyle=none,fillstyle=solid,fillcolor=curcolor]
{
\newpath
\moveto(116.46551876,853.08560597)
\lineto(111.52017867,860.50178677)
\lineto(114.19509877,860.50178677)
\lineto(117.88779335,854.78428426)
\lineto(121.54134275,860.50178677)
\lineto(124.19016607,860.50178677)
\lineto(119.24482598,853.08560597)
\lineto(124.46418228,845.33797575)
\lineto(121.78926218,845.33797575)
\lineto(117.88779335,851.38692769)
\lineto(113.93413096,845.33797575)
\lineto(111.28530764,845.33797575)
\lineto(116.46551876,853.08560597)
\closepath
}
}
{
\newrgbcolor{curcolor}{0 0 0}
\pscustom[linestyle=none,fillstyle=solid,fillcolor=curcolor]
{
\newpath
\moveto(134.03060907,846.26119997)
\curveto(134.03060907,843.98110434)(133.68852378,842.28449775)(133.00435322,841.1713802)
\curveto(132.32583695,840.05826265)(131.29109965,839.50170388)(129.90014131,839.50170388)
\curveto(128.55441739,839.50170388)(127.53381584,840.07322391)(126.83833667,841.21626397)
\curveto(126.1428575,842.36528854)(125.79511792,844.04693387)(125.79511792,846.26119997)
\curveto(125.79511792,848.5472801)(126.12872175,850.24388669)(126.7959294,851.35101973)
\curveto(127.46879137,852.45815278)(128.50352867,853.0117193)(129.90014131,853.0117193)
\curveto(131.24586523,853.0117193)(132.26929393,852.43720702)(132.97042741,851.28818245)
\curveto(133.67721518,850.13915788)(134.03060907,848.46349706)(134.03060907,846.26119997)
\closepath
\moveto(127.33026096,846.26119997)
\curveto(127.33026096,844.35812803)(127.53381584,842.9846846)(127.9409256,842.14086968)
\curveto(128.35368966,841.29705476)(129.00676156,840.87514731)(129.90014131,840.87514731)
\curveto(130.80482966,840.87514731)(131.46072872,841.30603152)(131.86783848,842.16779994)
\curveto(132.28060254,843.03555287)(132.48698457,844.40001955)(132.48698457,846.26119997)
\curveto(132.48698457,848.10442688)(132.28060254,849.4599168)(131.86783848,850.32766973)
\curveto(131.46072872,851.19542266)(130.80482966,851.62929912)(129.90014131,851.62929912)
\curveto(128.99545296,851.62929912)(128.3395539,851.20439941)(127.93244415,850.35459999)
\curveto(127.53098869,849.50480057)(127.33026096,848.1403339)(127.33026096,846.26119997)
\closepath
}
}
{
\newrgbcolor{curcolor}{0 0 0}
\pscustom[linestyle=none,fillstyle=solid,fillcolor=curcolor]
{
\newpath
\moveto(80.91160482,933.37718851)
\lineto(75.76020888,941.10237686)
\lineto(78.54658399,941.10237686)
\lineto(82.39314085,935.14664506)
\lineto(86.19892149,941.10237686)
\lineto(88.95811246,941.10237686)
\lineto(83.80671652,933.37718851)
\lineto(89.24354601,925.30674034)
\lineto(86.4571709,925.30674034)
\lineto(82.39314085,931.60773196)
\lineto(78.27474252,925.30674034)
\lineto(75.51555155,925.30674034)
\lineto(80.91160482,933.37718851)
\closepath
}
}
{
\newrgbcolor{curcolor}{0 0 0}
\pscustom[linestyle=none,fillstyle=solid,fillcolor=curcolor]
{
\newpath
\moveto(188.52512456,895.93417353)
\curveto(189.18660548,895.93417353)(189.77106465,895.87663023)(190.27850206,895.76154362)
\lineto(189.97947644,893.37349657)
\curveto(189.42673211,893.50776428)(188.88757985,893.57489813)(188.36201967,893.57489813)
\curveto(187.57367939,893.57489813)(186.83970741,893.34472492)(186.16010373,892.8843785)
\curveto(185.48956142,892.42403208)(184.96400124,891.78626048)(184.58342318,890.9710637)
\curveto(184.20284511,890.16545746)(184.01255608,889.268741)(184.01255608,888.28091431)
\lineto(184.01255608,879.8508205)
\lineto(181.56598281,879.8508205)
\lineto(181.56598281,895.64645702)
\lineto(183.57760972,895.64645702)
\lineto(183.8494512,892.75490607)
\lineto(183.95818779,892.75490607)
\curveto(184.52905488,893.80027607)(185.20412788,894.59149647)(185.98340677,895.1285673)
\curveto(186.76268566,895.66563812)(187.60992492,895.93417353)(188.52512456,895.93417353)
\closepath
}
}
{
\newrgbcolor{curcolor}{0 0 0}
\pscustom[linestyle=none,fillstyle=solid,fillcolor=curcolor]
{
\newpath
\moveto(120.84084073,887.13910502)
\curveto(118.90170488,887.13910502)(117.39751539,887.84880575)(116.32827226,889.26820721)
\curveto(115.25902913,890.68760867)(114.72440756,892.69682898)(114.72440756,895.29586814)
\curveto(114.72440756,897.8757262)(115.25902913,899.88974178)(116.32827226,901.33791489)
\curveto(117.40657678,902.78608801)(118.91982765,903.51017456)(120.86802488,903.51017456)
\curveto(122.87059041,903.51017456)(124.41102543,902.73333998)(125.48932995,901.17967081)
\lineto(125.6660269,901.17967081)
\curveto(125.63884276,901.37148182)(125.60259723,901.75510384)(125.55729031,902.33053686)
\curveto(125.5119834,902.91556044)(125.48932995,903.308773)(125.48932995,903.51017456)
\lineto(125.48932995,909.81116618)
\lineto(127.93590322,909.81116618)
\lineto(127.93590322,887.42682153)
\lineto(125.96505253,887.42682153)
\lineto(125.59806654,889.54153789)
\lineto(125.48932995,889.54153789)
\curveto(124.44727096,887.93991598)(122.89777456,887.13910502)(120.84084073,887.13910502)
\closepath
\moveto(121.23501087,889.28259303)
\curveto(122.7210776,889.28259303)(123.80391281,889.70937253)(124.48351649,890.56293151)
\curveto(125.17218156,891.42608105)(125.51651409,892.83589196)(125.51651409,894.79236424)
\lineto(125.51651409,895.26709649)
\curveto(125.51651409,897.46333253)(125.16765087,899.02659225)(124.46992442,899.95687564)
\curveto(123.78125935,900.89674958)(122.69389345,901.36668655)(121.20782673,901.36668655)
\curveto(119.94829456,901.36668655)(118.97419594,900.834411)(118.28553088,899.7698599)
\curveto(117.59686581,898.71489936)(117.25253327,897.20438767)(117.25253327,895.23832484)
\curveto(117.25253327,893.28185255)(117.59233512,891.80011251)(118.2719388,890.79310472)
\curveto(118.96060387,889.78609693)(119.94829456,889.28259303)(121.23501087,889.28259303)
\closepath
}
}
{
\newrgbcolor{curcolor}{1 1 1}
\pscustom[linestyle=none,fillstyle=solid,fillcolor=curcolor]
{
\newpath
\moveto(473.60427856,902.96489637)
\curveto(473.60427856,863.62206603)(440.29392509,831.7284004)(399.20352173,831.7284004)
\curveto(358.11311837,831.7284004)(324.80276489,863.62206603)(324.80276489,902.96489637)
\curveto(324.80276489,942.30772672)(358.11311837,974.20139235)(399.20352173,974.20139235)
\curveto(440.29392509,974.20139235)(473.60427856,942.30772672)(473.60427856,902.96489637)
\closepath
}
}
{
\newrgbcolor{curcolor}{0 0 0}
\pscustom[linewidth=1.32484317,linecolor=curcolor]
{
\newpath
\moveto(473.60427856,902.96489637)
\curveto(473.60427856,863.62206603)(440.29392509,831.7284004)(399.20352173,831.7284004)
\curveto(358.11311837,831.7284004)(324.80276489,863.62206603)(324.80276489,902.96489637)
\curveto(324.80276489,942.30772672)(358.11311837,974.20139235)(399.20352173,974.20139235)
\curveto(440.29392509,974.20139235)(473.60427856,942.30772672)(473.60427856,902.96489637)
\closepath
}
}
{
\newrgbcolor{curcolor}{0 0 0}
\pscustom[linewidth=1.32484317,linecolor=curcolor]
{
\newpath
\moveto(385.06506729,860.70596998)
\curveto(385.06506729,831.89901453)(361.71669068,808.54636305)(332.91500854,808.54636305)
\curveto(304.11332641,808.54636305)(280.7649498,831.89901453)(280.7649498,860.70596998)
\curveto(280.7649498,889.51292544)(304.11332641,912.86557692)(332.91500854,912.86557692)
\curveto(361.71669068,912.86557692)(385.06506729,889.51292544)(385.06506729,860.70596998)
\closepath
}
}
{
\newrgbcolor{curcolor}{1 1 1}
\pscustom[linestyle=none,fillstyle=solid,fillcolor=curcolor]
{
\newpath
\moveto(326.65699518,853.94452589)
\curveto(326.65699518,853.14433271)(326.03437178,852.49564796)(325.2663269,852.49564796)
\curveto(324.49828203,852.49564796)(323.87565863,853.14433271)(323.87565863,853.94452589)
\curveto(323.87565863,854.74471908)(324.49828203,855.39340382)(325.2663269,855.39340382)
\curveto(326.03437178,855.39340382)(326.65699518,854.74471908)(326.65699518,853.94452589)
\closepath
}
}
{
\newrgbcolor{curcolor}{0 0 0}
\pscustom[linewidth=1.32484317,linecolor=curcolor]
{
\newpath
\moveto(326.65699518,853.94452589)
\curveto(326.65699518,853.14433271)(326.03437178,852.49564796)(325.2663269,852.49564796)
\curveto(324.49828203,852.49564796)(323.87565863,853.14433271)(323.87565863,853.94452589)
\curveto(323.87565863,854.74471908)(324.49828203,855.39340382)(325.2663269,855.39340382)
\curveto(326.03437178,855.39340382)(326.65699518,854.74471908)(326.65699518,853.94452589)
\closepath
}
}
{
\newrgbcolor{curcolor}{1 1 1}
\pscustom[linestyle=none,fillstyle=solid,fillcolor=curcolor]
{
\newpath
\moveto(354.47034967,880.99024122)
\curveto(354.47034967,880.19004804)(353.84772628,879.54136329)(353.0796814,879.54136329)
\curveto(352.31163652,879.54136329)(351.68901312,880.19004804)(351.68901312,880.99024122)
\curveto(351.68901312,881.79043441)(352.31163652,882.43911915)(353.0796814,882.43911915)
\curveto(353.84772628,882.43911915)(354.47034967,881.79043441)(354.47034967,880.99024122)
\closepath
}
}
{
\newrgbcolor{curcolor}{0 0 0}
\pscustom[linewidth=1.32484317,linecolor=curcolor]
{
\newpath
\moveto(354.47034967,880.99024122)
\curveto(354.47034967,880.19004804)(353.84772628,879.54136329)(353.0796814,879.54136329)
\curveto(352.31163652,879.54136329)(351.68901312,880.19004804)(351.68901312,880.99024122)
\curveto(351.68901312,881.79043441)(352.31163652,882.43911915)(353.0796814,882.43911915)
\curveto(353.84772628,882.43911915)(354.47034967,881.79043441)(354.47034967,880.99024122)
\closepath
}
}
{
\newrgbcolor{curcolor}{1 1 1}
\pscustom[linestyle=none,fillstyle=solid,fillcolor=curcolor]
{
\newpath
\moveto(401.28953302,905.1382057)
\curveto(401.28953302,904.33801252)(400.66690963,903.68932777)(399.89886475,903.68932777)
\curveto(399.13081987,903.68932777)(398.50819647,904.33801252)(398.50819647,905.1382057)
\curveto(398.50819647,905.93839889)(399.13081987,906.58708363)(399.89886475,906.58708363)
\curveto(400.66690963,906.58708363)(401.28953302,905.93839889)(401.28953302,905.1382057)
\closepath
}
}
{
\newrgbcolor{curcolor}{0 0 0}
\pscustom[linewidth=1.32484317,linecolor=curcolor]
{
\newpath
\moveto(401.28953302,905.1382057)
\curveto(401.28953302,904.33801252)(400.66690963,903.68932777)(399.89886475,903.68932777)
\curveto(399.13081987,903.68932777)(398.50819647,904.33801252)(398.50819647,905.1382057)
\curveto(398.50819647,905.93839889)(399.13081987,906.58708363)(399.89886475,906.58708363)
\curveto(400.66690963,906.58708363)(401.28953302,905.93839889)(401.28953302,905.1382057)
\closepath
}
}
{
\newrgbcolor{curcolor}{1 1 1}
\pscustom[linestyle=none,fillstyle=solid,fillcolor=curcolor]
{
\newpath
\moveto(719.41207886,921.84929579)
\curveto(719.41207886,872.37068436)(676.55483437,832.26034468)(623.68774414,832.26034468)
\curveto(570.82065391,832.26034468)(527.96340942,872.37068436)(527.96340942,921.84929579)
\curveto(527.96340942,971.32790722)(570.82065391,1011.4382469)(623.68774414,1011.4382469)
\curveto(676.55483437,1011.4382469)(719.41207886,971.32790722)(719.41207886,921.84929579)
\closepath
}
}
{
\newrgbcolor{curcolor}{0 0 0}
\pscustom[linewidth=1.32484317,linecolor=curcolor]
{
\newpath
\moveto(719.41207886,921.84929579)
\curveto(719.41207886,872.37068436)(676.55483437,832.26034468)(623.68774414,832.26034468)
\curveto(570.82065391,832.26034468)(527.96340942,872.37068436)(527.96340942,921.84929579)
\curveto(527.96340942,971.32790722)(570.82065391,1011.4382469)(623.68774414,1011.4382469)
\curveto(676.55483437,1011.4382469)(719.41207886,971.32790722)(719.41207886,921.84929579)
\closepath
}
}
{
\newrgbcolor{curcolor}{1 1 1}
\pscustom[linestyle=none,fillstyle=solid,fillcolor=curcolor]
{
\newpath
\moveto(664.67134476,923.05669325)
\curveto(664.67134476,900.40826136)(646.21863084,882.04807966)(623.45605469,882.04807966)
\curveto(600.69347853,882.04807966)(582.24076462,900.40826136)(582.24076462,923.05669325)
\curveto(582.24076462,945.70512514)(600.69347853,964.06530684)(623.45605469,964.06530684)
\curveto(646.21863084,964.06530684)(664.67134476,945.70512514)(664.67134476,923.05669325)
\closepath
}
}
{
\newrgbcolor{curcolor}{0 0 0}
\pscustom[linewidth=1.40896142,linecolor=curcolor]
{
\newpath
\moveto(664.67134476,923.05669325)
\curveto(664.67134476,900.40826136)(646.21863084,882.04807966)(623.45605469,882.04807966)
\curveto(600.69347853,882.04807966)(582.24076462,900.40826136)(582.24076462,923.05669325)
\curveto(582.24076462,945.70512514)(600.69347853,964.06530684)(623.45605469,964.06530684)
\curveto(646.21863084,964.06530684)(664.67134476,945.70512514)(664.67134476,923.05669325)
\closepath
}
}
{
\newrgbcolor{curcolor}{1 1 1}
\pscustom[linestyle=none,fillstyle=solid,fillcolor=curcolor]
{
\newpath
\moveto(626.70084894,923.5396797)
\curveto(626.70084894,922.73948651)(626.07822554,922.09080177)(625.31018066,922.09080177)
\curveto(624.54213579,922.09080177)(623.91951239,922.73948651)(623.91951239,923.5396797)
\curveto(623.91951239,924.33987289)(624.54213579,924.98855763)(625.31018066,924.98855763)
\curveto(626.07822554,924.98855763)(626.70084894,924.33987289)(626.70084894,923.5396797)
\closepath
}
}
{
\newrgbcolor{curcolor}{0 0 0}
\pscustom[linewidth=1.32484317,linecolor=curcolor]
{
\newpath
\moveto(626.70084894,923.5396797)
\curveto(626.70084894,922.73948651)(626.07822554,922.09080177)(625.31018066,922.09080177)
\curveto(624.54213579,922.09080177)(623.91951239,922.73948651)(623.91951239,923.5396797)
\curveto(623.91951239,924.33987289)(624.54213579,924.98855763)(625.31018066,924.98855763)
\curveto(626.07822554,924.98855763)(626.70084894,924.33987289)(626.70084894,923.5396797)
\closepath
}
}
{
\newrgbcolor{curcolor}{1 1 1}
\pscustom[linestyle=none,fillstyle=solid,fillcolor=curcolor]
{
\newpath
\moveto(565.04789484,888.76659315)
\curveto(565.04789484,887.96639997)(564.42527144,887.31771522)(563.65722656,887.31771522)
\curveto(562.88918168,887.31771522)(562.26655829,887.96639997)(562.26655829,888.76659315)
\curveto(562.26655829,889.56678634)(562.88918168,890.21547108)(563.65722656,890.21547108)
\curveto(564.42527144,890.21547108)(565.04789484,889.56678634)(565.04789484,888.76659315)
\closepath
}
}
{
\newrgbcolor{curcolor}{0 0 0}
\pscustom[linewidth=1.32484317,linecolor=curcolor]
{
\newpath
\moveto(565.04789484,888.76659315)
\curveto(565.04789484,887.96639997)(564.42527144,887.31771522)(563.65722656,887.31771522)
\curveto(562.88918168,887.31771522)(562.26655829,887.96639997)(562.26655829,888.76659315)
\curveto(562.26655829,889.56678634)(562.88918168,890.21547108)(563.65722656,890.21547108)
\curveto(564.42527144,890.21547108)(565.04789484,889.56678634)(565.04789484,888.76659315)
\closepath
}
}
{
\newrgbcolor{curcolor}{0 0 0}
\pscustom[linestyle=none,fillstyle=solid,fillcolor=curcolor]
{
\newpath
\moveto(567.09542182,878.42517387)
\lineto(564.09294533,882.85741033)
\lineto(565.71697615,882.85741033)
\lineto(567.95893089,879.44037887)
\lineto(570.17711932,882.85741033)
\lineto(571.78530593,882.85741033)
\lineto(568.78282945,878.42517387)
\lineto(571.95167007,873.79484864)
\lineto(570.32763925,873.79484864)
\lineto(567.95893089,877.40996887)
\lineto(565.55853412,873.79484864)
\lineto(563.9503475,873.79484864)
\lineto(567.09542182,878.42517387)
\closepath
}
}
{
\newrgbcolor{curcolor}{0 0 0}
\pscustom[linestyle=none,fillstyle=solid,fillcolor=curcolor]
{
\newpath
\moveto(575.98837693,870.4141335)
\lineto(575.08208856,870.4141335)
\lineto(575.08208856,875.89706557)
\curveto(575.08208856,876.38348355)(575.0958202,876.8341355)(575.12328349,877.24902143)
\curveto(575.02716199,877.14172334)(574.92417468,877.04157846)(574.81432154,876.94858679)
\curveto(574.46416467,876.64457556)(574.08139515,876.3155281)(573.66601298,875.96144442)
\lineto(573.1819726,876.61059783)
\lineto(575.22112143,878.25762341)
\lineto(575.98837693,878.25762341)
\lineto(575.98837693,870.4141335)
\closepath
}
}
{
\newrgbcolor{curcolor}{0 0 0}
\pscustom[linestyle=none,fillstyle=solid,fillcolor=curcolor]
{
\newpath
\moveto(620.86786519,911.26641953)
\lineto(617.86538871,915.69865599)
\lineto(619.48941952,915.69865599)
\lineto(621.73137426,912.28162453)
\lineto(623.94956269,915.69865599)
\lineto(625.55774931,915.69865599)
\lineto(622.55527282,911.26641953)
\lineto(625.72411344,906.6360943)
\lineto(624.10008262,906.6360943)
\lineto(621.73137426,910.25121453)
\lineto(619.33097749,906.6360943)
\lineto(617.72279088,906.6360943)
\lineto(620.86786519,911.26641953)
\closepath
}
}
{
\newrgbcolor{curcolor}{0 0 0}
\pscustom[linestyle=none,fillstyle=solid,fillcolor=curcolor]
{
\newpath
\moveto(631.53220211,907.18785392)
\curveto(631.53220211,905.82516826)(631.32451103,904.81120137)(630.90912886,904.14595326)
\curveto(630.4971796,903.48070514)(629.86895698,903.14808108)(629.024461,903.14808108)
\curveto(628.20742831,903.14808108)(627.58778796,903.48964664)(627.16553997,904.17277778)
\curveto(626.74329198,904.85948551)(626.53216799,905.86451089)(626.53216799,907.18785392)
\curveto(626.53216799,908.55411619)(626.73470971,909.56808308)(627.13979315,910.22975459)
\curveto(627.54830949,910.89142611)(628.17653211,911.22226187)(629.024461,911.22226187)
\curveto(629.8414937,911.22226187)(630.4628505,910.878908)(630.8885314,910.19220027)
\curveto(631.31764521,909.50549253)(631.53220211,908.50404375)(631.53220211,907.18785392)
\closepath
\moveto(627.46420319,907.18785392)
\curveto(627.46420319,906.05049424)(627.58778796,905.2296639)(627.83495752,904.72536291)
\curveto(628.08555998,904.22106191)(628.48206115,903.96891142)(629.024461,903.96891142)
\curveto(629.57372668,903.96891142)(629.9719443,904.22642682)(630.21911385,904.74145762)
\curveto(630.46971632,905.26006502)(630.59501755,906.07553046)(630.59501755,907.18785392)
\curveto(630.59501755,908.28944758)(630.46971632,909.09954811)(630.21911385,909.61815552)
\curveto(629.9719443,910.13676292)(629.57372668,910.39606662)(629.024461,910.39606662)
\curveto(628.47519533,910.39606662)(628.07697771,910.14212783)(627.82980815,909.63425023)
\curveto(627.58607151,909.12637263)(627.46420319,908.3109072)(627.46420319,907.18785392)
\closepath
}
}
{
\newrgbcolor{curcolor}{0 0 0}
\pscustom[linestyle=none,fillstyle=solid,fillcolor=curcolor]
{
\newpath
\moveto(639.41016952,922.3744728)
\lineto(636.40769304,926.80670926)
\lineto(638.03172385,926.80670926)
\lineto(640.27367859,923.3896778)
\lineto(642.49186702,926.80670926)
\lineto(644.10005364,926.80670926)
\lineto(641.09757715,922.3744728)
\lineto(644.26641777,917.74414756)
\lineto(642.64238695,917.74414756)
\lineto(640.27367859,921.3592678)
\lineto(637.87328182,917.74414756)
\lineto(636.26509521,917.74414756)
\lineto(639.41016952,922.3744728)
\closepath
}
}
{
\newrgbcolor{curcolor}{0 0 0}
\pscustom[linestyle=none,fillstyle=solid,fillcolor=curcolor]
{
\newpath
\moveto(407.26460473,887.17995853)
\lineto(404.26212824,891.612195)
\lineto(405.88615906,891.612195)
\lineto(408.12811381,888.19516353)
\lineto(410.34630225,891.612195)
\lineto(411.95448886,891.612195)
\lineto(408.95201237,887.17995853)
\lineto(412.120853,882.54963328)
\lineto(410.49682218,882.54963328)
\lineto(408.12811381,886.16475353)
\lineto(405.72771703,882.54963328)
\lineto(404.11953041,882.54963328)
\lineto(407.26460473,887.17995853)
\closepath
}
}
{
\newrgbcolor{curcolor}{0 0 0}
\pscustom[linestyle=none,fillstyle=solid,fillcolor=curcolor]
{
\newpath
\moveto(417.92894169,883.10139291)
\curveto(417.92894169,881.73870724)(417.72125061,880.72474035)(417.30586844,880.05949223)
\curveto(416.89391918,879.39424411)(416.26569655,879.06162005)(415.42120057,879.06162005)
\curveto(414.60416787,879.06162005)(413.98452753,879.40318562)(413.56227954,880.08631675)
\curveto(413.14003155,880.77302449)(412.92890755,881.77804987)(412.92890755,883.10139291)
\curveto(412.92890755,884.46765518)(413.13144927,885.48162207)(413.53653271,886.14329359)
\curveto(413.94504906,886.80496511)(414.57327168,887.13580086)(415.42120057,887.13580086)
\curveto(416.23823327,887.13580086)(416.85959007,886.792447)(417.28527097,886.10573926)
\curveto(417.71438479,885.41903152)(417.92894169,884.41758274)(417.92894169,883.10139291)
\closepath
\moveto(413.86094275,883.10139291)
\curveto(413.86094275,881.96403322)(413.98452753,881.14320288)(414.23169708,880.63890188)
\curveto(414.48229955,880.13460089)(414.87880071,879.88245039)(415.42120057,879.88245039)
\curveto(415.97046625,879.88245039)(416.36868387,880.13996579)(416.61585343,880.65499659)
\curveto(416.86645589,881.173604)(416.99175713,881.98906944)(416.99175713,883.10139291)
\curveto(416.99175713,884.20298657)(416.86645589,885.0130871)(416.61585343,885.53169451)
\curveto(416.36868387,886.05030192)(415.97046625,886.30960562)(415.42120057,886.30960562)
\curveto(414.87193489,886.30960562)(414.47371727,886.05566682)(414.22654772,885.54778922)
\curveto(413.98281107,885.03991162)(413.86094275,884.22444619)(413.86094275,883.10139291)
\closepath
}
}
{
\newrgbcolor{curcolor}{0 0 0}
\pscustom[linestyle=none,fillstyle=solid,fillcolor=curcolor]
{
\newpath
\moveto(359.17740548,866.11755976)
\lineto(356.17492898,870.54979624)
\lineto(357.7989598,870.54979624)
\lineto(360.04091455,867.13276477)
\lineto(362.25910299,870.54979624)
\lineto(363.86728961,870.54979624)
\lineto(360.86481311,866.11755976)
\lineto(364.03365374,861.48723451)
\lineto(362.40962292,861.48723451)
\lineto(360.04091455,865.10235476)
\lineto(357.64051777,861.48723451)
\lineto(356.03233115,861.48723451)
\lineto(359.17740548,866.11755976)
\closepath
}
}
{
\newrgbcolor{curcolor}{0 0 0}
\pscustom[linestyle=none,fillstyle=solid,fillcolor=curcolor]
{
\newpath
\moveto(326.56441072,836.41555187)
\lineto(323.56193422,840.84778834)
\lineto(325.18596505,840.84778834)
\lineto(327.42791979,837.43075687)
\lineto(329.64610823,840.84778834)
\lineto(331.25429485,840.84778834)
\lineto(328.25181836,836.41555187)
\lineto(331.42065898,831.78522662)
\lineto(329.79662816,831.78522662)
\lineto(327.42791979,835.40034687)
\lineto(325.02752302,831.78522662)
\lineto(323.4193364,831.78522662)
\lineto(326.56441072,836.41555187)
\closepath
}
}
{
\newrgbcolor{curcolor}{0 0 0}
\pscustom[linestyle=none,fillstyle=solid,fillcolor=curcolor]
{
\newpath
\moveto(335.45736586,828.40451147)
\lineto(334.55107749,828.40451147)
\lineto(334.55107749,833.88744356)
\curveto(334.55107749,834.37386154)(334.56480913,834.82451349)(334.59227242,835.23939942)
\curveto(334.49615092,835.13210133)(334.39316361,835.03195646)(334.28331047,834.93896478)
\curveto(333.9331536,834.63495355)(333.55038408,834.30590609)(333.13500191,833.95182241)
\lineto(332.65096153,834.60097582)
\lineto(334.69011036,836.24800141)
\lineto(335.45736586,836.24800141)
\lineto(335.45736586,828.40451147)
\closepath
}
}
{
\newrgbcolor{curcolor}{0 0 0}
\pscustom[linestyle=none,fillstyle=solid,fillcolor=curcolor]
{
\newpath
\moveto(272.61325073,998.19456404)
\lineto(271.53903198,998.19456404)
\lineto(271.53903198,1004.43235701)
\curveto(271.53903198,1004.98574242)(271.55530802,1005.49843774)(271.58786011,1005.97044294)
\curveto(271.47392782,1005.84837263)(271.3518575,1005.73444034)(271.22164917,1005.62864607)
\curveto(270.80661011,1005.28278018)(270.35291545,1004.90843123)(269.86056519,1004.50559919)
\lineto(269.28683472,1005.24412459)
\lineto(271.7038269,1007.11790388)
\lineto(272.61325073,1007.11790388)
\lineto(272.61325073,998.19456404)
\closepath
}
}
{
\newrgbcolor{curcolor}{0 0 0}
\pscustom[linestyle=none,fillstyle=solid,fillcolor=curcolor]
{
\newpath
\moveto(278.5763855,1001.6186363)
\curveto(278.5763855,1000.54848656)(278.41769409,999.55157901)(278.10031128,998.62791365)
\curveto(277.78699748,997.70424828)(277.33330282,996.90061873)(276.73922729,996.21702498)
\lineto(275.71383667,996.21702498)
\curveto(276.27942912,996.98199893)(276.71277873,997.83038761)(277.0138855,998.76219099)
\curveto(277.31499227,999.69806339)(277.46554565,1000.65428084)(277.46554565,1001.63084334)
\curveto(277.46554565,1002.62368188)(277.31702677,1003.59007185)(277.01998901,1004.53001326)
\curveto(276.72702026,1005.46995466)(276.28756714,1006.33258487)(275.70162964,1007.11790388)
\lineto(276.73922729,1007.11790388)
\curveto(277.33737183,1006.40582706)(277.79310099,1005.57778344)(278.10641479,1004.63377302)
\curveto(278.4197286,1003.69383162)(278.5763855,1002.68878604)(278.5763855,1001.6186363)
\closepath
}
}
{
\newrgbcolor{curcolor}{0 0 0}
\pscustom[linestyle=none,fillstyle=solid,fillcolor=curcolor]
{
\newpath
\moveto(510.53420177,1002.58748904)
\lineto(504.63210216,1002.58748904)
\lineto(504.63210216,1003.50301638)
\lineto(506.96974864,1005.86507693)
\curveto(507.67368744,1006.57715375)(508.13958914,1007.08578005)(508.36745372,1007.39095584)
\curveto(508.5953183,1007.69613162)(508.76418223,1007.99113487)(508.87404552,1008.2759656)
\curveto(508.98797781,1008.56079633)(509.04494395,1008.86800662)(509.04494395,1009.19759646)
\curveto(509.04494395,1009.66960167)(508.90456309,1010.03784711)(508.62380137,1010.30233279)
\curveto(508.34303966,1010.56681847)(507.95851817,1010.6990613)(507.47023692,1010.6990613)
\curveto(507.10402598,1010.6990613)(506.75612559,1010.63802615)(506.42653575,1010.51595584)
\curveto(506.10101492,1010.39388552)(505.73276947,1010.17212445)(505.32179942,1009.85067263)
\lineto(504.73586192,1010.58919802)
\curveto(505.57000906,1011.28906781)(506.48146739,1011.63900271)(507.47023692,1011.63900271)
\curveto(508.30031505,1011.63900271)(508.95746023,1011.42537966)(509.44167247,1010.99813357)
\curveto(509.92995372,1010.57495649)(510.17409434,1009.99715701)(510.17409434,1009.26473513)
\curveto(510.17409434,1008.87411013)(510.10288666,1008.49569216)(509.9604713,1008.12948123)
\curveto(509.81805593,1007.7673393)(509.60443289,1007.39705935)(509.31960216,1007.01864138)
\curveto(509.03477143,1006.64429242)(508.5566627,1006.12956261)(507.88527598,1005.47445193)
\lineto(506.02370372,1003.63729373)
\lineto(506.02370372,1003.5884656)
\lineto(510.53420177,1003.5884656)
\lineto(510.53420177,1002.58748904)
\closepath
}
}
{
\newrgbcolor{curcolor}{0 0 0}
\pscustom[linestyle=none,fillstyle=solid,fillcolor=curcolor]
{
\newpath
\moveto(514.43434825,1006.0115613)
\curveto(514.43434825,1004.94141156)(514.27565684,1003.94450401)(513.95827403,1003.02083865)
\curveto(513.64496023,1002.09717328)(513.19126557,1001.29354373)(512.59719005,1000.60994998)
\lineto(511.57179942,1000.60994998)
\curveto(512.13739187,1001.37492393)(512.57074148,1002.22331261)(512.87184825,1003.15511599)
\curveto(513.17295502,1004.09098839)(513.32350841,1005.04720584)(513.32350841,1006.02376834)
\curveto(513.32350841,1007.01660688)(513.17498953,1007.98299685)(512.87795177,1008.92293826)
\curveto(512.58498302,1009.86287966)(512.14552989,1010.72550987)(511.55959239,1011.51082888)
\lineto(512.59719005,1011.51082888)
\curveto(513.19533458,1010.79875206)(513.65106374,1009.97070844)(513.96437755,1009.02669802)
\curveto(514.27769135,1008.08675662)(514.43434825,1007.08171104)(514.43434825,1006.0115613)
\closepath
}
}
\end{pspicture}


        \end{proof}
        \begin{theorem}[св-ва замк. мн-в]
            \begin{enumerate}
                \item $\{F_i\}_{i \in A} - $ замкн. $$\Rightarrow \bigcap_{i \in A} F_i - \text{замк.}$$ 
                \item $F_1, ..., F_n - $ замк. $$\Rightarrow \bigcup_{i = 1}^n F_i - \text{замк.}$$
                \item $\varnothing$ и $X$ замк.
            \end{enumerate}
                $F_i = X \setminus U_i, \quad U_i$ - откр.\\
                $\bigcap F_i = \bigcap (X \setminus U_i) = X \setminus \bigcup U_i$

        \end{theorem}   
    \end{question}

    \begin{question}{3. Внутренность и вшеность множества.}
        \begin{definition} 
            X - м. пространство $A \subset X \quad x_0 \in X$\\
            $x_0$ - назыв. внутр. относительно A (в X), если $\exists \; \mathcal{E} > 0:$\\
            $B(x_0, \; \mathcal{E}) \subset A$
        \end{definition}

        \begin{definition} 
            $x_0$ - назыв. внешней, если $x_0$ - внутр. для $\overline{A} = X \setminus A$\\
            $\exists \; \mathcal{E} > 0 : B(x_0, \mathcal{E}) \; \cap \; A = \varnothing$
        \end{definition}

        \begin{definition} 
            Остальные точки - граничные\\
            $x_0$ - гранич., если $\forall \mathcal{E} > 0 \; B(x_0, \mathcal{E}) \; \cap \; A \neq \varnothing$ и\\
            $B(x_0, \mathcal{E}) \not\subset A$\\
            Int A - внутренность A - мн-во внутр. т.\\
            Ex A - внешность A - мн-во внешних т.\\
            $\partial A = Fr A$ - граница A - мн-во гр.т.
        \end{definition}

        \begin{theorem} 
            След. описания Int эквив.\\
            \begin{enumerate}
                \item Int A - мн-во внутр. т.
                \item Наибольшее (по включению) откр. мн-во, содерж. в A
                \item max (по включению) откр. мн-во, содерж. в A
                \item Int A = $\bigcup U_i, \quad U_i - \text{откр.} \quad U_i \subset A$
                \item Int A = $(X \setminus Ex A) \setminus \partial A$ 
            \end{enumerate}
        \end{theorem}

        \begin{proof} 
            $(2) \Leftrightarrow (4) \Leftrightarrow (3)$ т.к объед. откр. - откр.\\
            $(1) \Leftrightarrow (4):$\\
            $\Rightarrow\\ x_0 \in$ мн-во внутр. т. $\subset \bigcup U_i, \quad U_i \text{- откр.} \quad U_i \subset A$\\
            $\exists \; \mathcal{E} > 0: B(x_0, \mathcal{E}) \text{- откр.} \subset A$ (по определению Int A)\\
            $\Leftarrow\\ \exists \; i: x_0 \in U_i \subset A \quad x_0 \in \bigcup U_i$ \\
            $\exists \; \mathcal{E}: B(x_0, \mathcal{E}) \subset U_i \subset A \Rightarrow x_0$ - внутр. т. A
        \end{proof}

        \begin{theorem}[равносильные определения внешности]
            \begin{enumerate}
                \item Ex A - мн-во внеш. т.
                \item Ex A = Int $(X \setminus A)$
                \item Ex A - max (по вкл.) откр. мн-во, не пересек. с A
                \item Ex A = $\bigcup U_i, \quad U_i - \text{откр.} \quad U_i \cap A = \varnothing$
            \end{enumerate}
        \end{theorem}

        Относительно внутр.\\
        $A \subset X \Rightarrow (A, \rho) - \text{метр. пр-во} \\
        B \subset A \quad Int_A B \neq Int_X B$
        \begin{example} 
            $X = \R, \quad \rho - \text{станд.} \\
            A = [0, 1] \quad B = [0, \frac{1}{2}) \\
            Int_X B = (0, \frac{1}{2}) \quad Int_A B = [0, \frac{1}{2})$
        \end{example}
    \end{question}

    \begin{question}{4. Замыкание множества.}
        \begin{definition} 
            Замыкание A $\quad Cl A = \{x \in X \; | \; \forall \mathcal{E} > 0 \quad B(x, \mathcal{E}) \cap A \neq \varnothing\}$
        \end{definition}

        \begin{theorem} 
            \begin{enumerate}
                \item $Cl A = \{x \in X \; | \; \forall \mathcal{E} > 0 \quad B(x, \mathcal{E}) \cap A \neq \varnothing\}$
                \item $Cl A = Int A \cup \partial A$
                \item $Cl A = \cap F_i, \quad F_i - \text{замк} \quad F_i \supset A$
                \item $Cl A = min \text{(по вкл.) замк.} \supset A$
            \end{enumerate}

            \begin{proof} 
                $(3) \Leftrightarrow (4)$ - пересеч. замк. - замк.\\
                $(1) \Leftrightarrow (2)$ - очев. \\
                $(1) \Rightarrow (3):$
                \[\forall \mathcal{E} > 0 \quad x : B(x, \mathcal{E}) \cap A \neq \varnothing\]
                \[\sqsupset x \not \in F \text{- замк.} \quad F \supset A \quad x \in X \setminus F \text{- откр.}\] 
                \[\exists \; \mathcal{E} > 0: \; B(x, \mathcal{E}) \; \subset \; X \setminus F \; \subset \; X \setminus A\]
                \[\Rightarrow x \text{ - внеш. \quad противореч.}\]
                $(3) \Leftarrow (1):$
                \[x \in \cap F_i\]
                \[\sqsupset \exists \mathcal{E} > 0: B(x, \mathcal{E}) \cap A = \varnothing\] 
                \[B(x, \mathcal{E})$ - откр. (по л.) \quad замк - $ F = X \setminus B(x, \mathcal{E}) \quad F \supset A\]
                \[x \not \in F \text{ - противореч.}\]
            \end{proof}
        \end{theorem}

        \begin{remark} 
            \begin{enumerate}
                \item A - откр. $\Leftrightarrow  A = Int A$
                \item A - замк. $\Leftrightarrow  A = Cl A$
                \item $Int A \subset A \subset Cl A$\\
                      $\partial A = Cl A \setminus Int A$
            \end{enumerate}
        \end{remark}

        \begin{example} 
            $X = \R; \quad A = \varnothing \\
            Int A = \varnothing \quad Ex A = \varnothing \quad \partial A = \R$
        \end{example}

        \begin{example} 
            Кантор. мн-во - замк. \\
            %LaTeX with PSTricks extensions
%%Creator: inkscape 0.91
%%Please note this file requires PSTricks extensions
\psset{xunit=.5pt,yunit=.5pt,runit=.5pt}
\begin{pspicture}(744.09448819,1052.36220472)
{
\newrgbcolor{curcolor}{0 0 0}
\pscustom[linewidth=1,linecolor=curcolor]
{
\newpath
\moveto(60.866413,858.77905472)
\lineto(643.44494,858.77905472)
}
}
{
\newrgbcolor{curcolor}{0 0 0}
\pscustom[linestyle=none,fillstyle=solid,fillcolor=curcolor]
{
\newpath
\moveto(163.00162162,836.45430282)
\lineto(149.554356,836.45430282)
\lineto(149.554356,888.77852157)
\lineto(163.00162162,888.77852157)
\lineto(163.00162162,884.47188095)
\lineto(154.710606,884.47188095)
\lineto(154.710606,840.79024032)
\lineto(163.00162162,840.79024032)
\lineto(163.00162162,836.45430282)
\closepath
}
}
{
\newrgbcolor{curcolor}{0 0 0}
\pscustom[linestyle=none,fillstyle=solid,fillcolor=curcolor]
{
\newpath
\moveto(181.63443412,840.79024032)
\lineto(189.92544975,840.79024032)
\lineto(189.92544975,884.47188095)
\lineto(181.63443412,884.47188095)
\lineto(181.63443412,888.77852157)
\lineto(195.08169975,888.77852157)
\lineto(195.08169975,836.45430282)
\lineto(181.63443412,836.45430282)
\lineto(181.63443412,840.79024032)
\closepath
}
}
{
\newrgbcolor{curcolor}{0 0 0}
\pscustom[linestyle=none,fillstyle=solid,fillcolor=curcolor]
{
\newpath
\moveto(202.34732475,862.3820372)
\curveto(202.34732475,867.55781845)(203.09927787,872.40156845)(204.60318412,876.9132872)
\curveto(206.12662162,881.42500595)(208.31412162,885.38008407)(211.16568412,888.77852157)
\lineto(216.14615287,888.77852157)
\curveto(213.33365287,885.00899032)(211.21451225,880.86836532)(209.788731,876.35664657)
\curveto(208.382481,871.84492782)(207.679356,867.20625595)(207.679356,862.44063095)
\curveto(207.679356,857.75313095)(208.40201225,853.17305282)(209.84732475,848.70039657)
\curveto(211.29263725,844.22774032)(213.37271537,840.14570907)(216.08755912,836.45430282)
\lineto(211.16568412,836.45430282)
\curveto(208.29459037,839.77461532)(206.10709037,843.65156845)(204.60318412,848.0851622)
\curveto(203.09927787,852.51875595)(202.34732475,857.28438095)(202.34732475,862.3820372)
\closepath
}
}
{
\newrgbcolor{curcolor}{0 0 0}
\pscustom[linestyle=none,fillstyle=solid,fillcolor=curcolor]
{
\newpath
\moveto(249.163731,862.3820372)
\curveto(249.163731,857.24531845)(248.40201225,852.4601622)(246.87857475,848.02656845)
\curveto(245.3746685,843.5929747)(243.19693412,839.73555282)(240.34537162,836.45430282)
\lineto(235.42349662,836.45430282)
\curveto(238.13834037,840.12617782)(240.2184185,844.19844345)(241.663731,848.6710997)
\curveto(243.1090435,853.1632872)(243.83169975,857.75313095)(243.83169975,862.44063095)
\curveto(243.83169975,867.20625595)(243.11880912,871.84492782)(241.69302787,876.35664657)
\curveto(240.28677787,880.86836532)(238.17740287,885.00899032)(235.36490287,888.77852157)
\lineto(240.34537162,888.77852157)
\curveto(243.21646537,885.36055282)(245.40396537,881.38594345)(246.90787162,876.85469345)
\curveto(248.41177787,872.3429747)(249.163731,867.51875595)(249.163731,862.3820372)
\closepath
}
}
{
\newrgbcolor{curcolor}{0 0 0}
\pscustom[linestyle=none,fillstyle=solid,fillcolor=curcolor]
{
\newpath
\moveto(269.87662162,836.45430282)
\lineto(256.429356,836.45430282)
\lineto(256.429356,888.77852157)
\lineto(269.87662162,888.77852157)
\lineto(269.87662162,884.47188095)
\lineto(261.585606,884.47188095)
\lineto(261.585606,840.79024032)
\lineto(269.87662162,840.79024032)
\lineto(269.87662162,836.45430282)
\closepath
}
}
{
\newrgbcolor{curcolor}{0 0 0}
\pscustom[linestyle=none,fillstyle=solid,fillcolor=curcolor]
{
\newpath
\moveto(288.50943412,840.79024032)
\lineto(296.80044975,840.79024032)
\lineto(296.80044975,884.47188095)
\lineto(288.50943412,884.47188095)
\lineto(288.50943412,888.77852157)
\lineto(301.95669975,888.77852157)
\lineto(301.95669975,836.45430282)
\lineto(288.50943412,836.45430282)
\lineto(288.50943412,840.79024032)
\closepath
}
}
{
\newrgbcolor{curcolor}{0 0 0}
\pscustom[linestyle=none,fillstyle=solid,fillcolor=curcolor]
{
\newpath
\moveto(309.22232475,862.3820372)
\curveto(309.22232475,867.55781845)(309.97427787,872.40156845)(311.47818412,876.9132872)
\curveto(313.00162162,881.42500595)(315.18912162,885.38008407)(318.04068412,888.77852157)
\lineto(323.02115287,888.77852157)
\curveto(320.20865287,885.00899032)(318.08951225,880.86836532)(316.663731,876.35664657)
\curveto(315.257481,871.84492782)(314.554356,867.20625595)(314.554356,862.44063095)
\curveto(314.554356,857.75313095)(315.27701225,853.17305282)(316.72232475,848.70039657)
\curveto(318.16763725,844.22774032)(320.24771537,840.14570907)(322.96255912,836.45430282)
\lineto(318.04068412,836.45430282)
\curveto(315.16959037,839.77461532)(312.98209037,843.65156845)(311.47818412,848.0851622)
\curveto(309.97427787,852.51875595)(309.22232475,857.28438095)(309.22232475,862.3820372)
\closepath
}
}
{
\newrgbcolor{curcolor}{0 0 0}
\pscustom[linestyle=none,fillstyle=solid,fillcolor=curcolor]
{
\newpath
\moveto(402.7965435,862.3820372)
\curveto(402.7965435,857.24531845)(402.03482475,852.4601622)(400.51138725,848.02656845)
\curveto(399.007481,843.5929747)(396.82974662,839.73555282)(393.97818412,836.45430282)
\lineto(389.05630912,836.45430282)
\curveto(391.77115287,840.12617782)(393.851231,844.19844345)(395.2965435,848.6710997)
\curveto(396.741856,853.1632872)(397.46451225,857.75313095)(397.46451225,862.44063095)
\curveto(397.46451225,867.20625595)(396.75162162,871.84492782)(395.32584037,876.35664657)
\curveto(393.91959037,880.86836532)(391.81021537,885.00899032)(388.99771537,888.77852157)
\lineto(393.97818412,888.77852157)
\curveto(396.84927787,885.36055282)(399.03677787,881.38594345)(400.54068412,876.85469345)
\curveto(402.04459037,872.3429747)(402.7965435,867.51875595)(402.7965435,862.3820372)
\closepath
}
}
{
\newrgbcolor{curcolor}{0 0 0}
\pscustom[linestyle=none,fillstyle=solid,fillcolor=curcolor]
{
\newpath
\moveto(423.50943412,836.45430282)
\lineto(410.0621685,836.45430282)
\lineto(410.0621685,888.77852157)
\lineto(423.50943412,888.77852157)
\lineto(423.50943412,884.47188095)
\lineto(415.2184185,884.47188095)
\lineto(415.2184185,840.79024032)
\lineto(423.50943412,840.79024032)
\lineto(423.50943412,836.45430282)
\closepath
}
}
{
\newrgbcolor{curcolor}{0 0 0}
\pscustom[linestyle=none,fillstyle=solid,fillcolor=curcolor]
{
\newpath
\moveto(442.14224662,840.79024032)
\lineto(450.43326225,840.79024032)
\lineto(450.43326225,884.47188095)
\lineto(442.14224662,884.47188095)
\lineto(442.14224662,888.77852157)
\lineto(455.58951225,888.77852157)
\lineto(455.58951225,836.45430282)
\lineto(442.14224662,836.45430282)
\lineto(442.14224662,840.79024032)
\closepath
}
}
{
\newrgbcolor{curcolor}{0 0 0}
\pscustom[linestyle=none,fillstyle=solid,fillcolor=curcolor]
{
\newpath
\moveto(462.85513725,862.3820372)
\curveto(462.85513725,867.55781845)(463.60709037,872.40156845)(465.11099662,876.9132872)
\curveto(466.63443412,881.42500595)(468.82193412,885.38008407)(471.67349662,888.77852157)
\lineto(476.65396537,888.77852157)
\curveto(473.84146537,885.00899032)(471.72232475,880.86836532)(470.2965435,876.35664657)
\curveto(468.8902935,871.84492782)(468.1871685,867.20625595)(468.1871685,862.44063095)
\curveto(468.1871685,857.75313095)(468.90982475,853.17305282)(470.35513725,848.70039657)
\curveto(471.80044975,844.22774032)(473.88052787,840.14570907)(476.59537162,836.45430282)
\lineto(471.67349662,836.45430282)
\curveto(468.80240287,839.77461532)(466.61490287,843.65156845)(465.11099662,848.0851622)
\curveto(463.60709037,852.51875595)(462.85513725,857.28438095)(462.85513725,862.3820372)
\closepath
}
}
{
\newrgbcolor{curcolor}{0 0 0}
\pscustom[linestyle=none,fillstyle=solid,fillcolor=curcolor]
{
\newpath
\moveto(509.6715435,862.3820372)
\curveto(509.6715435,857.24531845)(508.90982475,852.4601622)(507.38638725,848.02656845)
\curveto(505.882481,843.5929747)(503.70474662,839.73555282)(500.85318412,836.45430282)
\lineto(495.93130912,836.45430282)
\curveto(498.64615287,840.12617782)(500.726231,844.19844345)(502.1715435,848.6710997)
\curveto(503.616856,853.1632872)(504.33951225,857.75313095)(504.33951225,862.44063095)
\curveto(504.33951225,867.20625595)(503.62662162,871.84492782)(502.20084037,876.35664657)
\curveto(500.79459037,880.86836532)(498.68521537,885.00899032)(495.87271537,888.77852157)
\lineto(500.85318412,888.77852157)
\curveto(503.72427787,885.36055282)(505.91177787,881.38594345)(507.41568412,876.85469345)
\curveto(508.91959037,872.3429747)(509.6715435,867.51875595)(509.6715435,862.3820372)
\closepath
}
}
{
\newrgbcolor{curcolor}{0 0 0}
\pscustom[linestyle=none,fillstyle=solid,fillcolor=curcolor]
{
\newpath
\moveto(530.38443412,836.45430282)
\lineto(516.9371685,836.45430282)
\lineto(516.9371685,888.77852157)
\lineto(530.38443412,888.77852157)
\lineto(530.38443412,884.47188095)
\lineto(522.0934185,884.47188095)
\lineto(522.0934185,840.79024032)
\lineto(530.38443412,840.79024032)
\lineto(530.38443412,836.45430282)
\closepath
}
}
{
\newrgbcolor{curcolor}{0 0 0}
\pscustom[linestyle=none,fillstyle=solid,fillcolor=curcolor]
{
\newpath
\moveto(549.01724662,840.79024032)
\lineto(557.30826225,840.79024032)
\lineto(557.30826225,884.47188095)
\lineto(549.01724662,884.47188095)
\lineto(549.01724662,888.77852157)
\lineto(562.46451225,888.77852157)
\lineto(562.46451225,836.45430282)
\lineto(549.01724662,836.45430282)
\lineto(549.01724662,840.79024032)
\closepath
}
}
{
\newrgbcolor{curcolor}{0 0 0}
\pscustom[linestyle=none,fillstyle=solid,fillcolor=curcolor]
{
\newpath
\moveto(152.76537244,807.86076371)
\curveto(152.76537244,804.14006058)(152.17455212,801.3715059)(150.9929115,799.55509965)
\curveto(149.8210365,797.7386934)(148.03392712,796.83049027)(145.63158337,796.83049027)
\curveto(143.30736462,796.83049027)(141.54466931,797.76310746)(140.34349744,799.62834183)
\curveto(139.14232556,801.50334183)(138.54173962,804.24748246)(138.54173962,807.86076371)
\curveto(138.54173962,811.59123246)(139.1179115,814.35978715)(140.27025525,816.16642777)
\curveto(141.43236462,817.9730684)(143.219474,818.87638871)(145.63158337,818.87638871)
\curveto(147.95580212,818.87638871)(149.72338025,817.93888871)(150.93431775,816.06388871)
\curveto(152.15502087,814.18888871)(152.76537244,811.45451371)(152.76537244,807.86076371)
\closepath
\moveto(141.19310681,807.86076371)
\curveto(141.19310681,804.75529496)(141.54466931,802.51408402)(142.24779431,801.1371309)
\curveto(142.96068494,799.76017777)(144.08861462,799.07170121)(145.63158337,799.07170121)
\curveto(147.19408337,799.07170121)(148.32689587,799.77482621)(149.03002087,801.18107621)
\curveto(149.7429115,802.59709183)(150.09935681,804.82365433)(150.09935681,807.86076371)
\curveto(150.09935681,810.86857621)(149.7429115,813.08049027)(149.03002087,814.4965059)
\curveto(148.32689587,815.91252152)(147.19408337,816.62052933)(145.63158337,816.62052933)
\curveto(144.06908337,816.62052933)(142.93627087,815.92716996)(142.23314587,814.54045121)
\curveto(141.5397865,813.15373246)(141.19310681,810.92716996)(141.19310681,807.86076371)
\closepath
}
}
{
\newrgbcolor{curcolor}{0 0 0}
\pscustom[linestyle=none,fillstyle=solid,fillcolor=curcolor]
{
\newpath
\moveto(567.26087594,802.01565602)
\lineto(564.68275094,802.01565602)
\lineto(564.68275094,816.98635915)
\curveto(564.68275094,818.31448415)(564.72181344,819.5449529)(564.79993844,820.6777654)
\curveto(564.52650094,820.38479665)(564.23353219,820.11135915)(563.92103219,819.8574529)
\curveto(562.92493844,819.02737477)(561.83607125,818.12893727)(560.65443063,817.1621404)
\lineto(559.2774775,818.93460133)
\lineto(565.07825875,823.43167165)
\lineto(567.26087594,823.43167165)
\lineto(567.26087594,802.01565602)
\closepath
}
}
{
\newrgbcolor{curcolor}{0 0 0}
\pscustom[linestyle=none,fillstyle=solid,fillcolor=curcolor]
{
\newpath
\moveto(317.53044594,801.52614702)
\lineto(314.95232094,801.52614702)
\lineto(314.95232094,816.49685015)
\curveto(314.95232094,817.82497515)(314.99138344,819.0554439)(315.06950844,820.1882564)
\curveto(314.79607094,819.89528765)(314.50310219,819.62185015)(314.19060219,819.3679439)
\curveto(313.19450844,818.53786577)(312.10564125,817.63942827)(310.92400062,816.6726314)
\lineto(309.5470475,818.44509233)
\lineto(315.34782875,822.94216265)
\lineto(317.53044594,822.94216265)
\lineto(317.53044594,801.52614702)
\closepath
}
}
{
\newrgbcolor{curcolor}{0 0 0}
\pscustom[linestyle=none,fillstyle=solid,fillcolor=curcolor]
{
\newpath
\moveto(401.24992781,801.52614602)
\lineto(387.08488875,801.52614602)
\lineto(387.08488875,803.72341165)
\lineto(392.69524031,809.39235696)
\curveto(394.38469344,811.10134133)(395.5028575,812.32204446)(396.0497325,813.05446633)
\curveto(396.5966075,813.78688821)(397.00188094,814.49489602)(397.26555281,815.17848977)
\curveto(397.53899031,815.86208352)(397.67570906,816.59938821)(397.67570906,817.39040383)
\curveto(397.67570906,818.52321633)(397.338795,819.4070054)(396.66496688,820.04177102)
\curveto(395.99113875,820.67653665)(395.06828719,820.99391946)(393.89641219,820.99391946)
\curveto(393.01750594,820.99391946)(392.182545,820.84743508)(391.39152938,820.55446633)
\curveto(390.61027938,820.26149758)(389.72649031,819.72927102)(388.74016219,818.95778665)
\lineto(387.33391219,820.73024758)
\curveto(389.33586531,822.40993508)(391.52336531,823.24977883)(393.89641219,823.24977883)
\curveto(395.88859969,823.24977883)(397.46574813,822.73708352)(398.6278575,821.7116929)
\curveto(399.7997325,820.6960679)(400.38567,819.30934915)(400.38567,817.55153665)
\curveto(400.38567,816.61403665)(400.21477156,815.70583352)(399.87297469,814.82692727)
\curveto(399.53117781,813.95778665)(399.0184825,813.06911477)(398.33488875,812.16091165)
\curveto(397.651295,811.26247415)(396.50383406,810.02712258)(394.89250594,808.45485696)
\lineto(390.4247325,804.04567727)
\lineto(390.4247325,803.92848977)
\lineto(401.24992781,803.92848977)
\lineto(401.24992781,801.52614602)
\closepath
}
}
{
\newrgbcolor{curcolor}{0 0 0}
\pscustom[linestyle=none,fillstyle=solid,fillcolor=curcolor]
{
\newpath
\moveto(321.67595375,785.19751902)
\curveto(321.67595375,783.83033152)(321.29021156,782.70728465)(320.51872719,781.8283784)
\curveto(319.75700844,780.94947215)(318.66325844,780.36353465)(317.23747719,780.0705659)
\lineto(317.23747719,779.9533784)
\curveto(318.94646156,779.73853465)(320.22087562,779.19165965)(321.06071937,778.3127534)
\curveto(321.91032875,777.43384715)(322.33513344,776.29126902)(322.33513344,774.88501902)
\curveto(322.33513344,772.8440034)(321.62712562,771.26685496)(320.21111,770.15357371)
\curveto(318.79509437,769.05005808)(316.76384437,768.49830027)(314.11736,768.49830027)
\curveto(312.9845475,768.49830027)(311.93962562,768.58130808)(310.98259437,768.74732371)
\curveto(310.03532875,768.91333933)(309.11736,769.21607371)(308.22868812,769.65552683)
\lineto(308.22868812,772.11646433)
\curveto(309.13689125,771.66724558)(310.10857094,771.3205659)(311.14372719,771.07642527)
\curveto(312.17888344,770.83228465)(313.17986,770.71021433)(314.14665687,770.71021433)
\curveto(317.75993812,770.71021433)(319.56657875,772.12134715)(319.56657875,774.94361277)
\curveto(319.56657875,777.46314402)(317.56950844,778.72290965)(313.57536781,778.72290965)
\lineto(311.49528969,778.72290965)
\lineto(311.49528969,780.96412058)
\lineto(313.60466469,780.96412058)
\curveto(315.28435219,780.96412058)(316.59782875,781.3205659)(317.54509437,782.03345652)
\curveto(318.49236,782.75611277)(318.96599281,783.75220652)(318.96599281,785.02173777)
\curveto(318.96599281,786.0471284)(318.61931312,786.84790965)(317.92595375,787.42408152)
\curveto(317.24236,788.0002534)(316.29021156,788.28833933)(315.06950844,788.28833933)
\curveto(314.12224281,788.28833933)(313.24333656,788.1565034)(312.43278969,787.89283152)
\curveto(311.63200844,787.63892527)(310.70915687,787.17017527)(309.664235,786.48658152)
\lineto(308.34587562,788.28833933)
\curveto(309.22478187,788.98169871)(310.23552406,789.5236909)(311.37810219,789.9143159)
\curveto(312.53044594,790.31470652)(313.75114906,790.51490183)(315.04021156,790.51490183)
\curveto(317.13005531,790.51490183)(318.75603187,790.03638621)(319.91814125,789.07935496)
\curveto(321.09001625,788.13208933)(321.67595375,786.83814402)(321.67595375,785.19751902)
\closepath
}
}
{
\newrgbcolor{curcolor}{0 0 0}
\pscustom[linestyle=none,fillstyle=solid,fillcolor=curcolor]
{
\newpath
\moveto(401.46722375,785.19751802)
\curveto(401.46722375,783.83033052)(401.08148156,782.70728365)(400.30999719,781.8283774)
\curveto(399.54827844,780.94947115)(398.45452844,780.36353365)(397.02874719,780.0705649)
\lineto(397.02874719,779.9533774)
\curveto(398.73773156,779.73853365)(400.01214563,779.19165865)(400.85198938,778.3127524)
\curveto(401.70159875,777.43384615)(402.12640344,776.29126802)(402.12640344,774.88501802)
\curveto(402.12640344,772.8440024)(401.41839563,771.26685396)(400.00238,770.15357271)
\curveto(398.58636438,769.05005708)(396.55511438,768.49829927)(393.90863,768.49829927)
\curveto(392.7758175,768.49829927)(391.73089563,768.58130708)(390.77386438,768.74732271)
\curveto(389.82659875,768.91333833)(388.90863,769.21607271)(388.01995813,769.65552583)
\lineto(388.01995813,772.11646333)
\curveto(388.92816125,771.66724458)(389.89984094,771.3205649)(390.93499719,771.07642427)
\curveto(391.97015344,770.83228365)(392.97113,770.71021333)(393.93792688,770.71021333)
\curveto(397.55120813,770.71021333)(399.35784875,772.12134615)(399.35784875,774.94361177)
\curveto(399.35784875,777.46314302)(397.36077844,778.72290865)(393.36663781,778.72290865)
\lineto(391.28655969,778.72290865)
\lineto(391.28655969,780.96411958)
\lineto(393.39593469,780.96411958)
\curveto(395.07562219,780.96411958)(396.38909875,781.3205649)(397.33636438,782.03345552)
\curveto(398.28363,782.75611177)(398.75726281,783.75220552)(398.75726281,785.02173677)
\curveto(398.75726281,786.0471274)(398.41058313,786.84790865)(397.71722375,787.42408052)
\curveto(397.03363,788.0002524)(396.08148156,788.28833833)(394.86077844,788.28833833)
\curveto(393.91351281,788.28833833)(393.03460656,788.1565024)(392.22405969,787.89283052)
\curveto(391.42327844,787.63892427)(390.50042688,787.17017427)(389.455505,786.48658052)
\lineto(388.13714563,788.28833833)
\curveto(389.01605188,788.98169771)(390.02679406,789.5236899)(391.16937219,789.9143149)
\curveto(392.32171594,790.31470552)(393.54241906,790.51490083)(394.83148156,790.51490083)
\curveto(396.92132531,790.51490083)(398.54730188,790.03638521)(399.70941125,789.07935396)
\curveto(400.88128625,788.13208833)(401.46722375,786.83814302)(401.46722375,785.19751802)
\closepath
}
}
{
\newrgbcolor{curcolor}{0 0 0}
\pscustom[linestyle=none,fillstyle=solid,fillcolor=curcolor]
{
\newpath
\moveto(321.98167997,795.52869906)
\lineto(306.31029325,795.52869906)
\lineto(306.31029325,797.76746859)
\lineto(321.98167997,797.76746859)
\lineto(321.98167997,795.52869906)
\closepath
}
}
{
\newrgbcolor{curcolor}{0 0 0}
\pscustom[linestyle=none,fillstyle=solid,fillcolor=curcolor]
{
\newpath
\moveto(403.21164297,795.25881306)
\lineto(387.54025625,795.25881306)
\lineto(387.54025625,797.49758259)
\lineto(403.21164297,797.49758259)
\lineto(403.21164297,795.25881306)
\closepath
}
}
\end{pspicture}

        \end{example}
    \end{question}

    \begin{question}{5. Топологические пространства. Примеры.}
        \begin{definition} 
            X - мн-во\\
            $\Omega \subset 2^X = \{A \subset X\}$ - мн-во подмн. X\\
            $(X, \Omega)$ - назыв. тополог. пр-вом, если\\
            \begin{enumerate}
                \item \[\forall \{U_i\}_{i \in I} \in \Omega \Rightarrow\bigcup_{i \in I} U_i \in \Omega\]        
                \item $U_1, U_2, ..., U_n \Rightarrow U_1 \cap U_2 \cap ... \cap U_n \in \Omega$
                \item $\varnothing; \; X \in \Omega$\\\\
                $\Omega$ - тополог. на X\\
                $U \in \Omega$ - назыв. открытым мн-вом
            \end{enumerate}
        \end{definition}

        \begin{definition} 
            $(X, \Omega)$ - топ. пр-во; $F \subset X$ \\
            F - назыв. замк., если $X \setminus F \in \Omega$
        \end{definition}

        \begin{theorem} 
            \begin{enumerate}
                \item \[\bigcap_{i \in I} F_i \text{- замк, если } F_i - \text{замк}\]
                \item $F_1 \cup F_2$ - замк ($F_1, F_2$ - замк.)
                \item $\varnothing, X$ - замк.
            \end{enumerate}
        \end{theorem}

        \begin{examples} 
            \begin{enumerate}
                \item $(X, \rho)$ - топ. пр-во
                \item дискр. пр-во: $\Omega = 2^X$
                \item антидискр. пр-во: $\Omega = \{\varnothing, X\}$
        
            \begin{definition} 
                $(X, \Omega)$ - метризуемо, если $\exists$ метрика $\rho: X \times X \rightarrow \R_X$\\
                $\Omega = $ мн-во откр. подмн. в $\rho$\\
                Антидискр. - не метризуемо, если |X| > 1
            \end{definition}
                \item Стрелка\\
                      $X = \R  \;$ или $\;  \R_+ = \{x \geq 0\}$\\
                      $\Omega = \{(a, +\infty)\} \cup \{\varnothing\} \cup \{X\}$
                \item Связное двоеточие\\
                      $X = \{a, b\}$\\
                      $\Omega = \{\varnothing, X, \{a\}\}$
                \item Топология конечных дополнений (Зариского)\\
                      X - беск. мн-во\\
                      Замкнутые конечные мн-ва и X \\
                      $\Omega = \{A \; | \; X \setminus A \text{ конечно}\}$
            \end{enumerate}
        \end{examples}
    \end{question}

    \begin{question}{6. База топологии. Критерий базы.}
        \begin{definition} 
            X - топ. пр-во; $\quad A \subset X$ \\
            $Int A = \cup U, \quad U \in \Omega \quad U \subset A$\\
            $Cl A = \cap F, \quad F - $ замк. $F \supset A$ \\
            $\partial A = Cl A \setminus Int A$
        \end{definition}

        \begin{definition} 
            $x_0 \in X$\\
            окр. $x_0$ назыв. $\forall U \in \Omega: x_0 \in U$
        \end{definition}

        \begin{definition} 
            $x_0$ назыв. внутр. т. A, если $\exists U_{x_0} \subset A$\\
            $x_0$ назыв. внеш. т. A, если $\exists U_{x_0} \cap A = \varnothing$\\
            $x_0$ назыв. граничной, если $\forall U_{x_0} \quad (U_{x_0} \not \subset A)$ и $(U_{x_0} \cap A \neq \varnothing)$
        \end{definition}

        \begin{definition} 
            $(X, \Omega)$ - топ. пр-во\\
            $\mathcal{B} \subset \Omega \quad \mathcal{B}$ назыв. базой топологии, если\\
            \[\forall U \in \Omega \quad \exists \{V_i\} \in \mathcal{B}: \quad U = \bigcup_{i \in I} V_i\] 
        \end{definition}

        \begin{example} 
            $X = \R^n$ или другое метр. пр-во\\
            $\B = \{B(x_0, \mathcal{E}) \; | \; x_0 \in X, \mathcal{E} > 0\}$ - база топологии\\
            $\forall U$ - откр. $\quad \forall x_0 \in U \quad \exists \mathcal{E}: B(x_0, \mathcal{E}) \subset U$\\
            \[\bigcup_{x_0 \in U} B(x_0, \mathcal{E}) = U\]
        \end{example}

        \begin{theorem}[Критерий базы] 
            X - мн-во $\B$ - нек. совокупность подмн-в X\\
            $\B$ - база $\Omega \Leftrightarrow$ \begin{enumerate}
                \item \[\bigcup_{U_i \in \B} U_i = X\]
                \item $\forall U, V \in \B \quad \forall x \in U \cap V \quad \exists W \in \B : x \in W; W \subset U\cap V$
            \end{enumerate}
        \end{theorem}

        \begin{proof} 
            $\ra$ очев\\
            \[\la \Omega = \{\bigcup_{i \in I} U_i | \quad U_i \in \B\}\]\\
            \begin{enumerate}
                \item \[\bigcup_{j \in J}(\bigcup_{i \in I_j}) = \bigcup_{i, j} U_i \]
                \item \[(\bigcup_j U_j) \cap (\bigcup_i U_i)  =  \bigcup_{i, j} (U_i \cap U_j) = 
                \bigcup_{i, j} (\bigcup_{x \in U_i \cap U_j} W_x)\]\\
                \[x \in W_x \subset U_i \cap U_j\]
                \[\bigcup_{x \in U_i \cap U_j} W_x = U_i \cap U_j \q W_x \in \B\]
                \item \[\varnothing = \bigcup_{i \in \varnothing} U_i \q X = \bigcup_{U_i \in \B} U_i\]
            \end{enumerate}
        \end{proof}

        \begin{theorem} [База окр. точки]
                X - мн-во $\forall x \in X \q \exists \B_x \subset 2^x$
                \begin{enumerate}
                    \item $x \in U \q \forall U \in \B_x$
                    \item $U, V \in \B_x \ra \exists W \in \B_x: \q W \subset U \cap V$
                    \item $y \in U \q(U \in \B_x) \ra \exists V \in \B_y: \q V \subset U$
                    \setItemnumber{0}
                    \item \[\B_x \neq \varnothing \ra \bigcup_{x \in X} \B_x - \text{база нек. топологии}\] 
                \end{enumerate}
        \end{theorem}
    \end{question}

    \begin{question}{7. Топология произведения пространств.}
        \begin{example} [- конструкция]
            $X, Y$ - топ. пр-ва\\
            $(X, \Omega_X); \q (Y, \Omega_Y)$ \\
            Введем базу топ. на $X \times Y$\\
            $\B = \{U \times V | \q U \in \Omega_X; \q V \in \Omega_Y\}$\\
            \[\Omega_{X \times Y} = \{\bigcup_{i \in I} U_i \times V_i | \q U_i \in \Omega_X; \q V_i \in \Omega_Y\}\]
            \[(\bigcup_{i \in I} U_i \times V_i) \cap (\bigcup_{j \in J} S_j \times T_j) = 
            \bigcup_{i \in I \; j  \in J} ((U_i \cap S_j) \times (V_i \cap T_j)\]
            $(U_i \cap S_j) \in \Omega_X \q (V_i \cap T_j) \in \Omega_Y$
        \end{example}
    \end{question}

    \begin{question}{8. Равносильные определения непрерывности.}
        \begin{definition} 
            $(X, \rho); \q (Y, d)$ - метр. пр-ва $\q f: X \rightarrow Y$\\
            f - назыв. непр. в т. $x_0$, если\\
            $\forall \mathcal{E} > 0 \q \exists \; \delta > 0 :$\\
            Если $\rho(x, x_0) < \delta \ra d(f(x), f(x_0)) < \mathcal{E}$\\
            f - непр, если она непр. в каждой точке
        \end{definition}

        \begin{theorem} 
            f - непр в $x_0 \rla \forall U - \text{откр.} \subset Y: U \ni f(x_0)$\\
            $\exists V - \text{откр.} \subset X \q x_0 \in V$ и $f(V) \subset U$
        \end{theorem}

        \begin{proof} 
            f - непр. в $x_0 \rla \forall \mathcal{E} > 0 \q \exists \delta > 0$\\
            $f(B(x_0, \delta)) \subset B(f(x_0), \mathcal{E})$\\
            $\ra \forall U -$ откр. $\subset Y: \q f(x_0) \in U \ra \exists \mathcal{E} > 0:$\\
            $f(x_0) \in B(f(x_0), \mathcal{E}) \subset U \ra \exists \delta > 0$ \\
            $f(B(x_0, \delta)) \subset B(f(x_0), \mathcal{E}) \subset U \q B(x_0, \delta) = V$\\
            $\la \forall$ обрывается
        \end{proof}
    \end{question}

    \begin{question}{9. Прообраз топологии. Индуцированная топология.}
        \begin{definition} 
            $f: X \ra Y$ - отобр. мн-в\\
            $(Y, \Omega_Y)$ - топ. пр-во\\
            $\Omega_X$ - самая слабая топ.\\
            f - непр.\\
            $\forall U \in \Omega_Y \q f^{-1}(U)$ должен быть открытым в X
        \end{definition}
        \begin{theorem} 
            $\{f^{-1}(U)\}$ - топология на X и она назыв. прообразом $\Omega_Y$
        \end{theorem}

        \begin{proof} 
            \begin{enumerate}
                \item \[f^{-1}(\bigcup_{i \in I} U_i) = \bigcup_{i \in I} f^{-1}(U_i) \q (*)\]
                \item $f^{-1}(U_1 \cap U_2) = f^{-1}(U_1) \cap f^{-1}(U_2)$
                \item $f^{-1}(\varnothing) = \varnothing \q\q f^{-1}(Y) = X$
            \end{enumerate}
            \[(*): \q f^{-1}(\bigcup_{i \in I} U_i) = \{x | \ f(x) \in \bigcup_{i \in I} U_i\} = 
            \{x | \ \exists \; i \in I: f(x) \in U_i\}\]
        \end{proof}

        \begin{definition} 
            $(X, \Omega_X)$ - топ. пр-во\\
            $A \subset X$\\
            $\Omega_A = \{U \cap A | \ U \in \Omega_X\}$ - индуцированная топология на A
        \end{definition}
    \end{question}

    \begin{question}{10. Инициальная топология. Топология произведения как инициальная.}
        \begin{definition} 
            $\forall i \in I \q f_i: X \rightarrow Y_i$\\
            $(Y_i, \Omega_i)$ - топ. пр-во\\
            $\{f_{i1}^{-1}(U_1) \cap f_{i2}^{-1}(U_2) \cap ... \cap f_{ik}^{-1}(U_k) \  | \ U_j \in \Omega_{i j}; \q j = 1, ..., k; \q k \in \N \}$ 
            - база нек. топологии\\
            $\Omega_X$ - соотв. топология (инициальная топология)
        \end{definition}

        \begin{definition} 
            $\{f_i^{-1}(U)\}$ - предбаза топологии
        \end{definition}

        \begin{theorem} 
            Топология произведения совпадает с инициальной
        \end{theorem}

        \begin{definition} 
            \[\prod_{i \in I} x_i = \{f: I \rightarrow \bigcup_{i \in I} x_i \ | \ f(i) \in X_i \}\]\\
            \[p_k : \prod_{i \in I} x_i \rightarrow X_k \q k \in I\]\\
            \[p_k(f) = f(k) \ra  \text{если } x_i \text{- топ.} \ra \prod_{i \in I} x_i - \text{топ.}\]
        \end{definition}
    \end{question}

    \begin{question}{11. Финальная топология. Фактортопология. Приклеивание.}
        \begin{definition} 
            $\forall i \in I \q f_i: \ X_i \rightarrow Y$ - отобр.\\
            $(X_i, \Omega_i)$\\
            Хотим завести на Y топологию:\\
            $\forall f_i$ - непр. Топ на Y самая сильная \\
            $U \subset Y \q \forall i \in I \q f_i^{-1}(U) \in \Omega_i$\\
            $\Omega_Y = \{U \ | \ \forall i \ f_i^{-1}(U) \in \Omega_i\}$\\
            $\varnothing, Y \in \Omega_Y$\\
            $f_i^{-1}(U_1 \cap U_2) = f_i^{-1}(U_1) \cap f_i^{-1}(U2)$\\
            \[f_i^{-1}(\bigcup_{k \in K} U_k) = \bigcup_{k \in K} f_i^{-1}(U_k)\]
        \end{definition}

        \begin{example} 
            Приклеивание\\
            $X, Y$ - пр-ва\\
            $A \subset X \q f: A \rightarrow Y$ - отобр.\\
            Хотим получить $X \cup_f Y$ - приклеивание\\
            $X \cup_f Y = X \cup Y /\sim \q \forall a \ a \sim f(a)$\\
            %LaTeX with PSTricks extensions
%%Creator: inkscape 0.91
%%Please note this file requires PSTricks extensions
\psset{xunit=.5pt,yunit=.5pt,runit=.5pt}
\begin{pspicture}(744.09448819,1052.36220472)
{
\newrgbcolor{curcolor}{0 0 0}
\pscustom[linestyle=none,fillstyle=solid,fillcolor=curcolor]
{
\newpath
\moveto(102.24414062,837.34499355)
\lineto(97.14160156,837.34499355)
\lineto(87.69335938,852.84792324)
\lineto(78.09863281,837.34499355)
\lineto(73.33789062,837.34499355)
\lineto(85.203125,855.94850918)
\lineto(74.09472656,873.03835293)
\lineto(79.09960938,873.03835293)
\lineto(87.83984375,859.04909512)
\lineto(96.65332031,873.03835293)
\lineto(101.38964844,873.03835293)
\lineto(90.35449219,856.04616543)
\lineto(102.24414062,837.34499355)
\closepath
}
}
{
\newrgbcolor{curcolor}{0 0 0}
\pscustom[linestyle=none,fillstyle=solid,fillcolor=curcolor]
{
\newpath
\moveto(245.97968475,970.58970108)
\lineto(254.35126678,986.38804092)
\lineto(258.70185272,986.38804092)
\lineto(248.00116912,966.72251358)
\lineto(248.00116912,954.26401748)
\lineto(243.93622772,954.26401748)
\lineto(243.93622772,966.54673233)
\lineto(233.23554412,986.38804092)
\lineto(237.65204803,986.38804092)
\lineto(245.97968475,970.58970108)
\closepath
}
}
{
\newrgbcolor{curcolor}{0 0 0}
\pscustom[linestyle=none,fillstyle=solid,fillcolor=curcolor]
{
\newpath
\moveto(272.90917125,966.49367138)
\lineto(272.90917125,968.05617138)
\lineto(277.75292125,968.05617138)
\lineto(277.75292125,966.49367138)
\lineto(272.90917125,966.49367138)
\closepath
}
}
{
\newrgbcolor{curcolor}{0 0 0}
\pscustom[linestyle=none,fillstyle=solid,fillcolor=curcolor]
{
\newpath
\moveto(290.21385875,961.91359326)
\lineto(288.40721812,961.91359326)
\lineto(288.40721812,974.60890576)
\lineto(283.9443275,974.60890576)
\lineto(283.9443275,976.19093701)
\lineto(294.6474525,976.19093701)
\lineto(294.6474525,974.60890576)
\lineto(290.21385875,974.60890576)
\lineto(290.21385875,961.91359326)
\closepath
}
}
{
\newrgbcolor{curcolor}{0 0 0}
\pscustom[linestyle=none,fillstyle=solid,fillcolor=curcolor]
{
\newpath
\moveto(305.23339,967.29445263)
\curveto(305.23339,965.54315055)(304.79068167,964.17596305)(303.905265,963.19289013)
\curveto(303.01984833,962.20981721)(301.79589,961.71828076)(300.23339,961.71828076)
\curveto(299.26333792,961.71828076)(298.40396292,961.94289013)(297.655265,962.39210888)
\curveto(296.90656708,962.84783805)(296.33039521,963.49887971)(295.92674937,964.34523388)
\curveto(295.52310354,965.19158805)(295.32128062,966.17466096)(295.32128062,967.29445263)
\curveto(295.32128062,969.0392443)(295.75747854,970.39666617)(296.62987437,971.36671826)
\curveto(297.50227021,972.34328076)(298.72622854,972.83156201)(300.30174937,972.83156201)
\curveto(301.81216604,972.83156201)(303.01008271,972.33351513)(303.89549937,971.33742138)
\curveto(304.78742646,970.34783805)(305.23339,969.0001818)(305.23339,967.29445263)
\closepath
\moveto(297.13768687,967.29445263)
\curveto(297.13768687,965.94679638)(297.40135875,964.92140576)(297.9287025,964.21828076)
\curveto(298.46255667,963.52166617)(299.24706187,963.17335888)(300.28221812,963.17335888)
\curveto(301.30435354,963.17335888)(302.08234833,963.52166617)(302.6162025,964.21828076)
\curveto(303.15005667,964.92140576)(303.41698375,965.94679638)(303.41698375,967.29445263)
\curveto(303.41698375,968.64210888)(303.14680146,969.65447867)(302.60643687,970.33156201)
\curveto(302.07258271,971.01515576)(301.2880775,971.35695263)(300.25292125,971.35695263)
\curveto(298.17609833,971.35695263)(297.13768687,970.00278596)(297.13768687,967.29445263)
\closepath
}
}
{
\newrgbcolor{curcolor}{0 0 0}
\pscustom[linestyle=none,fillstyle=solid,fillcolor=curcolor]
{
\newpath
\moveto(309.78417125,961.91359326)
\lineto(308.02635875,961.91359326)
\lineto(308.02635875,972.63624951)
\lineto(317.08885875,972.63624951)
\lineto(317.08885875,961.91359326)
\lineto(315.33104625,961.91359326)
\lineto(315.33104625,971.13234326)
\lineto(309.78417125,971.13234326)
\lineto(309.78417125,961.91359326)
\closepath
}
}
{
\newrgbcolor{curcolor}{0 0 0}
\pscustom[linestyle=none,fillstyle=solid,fillcolor=curcolor]
{
\newpath
\moveto(120.61376325,853.24922638)
\lineto(120.61376325,854.81172638)
\lineto(125.45751325,854.81172638)
\lineto(125.45751325,853.24922638)
\lineto(120.61376325,853.24922638)
\closepath
}
}
{
\newrgbcolor{curcolor}{0 0 0}
\pscustom[linestyle=none,fillstyle=solid,fillcolor=curcolor]
{
\newpath
\moveto(137.91845075,848.66914826)
\lineto(136.11181013,848.66914826)
\lineto(136.11181013,861.36446076)
\lineto(131.6489195,861.36446076)
\lineto(131.6489195,862.94649201)
\lineto(142.3520445,862.94649201)
\lineto(142.3520445,861.36446076)
\lineto(137.91845075,861.36446076)
\lineto(137.91845075,848.66914826)
\closepath
}
}
{
\newrgbcolor{curcolor}{0 0 0}
\pscustom[linestyle=none,fillstyle=solid,fillcolor=curcolor]
{
\newpath
\moveto(152.937982,854.05000763)
\curveto(152.937982,852.29870555)(152.49527367,850.93151805)(151.609857,849.94844513)
\curveto(150.72444033,848.96537221)(149.500482,848.47383576)(147.937982,848.47383576)
\curveto(146.96792992,848.47383576)(146.10855492,848.69844513)(145.359857,849.14766388)
\curveto(144.61115908,849.60339305)(144.03498721,850.25443471)(143.63134138,851.10078888)
\curveto(143.22769554,851.94714305)(143.02587263,852.93021596)(143.02587263,854.05000763)
\curveto(143.02587263,855.7947993)(143.46207054,857.15222117)(144.33446638,858.12227326)
\curveto(145.20686221,859.09883576)(146.43082054,859.58711701)(148.00634138,859.58711701)
\curveto(149.51675804,859.58711701)(150.71467471,859.08907013)(151.60009138,858.09297638)
\curveto(152.49201846,857.10339305)(152.937982,855.7557368)(152.937982,854.05000763)
\closepath
\moveto(144.84227888,854.05000763)
\curveto(144.84227888,852.70235138)(145.10595075,851.67696076)(145.6332945,850.97383576)
\curveto(146.16714867,850.27722117)(146.95165388,849.92891388)(147.98681013,849.92891388)
\curveto(149.00894554,849.92891388)(149.78694033,850.27722117)(150.3207945,850.97383576)
\curveto(150.85464867,851.67696076)(151.12157575,852.70235138)(151.12157575,854.05000763)
\curveto(151.12157575,855.39766388)(150.85139346,856.41003367)(150.31102888,857.08711701)
\curveto(149.77717471,857.77071076)(148.9926695,858.11250763)(147.95751325,858.11250763)
\curveto(145.88069033,858.11250763)(144.84227888,856.75834096)(144.84227888,854.05000763)
\closepath
}
}
{
\newrgbcolor{curcolor}{0 0 0}
\pscustom[linestyle=none,fillstyle=solid,fillcolor=curcolor]
{
\newpath
\moveto(157.48876325,848.66914826)
\lineto(155.73095075,848.66914826)
\lineto(155.73095075,859.39180451)
\lineto(164.79345075,859.39180451)
\lineto(164.79345075,848.66914826)
\lineto(163.03563825,848.66914826)
\lineto(163.03563825,857.88789826)
\lineto(157.48876325,857.88789826)
\lineto(157.48876325,848.66914826)
\closepath
}
}
{
\newrgbcolor{curcolor}{0 0 0}
\pscustom[linewidth=1,linecolor=curcolor]
{
\newpath
\moveto(176.97293,855.19868472)
\lineto(256.25271,854.17571472)
}
}
{
\newrgbcolor{curcolor}{0 0 0}
\pscustom[linewidth=1,linecolor=curcolor]
{
\newpath
\moveto(235.79342,874.63501472)
\lineto(255.74123,855.19868472)
}
}
{
\newrgbcolor{curcolor}{0 0 0}
\pscustom[linewidth=1,linecolor=curcolor]
{
\newpath
\moveto(256.25271,853.15275472)
\lineto(238.86231,836.78531472)
}
}
{
\newrgbcolor{curcolor}{0 0 0}
\pscustom[linewidth=1,linecolor=curcolor]
{
\newpath
\moveto(269.03978,943.68514472)
\lineto(308.42393,896.62876472)
\lineto(313.02727,920.15695472)
}
}
{
\newrgbcolor{curcolor}{0 0 0}
\pscustom[linewidth=1,linecolor=curcolor]
{
\newpath
\moveto(308.42393,895.60579472)
\lineto(282.33832,901.23210472)
}
}
{
\newrgbcolor{curcolor}{0 0 0}
\pscustom[linestyle=none,fillstyle=solid,fillcolor=curcolor]
{
\newpath
\moveto(300.88330063,828.65679975)
\lineto(295.78076156,828.65679975)
\lineto(286.33251938,844.15972944)
\lineto(276.73779281,828.65679975)
\lineto(271.97705063,828.65679975)
\lineto(283.842285,847.26031538)
\lineto(272.73388656,864.35015913)
\lineto(277.73876938,864.35015913)
\lineto(286.47900375,850.36090132)
\lineto(295.29248031,864.35015913)
\lineto(300.02880844,864.35015913)
\lineto(288.99365219,847.35797163)
\lineto(300.88330063,828.65679975)
\closepath
}
}
{
\newrgbcolor{curcolor}{0 0 0}
\pscustom[linestyle=none,fillstyle=solid,fillcolor=curcolor]
{
\newpath
\moveto(365.72094875,847.27863408)
\lineto(374.09253078,863.07697392)
\lineto(378.44311672,863.07697392)
\lineto(367.74243312,843.41144658)
\lineto(367.74243312,830.95295048)
\lineto(363.67749172,830.95295048)
\lineto(363.67749172,843.23566533)
\lineto(352.97680812,863.07697392)
\lineto(357.39331203,863.07697392)
\lineto(365.72094875,847.27863408)
\closepath
}
}
{
\newrgbcolor{curcolor}{0 0 0}
\pscustom[linestyle=none,fillstyle=solid,fillcolor=curcolor]
{
\newpath
\moveto(344.15188599,828.6783058)
\lineto(340.43240356,828.6783058)
\lineto(340.43240356,815.81453627)
\lineto(338.01541138,815.81453627)
\lineto(338.01541138,828.6783058)
\lineto(335.43728638,828.6783058)
\lineto(335.43728638,829.80623549)
\lineto(338.01541138,830.63875502)
\lineto(338.01541138,831.48470228)
\curveto(338.01541138,833.27506687)(338.40481567,834.6178403)(339.18362427,835.5130226)
\curveto(339.97138468,836.40820489)(341.17988078,836.85579603)(342.80911255,836.85579603)
\curveto(343.74905396,836.85579603)(344.7024231,836.69018731)(345.66921997,836.35896986)
\lineto(345.02468872,834.46565931)
\curveto(344.19216919,834.734214)(343.46259562,834.86849135)(342.83596802,834.86849135)
\curveto(342.02135213,834.86849135)(341.41710409,834.59993666)(341.02322388,834.06282728)
\curveto(340.62934367,833.52571791)(340.43240356,832.67529473)(340.43240356,831.51155775)
\lineto(340.43240356,830.55818861)
\lineto(344.15188599,830.55818861)
\lineto(344.15188599,828.6783058)
\closepath
}
}
{
\newrgbcolor{curcolor}{0 0 0}
\pscustom[linestyle=none,fillstyle=solid,fillcolor=curcolor]
{
\newpath
\moveto(336.25769885,865.24112633)
\lineto(336.25769885,843.39668904)
\curveto(336.25769885,839.54812716)(335.30771432,836.52315751)(333.40774524,834.32178011)
\curveto(331.50777616,832.12040271)(328.87172635,831.01971401)(325.49959581,831.01971401)
\curveto(322.16521299,831.01971401)(319.57949349,832.12809983)(317.7424373,834.34487148)
\curveto(315.90538111,836.57703738)(314.98685301,839.62509839)(314.98685301,843.48905453)
\lineto(314.98685301,865.24112633)
\lineto(318.47851803,865.24112633)
\lineto(318.47851803,843.3505063)
\curveto(318.47851803,840.59493599)(319.07619042,838.4628327)(320.27153521,836.95419644)
\curveto(321.47946257,835.44556018)(323.2913536,834.69124205)(325.70720832,834.69124205)
\curveto(327.9846547,834.69124205)(329.73363285,835.43786306)(330.95414279,836.93110507)
\curveto(332.1872353,838.42434708)(332.80378156,840.57954174)(332.80378156,843.39668904)
\lineto(332.80378156,865.24112633)
\lineto(336.25769885,865.24112633)
\closepath
}
}
\end{pspicture}
\\
            U - откр. в $X \cup_f Y$, если $U \cap X$ - откр. в X и\\ $U \cap Y$ - откр. в Y 
            (если f - инъект.)
        \end{example}
    \end{question}

    \begin{question}{12. Гомеоморфизм.}
        \begin{definition} 
            $f: X \rightarrow Y$ - гомеоморфизм, если\\
            \begin{enumerate}
                \item f - непр.
                \item f - биекция
                \item $f^{-1}$ - непр.
            \end{enumerate}
        \end{definition}

        \begin{hypothesis} 
            $\simeq$ - отношение эквив.
        \end{hypothesis}

        \begin{theorem} 
            Если $(X, \Omega_X) \simeq (Y, \Omega_Y)$, то\\
            $f_*: \Omega_X \rightarrow \Omega_Y$ - биекция\\
            $f_*(U) = f(U)$
        \end{theorem}
    \end{question}

    \begin{question}{13. Связность топологического пространства и множества.}
        
    \end{question}

    \begin{question}{14. Связность отрезка.}
        
    \end{question}

    \begin{question}{15. Связность замыкания. Связность объединения.}
        \begin{theorem} 
            $(X, \Omega)$ - топ. пр-во\\
            $A \subseteq X$ - связно\\
            $A \subseteq B \subseteq Cl A\\ \ra B$ - связно
        \end{theorem}

        \begin{theorem} 
            Если A - связ., то ClA - связ.
        \end{theorem}

        \begin{theorem} 
            $(X, \Omega)$ - топ. пр-во\\
            $A, B \subseteq X$ - связны\\
            $A \cap B \neq \varnothing\\ \ra A \cup B $ - связно
        \end{theorem}
    \end{question}

    \begin{question}{16. Связность и непрерывные отображения.}
        \begin{theorem} 
            $(X, \Omega_X), (Y, \Omega_Y)$ - топ. пр-ва\\
            $f: X \rightarrow Y$ - непр.\\
            X - связно $\ra$ f(x) - связно
        \end{theorem}
    \end{question}

    \begin{question}{17. Связность произведения пространств}
        \begin{theorem} 
            X, Y - топ. пр-ва\\
            $X \times Y$ - связн. $\rla$ X, Y - связн.
        \end{theorem}

        \begin{remark} 
            Любое конечное произведение связных топ. пр-в связно
        \end{remark}

        \begin{theorem} 
            \[\prod_{i \in I} X_i \text{ - связно} \rla \forall i \in I \q X_i \text{ - связно}\] 
        \end{theorem}
    \end{question}

    \begin{question}{18. Компоненты Связности.}
        \begin{definition} 
            X - топ. пр-во\\
            Компонентой связности т. $x_0 \in X$ назыв. наиб. по включению
            связное множество, ее содерж.\\
            $K_{x_0} = \cup \{M \in 2^X  \mid x_0 \in M \text{ - связ.}\}$
        \end{definition}

        \begin{theorem} 
            \begin{enumerate}
                \item $\forall x, y \in X \q K_x = K_y$ или $K_x \cap K_y = \varnothing$
                \item компоненты связности - замк.
                \item Для любого связ. мн-ва $\exists$ компонента связности, в которой оно
                целиком содержится\\
                $\forall M \subseteq X \ (M - \text{связ.} \ra \exists x \in X: M \subseteq K_x)$
                \item $\forall x, y, z \in X \ (x, y \in K_z \rla \exists M \text{ - связ.}: 
                x, y \in M \text{ и } z \in M)$
            \end{enumerate}
        \end{theorem}

        \begin{definition} 
            X - топ. пр-во назыв. вполне несвязным, если $\forall x \in X: K_x = \{x\}$
        \end{definition}
    \end{question}

    \begin{question}{19. Линейная связность}
        \begin{definition} 
            Линейно связное пр-во - топ. пр-во, в котором любые две точки можно соединить непр. кривой\\
            $(X, \Omega)$ - лин. св., если $\exists f:$\\
            $f: [0, 1] \rightarrow X \text{(путь в X)} \mid f(0) = x \text{(нач. пути)}; \  
            f(1) = y \text{(кон. пути)},\\  \forall x, y \in X$
        \end{definition}

        \begin{theorem} 
            X - топ. пр-во\\
            X - лин.св. $\ra$ X - св.
        \end{theorem}

        \begin{theorem} 
            A, B - лин. св.  $\q A \cap B \neq \varnothing \ra A \cup B$ - лин.св.
        \end{theorem}

        \begin{theorem} 
            X, Y - топ. пр-во; $\q f: X \rightarrow Y$ - непр.\\
            X - лин. св. $\ra$ f(x) - лин. св.
        \end{theorem}
    \end{question}

    \begin{question}{20. Компактность. Примеры.}
        \begin{definition} 
            (X, $\Omega$) - топ. пр-во\\
            X - компакт, если из любого открытого покрытия X можно выбрать конечное подпокрытие\\
            $\forall \{U_i\}_{i \in I},\q U_i \in \Omega$\\
            \[(\bigcup_{i \in I}U_i = X \ra \exists n \in \N \q \exists \{i_1, ..., i_n\}_{ij \in I}:
            \bigcup_{k = 1}^n U_{ik} = X)\] 
        \end{definition}

        \begin{definition} 
            $(X, \Omega)$ - топ. пр-во\\
            $A \subseteq X$ - комп., если оно комп. в индуц. топ.
        \end{definition}

        \begin{theorem} 
            \begin{enumerate}
                \item конечное топ. пр-во всегда компактно
                \item дискретное бесконечное множ. не комп.
                \item антидискр. множ. комп.
                \item  $[0, 1]$ - компакт.
            \end{enumerate}
        \end{theorem}

        \begin{theorem} 
            X - комп. $A \subseteq X$ - замк. $\ra A$ - комп.
        \end{theorem}

        \begin{theorem} 
            X - комп $\q f:X \rightarrow Y \ra f(x)$ - комп.
        \end{theorem}

        \begin{consequence} 
            Комп. - топ. св-во
        \end{consequence}
    \end{question}

    \begin{question}{21. Простейшие свойства компактности.}
        
    \end{question}

    \begin{question}{22. Компактность произведения пространств.}
        \begin{theorem} 
            X, Y - комп $\rla X \times Y$ - комп.
        \end{theorem}
        
        \begin{theorem} 
            \[\{X_i\}_{i \in I} \text{ - комп.} \rla \prod_{i \in I} X_i \text{ - комп.}\]
        \end{theorem}
    \end{question}

    \begin{question}{23. Компактность и хаусдорфовость}
        \begin{definition} 
            X назыв хаусдорф., если\\
            $\forall x_1 \neq x_2 \in X \q \exists U_{x_1}, U_{x_2}: \q U_{x_1} \cap U_{x_2} = \varnothing$
        \end{definition}
        
        \begin{theorem}[1] 
            X - хаусдорф. A - комп $\in$ X $\ra A$ - замк.
        \end{theorem}

        \begin{theorem} 
            $f: X \rightarrow Y$ непр., биекция\\
            X - комп.\\
            Y - хаусдорф.\\
            $\ra f$ - гомеоморф.
        \end{theorem}

        \begin{proof} [1]
            $X \setminus A$ - откр?\\
            $x_0 \in X \setminus A$\\
            $\forall x_1 \in A \ra \exists U_{x_0} \ni x_0; \ V_{x_1} \ni x_1$\\
            $U_{x_0} \cap V_{x_1} = \varnothing$\\
            \[\bigcup_{x_1 \in A} V_{x_1} \subset A \ra x_1, x_2, ..., x_k: 
            \q \bigcup_{i = 1}^k V_{x_i} \supset A\]\\
            \[U_{x_0} = \bigcap_{i = 1}^k U_{x_i} \text{ - искомая окр.  } U_{x_0} \cap A = \varnothing\]\\
            (Иначе $U_{x_0} \cap V_{x_i} \neq \varnothing, \q U_{x_i} \cap V_{x_i} \neq \varnothing$)
        \end{proof}
    \end{question}

    \begin{question}{24. Лемма Лебега. Компактность отрезка.}
        \begin{theorem} [Лемма Лебега] 
            \[X = [0, 1] \subset \bigcup_{i \in I} U_i \q\q \{U_i\}_{i \in I} \text{ - откр. покр. X}\]\\
            $\ra \exists \mathcal{E} > 0: \forall x_0 \ \exists i \in I: B(x_0, \mathcal{E}) \subseteq U_i$
            \\($\mathcal{E}$ зависит от покр. \q $\mathcal{E}$ - число Лебега)
        \end{theorem}

        \begin{consequence} 
            Отрезок - комп.
        \end{consequence}
    \end{question}

    \begin{question}{25. Критерий компактности подмножеств евклидова пространства.}
        \begin{theorem} 
            $A \subset \R^n$\\
            A - комп. $\rla A$ - замк и огр.
        \end{theorem}

        \begin{definition} 
            A - огр., если $\exists N: A \subset B(0, N)$
        \end{definition}

        \begin{proof} 
            $\ra A$ - замк. т.к. $\R^n$ - хаусдорф.\\
            A - огр. \q $\{B(0, n)\}_{n \in \N}$\\
            $\la A \subset [-N, N] \times [-N, N] \times ... \times [-N, N] = K$ т.к. огр.\\
            K - комп.\\
            A - замк. в K $\ra$ A - комп.
        \end{proof}
    \end{question}

    \begin{question}{26. Теорема Вейерштрасса. Примеры.}
        \begin{theorem} [Вейерштрасса] 
            K - компакт.\\
            $f: K \rightarrow \R $ - непр. $\ra \exists x_0 \in K:$\\
            $\forall x \in K \q f(x) \leq f(x_0) \q (x_0 - max)$
        \end{theorem}

        \begin{proof} 
            f(K) - комп. $\subset \R \; \ra f(K)$ - замк. и огр $\ra$\\
            $\sup{f(K)} \in f(K)$ (замк.)\\
            $\sup{f(K)} \neq \infty$ (огр.)\\
            $\sup{f(K)} = f(x_0)$
        \end{proof}
    \end{question}

    \begin{question}{27. Вторая аксиома счётности и сепарабельность.}
        \begin{definition} 
            X - обл. II А.С., если в X $\exists$ счетная база
        \end{definition}

        \begin{definition} 
            X - назыв сепараб., если $\exists$ A $\subset$ X\\
            |A| $\leq \aleph_0$ и $Cl A = X$
        \end{definition}

        \begin{definition} 
            A - всюду плотно, если ClA = X
        \end{definition}

        \begin{theorem} 
            X - II А.С. $\ra$ X - сепараб.
        \end{theorem}
    \end{question}

    \begin{question}{28. Теорема Линделёфа.}
        \begin{theorem} 
            X - II А.С. $\ra$ из $\forall$ откр. покр. X можно извлечь не более чем счетное подпокрытие
        \end{theorem}
    \end{question}

    \begin{question}{29. Первая аксиома счётности.}
        \begin{definition} 
            База окр-тей точки\\
            $\forall x \q \exists \{U_{x_i}\}_{i \in I_x}$\\
            \begin{enumerate}
                \item $U_{x_i} \in \Omega; \q x \in U_{x_i}$
                \item $\forall U \in \Omega \ : \ x \in U \q \exists U_{x_i} \ : \ x \in U_{x_i} \subset U$
            \end{enumerate}
        \end{definition} 
        
        \begin{definition} 
            Если $\exists$  база окр-тей:\\
            $\forall x \ \{U_{x_i}\}_{i \in I_x}$ не более чем счетное $\ra$ X удовл. I А.С.
        \end{definition}
    \end{question}

    \begin{question}{30. Из компактности следует секвенциальная компактность (с первой АС).}
        
    \end{question}

    \begin{question}{31. Из секвенциальной компактности следует компкатность (со второй АС).}
        
    \end{question}

    \begin{question}{32. Полнота и вполне ограниченность метрических пространств.}
        \begin{definition} 
            Фунд. послед.\\
            $\{X_n\}$ - фунд., если $\forall \mathcal{E} > 0 \q \exists N: \forall n, m > N: \rho(X_n, X_m) < \mathcal{E}$
        \end{definition}

        \begin{definition} 
            X назыв. полным, если $\forall$ фунд. послед. сходится
        \end{definition}

        \begin{definition} 
            $\{X_i\}_{i \in I}$ - $\mathcal{E}$-сеть, если $\forall x \q \exists x_i: \rho(x, x_i) < \mathcal{E}$
        \end{definition}

        \begin{definition} 
            X назыв. вполне огранич., если $\forall \mathcal{E} > 0 \q \exists$ конечная $\mathcal{E}$-сеть
        \end{definition}
    \end{question}

    \begin{question}{33. Из полноты и вполне ограниченности следует компактность}
        \begin{theorem} [равносильные]
                \begin{enumerate}
                    \item X - компактно
                    \item X - секцвенц. комп.
                    \item X - полн. и вполне огр.
                \end{enumerate}
        \end{theorem}
    \end{question}

    \begin{question}{34. Аксиомы отделимости.}
        \begin{theorem} [Колмогорова]
            $\forall x, y \in X: x \neq y \ \ra \ \exists U \in \Omega$
        \end{theorem}

        \begin{theorem} [Тихонова] 
            $\forall x, y \in X: x \neq y \ \ra \ \exists U \in \Omega$
        \end{theorem}

        \begin{theorem} [Хаусдорфа]
            $\forall x, y \in X \q \exists U_x, U_y: \ U_x \cap U_y = \varnothing$
        \end{theorem}

        \begin{theorem} [3]
            $\forall x \in X$ и замкнуто $F \subseteq X, \  x \not \in F$\\
            $\exists U_x$ и $U_F : \ U_x \cap U_F = \varnothing$
        \end{theorem}

        \begin{theorem} [4]
            $F_1, F_2$ - замк. : $F_1 \cap F_2 = \varnothing$\\
            $\exists U_{F_1}$ и $U_{F_2}: \ U_{F_1} \cap U_{F_2} = \varnothing$\\
            $T_2 \ra T_1 \ra T_0$
        \end{theorem}
    \end{question}

    \begin{question}{35. Нормальность матрического пространства.}
        \begin{definition} 
            $(X, \Omega)$ - хаусдорф.\\
            $X$ - нормально $\rla$ $\forall F$ - замк., $\forall G \in \Omega \q F \subseteq G \ra 
            \exists G' \in \Omega:\\ \ F \subseteq G' \subseteq Cl G' \subseteq G$
        \end{definition}
    \end{question}
\end{document}  

