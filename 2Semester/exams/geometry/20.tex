\documentclass[geometry.tex]{subfiles}

\begin{document}
  \section{Компактность. Примеры.}

  \begin{definition}
      (X, $\Omega$) - топ. пр-во\\
      X - компакт, если из любого открытого покрытия X можно выбрать конечное подпокрытие\\
      $\forall \{U_i\}_{i \in I},\q U_i \in \Omega$
      \[\bigcup_{i \in I}U_i = X \ra \exists n \in \N \q \exists \{i_1, ..., i_n\}_{ij \in I}: \bigcup_{k = 1}^n U_{ik} = X)\]
  \end{definition}

  \begin{examples}
      \begin{enumerate}
          \item Конечное топ. пр-во всегда компактно
          \item Дискретное бесконечное множ. не комп.
          \item Антидискр. мн-во комп.
          \item Топология зарицкого - комп.\\
          (выберем окр. мн-во, оно покрывается конечным набором мн-в, для каждой из остальных также)
          \item $X = \R$ с топологией стрелки - не компакт\\
          ($U_n = (-n,\ \infty)$)
          \item ($\R$, станд.) - не компакт\\
          ($U_n = (n,\ \infty)$)
          \item $[0, 1]$ - компакт
      \end{enumerate}
  \end{examples}

  \begin{definition}
      $(X, \Omega)$ - топ. пр-во\\
      $A \subseteq X$ - компактно, если оно комп. в индуц. топ.
  \end{definition}

  \begin{theorem}
      X - комп. $A \subseteq X$ - замк. $\Ra A$ - комп.
  \end{theorem}

  \begin{proof}
    *здесь когда-нибудь будет док-во*
  \end{proof}
\end{document}
