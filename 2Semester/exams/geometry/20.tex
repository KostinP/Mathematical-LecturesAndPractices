\documentclass[geometry.tex]{subfiles}

\begin{document}
  \section{Компактность. Примеры.}

  \begin{definition}
      (X, $\Omega$) - топ. пр-во\\
      X - компакт, если из любого открытого покрытия X можно выбрать конечное подпокрытие\\
      $\forall \{U_i\}_{i \in I},\q U_i \in \Omega$\\
      \[(\bigcup_{i \in I}U_i = X \ra \exists n \in \N \q \exists \{i_1, ..., i_n\}_{ij \in I}:
      \bigcup_{k = 1}^n U_{ik} = X)\]
  \end{definition}

  \begin{definition}
      $(X, \Omega)$ - топ. пр-во\\
      $A \subseteq X$ - комп., если оно комп. в индуц. топ.
  \end{definition}

  \begin{theorem}
      \begin{enumerate}
          \item конечное топ. пр-во всегда компактно
          \item дискретное бесконечное множ. не комп.
          \item антидискр. множ. комп.
          \item  $[0, 1]$ - компакт.
      \end{enumerate}
  \end{theorem}

  \begin{theorem}
      X - комп. $A \subseteq X$ - замк. $\ra A$ - комп.
  \end{theorem}

  \begin{theorem}
      X - комп $\q f:X \rightarrow Y \ra f(x)$ - комп.
  \end{theorem}

  \begin{consequence}
      Комп. - топ. св-во
  \end{consequence}
\end{document}
