\documentclass[geometry.tex]{subfiles}

\begin{document}
  \section{Прообраз топологии. Индуцированная топология.}

  \begin{definition}
      Пусть заданы $(X, \Omega_1)$ и $(X, \Omega_2)$\\
      Тогда $\Omega_1$ сильнее (тоньше) $\Omega_2$, если $\Omega_1 \supset \Omega_2$\\ \ \\
      Или: $\id: (X, \Omega_1) \us{\text{непр.}}{\ra} (X, \Omega_2)$
  \end{definition}

  \begin{utv}
      $f: X \ra Y$ - отобр. мн-в, $(Y, \Omega_Y)$ - топ. пр-во\\
      Вопрос: можно ли ввести топологию на $X$, чтобы отображение стало непрерывным? Да можно, если $\Omega_X$ - дискретная\\
      $\Omega_X$ - самая слабая топ.: f - непр.\\
      $\forall U \in \Omega_Y \q f^{-1}(U)$ должен быть открытым в X\\
      Вопрос: не является ли совокупность $f^{-1}(U)$ уже топологией?
  \end{utv}

  \begin{theorem}
      $\{f^{-1}(U)\}$ - топология на X и она назыв. прообразом $\Omega_Y$
  \end{theorem}

  \begin{proof}
      \begin{enumerate}
          \item $\displaystyle f^{-1}(\bigcup_{i \in I} U_i) = \bigcup_{i \in I} f^{-1}(U_i) \q (*)$
          \item $f^{-1}(U_1 \cap U_2) = f^{-1}(U_1) \cap f^{-1}(U_2)$
          \item $f^{-1}(\varnothing) = \varnothing \q\q f^{-1}(Y) = X$
      \end{enumerate}
      \[(*): \q f^{-1}(\bigcup_{i \in I} U_i) = \{x | \ f(x) \in \bigcup_{i \in I} U_i\} =
      \{x | \ \exists \  i \in I: f(x) \in U_i\}\]
  \end{proof}

  \begin{definition}
      $(X, \Omega_X)$ - топ. пр-во\\
      $A \subset X$\\
      $\Omega_A = \{U \cap A | \ U \in \Omega_X\}$ - индуцированная топология на A
  \end{definition}
\end{document}
