\documentclass[geometry.tex]{subfiles}

\begin{document}
  \section{Критерий компактности подмножеств евклидова пространства.}

  \begin{theorem}
      $A \subset \R^n$
      \[\text{A - комп. $\rla A$ - замк и огр.}\]
  \end{theorem}

  \begin{definition}
      A - огр., если $\exists N: A \subset B(0, N)$
  \end{definition}

  \begin{proof}
      ($\Ra$):

      $A$ - замк. т.к. $\R^n$ - хаусдорф.

      A - огр. \q $\{B(0, n)\}_{n \in \N}$\\ \ \\
      ($\La$):

      $A \subset [-N, N] \times [-N, N] \times ... \times [-N, N] = K$, т.к. огр K - компакт\\
      (каждый отрезок компактен, произведение комп. компактно)

      A - замк. в K $\Ra$ A - комп.
  \end{proof}
\end{document}
