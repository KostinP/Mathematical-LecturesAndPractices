\documentclass[algebra]{subfiles}

\begin{document}
    \section{Матрица линейного отображения для данных базисов. Матрица суммы отображений. Изоморфизм пространства линейных отображений и пространства матриц.}

    \begin{Definition}
        \[\dim U = m < \infty \q\q \dim V = n < \infty\]
        \[u_1, ..., u_m \text{ - базис } U; \q v_1, ..., v_n \text{ - базис }V\]
        \[f : U \to V \text{ - лин. отобр.}\]
        \[f(u_j) = \sum^{n}_{i=1} a_{ij} v_i\]
        \[A = (a_{ij}) = \begin{pmatrix}
          a_{11} & a_{1j} \\
          a_{21} & \ddots\\
               & a_{nj} 		& a_{nm}
        \end{pmatrix}\]
        \[a_{1j} \text{ - коэфф разложения } f(u_j) \text{ по базису } \{v_1, ..., v_n\}\]
        \[A \text{ - матрица лин. отобр в базисах } \{u_1, ..., u_m\}, \{v_1, ..., v_n\}\]
        \[\begin{matrix}
          A = [f] &\{u_j\}\\
              &\{v_j\}
        \end{matrix}\]
        \[f(u) = c_1 f(u_1) + ... + c_m f(u_m) = \sum^{m}_{j=1} c_j f(u_j) = \]
        \[= \sum^{m}_{j=1} c_j \sum^{n}_{i=1} a_{ij} v_i = \sum^{n}_{i=1} ( \sum^{m}_{j=1} c_j a_{ij})v_i\]
        \[\text{где } u = c_1 u_1 + ... + c_m u_m\]
        \[\begin{pmatrix}
          c_1\\
          ...\\
          c_m
        \end{pmatrix}
        = [u]_{\{u_i\}} \q\q [v]_{\{v_i\}} = A \cdot [u]_{\{u_i\}}
        \]
        \[\begin{align*}
          [f + g]  &_{\{u_j\}}& = [f]&_{\{u_j\}}& +\q [g]&_{\{u_j\}}\\
               &_{\{v_i\}}&      &_{\{v_i\}}&     &_{\{v_i\}}
        \end{align*}\]
    \end{Definition}

    \begin{consequence}[лин. отобр. и матриц лин. отображения]
        $u_1,...,u_m$ - базис U, $v_1,...,v_m$ - базис V
        \begin{enumerate}
          \item $f: U \ra V$
          \[\alpha \in K\]
          \[(\alpha f) U \ra V\]
          \[(\alpha f)(u) = \alpha f(u)\]
          \[\alpha f \text{ - лин. отобр.}\]
          \[\alpha f(\gamma_1 u_1 + \gamma_2 u_2) = \alpha (\gamma_1 f(u_1) + \gamma_2 f(u_2))\]
          \[= \gamma_1 \alpha f(u_1) + \gamma_2 \alpha f(u_2) = \gamma_1 (\alpha f) (u_1) + \gamma_2 (\alpha f) (u_2)\]
          \[[\alpha f]_{\{u_j\}\ \{v_i\}} = \alpha[f]_{\{u_j\}\ \{v_i\}}\]
          \item $f,g: U \ra V$ $f,g$ - лин.
          \[(f+g)(u) := f(u) + g(u)\]
          $f+g$ - лин. отображение
          \[(f+g)(u_i) = f(u_i) + g(u_i)\]
          \[[f + g]_{\{u_j\}\ \{v_i\}} = [f]_{\{u_j\}\ \{v_i\}} + [g]_{\{u_j\}\ \{v_i\}}\]
        \end{enumerate}
    \end{consequence}

    \begin{definition}
        $U,V \text{ назыв. изоморфными, если } \exists f: U \to V:$
        \begin{enumerate}
            \item $f \text{ - лин.}$
            \item $f \text{ - биекция}$
        \end{enumerate}
    \end{definition}

    \begin{upr}
        $f^{-1}$ - тоже лин. отобр.
    \end{upr}

    \begin{remark}
        Пусть $\dim U = m < \infty, \q \dim V = n < \infty$, фикс. базисы в $U$ и $V$
        \[L(U, V) \ni f \ra [f]_{\{u_j\}\ \{v_i\}} \in M(n,m,K)\]
        \[\varphi: L(u,v) \ra M(n,m,k)\]
        \[\dim V = n,\q \dim U = m\]
        \[f \ra [f]_{\{u_i\}\ \{v_j\}}\]
        Это отобр. инъективно (разным отобр. отвечают разные матрицы)
        \[A_i = (a_{ij}) \qq 1 \leq i \leq n \qq 1 \leq j \leq m\]
        Определим лин. отобр. f
        \[f(c_1 u_1 + ... + c_n u_n) = d_1 v_1 + ... + d_n v_n\]
        \[\begin{pmatrix}
            d_1\\
            \vdots\\
            d_n
        \end{pmatrix} = A \begin{pmatrix}
            c_1\\
            \vdots\\
            c_m
        \end{pmatrix} \qq [u_i] = \begin{pmatrix}
            0\\
            \vdots\\
            1\\
            \vdots\\
            0
        \end{pmatrix} - i\]
        \[[f] = A\]
        $Au_i$ - i-ый столбец A
        \[\begin{pmatrix}
            d_1\\
            \vdots\\
            d_n
        \end{pmatrix}\]
        \[\Ra \varphi\text{ - сюръекция}\]
        Получаем: $L \cong M(n,m,K)$
    \end{remark}


    \section{Композиция линейных отображений. Матрица композиции.}

    \begin{Hypothesis}
        \[u_1, ..., u_m \q v_1, ..., v_n \q w_1, ..., w_k \text{ - базисы}\]
        \[[gf]_{\us{\{w_k\}}{\{u_i\}}} = [g]_{\us{\{w_k\}}{\{v_j\}}}  [f]_{\us{\{v_j\}}{\{u_i\}}} \]
    \end{Hypothesis}

    \begin{Proof}
        \[i \text{ - ый столбец } [gf] \text{ - это коорд. } (gf)(u_i) \text{ в базисе } \{w_1, ..., w_k\}\]
          \[f(u_i) \text{ - коорд. этого вектора в базисе } v_1, ..., v_n \text{ - это }i\text{ - ый столбец матрицы } [f]\]
        \[[gf(u_i)]_{\{w\}}   \text{ - это } i \text{ - ый столбец }[gf]\]
        \[[gf(u_i)]_{\{w\}} = [g] - i \text{ - ый столбец матр. } [f] = [g][f(u_i)]_{\{v_j\}} \]
        \[\text{т.о. } [gf] = [g][f]\]
    \end{Proof}

    \section{Преобразование матрицы линейного отображения при замене базисов.}
    \begin{Definition}
        \[f: U \to V \text{ - лин}\]
        \[\begin{matrix}
            u_1, ..., u_m\\
            u_1', ..., u_m'
        \end{matrix} \text{ - базисы } U
        \q\q
        \begin{matrix}
            v_1, ..., v_n\\
            v_1', ..., v_n'
        \end{matrix}
        \text{ - базисы } V
         \]

         \[\begin{matrix}
           A = [f]&_{\{u_i\}}& \q\q A' = [f]&_{\{u_i'\}} \\
              &_{\{v_j\}}& 			   &_{\{v_j'\}}
         \end{matrix}\]
         \\
         $C $ - матрица замены координат при переходе от $\{u_i\}$ к $\{u_i'\}$\\
         $D $ - матрица замены координат при переходе от $\{v_j\}$ к $\{v_j'\}$\\
         $i $ - ый столбец $C $ - это коорд. $u_i'$ в базисе $u_1, ..., u_m$\\
         $i $ - ый столбец $D $ - это коорд. $v_j'$ в базисе $v_1, ..., v_k$

         \[[u]_{\{u_i\}} = C[u]_{\{u_i'\}}\text{, аналогично для }D  \]
    \end{Definition}
    \begin{Theorem}
        \[A' = D^{-1}AC \]
    \end{Theorem}

    \begin{proof}
        i-ый столбец $A'$ - это коорд. $f(u_i')$ в базисе $v_1
        ,...,v_n'$\\
        Найдем коорд. $f(u_i')$ в базиск $v_1,...,v_n$
        \[[u_i']_{\{u_j\}} = \text{i-ый столбец C}\]
        \[[f(u_i')]_{\{v_j'\}} = D^{-1} [f(u_i')]_{\{v_j\}} = D^{-1} A [u_i']_{\{u_j\}}\]
        \[\text{i-ый столбец $A' = D^{-1}A$ - i-ый столбец C}\]
        \[\text{объединим для всех столбцов: $A' = D^{-1} A C$}\]
    \end{proof}


    \section{Ядро и образ линейного отображения, их свойства. Критерий инъективности и
      сюръективности линейного отображения в терминах ядра и образа.}
        \begin{Definition}
            \[f : U \to  V \q f \text{ - лин.}\]
            \[f(U) = \{v \in V \mid \exists u \in U : v = f(u)\} = \Im f \text{ (образ f)}\]
            \[f^{-1} (\{0_v\}) = \{u \in U : f(u) = 0_v\} = \Ker f \text{ (ядро f)}\]
        \end{Definition}

        \begin{Hypothesis}
            \[\Im f \subseteq V; \q \Ker f \subseteq U\]
        \end{Hypothesis}

        \begin{Proof}
            \[f(0_U) = 0_V \Ra \Im f \neq \varnothing,\q \Ker f \neq \varnothing\]
            Замкн-ть относ. слож. и умножение:
            \[v_1, v_2 \in \Im f;\q \alpha_1, \alpha_2 \in K\]
            \[\alpha_1 v_1 + \alpha_2 \in \Im f, \text{ т.е. } \e u_1, u_2 \in U: f(u_i) = v_i\]
            \[f(\alpha_1 u_1 + \alpha_2 u_2) = \alpha_1 f(u_1) + \alpha_2 f(u_2) = \alpha_1 v_1 + \alpha_2 v_2\]
            \[\text{зн. } \alpha_1 v_1 + \alpha_2 v_2 \in \Im f\]
            \[\Ra \Im f \subset V\]
            \[u_1, u_2 \in \Ker f,\q \alpha_1,\ \alpha_2 \in K\]
            \[f(\alpha_1 u_1 + \alpha_2 u_2) = \alpha_1 f(u_1) + \alpha_2 f(u_2) = \alpha_1 \cdot 0 + \alpha_2 \cdot 0 = 0\]
            \[\text{зн. } \alpha_1 u_1 + \alpha_2 u_2 \in \Ker f\]
            \[\Ra \Ker f \subset U\]
        \end{Proof}

        \begin{hypothesis}
            \begin{enumerate}
                \item лин. отобр. $f: U \to V \text{ сюръективно } \rla \Im f = V$
                \item инъективно $\rla \Ker f = \{0_u\}$
            \end{enumerate}
        \end{hypothesis}

        \begin{proof}
            \begin{enumerate}
              \item очевидно (опр. сюръект-ти)
              \item $(\Ra)$:
              \[\left. \begin{matrix}
                f(0_u) = 0_v\\
                u \in \Ker f \q f(u) = 0_v
              \end{matrix} \right| \Ra u = 0, \text{ т.к. f инъект.}\Ra \Ker f = \{0\}\]
              $(\La)$:
              \[u_1, u_2 \in U\]
              \[\begin{matrix}
                f(u_1) = f(u_2)\\
                f(u_1 - u_2) = 0
              \end{matrix} \left. \begin{matrix}
                u_1 -u_2 \in \Ker f\\
                \Ker f = \{0\}
              \end{matrix} \right| \Ra \begin{matrix}
                u_1 - u_2 = 0\\
                u_1 = u_2
              \end{matrix} \Ra \text{f - инъект.}\]
            \end{enumerate}
        \end{proof}


    \section{Выбор базисов, для которых матрица линейного отображения имеет почти единичный вид. Следствие для матриц. Теорема о размерности ядра и образа.}
        \begin{Theorem}
            \[U, V \text{ - конечномерные; } f: U \to V \text{ - лин. Тогда } \exists \text{ базисы пр-в } U \text{ и } V,\]
            \[\text{в которых матрица f - почти единичная}\]
            \[\begin{matrix}
              [f]&_{\{u_i\}}\\
                 &_{\{v_j\}}
            \end{matrix} =
            \begin{pmatrix}
              E_2 & 0\\
              0	& 0
            \end{pmatrix}\]
        \end{Theorem}

        \begin{Proof}
            \[\Ker f \subset U \qq \Ker f \text{ конечномерно}\]
            \[r = \dim U - \dim \Ker f\]
            \[\dim U = m,\q \dim V = n\]
            Выберем базис в $Ker f$: $u_1,...,u_m$ - л.н. в U\\ \ \\
            Дополним до базиса U
            \[\ub{\text{базис U}}{u_1,...,u_r: \ub{\text{базис в }\Ker f}{u_{r+1},...,u_m}}\]
            \[v_i = f(u_i)_{i=1...r} \text{ - базис }\Im f\]
            \[v \in \Im f \qq \e u \in U \qq v = f(u)\]
            \[u = a_1 u_1 + ... + a_r u_r + a_{r+1} u_{r+1} + ... + a_m u_m\]
            \[v = f(u) = a_1 f(u_1) + ... + a_r f(u_r) + a_{r+1} f(u_{r+1}) + ... + a_m f(u_m) = a_1 v_1 + ... + a_r v_r \text{ (остальные 0)}\]
            \[v_1,...,v_r \text{ - сем-во обр. $\Im f$}\]
            \[d_1 v_1 + ...+ d_r v_r = 0\]
            \[d_1 f(u_1) + ... + d_r f(u_r) = 0\]
            \[f(d_1 u_1 + ... + d_r u_r) = 0\]
            \[d_1 u_1 + ... + d_r u_r \in \Ker f\]
            \[d_1 u_1 + ... + d_r u_r = d_{r+1} u_{r+1} + ... + d_m u_m \text{ (по базису ядра)}\]
            \[d_1 u_1 + ... + d_r u_r - d_{r+1} u_{r+1} + ... - d_m u_m = 0\]
            \[\text{т.к. $u_1,...,u_m$ - базис U, то $\alpha_1 = ... = \alpha_r = \alpha_{r+1} = ... = \alpha_m = 0$}\]
            \[\alpha_1 = ... = \alpha_r = 0 \Ra v_1,...,v_r \text{ - ЛНЗ} \Ra v_1,...,v_r \text{ - базис }\Im f\]
            \[\text{Дополним до базиса V }\ub{\text{базис V}}{\ub{\text{базис }\Im f}{v_1,...,v_r}, v_{r+1},...,v_n}\]
            \[[f]_{\{u_i\}\ \{v_j\}}\]
            \[i=1...r \q f(u_i) = v_i \text{ (по постр. базисов)}\]
            \[[f(u_i)]_{\{v_j\}} = \begin{pmatrix}
              0\\
              \vdots\\
              1\\
              \vdots\\
              0
            \end{pmatrix} - i\]
            \[i > r \qq f(u_i) = 0\]
            \[f(u_i)_{\{v_j\}} = \begin{pmatrix}
              0\\
              \vdots\\
              0
            \end{pmatrix}\]
            \[[f]_{\{u_i\}\ \{v_j\}} = \os{r}{\begin{pmatrix}
              \begin{matrix}
                1 & & 0\\
                 & \ddots & \\
                 0 & & 1
              \end{matrix} & 0\\
              0 & 0
            \end{pmatrix}}\]
        \end{Proof}

        \begin{Consequence} [1]
          \[A \in M(n, m, K) \text{ Тогда } \exists \text{ обрат. матрицы } C \in M(m, n, K) \text{ и } \]
          \[D \in M(n, m, K) \text{, такие, что } D^{-1} AC = \begin{pmatrix}
            E_r & 0\\
            0   & 0
          \end{pmatrix}\]
        \end{Consequence}

        \begin{proof}
            $U = K^m$ - пр-во векторов-столбцов размера m, $V = K^{\sim}$
            \[f: U \ra V \qq x \mapsto Ax\]
            \[e_1,...,e_m \text{ - станд. базис }K^m\]
            \[\E_1,...,\E_m \text{ - станд. базис }K^{\sim}\]
            \[e_i = \begin{pmatrix}
              0\\
              \vdots\\
              1\\
              \vdots\\
              0
            \end{pmatrix} - i \qq \R_i = \begin{pmatrix}
              0\\
              \vdots\\
              1\\
              \vdots\\
              0
            \end{pmatrix} - i\]
            \[[f]_{\{e_i\}\ \{\E_i\}} = A\]
            \[\e\text{базисы } \begin{matrix}
              u_1,...,u_m\\
              v_1,...,v_n
            \end{matrix}: [f]_{\{u_i\}\ \{v_j\}} = \begin{pmatrix}
              E_r & 0\\
              0 & 0
            \end{pmatrix}\]
            C - матрица преобр. коорд. при переходе от $\{e_i\}$ к базису $\{u_i\}$\\
            D - матрица преобр. коорд. при переходе от $\{\E_i\}$ к базису $\{v_i\}$
            \[D^{-1} A C = \begin{pmatrix}
              E_r & 0\\
              0 & 0
            \end{pmatrix}\]
        \end{proof}

        \begin{Consequence} [2]
            \[\dim U < \infty; \q V \text{ - произв.}\]
            \[f: U \to V\]
            \[\text{Тогда } \dim U = \dim \Ker f + \dim \Im f\]
        \end{Consequence}

        \begin{proof}
            В теореме мы доказали, что $v_1,...,v_r$ - базис $\Im f$ не используя предположения о конечности V
            \[v_1,...,v_r \text{ - базис }\Im f\]
            \[r = \dim \Im f - \dim U - \dim \Ker f\]
            \[\Ra \dim U = r + \dim \Ker f = \dim \Im f + \dim \Ker f\]
        \end{proof}


    \section{Критерий изоморфности конечномерных пространств}
      \begin{Definition}
        \[U, V \text{ изоморфны, если } \exists \text{ биект. лин. отображение (изоморфизм) } f: U \to V\]
        \[U \cong V\]
      \end{Definition}
      \begin{Theorem}
          \[U, V \text{ - конечномерные в.п. над }K\]
          \[U \cong V \rla \dim U = \dim V\]
      \end{Theorem}
      \begin{Proof}
          \[\Ra f: U \to  V, \q f \text{ - биекция, лин.}\]
          \[f \text{ - инъект. } \Ra \Ker f = \{0\}\]
          \[f \text{ - сюръект. } \Ra \Im f = V\]
          \[\dim V = \dim \Im f = \dim U - \dim \Ker f = \dim U - 0 = \dim U \]
          \[\La\dim U = \dim V = n\]
          \[u_1, ..., u_n \text{ - базис } U\]
          \[v_1, ..., v_n \text{ - базис } V\]
          \[\text{Любой } u \in U \text{ единственным образом раскладывается в сумму }\]
          \[u = \alpha_1 u_1 + ... \alpha_n u_n \q \alpha_i \in K\]
          \[f(u) = \alpha_1 v_1 + ... + \alpha_n v_n\]
          \[\widetilde{u} = \widetilde{\alpha_1}u_1 + ... + \widetilde{\alpha_n}u_n\]
          \[u + \widetilde{u} = (\alpha_1 + \widetilde{\alpha})u_1 + ... + (\alpha_n + \widetilde{\alpha_n})u_n\]
          \[f(\widetilde{u}) = \widetilde{\alpha}_1 v_1 + ... + \widetilde{\alpha}_n v_n\]
          \[f(u + \widetilde{u}) = (\alpha_1 + \widetilde{\alpha}_1) v_1 + ... + (\alpha_n + \widetilde{\alpha}_n) v_n\]
          \[f(u + \widetilde{u}) = f(u) + f(\widetilde{u})\]
          \[\text{Аналогично } f(\alpha u) = \alpha f(u)\]
          \[\text{Значит } f \text{ - лин. отобр}\]
          \[\text{т.к. } v_1, ..., v_2 \text{ - сем-во образующих } \Ra f \text{ - сюръект.}\]
          \[v \in V \q v = \alpha_1 v_1 + ... + \alpha_n v_n\]
          \[u = \alpha_1 u_1 + ... + \alpha_n u_n \q f(u) = v\]
          \[\text{т.к. } v_1, ..., v_n \text{ - ЛНЗ, то f - инъект.}\]
          \[\text{достаточно проверить, что } \Ker f = \{0\}\]
          \[u = \alpha_1 u_1 + ... + \alpha_n u_n\]
          \[0 = f(u) = \alpha_1 v_1 + ... + \alpha_n v_n \Ra \alpha_1, ..., \alpha_n = 0, u = 0 \Ra \Ker f = \{0\}\]
          \[\Ra f \text{ - изоморфизм}\]
      \end{Proof}


    \section{Двойственное пространство. Двойственный базис. Изоморфность конечномерного пространства и его двойственного. Пример пространства не изоморфного своему двойственному.}

    \begin{Definition}
        \[V \text{ - в.п. над } K\]
        \[V^* = L(V, K) \text{ - двойственное пр-во к } V\]
        \[(\text{пр-во линейных отображений из V \text{ в } K } )\]
        \[\text{элементы } V^* \text{ - лин. функционалы } V \text{ (лин. отобр)}\]
    \end{Definition}

    \begin{Example}
        \[V_\R = C([0;1] \to \R)\]
        \[f \to  \int_{0}^1 f(x)dx\]
        \[a \in [0; 1] \q f \to f(a)\]
    \end{Example}

    \begin{Definition}
        \[e_1, ..., e_n \text{ - базис }V\]
        \[c_1, ..., c_n \text{ - двойственный базис } V \text{, если}\]
        \[f(e_i, c_j) = \begin{cases}
            1, & i = j\\
            0, & i \neq j
        \end{cases}\]
    \end{Definition}

    \begin{Theorem}
        \[\dim V = n < \infty \Ra V^* \cong V\]
    \end{Theorem}
    \begin{Proof}
        \[v_1, ..., v_n \text{ - базис } V\]
        \[v_i^*(\alpha_1 v_1 + ... + \alpha_n v_n) = \alpha_i\]
        Проверим, что $v_1^*,...,c_n^*$ - базис $V^*$
        \begin{enumerate}
            \item $v_1^*,...,v_n^*$ - мн-во обр.
            \[\varphi \in V^*;\q v \in V\q v = \alpha_1 v_1 + ... + \alpha_n v_n\]
            \[\varphi(v) = \varphi(\alpha_1 v_1 + ... + \alpha_n) = \alpha_1 \varphi(v_1) + ... + \alpha_n \varphi(v_n) = \]
            \[= v_1^*(v) \varphi(v_1) + ... + v_n^*(v) \varphi(v_n) = \varphi(v_1) v_1^*(v)+ ... + \varphi(v_n) v_n^*(v) = (\varphi(v_1) v_1^* + ... + \varphi(v_n)v_n^*)&n (v) \Ra\]
            \[\varphi = \varphi(v_1)v_1^* + ... + \varphi(v_n) v_n^*\]
            т.к. $\varphi$ произв, то $v_1^*,...,v_n^*$ - сем-во обр. $V^*$
            \item $v_1^*,...,v_n^*$ - ЛНЗ
            \[\beta_1 v_1^* + ... + \beta_n v_n^* = 0_{v^*}\]
            (пусть есть такое отобр., которое любой вектор)
            \[\forall v \in V\q (\beta_1 v_1^* + ... + \beta_n v_n^*)(v = 0)\]
            \[v = \alpha_1 v_1 + ... + \alpha_n v_n\]
            $\beta_1 \alpha_1 + ... + \beta_n \alpha_n = 0$ верно для $\forall \alpha_1,...,\alpha_n$
            \[\alpha_i = 1,\q \alpha_j = 0 \q i \neq j \Ra \beta_i = 0 \q \forall i = 1...n\]
            (предположим, что лин. к функций = 0, получим что все коэф. = 0 $\Ra$ они действительно ЛНЗ)
            \[v_1^*,...,v_n^* \text{ - базис $V^*$}\]
            \[\dim V^* = n = \dim V\]
            (по теореме о конечномерном пр-ве)
            \[\Ra V^* \cong V\]
        \end{enumerate}
    \end{Proof}

    \begin{definition}
        $v_1,...,v_n$ - базис V, $v_1^*,...,v_n^*$ - двойственный базис
    \end{definition}

    \begin{remark}
        Изоморфизм между $V$ и $V^*$
        \[V_i \mapsto V_i^*\]
        Далее продолжаем по линейности, изоморфизм не единственный, зависит от выбора базиса
    \end{remark}

    \begin{Example}
        \[V = \{(b_0, b_1,...): b_i \in K\}\]
        \[V_0 = \{(a_0, a_1,...): a_i \in K \text{, п.в. }a_i = 0\}\]
        Базис $V_0: e_i = (0,...,0,\os{i}{1},0,...,0),\q i = 0,1,...$
        \[V_0 \cong K[x]\]
        $\varphi \in V_0^*$ - полность. опр. значениями $b_i = \varphi(e_i),\q i=0,1,...$
        \[\varphi \ra (b_0,b_1,...)\]
        Изоморфизм между $V_0^*$ и $V$
        \[(b_0, b_1, ...)(\us{\text{п.в. }a_i = 0}{a_0,a_1},...) = \sum^{\infty}_{\text{п.в. }a_i = 0} a_i b_i\]
        \[V_0^* \cong V \neq V_0\]
        Рассмотрим $(b_0,b_1,...)$ как эл-ты двойственного пр-ва и смотрим как $(b_0,b_1,...)$ действуют на $(a_0,a_1,...)$
        \[(a_0, a_1,...) = a_0 b_0 + a_1 b_1 + ... = \sum_{\text{п.в. }a_i = 0} a_i b_i\]
        \[\dim V = n < \infty\]
        \[V = K^n \q e_i = \begin{pmatrix}
          0\\
          \vdots\\
          1\\
          \vdots\\
          0
        \end{pmatrix}\q i = 1,...,n\]
        Двойственное пр-во в этом случае удобно представлять как пр-во векторов строк
        \[V^* \cong {}^{\sim}K\]
        \[e_i^* = (0,...,1,..,0)\]
        \[e_i^*(e_i) = \begin{cases}
          1, & i = j\\
          0, & i \neq j
        \end{cases} = e_i^* e_j\]
        Всякая ${}^*$ строка $(b_1,...,b_n)$ задает лин. функц. на V
        \[(b_1,...,b_n) \begin{pmatrix}
          a_1\\
          \vdots\\
          a_n
        \end{pmatrix} = b_1 a_1 + ... + b_n a_n\]
    \end{Example}

    \section{Каноническое отождествление конечномерного пространства со вторым двойственным.}

    V - в.п. над K
    \[\dim U = n < \infty\]
    \[V \cong V^* \cong V^{**}\]
    \[u \in B \q u \mapsto \varphi_u \in V^{**}\]
    \[\varphi_u(f) = f(u),\q f \in V^*\]
    $\varphi_u \in V^{**}$ линеен по f по опр.
    \[\varphi_n(\alpha_1 f_1 + \alpha+2 f_2) = (\alpha_1 f_1 + \alpha_2 f_2)(u) = \alpha_1 f_1(u) + \alpha_2 f_2(u) = \alpha_1 \varphi_u(f_1) + \alpha_2 \varphi_u(f_2)\]
    Отобр.: $V \ra V^{**}$
    \[u \ra \varphi_u\]
    лин. отобр. с в.п.?
    \[\alpha_1 u_1 + \alpha_2 v_2 = \varphi_{\alpha_1 u_1 + \alpha_2 u_2}\]
    \[v \in V^* \q \varphi_{\alpha_1 u_1 + \alpha_2 u_2} (f) = f(\alpha_1 u_1 + \alpha_2 u_2) = \alpha_1 f(u_1) + \alpha_2 f(u_2) = \alpha_1 \varphi_{u_1}(f) + \alpha_2 \varphi_{u_2}(f)\]
    \[\varphi_{\alpha_1 u_1 + \alpha_2 u_2} = \alpha_1 \varphi_{u_1} + \alpha_2 \varphi_{u_2}\]
    $\Ra$ действительно отобр. в.п.\\ \ \\
    Проверим, что отобр $V \ra V^*,\q u \ra \varphi_n$ инъективно\\
    Для этого достаточно убедиться, что его ядро тривиально
    \[\varphi_n \equiv 0_{V^{**}}\]
    \[\forall f \in V^* \q f(u) = \varphi_u(f) = 0\]
    Если бы $u \neq 0$, то можно было бы дополнить $\{u\}$ до всего базиса V: $u_1,u_2,...$
    \[u_1^*,u_2^*,... \text{ - двойственный базис}\]
    \[u^*(u) = 1 \neq 0 \q \varphi_n(u^*) \neq 0 \Ra u = 0_u\]
    $\Ra$ в нулевое отображение переходит только нулевой вектор
    \[U \ra V^** \q u \ra \varphi_n \text{ линейно и инъективно}\]
    \[\dim V = \dim V^{**} = n\]
    $\dim$ образа = $n - \us{=0}{\dim}$ ядра = $n$\\
    $\Ra$ образ совпадает по всем $V^{**}$

    \section{Линейные операторы. Кольцо линейных операторов. Изоморфность кольца линейных операторов и кольца матриц.}

    \begin{Definition}
        \[V \text{ - в.п. над } K\]
        \[L(v, v) \text{ эл-ты этого пр-ва назыв. линейными операторами на }V\]
        \[\End(V) = L(V, V)\]
        \[\text{На } \End(V) \text{ определена композиция (умножение операторов)}\]
        \[\sqsupset \dim V = n\]
        \[\text{зафиксируем базис } v_1, ..., v_n \text{ пр-ва } V\]
        \[\End(V) \to M_n (K) \text{ изморфизм в.п.}\]
        \[f \to [f]_{\{v_i\}} \text{ - матрица оператора в базисе} \]
        Так как $[g \circ f]_{\{v_i\}} = [g]_{\{v_i\}} \cdot [f]_{\{v_i\}}$, то это отображение сопоставляет оператору матрицу в базисе $v_1,...,v_k$. Это ещё и изоморфизм колец
        \[\dim V = n < \infty \q \End(V) \cong M_n(K)\]
    \end{Definition}

    \begin{Theorem}
        \[(\End(V), \cdot, +) \text{ - кольцо}\]
    \end{Theorem}

    \begin{proof}
        Пусть $\varphi_1, \varphi_2 \in \End(V)$
        \begin{enumerate}
            \item $\varphi_1(x) + \varphi_2(x) = \varphi_2(x) + \varphi_1(x)$
            \item $-\varphi_1(x) = (-1) \varphi_1(x)\q -[f]_{\{v_i\}} = (-1) [f]_{\{v_i\}}$
            \item $(\varphi_1(x) + \varphi_2(x)) + \varphi_3(x) = \varphi_1(x) + (\varphi_2(x) + \varphi_3(x))$
            \item $\id = 1 \q \varphi_1(x) \cdot \id = \id \cdot \varphi_1(x) = \varphi_1(x)$
            \item $0_v + \varphi = \varphi + 0_v = \varphi$
            \item $\varphi_1(x) (\varphi_2(x) \cdot \varphi_3(x)) = (\varphi_1(x) \varphi_2(x)) \cdot \varphi_3(x)$
            \item $\varphi_1(\varphi_2 + \varphi_3) = \varphi_1 \varphi_2 + \varphi_1 \varphi_3$
        \end{enumerate}
    \end{proof}

    \section{Многочлены от оператора. Коммутирование многочленов от одного оператора.}
    \begin{Definition}
      \[V \text{ - в.п. над } K \q\q \varphi \in \End(V)\]
      \[h = a_0 + a_1 t + .... a_m t^m \in K[t]\]
      \[h(\varphi) = a_0 id + a_1 \varphi + ... + a_m \varphi^m \in \End(V)\]
      Умножение = композиция операторов
      \[A \in M_n(K)\]
      \[h(A) = a_0 E + a_1 A + ... + a_m A^m \text{ - мн-н от матрицы}\]
      \[(hg)(\varphi) = h(\varphi) \cdot g(\varphi)\]
    \end{Definition}

    \begin{proof}
        Д-ть для мономов, а затем прийти к мн-ам, пользуясь их св-ва
        \begin{enumerate}
          \item $h(\varphi) = \varphi^n,\q g(\varphi) = \varphi^m$
          \[(hg)(\varphi) = \varphi^{n+m}\]
          \item $h(\varphi) = \varphi^n,\q g(\varphi) = \sum_{k=0} b_k \varphi^k$
          \[(hg)(\varphi) = \sum_{k=0}^m b_k \varphi^k \varphi^n = \sum_{k=0}^m b_k \varphi^{n+m} \us{\text{док-ли}}{=} \sum_{k>0} b_k (hg) (\varphi)\]
        \end{enumerate}
    \end{proof}

    \begin{Proof}[тру д-во от Демченко]
        \[f(\varphi) = \sum a_i \varphi^i,\q g(\varphi) = \sum b_j \varphi^j\]
        \[f(\varphi) \cdot g(\varphi) = \sum a_i \varphi^i (\sum b_j \varphi^j) = \sum a_i b_j \varphi^{i+j}\]
        И наоборот то же самое
        \[\varphi^i = \ub{i}{\varphi \circ ... \varphi}\]
        \[(fg) (\varphi) = \sum a_i b_j \varphi^{i+j}\]
    \end{Proof}
\end{document}
