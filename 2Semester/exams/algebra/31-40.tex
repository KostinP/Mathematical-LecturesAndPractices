\documentclass[algebra]{subfiles}

\begin{document}
\section{Факториальные кольца. Содержание многочлена над факториальным  кольцом. Содержание произведения многочленов.}
    \begin{definition}
        R - комм. с 1, $a,b \in R$
        \[\text{a и b ассоциированы ($\sim$), если $a|b$ и $b|a$}\]
    \end{definition}

    \begin{utv}
        R - комм. кольцо с 1
        \begin{enumerate}
            \item $\E \in R^*\qq b = \E a \ \Ra\ a \sim b$
            \item Если R - обл. ц. и $a \sim b$, то $\e \E \in R^*: b = \E a$
        \end{enumerate}
    \end{utv}

    \begin{proof}
        \begin{enumerate}
            \item $b = \R a\ \Ra\ \E^{-1} b = a$
            \item $\e c,d: a = b d,\q b = c a$\\
            Если a или b $=0$, то тогда и второй $=0$
            \[0 = q \cdot 0 \Ra \E = 1\]
            Пусть $a \neq 0, \q b \neq 0$
            \[a = c a d = c d a\]
            \[a (1 - cd) = 0 \text{ R - обл. ц. }\Ra 1 - cd = 0, \q cd = 1\]
            \[\Ra c, d \in R^* \q \E = c \qq b = ca\]
        \end{enumerate}
    \end{proof}

    \begin{definition}
        R - о.ц
        \[a \not \in \{0\} \cup R^*\]
        назыв неприводимым, если
        \[a = bc \ \Ra \ \begin{matrix}
           & b \in R^* \text{ и } c \sim a\\
          \text{или} & \\
          & c \in R^* \text{ и } b \sim a
        \end{matrix}\]
        (все делители a есть либо обр. элем R либо ассоц. с a)
    \end{definition}
  \begin{definition}
      О.ц. R называется \ul{факториальным кольцом}, если в нем справедлива т-ма об однозначном разложении на множ.,
      а именно, всякий ненулевой необр. элемен R есть произведение неприводимых элементов, причем это разложение ед. с точностью
      до порядка сомножителей и ассоциированности
      \[a = p_1 \cdot ... \cdot p_n = q_1 \cdot ... \cdot q_m \q\q q_i, p_i \text{ - неприв } \Ra n = m \text{ и}\]
      \[\exists \text{ биекция } \sigma \text{ на } \{1,...,n\}\]
      \[p_i = q_{\sigma(i)} \]
      \[\Z, K[x] \text{ - факт. кольца}\]
      В факториальных кольцах можно определить НОД
      \[a = \mathcal{E}_1 \prod_{i = 1}^k q_i^{k_i} \q\q b= p_1 \prod_{i = 1}^n q_i^{l1} \q\q \mathcal{E}_1, p_1 \in R^* \q q_i
      \text{ - попарно ассоц. неприв}   \]
      \[\gcd (a,b) = \prod_{i = 1}^n q_i^{\min(k_i, l_i)}  \]
      \[ab = \mathcal{E}_1p_1 \prod_{i = 1}^n q_i^{(k_i + l_i)}  \]
  \end{definition}

  \begin{definition}
      Содержание многочлена f
      \[\text{cont}(f) = \gcd(a_1, a_2, ..., a_n)\]
  \end{definition}

  \begin{Definition}
      \[f \in R[x] \text{ называется примитивным, если  cont}(f) \sim 1\]
      В факториальном кольце $\gcd(ca_0,...,ca_n) = c \gcd(a_0,...,a_n)$
      \[0 \neq f \in R[x] \q f = \cont(f) f_1 \text{ - примитивный}\]
      \[0 \neq g \in R[x] \text{. В классе асоц. с g в $K[x]$ есть прим. мн-н с коэф. из R}\]
      g домножим на НОК знаменателей коэф.
      \[g_1 \in R[x] \q g_1 = \cont(g_1) g_2 \text{ - прим.}\]
      \[g \sim g_1 \sim g_2 \text{ в $K[x]$}\]
      \[0 \neq f,\q g \in R[x]\]
      \[f = \cont(f)  f_1,\q g = \cont(g) g_1,\q f,g \text{ - примит}\]
      \[f \sim g \text{ в $R[x]$} \lra \cont(f) \sim \cont(g) \text{ в R и $f_1 \sim g_1$ в $R[x]$}\]
  \end{Definition}

  \begin{lemma}[Гаусса]
      $f, g \in R[x]$
      \[\text{cont}(f\cdot g) = \text{cont}(f) \cdot \text{cont}(g)\]
  \end{lemma}

  \begin{lemma}[Гаусса, частный случай]
      $f, g \in R[x]$
      \[\text{f, g - прим. $\Ra$ $f \cdot g$ - прим.}\]
  \end{lemma}

  \begin{proof}
      Надо д-ть, что $\cont(f g)$ не делится на никакой неприводимый из R
      \[q \in R \text{ непр.}\]
      \[f = a_0 + a_1 x + ... + a_n x^n\]
      \[g = b_0 + b_1 x + ... + b_m x^m\]
      f примит. $\Ra$ q не может делить все коэф-ты f
      \[\e i: q \neq | a_i,\ q | a_{i+1},\ ...,\ q | a_n\]
      \[\text{для g } \e j: q \not |  b_j,\ q | b_{j+1},\ ...,\ q | b_m\]
      \[f g = c_0 + ... + c_{n+m} x^{n+m}\]
      \[c_{i+j} = a_i b_j (q) \qq c_{i+j} = \ob{\not \devided q}{a_i b_j} + (\ub{\devided q}{...})\]
      \[q \not | c_{i+j}\]
      \[\text{но } q | c_{i + j + 1},\ ...,\ c_{n+m} \Ra q \not | \cont(fg)\]
  \end{proof}


  \section{Теорема Гаусса о факториальности кольца многочленов над факториальным кольцом. Факториальность колец $K[x_1, ..., x_n], \Z[x_1, ..., x_n]$}

    \begin{Theorem}
        \[R \text{ - факториальное кольцо } \Ra R[x] \text{ - факториальное}\]
    \end{Theorem}

    \begin{proof}
        \begin{enumerate}
            \item Существование
            \[f = 0 \q f \in R[x]\]
            \[f \in K[x] \Ra f = \frac{a}{b} \prod_{i=1}^n g_i \q g_i\text{ - непр. над K}\q g_i \in K[x],\q \frac{a}{b} \in K\]
            \[\e h_i \in R[x] \q h_i \sim g_i \text{ в $K[x]$, $h_i$ - прим.}\]
            \[g_i = \frac{a_i}{b_i} h_i \qq a_i, b_i \in R \qq b b_1 \cdot ... \cdot b_n f = a a_1 \cdot ... \cdot a_n \cdot \ub{\text{примит. по лемме Гаусса}}{h_i \cdot ... \cdot h_n}\]
            \[\cont(a a_1 \cdot ... \cdot a_n \cdot h_i \cdot ... \cdot h_n) \sim a \cdot a_1 \cdot ... \cdot a_n\]
            \[\cont(b b_1 \cdot ... \cdot b_n f) \sim b - b_1 \cdot ... \cdot b_n \cont(f)\]
            \[a - a_1 \cdot ... \cdot a_n \devided b_1 \cdot b_1 \cdot ... \cdot b_n\]
            \[f = \cont(f) \cdot f_1 = \frac{a \cdot a_1 \cdot ... \cdot a_n}{b \cdot b_1 \cdot ... \cdot b_n} \cdot h_1 \cdot ... \cdot h_n\]
            \[\Ra f = \cont(f) \cdot h_1 \cdot ... \cdot h_n\]
            \[h_i \text{ - неприв. над K, примит $\Ra$ неприв. над R}\]
            \[\cont(f) \ni R \text{ раскл. в произв. неприв. в R}\]
            \[(a - a_1 \cdot ... \cdot a_n = b \cdot b_1 \cdot ... \cdot b_n - c\q c \in R \text{ т.к. обл. цел-ти,})\]
            \[\text{то на нулевой множитель можем сократить}\]
            \[\text{(f раскл. на неприв. в $R[x]$)}\]
            \item Единственность
            \[f = \prod_{i=1}^n P_i \cdot \prod_{j=1}^m g_j = \prod_{i=1}^{n'} p_i' \cdot \prod_{j=1}^{m'} g_j'\]
            \[p_i, p_i'\text{ - неприв. в R; $g_j, g_j'$ - неприв в $R[x]$}\]
            \[\text{В частности } \cont(g_j) \sim 1,\q \cont(g_j') \sim 1\]
            \[\text{По лемме Гаусса $\prod g_j$ и $\prod g_j'$ - примит.}\]
            \[\prod_{i=1}^n p_i \sim \prod_{i=1}^{n'} p_i' \text{ т.к. R - факт} \Ra n=n' \text{ и после перенумерации } p_i \sim p_i'\]
            \[\text{Многочлен из $R[x]$ приводим в $K[x] \Ra$ приводим и в $R[x]$}\]
            \[\text{Неприв. в $R[x] \Ra $ неприв. в $K[x]$}\]
            \[g_i,\ g_i' \text{ неприв. в }K[x] \Ra m=m' \text{ (т.к. $K[x]$ - факт.)}\]
            \[\text{после перенумерации $g \sim g'$ в K[x]}\]
            \[\e a,b \in R \qq g_i = \frac{a}{b} g_i'\]
            \[b g_i = a g_i' \qq a = b \E \q \E \in R^*\]
            \[b \sim \cont(bg) = \cont(ag') \sim a\]
            \[g_i = \E g'_i,\q \E \in R^* \Ra g_i \sim g_i'\]
        \end{enumerate}
    \end{proof}

    \begin{Consequence}
        \[\Z[x_1, ..., x_n], K[x_1, ..., x_n] \text{ - факториальны}\]
    \end{Consequence}


  \section{Неприводимость над $\Q $ и над $\Z$. Методы доказательства неприводимости многочленов с целыми коэффициентами (редукция по одному или нескольким простым модулям).}

    \[f \in \Q[x]\]
    \[\text{Хотим доказать, что } f \text{ неприв над } \Q\]
    \[\text{Не умоляя общности } f \in \Z[x] \text{ (можно домножить на знаменатель)}\]
    \[\text{cont}(f) = 1 \q \text{коэфф. в совокупности вз. просты}\]
    Идея:
    \[f = a_0 + ... + a_n x^n\]
    \[p \text{ - простое } p \nmid a_n\]
    \[\Z[x] \to \Z_{/p}\Z[x] \]
    каждый коэфф. заменяем на соотв. вычет
    \[f \to \ol{f} = [a_0] + ... + [a_n] \cdot x^n\]
    \[\text{Если } p \nmid a_n \q \deg(\ol{f}) = \deg f\]
    \[\text{Если } f \text{ приводим над } \Q \text{, то по т. Гаусса}\]
    \[f = gh \q g, h \in \Z[x]\]
    \[\deg g, \deg h < \deg f\]
    \[\ol{f} = \ol{g} \cdot \ol{h}\]
    Если $p$ не делит страш. коэфф $f$, то $p \nmid$ страш. коэфф. $g$ и $h$
    \[\deg \ol{g} = \deg g \q \text{и} \q \deg{\ol{h}} = \deg h\]
    Тогда приводимость $f$ влечет приводимость $\ol{f}$

    \begin{Hypothesis}
        \[\text{Если } p \nmid a_n \q f=a_0 + ... + a_n x^n \q\q \text{cont } f = 1\]
        \[\text{и } \ol{f} \text{ - неприводим над } \Z_{/p}\Z \text{, то } f \text{ неприводим над } \Z (\Ra \text{ и над } \Q) \]
    \end{Hypothesis}

    \begin{examples}
        \begin{enumerate}
            \item $f(x) = x^4 + x^3 + 2x^2 - 4x + 1$
            \[p = 2 \qq \ol{f} = x^4 + x^3 + 1\]
            \[\ol{f}(0) \os{\mod 2}{\neq} 0 \q \text{и} \q \ol{f}(1) \os{\mod 2}{\neq} 0\]
            Значит не делится на линейные\\
            Неприводимые степени 2 над $\Z/2\Z$:
            \[x^2 + \alpha x + \beta \qq \alpha, \beta \in \Z/2\Z\]
            \[\beta \neq 0 \qq \text{(иначе $[0]$ - корень)} \Ra \beta = 1\]
            \[1 + \alpha + \beta \neq 0 \qq \text{(иначе $[1]$ - корень)} \Ra \alpha = 1\]
            \[(x^2 + x + 1)^2 = x^4 + x^2 + 1 \neq \ol{f}\]
            Значит $\ol{f}$ не явл. произведением неприводимых степени $\leq 2$
            \[\deg \ol{f} = 4 \Ra \ol{f} \text{ - непр. над }\Z/2\Z \Ra f \text{ - непр. над }\Z\]
            \item $f(x) = x^4 - x^2 - 2x - 3$
            \[\mod 2 : \ol{f} = x^4 + x^2 + 1 = (x^2 + x + 1)^2\]
            \[\mod 3 : \ol{f} = x^4 - x^2 + x = (x^3 - x + 1)x\]
            Разложили на произведение неприводимых разных степеней
            \[f = gh \Ra \forall p \q \ol{f} = \ol{g} \ol{h} \q \text{(старш. к. 1)}\]
            Из $\mod 2 \Ra$ у f нет линейных сомножителей\\
            Из $\mod 3 \Ra$ у f нет квадратичных сомножителей
            \[\deg f = 4 \Ra \text{неприводим}\]
        \end{enumerate}
    \end{examples}


  \section{Критерий неприводимости Эйзенштейна.}

      \begin{Theorem}
          \[f \in \Z[x] \q f = a_0 + a_1x + ... + a_n x^n \q \text{ cont}(f) = 1\]
          \[p \text{ - простое}\]
          \[\begin{matrix}
              \text{Если } &* p \nmid a_n\\
                     &* p \mid a_i\\
                     &* p^2 \nmid a_0
            \end{matrix} \q i=0, ..., n - 1 \text{, то } f \text{ неприводим над } \Z (\Ra \text{ и над } \Q) \]
      \end{Theorem}

      \begin{Proof}
          \[\sqsupset f = gh \q\q g, h \in \Z[x] \q\q \deg g, \deg h < n\]
          \[\ol{f} = \ol{g} \cdot \ol{h}\]
          \[\ol{f} = [a_n]x^n\]
          \[\ol{g} \sim x^m \q \ol{h} \sim x^{n - m} \q 0 < m < n\]
          \[g = b_m x^m + ... + b_0 \q \q b_m \not \vdots \  p, \q b_{m - 1}, ... , b_0 \ \vdots \ p \]
          \[h = c_{n - m}x^{n - m} + ... + c_0  \]
          \[c_{n - m} \ \not \vdots \ p \q\q c_{n-m}, ..., c_0 \ \vdots \ p\]
          \[\text{по усл. } \us{\us{p^2}{\not \dotsb}}{a_0} = \us{\us{p}{\dotsb}}{b_0} \cdot \us{\us{p}{\dotsb}}{c_0}
          \text{ - противоречие}\]
      \end{Proof}

      \begin{examples}
          \begin{enumerate}
              \item $f(x) = x^5 + 2x^4 + 4x^3 - 6x^2 + 8x + 10$
              \[p = 2 \qq \text{f - непр.}\]
              \item $f(x) = x^{p-1} + x^{p-2} + ... + x + 1$, p - простое, $p \neq 2$\\
              \[f(x) = \frac{x^p - 1}{x - 1} \os{y = x - 1}{=} \frac{(y+1)^p - 1}{y + 1 - 1} = y^{p-1} + C_p^1 y^{p-1} + ... + C^{p-1}_p y + C^{p-1}_p\]
              \[0 < k < p \qq C^k_p = \frac{p!}{k! (p-k)!} \devided p \qq C^{p-1}_p = p \not \devided p^2 \Ra f(x) \text{ - непр.}\]
          \end{enumerate}
      \end{examples}


  \section{Рациональные корни многочлена с целыми коэффициентами.}

  \begin{Theorem}
      \[f \in \Z[x]\]
      \[f = a_0 + ... + a_n x^n \qq a_i \in \Z\]
      \[a_n \neq 0 \q a_0 \neq 0\]
      \[\text{Если некор. дробь } \frac{p}{q} \text{ - корень } f \text{, то } q \mid a_n ; \q p \mid a_0\]
  \end{Theorem}

  \begin{Proof}
      \[f \Br{\frac{p}{q}} = 0\]
      \[a_0 + a_1 \frac{p}{q} + ... + a_n \frac{p^n}{q^n} = 0 \qq \bigg| \cdot q^n\]
      \[q^n a_0 + a_1 pq^{n-1} + ... + a_{n - 1} p^{n-1}q  + a_n p^n = 0\]
      \[q^n a_0 + a_1 pq^{n-1} + ... + a_{n - 1} p^{n-1}q  = - a_n p^n \]
      \[(p, q) = 1 \qq (q, p^n) = 1 \Ra q \mid a_n\]
      \[\text{Аналогично } p \mid a_0\]
  \end{Proof}

  \section{Верхняя оценка модуля корня многочлена с комплексными коэффициентами.}

    \begin{Theorem}
        \[f = a_0 + a_1 x + ... + a_n x^n \in \CC [x] \q a_n \neq 0\]
        \[M = \us{i = 0, ..., n}{\max} \left\{\frac{\abs{a_i}}{\abs{a_n}}\right\}\]
        Тогда все корни многочлена $f$ лежат в круге $\abs{z} < M + 1$
    \end{Theorem}

    \begin{proof}
        Возьмем $z :  \q \abs{z} \geq M + 1$, покажем, что $z$  не корень $f$
        \[z^n \frac{a_n}{a_n} + z^{n - 1}  \frac{a_{n - 1} }{a_n} + ... + \frac{a_0}{a_n} \neq 0\]
        \[\abs{\frac{a_{n - 1} }{a_n} z^{n - 1} + ... + \frac{a_1}{a_n} z  + \frac{a_0}{a_n}} \leq M(\abs{z}^{n - 1} + ... + \abs{z} + 1)
        = M \frac{\abs{z}^n - 1}{\abs{z} - 1} \leq\]
        \[\leq M \frac{\abs{z}^n - 1}{M} < \abs{z}^n\]
        Нам нужно было бы получить $\abs{...} = \abs{z}^n$, получилось, что в первом выражении мы 0 не получим
    \end{proof}

    \section{Симметрические функции. Коэффициенты многочлена из C[x] как симметрические функции корней.}

    \begin{definition}
        $f \in K[u_1, ..., u_n]$ - симметрическая функция, если $f$ не меняется при любой перестановке переменных
        \begin{Example}
            \[S_0 = 1 \qq S_1 = u_1 + ... + u_n\]
            \[S_k = \sum_{i_1 < i_k}^n u_{i_1} \cdot ... \cdot u_{i_k}\]
            \[S_n = u_1 \cdot ... \cdot u_n\]
            $S_1,...,S_n$ - базисные симметрические функции
        \end{Example}
    \end{definition}

    \begin{Theorem}[Виета]
        \[f(x) = c_0 + ... + c_n x^n\]
        \[\letus f(x) = c_n \prod_{i=1}^n (x - x_i)^{n-1}\]
        \[\text{Тогда } \frac{c_{n-i}}{c_n} = (-1)^i S_i (x_1,...,x_n)\]
    \end{Theorem}

    \begin{Proof}
        \[\frac{c_0}{c_n} + ... + \frac{c_{n-1}}{c_n}x^{n-1} + x^n = \prod_{i=1}^n (x - x_i)\]
        \[\frac{c_{n-i}}{c_n} = \sum_{j_1 < ... < j_k} (-x_{j_1}) \cdot ... \cdot (-x_{j_i}) = (-1)^j S_i(x_1,...,x_n)\]
    \end{Proof}

    \section{Алгоритм разложения на неприводимые множители многочлена с целыми коэффициентами.}

    \begin{Alg}
        \[f \in \Q[x],\q \deg f = n\]
        \[\text{Н.у.о., } f \in \Z[x] \text{ и примитивный}\]
        По теореме Гаусса, если f раскл. на множетели в $\Q[x]$, то f раскл. на множетели и в $\Z[x]$
        \[f = gh \q g,h \in \Z[x]\]
        \[1 \leq \deg g \leq \deg h \leq h - 1\]
        \[f(x) = a_n x^n + ... a_0,\q a_n \neq 0,\q a_n \in \Z\]
        \[g(x) = b_m x^m + ...\]
        \[h(x) = c_{n-m} x^{n - m} + ...\]
        \[b_m \cdot c_{n-m} = a_n \text{ старшие коэффициенты g и h должны быть делителями $a_n$}\]
        \[\text{в частности: } |b_m| \leq a_n \qq |c_{n-m}| \leq |a_m|\]
        \[f,g,h \in \C[x]\]
        \[\text{z - корень f} \q |z| < \max_{i = 0,...,n-1} \Br{\frac{a_i}{a_n}} + 1\]
        \[\text{Корни g,h - корени }f = M + 1\]
        \[\text{Пусть $x_1,...,x_n$ - корни g}\]
        \[\abs{\frac{b_{m-i}}{b_m}} = \abs{g_i(x_{i_1},...,x_{i_n})} \leq C^i_m (M + 1)^i\]
        \[\text{коэф. g - огр.}\]
        Аналогично для h перебираем m от 1 до $\left[\frac{n}{2}\right]$
        \[\abs{b_{m-i}} \leq |a_n| C_m^i (m+1)^i\]
        Оцениваем коэф. g. Перебираем все g с коэф., удовл. этим оценкам и смотрим, делит ли g f (является ли g делителем f)
    \end{Alg}

  \section{Линейные отображения векторных пространств. Линейное отображение полностью задается своими значениями на базисных векторах.}
      \begin{Definition}
          \[K \text{ - поле,} \q V \text{ - в.п. над K},\q f: U \to V\]
          \[f \text{ - линейное, если}\q \forall u_1, u_2 \in U \q \forall \alpha_1, \alpha_2 \in K\]
          \[(*) \q f(\alpha_1 u_1 + \alpha_2 u_2) = \alpha_1 f(u_1) + \alpha_2 f(u_2)\]
          \[(*) \lra
          \begin{matrix}
              1) & \forall u_1, u_2 \in U & f(u_1 + u_2) = f(u_1) + f(u_2)\\
              2) & \forall u \in U & \forall \alpha \in K \q f(\alpha u) = \alpha f(u)
          \end{matrix}\]
          лин. отобр $\equiv$ гомеоморфизм векторных пр-в
      \end{Definition}
      \begin{Theorem} [св-ва]
          \[f \text{ - лин. отобр. }\]
          \[f(0_u) = 0_v\]
          \[f(-u) = - f(u)\]
      \end{Theorem}

      \begin{examples}
        \begin{enumerate}
          \item $U = K^m, \q V = K^n, \qq A \in M(n,\ m,\ K)$
          \[U \ra V\]
          \[x \mapsto Ax \text{ - лин. отображение}\]
          \item $K[x] \to K[x]$
          \[f \to f'\]
          \item $U \in C([0,1] \ra \R), \q V = \R$
          \[f \ra \int_0^1 f(t) dt\]
        \end{enumerate}

      \end{examples}

      \begin{Utv}
        \[U \text{ - в.п } \q\q \{u_i\}_{i \in I} \text{ - базис } U \]
        Достаточно задать лин. отобр. на базисных векторах
        \[f \text{ - лин. отобр } \q f: U \to V\]
        \[u \in U \q u = \sum \alpha_i u_i\]
        \[f(u) = f(\sum \alpha_i u_i) = \us{\alpha_i \neq 0}{f(\sum \alpha_i u_i)} = \us{\alpha_i \neq 0}{\sum \alpha_i f(u_i)}\]
      \end{Utv}

  \section{Сумма линейных отображений, умножение на скаляр. Пространство линейных отображений.}
    \begin{utv}
        $\text{Пусть задано отобр. } \q h : \us{\text{базис}}{\{u_i\}_{i \in I} }\to  V$
        \[\exists \text{ единств. лин. отобр. } f : U \to V \text{, такое что } \forall i \in I \q f(u_i) = h(u_i)\]
    \end{utv}

    \begin{Proof}
        \[u = \sum_{\text{п.в.} \alpha_i = 0} \alpha_i h(u_i) \text{ - такое разл. единств (т.к. $u_i$ - базис)}\]
        \[f(u) = \sum_i \alpha_i h(u_i)\]
        \[f: V \ra U \text{ построили}\]
        \[\text{По постр. } f(u_i) = h(u_i)\]
        Проверим, что f - линейное
        \[u,w \in U \qq u = \sum \alpha_i u_i \qq \sum \beta_i u_i\]
        \[u + w = \sum(\alpha_i + \beta_i) u_i\]
        \[f(u) = \sum \alpha_i h(u_i)\]
        \[f(w) = \sum \beta_i h(u_i)\]
        \[f(u + w) = \sum (\alpha_i + \beta_i) h(u_i)\]
        \[f(u + w) = f(u) + f(w)\]
        \[f(\alpha u) = \sum \alpha \alpha_i h(u_i)\]
        \[f(\alpha u) = \alpha f(u)\]
        В описанной ситуации говорим, что h продолжается до f по линейности\\
        $u_1,...,u_m$ - базис $U$, $v_1,...,v_m$ - базис $V$
        \begin{enumerate}
          \item $f: U \ra V$
          \[\alpha \in K\]
          \[(\alpha f) U \ra V\]
          \[(\alpha f)(u) = \alpha f(u)\]
          \[\alpha f \text{ - лин. отобр.}\]
          \[\alpha f(\gamma_1 u_1 + \gamma_2 u_2) = \alpha (\gamma_1 f(u_1) + \gamma_2 f(u_2))\]
          \[= \gamma_1 \alpha f(u_1) + \gamma_2 \alpha f(u_2) = \gamma_1 (\alpha f) (u_1) + \gamma_2 (\alpha f) (u_2)\]
          \[[\alpha f]_{\{u_j\}\ \{v_i\}} = \alpha[f]_{\{u_j\}\ \{v_i\}}\]
          \item $f,g: U \ra V$ $f,g$ - лин.
          \[(f+g)(u) := f(u) + g(u)\]
          $f+g$ - лин. отображение
          \[(f+g)(u_i) = f(u_i) + g(u_i)\]
          \[[f + g]_{\{u_j\}\ \{v_i\}} = [f]_{\{u_j\}\ \{v_i\}} + [g]_{\{u_j\}\ \{v_i\}}\]
        \end{enumerate}
    \end{Proof}

    \begin{Definition}
        \[U, V \text{ - в.п. над } K\]
        \[L(U, V) \text{ - мн-во всех линейных отобр. из } U \text{ в } V\]
        \[+: L(U, V) + L(U, V) \to L(U, V)\]
        \[*: K \times L(U, V) \to L(U, V)\]
    \end{Definition}

    \begin{Theorem}
        \[L(U, V) \text{ - в.п. над } K\]
    \end{Theorem}

    \begin{proof}
        \begin{enumerate}
          \item $(L(U, V) + L(U, V)) + L(U, V) = L(U, V) + (L(U,V) + L(U,V))$
          \item $0 = id \q L(U,V) + id = id + L(U,V) = L(U,V)$
          \item $-L(U,V) + L(U,V) = 0$
          \item $L_1(U,V) L_2(U,V) = L_2(U,V) + L_1(U,V)$
          И так далее
        \end{enumerate}
    \end{proof}
\end{document}
