\documentclass[12pt, fleqn]{article}

\usepackage{../../../template/template}
\usepackage{../../../template/fortickets}

\begin{document}
%\tableofcontents

\section{Базис векторного пространства. Четыре эквивалентных переформулировки определения базиса.}
\begin{definition}
        Пусть $V$ - векторное пространство над полем $K$, тогда:

        \begin{enumerate}
          \item $\{v_\alpha\}_{\alpha \in A}$ - линейно независима, если $\us{\text{почти все $c_\alpha=0$}}{\sum c_\alpha v_\alpha}  =0 \Ra$ все $c_\alpha=0$
          \item $\{v_\alpha\}$ -семейство образующих $V$, если любой $v \in V$ - есть линейная комбинация $\{v_\alpha\}$, если любой $v \in V$ есть $\us{\text{почти все $c_\alpha=0$}}{\sum c_\alpha v_\alpha}$
        \end{enumerate}
    \end{definition}

    \begin{definition}
        Базис - лин. незав. сем-во образующих $(\ol{0} \not \in \text{ базису})$
    \end{definition}

    \begin{definition}
        Линейно независимое семейство векторов называется максимальным (по включению), если при добавлении $\forall$ вектора новое семейство ЛЗ
    \end{definition}

    \begin{definition}
        Сем-во образующих называется минимальным по включению, если при выбрасывании $\forall$ вектора сем-во не является семейством образующих
    \end{definition}

    \begin{theorem} [Равносильные утверждения]
        $V$ - в.п. над K, $\{v_\alpha\}_{\alpha \in A}$, следующие условия равносильны:
        \begin{enumerate}
            \item $\{v_\alpha\}$ - базис V над K
            \item $\{v_\alpha\}$ - max ЛН семейство
            \item $\{v_\alpha\}$ - min семейство образующих
            \item $\forall v \in V $ единственным образом представим в виде лин. комбинации векторов из  $\{v_\alpha\}$
        \end{enumerate}
    \end{theorem}

    \begin{proof}
        $(1 \Ra 2)$:

        Базис $\Ra$ ЛН.

        Добавим $v \in V$ к $\{v_\alpha\}$: $v=\us{\text{Почти все $c_\alpha = 0$}}{\sum c_\alpha v_\alpha}$,

        но тогда $-v+\us{\text{Почти все $c_\alpha = 0$}}{\sum c_\alpha v_\alpha = 0}$ $\Ra$ новое семейство ЛЗ $\Ra$ $\{v_\alpha\}$ - ЛЗ
        \\ \\
        $(2 \Ra 1)$:

        $\{v_\alpha\}$ - max ЛН

        $\Ra$ при добавлении $\forall v \in V\ \exists c \neq 0:$ $0 = c v + \us{\text{Почти все $c_\alpha = 0$}}{\sum c_\alpha v_\alpha}$

        $\Ra$ $v=\us{\text{Почти все $c_\alpha = 0$}}{\sum (c^{-1} c_\alpha) v_\alpha}$ в силу произвольности $v$, $\{v_\alpha\}$ - базис.
        \\ \\
        $(1 \Ra 3)$:

        $\{v_\alpha\}$ - базис $\Ra$ семейство образующих. Пусть $v \in \{v_\alpha\}$.

        Если бы $\{v_\alpha\}$ без v было бы семейством образующих,

        то $v=\us{\text{п.в. $c_\alpha = 0$, $v \notin \{v_\alpha\}$}}{\sum c_\alpha v_\alpha}$, но тогда $0=-v+\us{\text{п.в. $c_\alpha = 0$, $v \notin \{v_\alpha\}$}}{\sum c_\alpha v_\alpha}$
        \\ \\
        $(3 \Ra 1)$:

        $\{v_\alpha\}$ - min семейство образующих, нужно проверить что ЛН.

        Пусть ЛЗ, тогда $\us{\text{п.в. $c_\alpha = 0$}}{\sum c_\alpha v_\alpha = 0}$ $\Ra$ $c_{\alpha_0} \neq 0$.

        Но тогда $v_{\alpha_0} = \us{\text{п.в. $c_\alpha = 0$}}{\sum (c_{\alpha_0}^{-1} c_\alpha) v_\alpha}$, противоречение с min сем-ом обр.
        \\ \\
        $(4 \Ra 1)$:

        4 формально сильнее
        \\ \\
        $(1 \Ra 4)$:

        $v=\us{\text{п.в. $c_\alpha=0$}}{\sum c_\alpha v_\alpha}=\us{\text{п.в. $c'_\alpha=0$}}{\sum c'_\alpha v_\alpha} \Ra 0 = \us{\text{п.в. $c_\alpha-c'_\alpha=0$}}{\sum c_\alpha v_\alpha}$

        В силу единственности разложения нуля получаем $c_\alpha=c'_\alpha$ $\forall \alpha$
    \end{proof}

\section{Конечномерные пространства. Всякое линейно независимое семейство конечномерного пространства можно дополнить
    до базиса. Существование базиса конечномерного пространства.}

    \begin{definition}
        V - в.п. над полем K, V называется конечномерным, если в V есть конечное сем-во образующих.
    \end{definition}

    \begin{example}
        $\mathds{C}$ - ВП не являющееся конечномерным.

        $V=\{(c_1,c_2,...)$, не все $c_i=0\}$

        Сложение, умножение на скаляр - некоординатно.

        V - ВП над $\mathds{C}$, пусть $v_1,...,v_k \in V$, $v_i=(c_{i_1}, c_{i_2},...)$, почти все $c_{i_j}=0$

        $\exists N:$ $\forall j > N$, $\forall i$ $c_{i_j}=0$
    \end{example}

    \begin{theorem}
        Всякое линейно независимое сем-во конечномерного пространства можно дополнить до базиса.
    \end{theorem}

    \begin{proof}
        1) $\{v_\alpha\}$ - ЛН $\Rightarrow$ либо порождает V, либо можно дополнить с сохранением условия ЛН.

        То есть линейная оболочка $\{\sum c_\alpha v_\alpha\}$ либо равна $\forall v \in V$, тогда $\{v_\alpha\}$ - семейство образующих V, либо неравна, тогда $v$ и $\{v_\alpha\}$ ЛН и можно им дополнить
        \\
        2) V - конечномерно, пусть $u_1,u_2,...,u_m$ - конечное семейство образующих V, тогда если $v_1,v_2,...,v_n$ - его ЛК и m > n, то $\{u_\alpha\}$ - ЛЗ $\Rightarrow$ всякое ЛН семейство из V содержит $\leqslant m$ векторов. Значит добавление векторов оборвётся.
    \end{proof}


    \begin{consequence}
        Во всяком конечномерном в.п. есть базис.
    \end{consequence}

    \begin{proof}
        Пустое сем-во ЛН\\
        Дополним до базиса
    \end{proof}



\section{Всякое семейство образующих конечномерного пространства содержит базис. Существование базиса конечномерного пространства.}
    \begin{theorem}
        V - конечномерное в.п. над K

        Всякое конечномерное сем-во образующих содержит базис.
    \end{theorem}

    \begin{proof}
      Пусть $v_1,v_2,...,v_k$ - семейство образующих V. Если оно ЛН, то базис.

      Если ЛЗ, то $\exists i$: $v_i$ - линейная комбинация остальных

      $\Rightarrow$ $\{v_1,...,v_{i-1},v_{i+1},...,v_k\}$ - семейство образующих, а т.к. семейство конечно, то процесс выкидывания $"$оборвётся$"$ и на каком-то шаге получится ЛН зависимое семейство, то есть базис.
    \end{proof}

    \begin{theorem}
        Во всяком конечномерном в.п. есть базис
    \end{theorem}

    \begin{proof}
        Возьмём конечное семейство образующих, по теореме оно содержит базис.
    \end{proof}


\section{Подпространства векторного пространства. Подпространство конечномерного пространства конечномерно.}
     \begin{definition}
     	V - в.п над полем K, $U \neq \varnothing$ - подпр-во V (записывается $U \subseteq V$),

      если U - само явл. в.п. над K
     \end{definition}
	 \begin{Hypothesis}[1]
	 		\[\varnothing \neq U \subseteq V \q U \text{ - подпр-во } V \rla \]
			\begin{enumerate}
				\item $\forall u_1, u_2 \in U: \q u_1 + u_2 \in U$
				\item $\forall u \in U, \ \forall a \in K \q au \in K$
			\end{enumerate}
	 \end{Hypothesis}

   \begin{proof}
      $(\Ra)$

      По определению ВП.\\
      $(\La)$

      Операции сложения и умножения на скаляр определены на U. Осталось проверить аксиомы ВП:
      \begin{enumerate}
        \item $\forall x, y \in U$ $x+y=y+x$ по опр. сложения
        \item $\forall x, y, z \in U$ $(x+y)+z=z+(y+z)$, аналогично
        \item Т.к. $U \neq \varnothing$, то $\exists u \in U$. $0_V=u+(-1)u$.

        По условию теоремы следует, что $0 \in U$, так как $u,\ (-1)u,\ u+(-1)u \in U$. $\forall u \in U$: $0+u=u$, $u+0=u$
        \item $\forall u \in U$ $\exists -u=(-1)u$, $u-u=0$
      \end{enumerate}
      Остальные 4 аналогично.
  \end{proof}

	 \begin{hypothesis}[2]
	 		V - конечномерное в.п над K
			\[U \subseteq V \Ra U \text{ - конечномерное}\]
	 \end{hypothesis}

	 \begin{proof}
     $\{\}$ - пустое семейство.

    Будем добавлять к нему вектора из U с сохранением ЛН, пока не получим семейство образующих. Причем в V есть конечное семейство ЛН образующих.

    Значит так как векторов в семействе U не может быть больше, чем в семействе V, то там тоже их конечное количество.
	 \end{proof}


\section{Теорема о мощности базиса конечномерного пространства. Размерность пространства.}
		\begin{theorem}
				V - конечномерное пространство
				\[\{v_1, ..., v_n\}, \{u_1, ..., u_m\} \text{ - базисы } V \text{ над } K\]
				\[\Ra n = m\]
		\end{theorem}

		\begin{proof}
				$u_1, ..., u_m$ - лин.комб $v_1, ..., v_n$
				\[\Ra \text{по т. о линейной зависимости лин. комбинаций}\]
				\[m \leq n \text{ и аналогично } m \geq n \Ra m = n\]
		\end{proof}

		\begin{definition}
				Размерноесть конечномерного пространства - размерность векторов в его базисе.\\
        Обозначаем как $\dim_K V = \dim V$\\
        Если пространство не конечно, то пишем $\dim V = \infty$
		\end{definition}


\section{Координаты вектора в данном базисе. Матрица перехода от одного базиса к другомую. Преобразование координат при замене базиса. Матрица преобразования координат.}
        \begin{theorem}
        Пусть V - ВП над K, $n = dim_K V < \infty$, $v_1, ..., v_n$ - базис V над K.

        Тогда если $v \in V$, то $\exists!$ набор $\alpha_1, ..., \alpha_n \in K:$ $v=\alpha_1 v_1+...+\alpha_n v_n$
        \end{theorem}

        \begin{definition}
        $\alpha_1,...,\alpha_n$ будем называть координатами v в базисе $\{v_1,...,v_n\}$ и записывать как
        $\begin{pmatrix}
        \alpha_1\\
        ...\\
        \alpha_n
        \end{pmatrix}$, причем $v=
        \begin{pmatrix} \alpha_1&...&\alpha_n \end{pmatrix}
        \begin{pmatrix}
        v_1\\
        ...\\
        v_n
        \end{pmatrix}$
        \end{definition}

        \begin{Proof}

        \end{Proof}
				\[\text{Пусть } v_1, ..., v_n \text{ - базис V}\]
				\[v_1', ..., v_n' \text{ - другой базис } V\]
				\[v_i' = c_{1i}v_1 + ... + c_{ni}v_n\]
				\[c = \begin{pmatrix}
					c_{11} & c_{21} & ... & c_{n1}\\
					c_{12} & \ddots \\
						   &        &  \ddots  &\\
					c_{1n} & & & c_{nn}
				\end{pmatrix} \text{ - матрица перехода от базиса } \]
				\[(v_1, ..., v_n) \text{ к базису } (v_1', ..., v_n') \]
				\[\begin{pmatrix}
					v_1'\\
					\vdots\\
					v_n'
				\end{pmatrix} = C
				\begin{pmatrix}
					v_1\\
					\vdots\\
					v_n
				\end{pmatrix} \q\q\q
				\begin{pmatrix}
					v_1\\
					\vdots\\
					v_n
				\end{pmatrix} = B
				\begin{pmatrix}
					v_1'\\
					\vdots\\
					v_n'
				\end{pmatrix}\]
				\[v_i = b_{1i}v_1' + ... b_{ni}v_n'\]
				\[B = \begin{pmatrix}
					b_{11} & \q & b_{n1}\\
					b_{12} &\\
					\\
					b_{1n}& & b_{nn}
				\end{pmatrix} \text{ - матрица перехода от базиса} (v_1', ..., v_n')\]
				\[\text{к базису } (v_1, ..., v_n)\]
				\[v = a_1v_1 + ... + a_n v_n\]
				\[v = a_1'v_1' + ... + a_n'v_n'\]
				\[C \text{ - матрица перехода от } (v_1, ..., v_n) \text{ к } (v_1', ..., v_n')  \]
				\[C^T = \begin{pmatrix}
					c_{11} & c_{1i} &       & c_{1n}\\
						   & \ddots\\
					c_{n1} &        & \ddots& c_{nn}
				\end{pmatrix} = D \text{ - матрица преобразования координат}\]

			\begin{theorem} [в указанных выше обозначениях]
					\[\begin{pmatrix}
						a_1\\
						\vdots\\
						a_n
					\end{pmatrix} = D
				   \begin{pmatrix}
				   		a_1'\\
						\vdots\\
						a_n'
				   \end{pmatrix}\]
			\end{theorem}
			\begin{proof}
				\[v = (a_1', ..., a_n') \begin{pmatrix}
					v_1'\\
					\vdots\\
					v_n'
				\end{pmatrix} =
				(a_1', ..., a_n') \cdot C \begin{pmatrix}
					v_1 \\
					\vdots\\
					v_n
				\end{pmatrix} \]

				\[v = (a_1, ..., a_n) \begin{pmatrix}
					v_1\\
					\vdots\\
					v_n
				\end{pmatrix} \]
			\end{proof}
			В силу единственности разложения по базису
			\[(a_1, ..., a_n) = (a_1', ..., a_n') \cdot C  \]
			\[\begin{pmatrix}
				a_1\\
				\vdots\\
				a_n
			\end{pmatrix} = C^T
			\begin{pmatrix}
				a_1'\\
				\vdots\\
				a_n'
			\end{pmatrix}\]


\section{Сумма и пересечение подпространств. Теорема о размерностях суммы и пересечения.}
  \begin{definition}
    V - ВП над K, $U_1,...,U_m \subseteq V$

    Пересечение: $\underset{i=1}{\overset{n}{\cap}} U_i = \{ v \in V \textpipe v \in U_1,...,v \in U_n\}$

    Сумма: $U_1+...+U_n=\{v \in V \textpipe \exists u_1 \in U_1,...,u_n \in U_n: v=u_1+...u_n \}$
  \end{definition}

	\begin{theorem}
			\begin{enumerate}
				\item Сумма $U_1 + ... + U_m$ является подпространством
					\[0 = 0 + ... + 0 \in U_1 + ... + U_m \Ra \text{ сумма } \neq \varnothing\]
					$\forall u, v \in U_1 + ... + U_m$:
					\[u = u_1 + u_2 + ... + u_m\]
					\[v = v_1 + v_2 + ... + v_m\]
					\[u + v = \us{\in U_1}{(u_1 + v_1) } + \us{\in U_2}{(u_2 + v_2)} + ... + \us{\in U_m}{(u_m + v_m)} \in
					U_1 + ... + U_m\]
					умножение на скаляр аналогично
				\item Пересечение является подпространством
					\[\bigcap_{i = 1}^n U_i \ni u, v \q a \in K\]
					\[\forall i \q u, v \in U_i \begin{align}
							&\q u + v \in U_i \q &u+v \in \bigcap_{i = 1}^n U_i\\
							&\q au \in U_i & au \in \bigcap_{i = 1}^n U_i
					\end{align} \]
					не пусто, т.к.:
					\[0_V \in \bigcap_{i = 1}^n U_i \Ra \bigcap_{i = 1}^n U_i \subseteq V\]
					\[\bigcap_{i=1}^n U_i \subseteq U_1 \subseteq U_1 + U_2 \supseteq U_2 \supset \bigcap_{i=1}^n U_i \]
			\end{enumerate}
	\end{theorem}
	\begin{theorem}
			$U_1, U_2 \subseteq V \q U_1, U_2 \text{ - конечномерные}$
			\[\text{Тогда } U_1 \cap U_2 \ \text{ и } \ U_1 + U_2 \text{ - конечномерны}\]
			\[\text{и } \dim(U_1 \cap U_2) + \dim(U_1 + U_2) = \dim(U_1) + \dim(U_2)\]
	\end{theorem}
	\begin{proof}
			$U_1 \cap U_2 \subseteq U_1,\q U_1 \text{ - конечномерно}$
			\[\Ra U_1 \cap U_2 \text{ - конечномерно}\]
			\[w_1, ..., w_r \text{ - базис } U_1 \cap U_2 \text{, ЛНЗ сем-во в } U_1\]
			Дополним до базиса $U_1$:
			\[w_1, ..., w_r, u_1, ..., u_s \text{ - базис } U_1\]
			Аналогично $w_1, ..., w_r$ дополним до базиса $U_2$:
			\[w_1, ..., w_r, v_1, ..., v_t \text{ - базис } U_2\]
			Проверим, что $w_1, ..., w_r, u_1, ..., u_s, v_1, ..., v_t$ - базис $U_1 + U_2$:
			\begin{enumerate}
				\item Семейство образующих
					\[z \in U_1 + U_2 \q z = z_1 + z_2 \q\q z_1 \in U_1 \ z_2 \in U_2\]
					\[z_1 = a_1w_1 + ... + a_rw_r + b_1u_1 + ... + b_su_s\]
					\[z_2 = c_1w_1 + ... + c_rw_r + d_1v_1 + ... + d_tv_t\]
					\[z = (a_1 + c_1) w_1 + ... + (a_r + c_r)w_r + b_1u_1 + ... + b_su_s + d_1v_1 + ... + d_tv_t\]
					\[\Ra w_1, ..., w_r, u_1, ..., u_s, v_1, ..., v_t \text{ - сем-во образующих}\]
				\item ЛНЗ
					\[(*) 0 = a_1w_1 + ... + a_rw_r + b_1u_1 + ... + b_su_s + c_1v_1 + ... + c_tv_t\]
					\[z = \underbrace{a_1w_1 + ... + a_2w_2 + b_1u_1 + ... + b_su_s}_{\in U_1} = \underbrace{-c_1v_1 - ... - c_tv_t}_{\in U_2}  \]
					\[z \in U_1 \cap U_2 \Ra	z = d_1w_1 + ... + d_rw_r = \]
					\[= d_1w_1 + ... + d_2w_2 + 0 \cdot u_1 + 0 \cdot u_2 + ... + 0 \cdot U_s\]
					В силу единственности разложения по базису $U_1$
					\[b_1 = b_2 = ... = b_s = 0\]
					\[\text{Из } (*) \Ra a_1w_1 + ... + a_2w_r + c_1v_1 + ... + c_tv_t = 0\]
					\[\text{т.к. } w_1, ..., w_r, v_1, ..., v_t \text{ - базис } U_2, \text{ то}\]
					\[a_1 = ... = a_r = c_1 = ... = c_t = 0\]
					\[\Ra w_1, ..., w_r, u_1, ..., u_s, v_1, ..., v_t \text{ - ЛНЗ}\]
          Знаем,
          \[\dim(U_1) = r + s\]
          \[\dim(U_2) = r + t\]
          \[\dim(U_1 \cap U_2) = r\]
          \[\dim(U_1 + U_2)= r + t + s\]
          Значит,
          \[\dim(U_1 \cap U_2) + \dim(U_1 + U_2) = \dim(U_1) + \dim(U_2)\]
			\end{enumerate}
	\end{proof}


\section{Прямая сумма подпространств. Эквивалентные переформулировки понятия прямой суммый подпротранств.}
    $V \text{ - в.п. над } K,\q U_1, ..., U_m \subseteq V$
		\begin{definition}
				$U_1 + ... + U_m \text{ назыв. прямой суммой, если любой } z \in U_1 + ... + U_m$

				едиственным образом представим в виде суммы:
				\[z = u_1 + u_2 + ... + u_m \q\q u_i \in U_i \q i=1, ..., m\]
				Обозначение: $U_1 \bigoplus U_2 \bigoplus ... \bigoplus U_m $
		\end{definition}
		\begin{remark}
				Сумма $U_1 + ... + U_m$ - прямая $\rla$
				\[\rla 0 = u_1 + ... + u_m \q u_i \in U_i\ \Ra \ u_1 = ... = u_m = 0\]
		\end{remark}
		\begin{proof}
				$(\Ra)$
        \[\text{очевидно}\]
        $(\La)$
				\[\begin{align}
						  & z \in U_1 + ... + U_m\\
						  & z = u_1 + ... + u_m = v_1 + ... + v_m\\
						  & 0 = z - z = \us{\in U_1}{(u_1 - v_1)} + ... + \us{\in U_m}{(u_m - v_m)}\\
						  & \forall i \q u_i - v_i = 0 \text{ т.е. } u_i = v_i
				\end{align}\]
		\end{proof}

		\begin{hypothesis}[1]
				Сумма $U_1 + U_2$ - прямая $\rla U_1 \cap U_2 = \{0\}$
		\end{hypothesis}
		\begin{hypothesis}[2]
				Сумма $U_1 + U_2$ - прямая $\rla$

        $\rla$ объединение базисов $U_1$ и $U_2$ - есть базис $U_1 + U_2$
		\end{hypothesis}
		\begin{hypothesis}[3]
			$U_1 + ... + U_m$ - прямая $\rla$

			$\rla \forall i = 1, ..., m \q U_i \cap (U_i + ... + U_{i-1} + U_{i + 1} + ... + U_m) = \{0\}$
		\end{hypothesis}
		\begin{hypothesis}[4]
				Сумма $U_1 + ... + U_m$ - прямая $\rla$

				$\rla$ объединение базисов $U_i \q i=1, ..., m$ - базис $U_1 + ... + U_m$
		\end{hypothesis}


\section{Построение кольца многочленов.}
	\begin{definition}
		          R - комм. кольцо с 1
		\[R[x] := \{(a_0, a_1, a_2 ...): a_i \in R \q i = 0, ...     \text{ п.в. } a_i = 0\}\]
		\[(a_0, a_1, ...), \  (b_0, b_1, ...) \in R[x]\]
    Сложение:
		\[(a_0, a_1, ...) + (b_0, b_1, ...) = (a_0 + b_0,\  a_1 + b_1, ...)\]
    Замечание:
    \[\begin{matrix}
      &\forall n > N \q a_i = 0\\
  		&\forall m > M \q b_i = 0
    \end{matrix}
		\Ra \forall i > \max(N, M) \q a_i + b_i = 0\]
    Умножение:
		\[(a_0, a_1, ...) \cdot (b_0, b_1, ...) = (c_0, c_1, ...) \]
		\[c_n = \sum_{i = 0}^n a_i b_{n - i} = a_0 b_n + a_1 b_{n-1} + ... + a_n b_0\]
    Замечание:
		\[\forall n > N \q a_n = 0\]
		\[\forall m > M \q b_m = 0\]
		\[\forall k > N + M \q c_k = \sum_{i = 0}^k a_i b_{k - i} = \sum_{i = 0}^N a_i b_{k - i} + \sum_{i = N + 1}^k a_i b_{k - i} = 0\]
		\[i \leq N \q k-i \geq k - N > N + M - N = M \]
	\end{definition}

	\begin{Theorem}
		\[(R[x], +, \cdot) - \text{комм. кольцо с 1}\]
	\end{Theorem}

  \begin{Proof}[ассоциативность умножения]
    \[A=(a_0,a_1,...),\q B=(b_0,b_1,...),\q C=(c_0,c_1,...)\]
    \[(AB)C\os{?}{=}A(BC)\]
    \[\text{Пусть }AB=D,\q BC=E,\q (AB)C=F,\q A(BC)=G\]
    \begin{multline*}
      $f_n=\sum\limits_{i=0}^n d_i c_{n-i}=\sum\limits_{i=0}^n (\sum\limits_{j=0}^i a_j b_{i-j}) c_{n-i} = \\
      \qq = \us{\text{напр. движения индекса изменилось}}{\sum\limits_{i=0}^n \sum\limits_{j=0}^i a_j b_{i-j} c_{n-i} = \sum\limits_{j=0}^n a_j (\sum\limits_{i=j}^n b_{i-j} c_{n-i})} \us{k=i-j}{=} \\
      = \sum\limits_{j=0}^n a_j (\sum\limits_{k=0}^{n-j} b_{k} c_{n-j-k}) = \sum\limits_{j=0}^n a_j e_{n-j}=g_n$
    \end{multline*}
  \end{Proof}
  \begin{upr}
    Остальное д-ть самостоятельно
  \end{upr}

	\begin{definition}
    Введем 0 и 1:
		\[0 = (0, 0, ...)\]
		\[1 = (1, 0, ...)\]
    Нетрудно проверить, что они уд-ют необходимым свойствам
		\[R[x] \supset \{(a, 0, ...); \ a \in R\} \text{ - подкольцо изоморфное R}\]
		\[(a, 0, ...) + (b, 0, ...) = (a + b, 0, ...)\]
		\[(a, 0, ...) \cdot (b, 0, ...) = (a b, 0, ...)\]
		\[(a, 0, ...) = a \text{ (обозначение)}\]
		\[x = (0, 1, 0, ...)\]
		\[x^i = (0, ..., 0, \us{i} 1, 0, ...)\]
		\[(a_0, a_1, ..., a_n, 0, ...) = (a_0, 0, ...) + (0, a_1, 0, ...) + ... + (0, ..., a_n, 0, ...) = \]
		\[= a_0 \cdot 1 + a_1 (0, 1, ...) + ... + a_n (0, ..., 1, ...) = \]
		\[= a_0 + a_1 x + a_2 x^2 + ... + a_n x^n = \sum_{i = 0}^n a_i x^i\]
	\end{definition}


\section{Степень многочлена. Свойства степени. Область целостности. Кольцо многочленов над областью целостности есть область целостности.}
	\begin{definition}
		$f = a_0 + a_1x + ... + a_nx^n \in R[x]$ \\ \\
		Наибольшее m, т.ч. $a_m \neq 0$ называется степенью f $(\deg f - degree)$\\
		$\deg 0 = -\infty$
	\end{definition}

	\begin{definition}
		Ком. кольцо R с 1 назыв. областью целостности (или кольцом без делителей 0)
		\[\text{Если } \forall a, b \in R \q \left(ab = 0 \Ra a = 0 \text{ или } b = 0 \right)\]
		\[\forall a, b \in R \left(a \neq 0 \q b \neq 0 \Ra ab \neq 0\right)\]
  \end{definition}

  \begin{examples}
    \begin{enumerate}
      \item $\Z$ - о.ц.
      \item Любое поле - о.ц
      \item $\Z_{/m} \Z$ - не всегда о.ц. \q\q $[a][b] = [m] = [0]$
    \end{enumerate}
  \end{examples}

	\begin{theorem} [Свойства степени]
		\begin{enumerate}
			\item  $\deg(f + g) \leq \max(\deg f,\ \deg g)$
			      \[\text{Если } \deg f \neq g \text{, то }  \deg(f,\ g) = \max(\deg f,\ \deg g) \]
			\item $\deg(fg) \leq \deg f + \deg g$
			      \[\text{Если } R - \text{о.ц, то } \deg(fg) = \deg f + \deg g\]
		\end{enumerate}
	\end{theorem}

	\begin{proof}
		1) $N = \deg f \q M = \deg g$
		\[f = \sum_{i = 0}^N a_i x^i \q\q g = \sum_{i = 0}^M b_i x^i\]
		\[\forall n > \max(N, M) \q a_n + b_n = 0 \Ra \deg(f + g) \leq \max(N, M)\]
		Равенства в общ. случае нет
		\[\text{Если } N = M \q a_N = -b_N \ \Ra \  a_N + b_N = 0\]
		Если $N \neq M \q \sqsupset N < M$
		\[a_M + b_M = 0 + b_M = b_M \neq 0\]
		2) $fg = \sum_{i = 0} c_i x^i \q c_i = 0 \text{  для всех  } \  i > N + M$
		\[\deg(fg) \leq N+M = \deg f + \deg g\]
		\[c_{N + M} = a_N b_M \q \text{в общем случае:}\]
		\[\text{Если R не о.ц, } a_N \neq 0 \q b_M \neq 0 \text{ то } a_N \cdot b_M \text{ м.б.} = 0\]
		\[\text{Если R - о.ц, то } a_N \neq 0 \q b_M \neq 0 \Ra c_{N+M} \neq 0\]
		\[\Ra \deg fg = \deg f + \deg g\]
	\end{proof}

	\begin{Consequence}
		\[\text{Если R - о.ц, то } R[x] - \text{о.ц} \]
  \end{Consequence}

  \begin{Proof}
    \[f, g \in R[x] \q f \neq 0 \q g \neq 0\]
		\[\deg f \geq 0 \q \deg g \geq 0\]
		\[\deg(fg) = \deg f + \deg g \geq 0\]
    \[\Ra \text{в fg есть хотя бы один ненулевой коэф.} \]
		\[\Ra fg \neq 0\]
  \end{Proof}

  \begin{Remark}
    \[\text{Если K - поле } \q K[x] \text{ - о.ц}\]
  \end{Remark}

	\begin{Remark}
		\[R \ra R[x_1] \text{ с помощью индукции сделаем вывод}\]
		\[R[x_1, x_2] = (R[x_1])[x_2]\]
		\[R[x_1, ..., x_n] = (R[x_1, ..., x_{n-1}])[x_n]\]
		\[\Ra R \text{ - о.ц} \Ra R[x_1, ..., x_n] \text{ - о.ц}\]
	\end{Remark}


\section{Теорема о делении с остатком в кольце многочленов.}
	\begin{theorem}
		$R \text{ - комм. к. с ед.},\q f, g \in R[x]$,
		\[g = a_0 + a_1 x + ... + a_n x^n, a_n \in R^* \text{ обр. элем.}\]
		$\text{Тогда } \exists ! \text{ мн-ны } q \text{ и } r \text{ такие, что:}$
		\[f = q \cdot g + r, \q \deg r < \deg g\]
	\end{theorem}

  \begin{proof}
    (Существование):

    Индукция по $m=\deg f$

    База. $\deg f < \deg g$
    \[h:=0,\q r:=f\]
    \[f=g \cdot 0+f\]

    Инд. переход. Пусть $m \geqslant n$ и утверждение доказано для всех многочленов меньшей степени $<m$
    \[f=b_0+b_1 x+...+b_m x^m\]
    \[f_1:=f-a_n^{-1} b_m x^{m-n} g= \cancel{b_m x^m} +...-(\cancel{a_n^{-1} b_m a_n x^m} +...) \Ra \deg f_1 < m\]
    \[f_1=g h_1+r_1,\q \text{по инд.п. } \deg r_1 < \deg g\]
    \[f=f_1-a_n^{-1} b_m x^{m-n} g = (\underbrace{h_1+a_1^{-1} b_m x^{m-n}}_{=h})g + \underbrace{r_1}_{=r}\]
    \[\deg r = \deg r_1 < g\]
    (Единственность):
    \[f=g h + r=g \w{h} + \w{r},\q \deg r < \deg g,\ \deg \w{r} < \deg g\]
    \[g(\w{h}-h)=r-\w{r} \q \deg(r-\w{r}) < \deg g\]

    Если $\w{h}-h \neq 0$, то положим $d=deg(\w{h}-h)$
    \[\w{h}-h=\us{\neq 0}{c_d} x^d+...\]
    \[g(\w{h}-h)=\us{\neq 0}{a_n c_d} x^{n+d}+...\]

    (Если $a_n c_d=0 \Ra c_d=a_n^{-1}a_n c_d=a_n^{-1} 0 = 0$, противоречние)
    \[\deg(r-\w{r}) = \deg g(\w{h}-h) \geqslant g \text{, но } \deg(r-\w{r}) < \deg g \]

  \end{proof}

	\begin{Example}
		\[\text{В кольце } \Z[x]\]
		\[x^2 + 1 \text{ нельзя поделить на } 2x + 1\]
	\end{Example}


\section{Корни многочлена. Теорема Безу.}
	\begin{definition}
		R - ком. кольцо с 1
		\[f \in R[x] \q f = a_0 + a_1 x + ... + a_n x^n\]
		Для данного мн-на определим отображение из R в R:
		\[c \ra a_0 + a_1 c + ... + a_n c^n = f(c)\]
	\end{definition}

	\begin{remark}
		Разные мн-ны могут задавать одно и то же отображение
		\[\Z_{/2}\Z \q f = 0 \q 0 \ra 0 \q 1 \ra 0\]
		\[f = x^2 + x \q 0 \ra 0 \q 1 \ra 0\]
    \[(f + g)(c) = f(c) + g(c)\]
		\[(f \cdot g)(c) = f(c) \cdot g(c)\]
	\end{remark}

	\begin{definition}
		$f \in R[x] \q c \text{ - корень f, если } f(c) = 0$
	\end{definition}

	\begin{theorem} [Безу]
		$f \in R[x] \q c \in R, \text{ тогда:}$
		\[\exists q \in R[x] \q f = (x - c)q + f(c)\]
	\end{theorem}

	\begin{proof}
		$g = x - c, \q \text{ по т. о делении с остатком:}$
		\[\exists q, r \in R[x]: f = (x - c)q + r\]
		\[\deg r < \deg g = 1\]
		\[\deg r \leq 0 \Ra r \in \R\]
		$f(c) = (c - c) \cdot q(c) + r = r \ \Ra\ f = (x - c)q + f(c)$
	\end{proof}

	\begin{Consequence}
		\[\text{c - корень f} \rla (x - c) \mid f\]
	\end{Consequence}

	\begin{proof}
    ($\Ra$):
		\[f(x) = (x - c)q(x) + f(c) = (x - c)q(x)\ \Ra\ (x - c) \mid f\]
    ($\La$):
		\[f(x) = (x - c)q(x)\ \Ra\ f(c) = (c - c)q(c) = 0\]
	\end{proof}


\section{Кратные корни многочлена. Теорема о числе корней многочлена над полем.}
	\begin{definition}
		$K \text{ - поле} \q f \in K[x]$\\
		$\text{Тогда a - корень f кратности k, если } (x - a)^k \mid f \text{ и } (x - a)^{k + 1} \nmid f$
		\[(\text{т.е. }f(x) = (x - a)^k \cdot g(x) \q (x-a) \nmid g\q (\lra g(a) \neq 0))\]
	\end{definition}

	\begin{remark}
		a - корень $f_1$ кратности $k_1$,\q\q a - корень $f_2$ кратности $k_2$
    \[\Ra \text{a - корень } f_1 \cdot f_2 \text{ кратности } k_1 + k_2\]
  \end{remark}
  \begin{proof}
		$f_1(x) = (x - a) ^{k_1} g_1(x) \q g_1(a) \neq 0$
		$f_2(x) = (x - a) ^{k_2} g_2(x) \q g_2(a) \neq 0$
		\[\Ra f_1(x) f_2(x) = (x - a)^{k_1 + k_2} g_1(x) g_2(x)\]
    \[\text{(поле K - о.ц.)}\]
	\end{proof}

	\begin{lemma}
		$f, g, h \in K[x],\q b \in K \q b \text{ - не корень h}$
		\[f(x) = h(x)g(x)\]
		\[b \text{ - корень f} \Ra b \text{ - корень g той же кратности}\]
	\end{lemma}

  \begin{proof}
    1) b - корень f кр. $l \geqslant 1 \Ra$ b - корень g кратности $\geqslant l$

    Индукция по l. Б.И.:
    \[l=1\q f(b)=0\q h(b)g(b)=0 \Ra g(b)=0\]
    \[\text{b - корень g $\Ra$ корень g кр. $\geqslant 1$}\]

    Инд. переход $(l \ra l+1)$
    \[\text{b - корень f кр. $l+1$} \lra f(x)=(x-b)^{l+1} f_1(x)\]

    По предп. b - корень g $g(x)=(x-b)g_1(x)$
    \[(x-b)^{l+1} f_1(x)=(x-b)g_1(x)h_1(x)\q (=f(x))\]

    В обл. целостности можем сократить на ненулевой множитель
    \[(x-b)^l f_1(x) = g_1(x) h(x)\]

    По инд. предп. b - корень кратности $\geqslant l$

    \[\Ra \text{b - корень g кр. } \geqslant l+1 \text{ (при перемножении кр-ти складываются)}\]
    2) $f(x)=h(x) g(x)$ и b - корень g кр-ти k
    \[(x-b)^k \mid g(x) \Ra (x-b)^k \mid f(x)\]

    b - корень кр-ти не больше кр-ти корня f
  \end{proof}

  \hypertarget{th:krat}{}
	\begin{theorem}
		$K \text{ - поле, } f \in K[x] \q f \neq 0$
		\[\Ra\text{число корней с учетом их кратности не превосходит }\deg f\]
	\end{theorem}

  \begin{proof}
      Индукция по $\deg f$\\
      Б.И.:

      $\deg f = 0$ корней нет\\
      И.П.:

      a - корень f кр. k $\Ra f(x)=(x-a)^k g(x)$\\
      Пусть $b \neq a \Ra b \text{ - корень f} \lra$\\
      $\lra$ b - корень g, причем кратности совпадают (по лемме, т.к. $(x-b)^k \neq 0)$\\
      По инд. предп. число корней g с учетом кратности $\leqslant \deg g$

      (а это в точности все корни f, отличные от a)

      Сумм. кр. корней $f=k+\text{сумм. кр. корней g} \leqslant k+\deg g = \deg f$
  \end{proof}

	\begin{remark}
		Теор. не верна для $f \in R[x]$ (в случае произвольного комм. кольца R)
		\[R = \Z_{/8}\Z\]
		\[x^2 = [1] \in R[x]\]
		корни 1, 3, 5, 7 $\q \deg f = 2$
	\end{remark}

  \hypertarget{co:deg}{}
	\begin{Consequence}
		\[\text{Если } f(a_1) = ... = f(a_n) = 0 \text{ для попарно различных } a_1, ..., a_n\]
    \[\text{И } n > \deg f,\q \text{тогда } f = 0\]
	\end{Consequence}


\section{Функциональное и формальное равенство многочленов.}
	\begin{consequence}[пред. \hyperlink{th:krat}{теореме}]
		$f, g \in K[x] \q |K| > \max(\deg f, \deg g),$

		если f и g совп. функционально, то f = g
	\end{consequence}

  \begin{proof}
    Функ. рав-во: $\forall a\in K\q f(a)=g(a) \Ra (f-g)(a)=0$
    \[\deg (f-g) \leqslant \max (\deg f,\ \deg g) < |k|\]
    \[\text{по пред. \hyperlink{co:deg}{сл.}}\q f-g=0\Ra f=g\]
  \end{proof}

	\begin{remark}
		Для беск. полей из функ. равенства мн-ов следует формальное
	\end{remark}


\section{Характеристика поля.}
	\begin{definition}
		$K \text{ - поле} \q 1 \in K$
		\[n \cdot 1 = \underbrace{ 1 + ... + 1}_{n}\]
		Если $n \cdot 1 \neq 0$ для всех $n \geq 1$, то говорят, что поле K имеет характеристику 0: \q $\Char K = 0$\\
		Если $\exists n \geq 1:\ n \cdot 1 = 0$, то наименьшее такое положительное n называют характеристикой K
	\end{definition}

	\begin{examples}
    \begin{enumerate}
      \item $\Char \Q = 0,\q \Char \R = 0,\q \Char\CC = 0$
      \item p - простое \q $\Char(\Z_{/p}\Z)=p$
    \end{enumerate}
	\end{examples}

	\begin{theorem}
		Характеристика поля либо 0, либо простое число
	\end{theorem}

	\begin{proof}
		1) не $\exists n \geq 1 \q n \cdot 1 = 0 \q \Ra \q \Char K = 0$\\
		2) $\e n:\ n \cdot 1 = 0$ возьмем наим. n и покажем, что n - простое\\
		\[\sqsupset \text{n - сост.} \q n = ab \q 1 < a, b < n\]
		\[0 = \underbrace{1 + ... + 1}_{n} = (\underbrace{1 + ... + 1}_{a})(\underbrace{1 + ... + 1}_{b})\]
		\[\Ra \underbrace{1 + ... + 1}_{a} = 0 \text{ или } \underbrace{1 + ... + 1}_{b} = 0\]
		противоречие с $\min n$\\
		$\Ra n \text{ не сост.}; 1 \neq 0 \Ra n \neq 1$\\
		$\Ra n$ - простое
	\end{proof}


\section{Производная многочлена. Свойства производной. Многочлены с нулевой производной.}
	\begin{definition}
		$\text{K - поле},\q f(x) \in K[x],\q f(x) = \sum\limits_{k = 0}^n a_k x^k$
		\[\text{Тогда } f^{'}(x) := \sum_{k = 1}^n (k a_k) x^{k - 1}\]
		\[k \cdot a_k = \underbrace{a_k \cdot ... \cdot a_k}_{k}\]
	\end{definition}

	\begin{theorem} [Свойства]
		\begin{enumerate}
			\item $(f + g)^{'} = f^{'} + g^{'}$
            \[f = \sum_{k=0}^n a_k x^k,\q g = \sum_{k=0}^n b_k x^k,\q f+g= \sum_{k=0}^n (a_k+b_k) x^k\]
            \[\text{Действительно, }k(a_k+b_k)=k a_k + k b_k\]
			\item $c \in K \q (c \cdot f)' = c f'$
            \[k(c a_k)=c(k a_k)\]
			\item $(f \cdot g)' = f'g + g'f$
            Док-во без $(\sum)'$:
			      \begin{enumerate}
			      	\item $f = x^n \q g = x^m$
			      	      \[(x^{n + m})' = (n + m) x^{n + m - 1}\]
			      	      \[(x^n)' x^m + x^n(x^m)' = nx^{n - 1} \cdot x^m + mx^n \cdot x^{m-1} = (n + m)x^{n + m - 1}\]
			      	\item $f = x^n \q g = \sum\limits_{k = 0}^m a_k x^k$
			      	      \[(f \cdot g)' = (\sum_{k = 0}^m a_k x^n x^k)' = \sum_{k=0}^m a_k (x^n \cdot x^k)' = \]
			      	      \[= \sum_{k = 0}^m a_k((x^n)' \cdot x^k + x^n (k x^{k - 1})) = \]
			      	      \[(x^n)' \sum_{k = 0}^m a_k x^k + x^n(\sum_{k = 0} k a_k x^k) = f'g + fg'\]
			      	\item $f, g \text{ - произвольные}$
			      	      \[f = \sum_{k = 0}^n b_k x^k\]
			      	      \[(fg)' = \sum_{k = 0}^n b_k (x^k g)' = (\sum_k b_k \cdot k x^{k - 1} \cdot g) + (\sum_k b_k x^k \cdot g') = \]
			      	      \[= f'g + fg'\]
            \end{enumerate}
      	\item Ф-ла Лейбница
      	      \[(f \cdot g)^{(k)} = \sum_{i = 0}^k C_k^i f^{(i)} g^{(k - i)}\]
      	\item Если  $\Char K = 0 \Ra f'= 0 \rla f \in K$\\
      	      Если  $\Char K = p > 0$, то $f' = 0 \rla f \in K[x^p]$
      	      \[(\text{т.е } f = a_0 + a_p x^p + ... + a_{kp} x ^{kp})\]
              *тут когда-нибудь будет док-во*
		\end{enumerate}
	\end{theorem}


\section{Теорема о кратности}
\begin{theorem}
  K - поле \q $char K = 0$
 \[f \in K[x] \q a \text{ - корень } f \text{ кр. }l \geq 1\]
  Тогда a - корень $f'$ кратности $l - 1$
\end{theorem}

\begin{remark}
  Если char K $ = p > 0$, то теор. не верна
  \[\Z_{/p}\Z \q f = x^{2p + 1} \q \text{ O - корень кр. p}\]
  \[\q\q f' = (2p + 1)x^{2p} + px^{p - 1} = x^{2p} \q \text{ O - корень кр. 2p}\]
\end{remark}

\begin{Proof}[теоремы]
  \[f(x) = (x - a)^l \cdot g(x) \q g(a) \neq 0\]
  \[f' = l(x - a)^{l - 1}  \cdot g(x) + (x - a)^l \cdot g'(x) = (x - a)^{l-1}(l g(x) + (x - a)g'(x))\]
  \[a \text{ - корень } f' \text{ кр } \geq l - 1\]
  \[lg(a) + (a - a)g'(a) = l \cdot g(a) \neq 0\]
  \[a \text{ - корень } f' \text{ кр } l - 1\]
\end{Proof}


\section{Интерполяционная задача. Существование и единственность решения.}
\begin{definition}[интерполяционная задача]
  K - поле. $a_1,...,a_n$ - попарно различны, $y_1,...,y_n \in K$\\
  Найти мн-н f, такой, что $f(a_i)=y_i$, где $i=1..n$
\end{definition}

\begin{theorem}
  Для интерполяционной задачи:
  \begin{center}
    \begin{tabular} {c | c}
      $x$ & $a_1 \  ... \  a_n$ \\
      \hline
      $f$ & $y_1 \  ... \  y_n$
    \end{tabular}
  \end{center}
  $\exists !$ решение $f$ степени $< n $
\end{theorem}

\begin{proof}
  1) Единственность
  \[f,\ h \text{ - решают одну и интер. задачу}\]
  \[\deg f, \ \deg h < n\]
  \[\forall i = 1, ..., n \q f(a_i) = h(a_i) = y_i \ \Ra\ f(a_i)-h(a_i) = 0\]
  \[f - h \text{ имеет } \geq n \text{ корней, а степ. } < n\]
  \[f - h = 0 \Ra f = h\]
  (теорема о числе корней мн-на)\\
  2) Существование
  \[f(x) = c_0 + c_1 x + ... + c_{n - 1} x^{n - 1}\]
  \[c_0 + c_1 a_i + ... + c_{n - 1} a_i^{n - 1} = y_i\]
  \[
    \begin{pmatrix}
      1 & a_1 & a_1^2 &...& a_1^{n - 1} \\
      \vdots &   &   &   & \vdots &   \\
      1 & a_n & a_n^2 & ... & a_n^{n - 1}
    \end{pmatrix}
    \begin{pmatrix}
      c_0       \\
      \vdots    \\
      c_{n - 1}
    \end{pmatrix}
    =
    \begin{pmatrix}
      y_1    \\
      \vdots \\
      y_n
    \end{pmatrix}
  \]
  \[
    A
    \begin{pmatrix}
      c_0       \\
      \vdots    \\
      c_{n - 1}
    \end{pmatrix}
    =
    \begin{pmatrix}
      y_1    \\
      \vdots \\
      y_n
    \end{pmatrix}
  \]
  \[\det A = \prod_{j > i}(a_j - a_i) \neq 0 \q\q \text{определитель Вандермонда}\]
  \[A \text{ - обр.}\]
  \[
    \begin{pmatrix}
      c_0       \\
      \vdots    \\
      c_{n - 1}
    \end{pmatrix}
    = A^{-1}
    \begin{pmatrix}
      y_1    \\
      \vdots \\
      y_n
    \end{pmatrix}
  \]
\end{proof}

\section{Интерполяционный метод Ньютона.}
	\begin{reminder}
    Дана интерполяционна задача:
    \begin{center}
      \begin{tabular} {c | c | c}
  			$x$    & $a_1$ & $a_i \  ... \  a_n$ \\
  			\hline
  			$f(x)$ & $y_1$ & $y_i \ ... \ y_n$
  		\end{tabular}
    \end{center}
  \end{reminder}

  \begin{Definition}[метод Ньютона]
    \[\text{Пусть }f_{i - 1} \text{ - интерпол. мн-н степени } \leq i - 1\]
		и решающий интерпол. задачу для первых i точек
		\[f_0(x) = y_1,\text{ где\q} f_0(a_1) = y_1 \text{ - так можно задать начальный}\]
		\[\sqsupset \text{ построли } f_{i - 1}. \q \text{Ищем $f_i:$}\]
		\[(f_i - f_{i - 1})(a_j) = 0 \q j = 1, ... , i \text{ - так должно быть}\]
		\[\Ra f_i(x) = f_{i - 1}(x) + c_i \cdot (x - a_1) ... (x - a_i)\]
		\[\deg f_i \leq i, \text{ найдем c:}\]
		\[y_{i + 1} = f_i (a_{i + 1}) = f_{i - 1}(a_{i + 1}) + c_i(a_{i + 1} - a_i) ... (a_{i + 1} - a_i)\]
		\[\Ra c_i = \frac{y_{i + 1} - f_{i - 1} (a_{i + 1})}{(a_{i + 1} - a_1)... (a_{i + 1} - a_i)}\]
  \end{Definition}

\section{Интерполяционный метод Лагранжа.}
	\begin{definition}
    Хотим построить функцию, такую что:
    \begin{center}
      \begin{tabular} {c | c | c c c | c}
  			$x$    & $a_1$ & $a_{j - 1} $ & $ a_j$ & $a_{j_i}$ & $a_n$ \\
  			\hline
  			$L_j(x)$ & $0$   & $0$          & $1$    & $0$       & $0$
  		\end{tabular}\\
    \end{center}

    Построим $M_j(x)$, который во всех точках кроме $a_j$ равен 0:
		\[M_j(x) := a_j (x - a_1) ... (x - a_{j - 1}) (x - a_{j + 1})...(x - a_n)\]
		\[L_j(a_j) = 1 \text{ - так должно быть}\]
		\[L_j(x) := \frac{(x - a_1) \cdot ... \cdot (x - a_{j - 1})(x - a_{j + 1}) \cdot ... \cdot (x - a_n)}
			{(a_j - a_1) \cdot ... \cdot (a_j - a_{j - 1})(a_j - a_{j + 1}) \cdot ... \cdot (a_j - a_n)}\]
			\[L_j(x) \text{ - интерп. мн-н Лагранжа}\]
			\[L_j(a) =
				\begin{cases}
					1, & i = j    \\
					0, & i \neq j
				\end{cases}
				\q\q\q
				\begin{align}
					\deg L_j(x) = n - 1 \\
					\deg f \leq n - 1
				\end{align}
			\]

      Теперь хотим решить интерполяционную задачу:
      \begin{center}
        \begin{tabular} {c | c  c}
  				$x$    & $a_1 \q$ & $a_n$ \\
  				\hline
  				$f(x)$ & $y_1 \q$ & $y_n$
  			\end{tabular}
      \end{center}
			\[f(x) = \sum_{j = 1}^n y_j L_j (x) \q\q f(a_i) = \sum_{j = 1}^n y_j L_j (a_j) = y_i L_i (a_i) = y_i\]
			Мн-н Лагранжа исп. в алгоритмах быстрого умножения\\
			$\forall \mathcal{E} > 0 \q \exists $ алг. умн., который для n-разрядных чисел требует $O(n^{1 + \mathcal{E}})$
			поразрядных операций
			\end{definition}


\section{Делимость и ассоциированность в кольце многочленов над полем.}
    \begin{definition}
        K - поле, $K[x]$
        \[f, g \in K[x] \text{ ассоциированы, если:}\]
        \[f \mid g \text{ и } g \mid f\]
        Обозначение: $f \sim g$
    \end{definition}
    \begin{Remark}
        \[0 \sim 0\]
        0 с другими не ассоц.
    \end{Remark}
    \begin{Proof}
        \[f \neq 0 \q g \neq 0 \q f \mid g \q g \mid f\]
        \[\deg f \leq \deg g \q \deg g \leq \deg f\]
        \[\Ra \deg f = \deg g\]
        \[f = c \cdot g \q c \in K^* = K \setminus \{0\}\]
        \[0 = 1 \cdot 0\]
        \[\text{Если } f = c \cdot g, c \in K^* \q g = c^{-1} f \Ra g \mid f, \q f \mid g\]
    \end{Remark}

    \begin{Consequence}
        \[f \sim g \rla \exists c \in K^* \q f = cg\]
        Если $f \neq 0$, то в классе ассоц. с f мн-нов всегда можно выбрать мн-ен со старшим коэф 1.\\
        Мн-н со старшим коэф. 1 называется унитарным, \ul{приведенным}
    \end{Consequence}

    \begin{Remark}
        \[f \mid g \q f \sim f_1 \q g \sim g_1 \Ra f_1 \mid g_1\]
    \end{Remark}
    \begin{Proof}
        \[g = f \cdot h\]
        \[cg = f(ch)\]
        \[g = (cf)(c^{-1} h)\]
    \end{Proof}


\section{Наибольший общий делитель в кольце многочленов над полем. \\Существование и линейное представление.}
    \begin{definition}
        $K \text{ - поле, } K[x],\q f_1, ..., f_n \in K[x]$\\
        $\text{Тогда }g = \gcd(f_1, ..., f_n) \text{, если:}$
        \[g \mid f_1, ..., g \mid f_n\]
        $\text{И } \forall h \q(h \mid f_1, ..., h \mid f_n) \Ra h \mid g$
    \end{definition}

    \begin{remark}
        НОД опред. не однозначно, а с точностью до ассоц.
        \[\text{НОД}(0, ..., 0) = 0\]
        Если хотя бы один $f_1 ... f_n \neq 0, $ то в классе ассоц. с НОД можно выбрать приведенный
    \end{remark}

    \begin{theorem}
        $\forall f_1, ..., f_n \in K[x]$\\
        $\text{Тогда существует } g = \text{НОД}(f_1, ..., f_n) \text{ и он допускает лин. предствление:}$
        \[g = f_1 h_1 + ... + f_n h_n \text{ для некоторых } h_1...h_n \in K[x]\]
    \end{theorem}

    \begin{proof}
        1) $f_1 = f_2 = ... = f_n = 0 \q\q \gcd(0, ..., 0) = 0$
        \[\text{Положим } h_1 = ... = h_n = 1\]
        2) $\exists i \q f_i \neq 0$
        \[I = \{ f_1 h_1 + ... + f_n h_n \ : \ h_1...h_n \in K[x]\}\]
        \[I \neq \{ 0 \} \q\q 0 \neq f_i \in I \]
        Пусть g - мн-ен наим. степени в $I \setminus \{ 0 \}$\\
        Утверждается, что $g = \gcd(f_1, ..., f_n)$
        \[f_j & = g \cdot u_j + r_j\qq r_j = 0 \text{ или } \deg r_j < \deg g\]
        \[r_j & = -g \cdot u_j + f_i = -h_1 u_j f_1 - h_2 u_j f_2 + (-h_ju_j  + 1) f_i -...\]
        \[g = h_1f_1 + ... + h_n f_n \q\q r_j \in I\]
        Т.к. степ. g - наименьшая в $I\setminus\{0\}$:\\
        \[\deg r_j < \deg g, \text{ то } r_j = 0\]
        \[f_j = g u_j\q g \mid f_j \q j = 1, ..., n\]
        \[h \mid f_i, ..., h \mid f_n\]
        \[g = (\us{\us{h}{\dots}}{f_1 h_1} + ... + \us{\us{h}{\dots}}{f_n h_n})
            \ \vdots \ h \Ra h \mid g
        \]
    \end{proof}


\section{Взаимно простые многочлены. Свойства взаимно простых многочленов. Если многочлен делит
    произведение двух многочленов и взаимно прост с первым сомножителем, то он делит второй сомножитель.}
    \begin{definition}
        $f_1, ..., f_n \in K[x] \text{ назыв. взаимно простыми, если } \gcd(f_1, ..., f_n) \sim 1$
    \end{definition}

    \begin{theorem} [Свойства НОД]
        \begin{enumerate}
            \item $\gcd(f,0) \sim 1$
            \item $\gcd(f_1,...,f_n) = \gcd(\gcd(f_1,...,f_n), f_n)$
            \item Если $g \sim \gcd(f_1, ..., f_n) \q$ $(\text{не все } f_i = 0)$
                    \[\text{то } \frac{f_1}{g}, ..., \frac{f_n}{g} \text{ - взаимно просты}\]
            \item $\gcd(f,g) \sim \gcd(f-gh,g)$
            \item $f_1, ... f_n$ - вз. просты $\rla 1 $ допускает лин. представление
                    \[1 = h_1 f_1 + ... + h_n f_n \q\q h_i, ..., h_n \in K[x]\]
        \end{enumerate}
    \end{theorem}

    \begin{proof}
        См. док-ва для $\Z$ (Спасибо, Всемирнов)
    \end{proof}

    \begin{Theorem}
        \[f \mid gh \text{ \  и \ \ } f \text{ и } g \text{ - вз. просты } \Ra f \mid h\]
    \end{Theorem}

    \begin{Proof}
        \[\exists u, v \in K[x]\]
        \[fu + gv = 1\]
        \[\us{\us{f}{...}}{fuh} + \us{\us{f}{...}}{ghv}  = h \q \Ra h \  \vdots \ f\]
    \end{Proof}


\section{Неприводимые многочлены. Теорме о разложении многочлена в произведение неприводимых (существование).}
    \begin{utv}
        $K[x] = \{0\} \cup K^* \cup \{\text{мн-ны ст. } \geq 1\}$\\
        т.к. обратимые эл-ты в кольце мно-ов - константы
    \end{utv}
    \begin{definition}
        $f \in K[x] \setminus K \text{ называются составными (или приводимым), если}$
        \[f = gh \q 1 \leq \deg g,\ \deg h < \deg f\]
        $\text{В противном случае } f \text{ - назыв. неприводимым}$
        \[f \text{ - неприводим, если }f = gh \Ra \deg h = 0 \text{ или } \deg g = 0\]
    \end{definition}

    \begin{definition}
        f - неприв. $\rla$ все делители f - это константы и мн-ны $\sim$ f
    \end{definition}

    \begin{examples}
        \begin{enumerate}
          \item $x - a \text{ неприводим при любом } a$
          \item $x^2 + 1 \text{ неприводим в } \R[x]$
          \item $x^2 + 1 \text{ в } \CC [x] \text{ приводим: } \q x^2 + 1 = (x + i)(x - i)$
          \item $\text{В } \R[x] \q (x^2 + 1)(x^2 +2) \text{ - приводим, но корней нет}$
          \item $\text{Если } gf \q \deg f \geq 2 \text{ есть корень в K, то }f \text{ - приводим в } K[x]$\\
          $f = (x - a)g \q \text{(по т. Безу)}$\\
          Обратное неверно. Но для мн-нов степени 2 и 3 неприводимость в $K[x]$ равносильна отсутствию корней в $K$
        \end{enumerate}
    \end{examples}

    \begin{Theorem}
        \[f \in K[x] \q f \text{ - неприводим}\]
        \[f \mid g_1 \cdot ... \cdot g_n \ \Ra \ \exists i : f \mid g_i\]
    \end{Theorem}

    \begin{proof}
      $n=1$:
      \[f \mid g \text{ - доказано}\]
      $n=2$:
      \[f \mid g_1 g_2\]
      \[\text{Если $f \mid g$ - всё доказано}\]
      \[\text{Пусть $f\ \cancel{\mid}\ g_1$. Общие делители f и g - константы}\]
      \[\gcd(f,g_1)=1,\q \text{по теореме из предыдущего билета, }f \mid g_2\]
      $n \geqslant 3 \text{ (индукция по n)}$:
      \[f \mid (g_1...g_{n-1})g_n\]
      \[\text{Аналогично $f \mid g_n$ или $f \mid g_1...g_{n-1}$}\]
      \[\Ra \e i: f \mid g_i\]
    \end{proof}

    \begin{theorem}[алгорим Евклида в {$K[x]$}]
      $f,g \in K[x]$, $r_0=f$, $r_1=g$\\
      До тех пор пока $r_i \neq 0$
      \[r_{i-1}=r_i q_i+r_{i+1} \q \deg r_{i+1} < \deg r_i\]
      Последний ненулевой остаток - это $\gcd (r_0,r_1)$
    \end{theorem}

    \begin{theorem} [основная теорема арифметики в кольце многочленов]
        Всякий ненулевой $f \in K[x]$ может быть представлен в виде \[c \cdot \prod_{i = 1}^n g_i\]\\
        $c \in K^*$, а все $g_i$ - приведенные неприводимые мн-ны. Причем такое произведение ед. с точностью до
        порядка сомножителей.
    \end{theorem}

    \begin{Remark}
        \[\text{Для } f = c \in K^* \q n = 0\]
    \end{Remark}

    \begin{lemma} [1]
        Всякий  f: $\deg f \geq 1$ делится хотя бы на один неприводимый.
    \end{lemma}

    \begin{proof}
        f - непр - все доказано\\
        Если приводим, то $f = f_1 \cdot g_1 \q\q 1 \leq \deg f_1 < \deg f$\\
        Если $f_1$ неприв, то делитель найден\\
        Если приводим $f_1 = f_2 g_2 \q\q q \leq \deg f_2 \leq \deg f_1$\\
        $\deg f> \deg f_1 > ...  \Ra$ процесс оборвется\\
        $\Ra $ найдем неприв. делитель f
    \end{proof}

    \begin{proof} [Существование]
        Индукция по $\deg f$:\\
        $\deg f = 0$:
        \[f = c \in K^* \q f = c \cdot (\prod\limits_{i = 1}^0 g_i)\]
        Инд. преход $\deg f > 0$:\\
        \[\text{По лемме $\exists$ неприв. $g_1$: $g_1 \mid f$}\]
        \[\text{Не умоляя общности $g_1$ - приведенный (с коэф. 1)}\]
        \[f = g_1 f_1 \q \deg f_1 < \deg f - \deg g_1 < \deg f\]
        По инд. предп.
        \[f_1 = c \prod_{i = 2}^n g_i \q g_i \text{ - приведенный неприводимый}\]
        \[f = f_1 g_1 = c \prod_{i = 1}^n g_i\]
    \end{proof}


\section{Теорема о разложении многочлена в произведение неприводимых (единственность).}
    \begin{theorem} [основная теорема арифметики в кольце многочленов]
        Всякий ненулевой $f \in K[x]$ может быть представлен в виде \[c \cdot \prod_{i = 1}^n g_i\]\\
        $c \in K^*$, а все $g_i$ - приведенные неприводимые мн-ны. Причем такое произведение ед. с точностью до
        порядка сомножителей.
    \end{theorem}
    \begin{Proof}[единственность]
        \[(*) \q f = c \prod_{i = 1}^n g_i = \widetilde{c} \prod_{i = 1}^m \widetilde{g_i}\]
        \[\Ra n = m \q c = \widetilde{c} \text{  иначе перенумеруем сомнож. } g_i = \widetilde{g_i}\]
        \[\text{Не умоляя общ. } n \leq m\]
        Индукция по n. База инд.:
        \[n = 0 \q c = \widetilde{c} \prod_{i = 1}^n \widetilde{g_i} \Ra m = 0 \q \widetilde{c} = c\]
        Инд. переход:
        \[g_n \mid \widetilde{c} \prod_{i = 1}^m \widetilde{g_i} \Ra \exists i \q g_n \mid \widetilde{g_i}\]
        \[\widetilde{c} \neq 0\]
        Не умоляя общности $i = m$ (иначе перенумеруем)
        \[g_n \mid \widetilde{g_m} \os{$g_n$ - непр.}{\us{\text{со ст. коэф. 1}}{\Ra}} g_n = \widetilde{g_m}\]
        В $(*)$ сократим на $g_n$
        \[c \prod_{i = 1}^{n - 1} g_i = \widetilde{c} \prod_{i = 1}^{m - 1}\widetilde{g_i} \q n-1 \leq m - 1\]
        По инд. предп. $n - 1 = m - 1 \q (\Ra n = m)$
        \[c = \widetilde{c} \text{ (после перенумерования)}\]
        \[g_i = \widetilde{g_i} \q i = 1, ..., n - 1\]
        \[g_n = \widetilde{g_n}\]
    \end{Proof}


\section{Алгебраически замкнутые поля. Эквивалентные переформулировки. Алегбраическая замкнутость поля
    комплексных чисел.(б.д.)}

    \begin{Theorem}
        \[\sqsupset K \text{ - поле, рассмотрим } K[x]\]
        Следующие условия равносильны
        \begin{enumerate}
            \item Все неприводимые в $K[x]$ - это в точности линейные мн-ны
            \item Всякий мн-н $f \in K[x],\ \deg f > 0$ расскладывается в произведение лин. множителей
            \item Всякий $f \in K[x],\ \deg f > 0$ делится на линейный
            \item Всякий $f \in K[x],\ \deg f > 0$ имеет в $K$ хотя бы $1$ корень
            \item Всякий $f \in K[x],\ \deg f > 0$ имеет в $K$ в точности $n = \deg f$ корней с учетом кратности
        \end{enumerate}
    \end{Theorem}

    \begin{definition}
        Если для $K$ и $K[x]$ выполнено любое из равносильных усл.овий теоремы, то $K$ называется алгебраически замкнутым
    \end{definition}

    \begin{proof}
      $(1 \Ra 2)$:
      \[f \in K[x],\ \deg f>0\q f \us{на неприв.}{\os{\text{т-ма о разлож.}}{=}} c \prod_{i=1}^n g_i,\q g_i \text{ - непр. мн-ль}\]

      (неприводимые - линейные)\\ \\
      $(2 \Ra 1)$:

      Если $\deg f > 1$, то тогда f - неприводим и произв. лин. сомножителей
      \[f = l h,\q \deg l = 1, \q \deg h = \deg f-1 \geqslant 1\]

      (линейные - неприводимые)\\ \\
      $(2 \Ra 3)$:

      2 формально сильнее 3\\ \\
      $(3 \Ra 2)$:

      Индукция по $\deg f$:
      \[\deg f = 1 \text{ - утверждение верно}\]
      \[\deg f > 1\q \e l \in K[x]: \deg l = 1\]
      \[f = l h \q \deg h = \deg f-1 \geqslant 1\]
      (по инд. предп. раскл. в произв. линейных)\\ \\
      $(3 \lra 4)$:

      По теореме Безу $(x-c) \mid f \lra f(c)=0$\\ \\
      $(5 \Ra 4)$:

      Есть n корней с учетом кратности \us{\deg f \geqslant 1}{$\Ra$} есть хотя бы 1 корень\\ \\
      $(2 \Ra 5)$:
      \[f = \prod_{i=1}^k (x-a_i)^{d_i},\q a_i \text{ попарно различны}\]
      \[\sum_{i=1}^k d_i = \deg f = n\]

      a - корень f кр. $d_i\ \Ra$ число корней f с учетом кр. $\geqslant n = \deg f$

      Но число корней f с учетом кратности есть $\deg f$
    \end{proof}

    \begin{examples}
      \begin{enumerate}
        \item $\R, \Q $ не алг. замкнуты
        \item   Любое конечное поле не алг. замкнуто
        \item $|F| = q \q \deg f = n > q$
      \end{enumerate}
    \end{examples}

    \begin{remark}
      В 3 семестре докажем, что над конечным полем есть неприводимые любой заданной степени
    \end{remark}

    \begin{theorem} [без д-ва]
        $\CC$ - алг. замк.
    \end{theorem}

    \begin{Consequence}
        \[f \in \CC[x],\q \deg f > 0\]
        \[f = c \prod_{i = 1}^k (x - a_i)^{d_i} \q\q a_i, c \in \CC\]
    \end{Consequence}


\section{Неприводимые многочлены над полем вещественных чисел. Теорема о разложении многочлена
     с вещественными коэффициентами в произведение неприводимых над $\R$.}
    \begin{example}
        Неприводимы:
        \[x - c, \q c \in \R\]
        \[x^2 + ax + b \q a^2 - 4b < 0 \q a, b \in \R \text{ (нет вещ. корней)}\]
    \end{example}

    \begin{theorem}
        Всякий неприв. в $\R[x]$ ассоциирован с линейным или с квадратичным с отриц. дискриминантом
    \end{theorem}

    \begin{Consequence}
        \[f \in \R[x] \q f \neq 0\]
        \[f = c \prod_{i = 1}^m (x - c_i)^{d_i} \prod_{j = 1}^k (x^2 + a_j x + b_j)^{l_j} \q a_j^2 - 4b_j < 0\]
    \end{Consequence}

    \begin{Lemma}
        \[f \in \R[x] \subseteq \CC[x]\]
        \[\text{Если } z \in \CC \text{ - корень } f \text{, то } \ol{z} \text{ - корень } f\]
    \end{Lemma}

    \begin{Proof}[леммы]
        \[f = a_0 + a_1 x + ... + a_n x^n\]
        \[a_0 + a_1z + ... + a_n z^n = 0\]
        \[\Ra \ol{a_0 + a_1 z + ... + a_n z^n} = \ol{0} = 0 \text{ (сопряжение)} \]
        \[=\ol{a_0}  +\ol{a_1} \ol{z} + ... + \ol{a_n}  (\ol{z})^n=a_0 + a_1\ol{z} + ... + a_n (\ol{z})^m = f(\ol{z})\]
    \end{Proof}
    \begin{proof}[теоремы]
      Осталось показать, что все остальные f с $\deg f > 0$ - неприводимы\\
      $\deg f = 2$:
      \[D = 0 \Ra f = a(x-c)^2\]
      \[D > 0 \Ra f = a(x-c_1)(x-c_2),\q c_1,c_2 \in \R\]
      $\deg f  \geqslant 3$:\\

      Посмотрим на него, как на мн-н с комил.(?) коэффициентами
      \[f \in \CC[x] \text{ z - корень в $\CC$}\]

      a) $z \in \R$
      \[\text{По теор. Безу } f(z)=0 \q f(x)=(x-z) h \q h \in \R[x],\ \deg h \geqslant 2\]
      б) $z \in \CC \setminus \R$
      \[\ol{z} \neq z\]
      \[f(x) = (x-z) h_1 = (x-z)(x-\ol{z}) h\]
      \[\ol{z}\text{ - корень f}\q \us{\neq 0}{(\ol{z}-z)} \Ra \ol{z} \text{ - корень }h_1\]
      \[(x-z)(z-\ol{z})=x^2-(z+\ol{z})x+z\ol{z}=x^2- \us{\in \R}{2 \real z} + \us{\in \R}{\abs{z}^2} \ra \in \R[x]\]
      \[D = 4 (\real z)^2 - \abs{z}^2 = -4(\im z)^2 < 0 \text{ (т.к. z - чисто компл. число)}\]
      \[g(x) = x^2 - 2 \real z x + \abs{z}^2\]
      \[f(x) = g(x) h(x)\]
      \[\deg h = \deg f - 2 \geqslant 1\]
      \[f g \in \R[x], \text{ поделим с остатком в $\R[x]$:}\]
      \[f = g q + r \q r=0 \text{ или } \deg r \leqslant 1\]

      Это равенство также верно и в $\CC[x]$:
      \[\begin{tabular}{c | c}
          f = gq + r &\\
          f = gh + 0
      \end{tabular} \Ra r = 0 \q h = q \in \R[x]\]
    \end{proof}


\section{Поле частных области целостности. Поле частных кольца многочленов (поле рациональных функций).}

    \begin{definition}
        R - комм. кольцо с $1$, о.ц.\\
        Хотим построить поле K, содержащее подкольцо изоморфное R, \\ состоящее из "дробей"
        \[X = R \times (R \setminus \{0\}) = \{(a, b) : \ a \in R, \  b \in R, \  b \neq 0\}\]
        На X введем отношение эквивалентности:
        \[(a, b) \sim (c, d) \text{ если } ad = bc\]
    \end{definition}

    \begin{Utv}
        \[\sim \text{ - отношение эквив.}\]
    \end{Utv}

    \begin{Proof}
        \[(a, b) \sim (a, b) \q\]
        \[(a, b) \sim (c, d) \Ra (c, d) \sim (a, b)\]

        \[\begin{align}
            (a,b) \sim (c, d)\\
            (c, d) \sim (e, f)
        \end{align}
        \Ra (a, b) \sim (e, f)\]
    \end{Proof}

    \begin{Definition}
        \[\frac{a}{b} = [(a, b)] \text{ - класс эквив.} \]
        \[K = X_{/\sim} \text{ На K введем структуру поля}\]
        \[\frac{a}{b} \cdot \frac{c}{d} = \frac{ac}{bd} \q \q b \neq 0 \q d \neq 0 \Ra bd \neq 0 \q (ac, bd) \in X\]
        \[\frac{a}{b} + \frac{c}{d} = \frac{ad + bc}{bd} \q\q (ad + bc, bd) \in X\]
    \end{Definition}

    \begin{proof}[корректность опредения]
       Корректность определения - это независимость от выбора представителя в классе
        \[\frac{a}{b} = \frac{a_1}{b_1} \q\q \frac{c}{d} = \frac{c_1}{d_1} \q\q\q
        \begin{align}
            ab_1 = ba_1\\
            cd_1 = dc_1
        \end{align}\]
        \[(ac, bd) \sim (a_1 c_1, b_1 d_1) \q\q a c b_1 d_1 = bd a_1 c_1\]
        \[(ad + bc, bd) \sim (a_1 d_1 + b_1 c_1, b_1 d_1)\]
        \[ad b_1 d_1 + bc b_1 d_1 = bd a_1 d_1 + bd b_1 c_1\]
        \[+ \begin{align}
            ab_1 = ba_1 & \ \mid \cdot dd_1\\
            cd_1 = dc_1 & \ \mid \cdot bb_1
        \end{align}\]
    \end{proof}

        \begin{theorem}
            $K, +, \cdot \text{ - поле} $
        \end{theorem}

        \begin{definition}
            Поле $K$ назыв. полем частных кольца $R$
        \end{definition}

        \begin{examples}
            $\Q$ - поле частных $\Z$\\
            $K[x]$ - о.ц\\
            Поле частных $K[x]$ обознач. $K(x)$ и назыв. полем рац. дробей или полем рац. функций\\
            Рац. функ. не есть функции в смысле отобр.
        \end{examples}




\section{Простейшие дроби. Разложение рациональной функции в сумму многочлена и простейших дробей. (существование).}
    \begin{Definition}
		\[K(x) \q K \text{ - поле}\]
		\[0 \neq \frac{f}{g} \in K(x) \q\q f, g \in K[x]\]
		\[\frac{f}{g} \text{ - правильная, если } \deg f < \deg g\]
  \end{Definition}

	\begin{Lemma} [1]
		\[\frac{f}{g};\q \frac{f_1}{g_1} \text{ - прав. дроби } \Ra \frac{f}{g} \cdot \frac{f_1}{g_1}; \q \frac{f}{g} + \frac{f_1}{g_1} \text{ - прав. дроби}\]
	\end{Lemma}

	\begin{Proof}
		\[\deg(f \cdot f_1) = \deg f + \deg f_1 < \deg g + \deg g_1 = \deg(g \cdot g_1)\]
		\[\frac{f}{g} + \frac{f_1}{g_1} = \frac{f g_1 + g f_1}{g g_1}\]
		\[\deg(fg_1 + gf_1) \leq \max \{\deg(fg_1), \deg(gf_1)\} < \deg(gg_1)\]
		\[\deg(fg_1) = \deg f + \deg g_1 < \deg g + \deg g_1 = \deg(gg_1)\]
		\[\deg(gf_1) = \deg g + \deg f_1 < \deg g + \deg g_1 = \deg(gg_1)\]
	\end{Proof}

	\begin{definition}
			Правильная дробь $\frac{f}{g}$ называется примарной, если $g = q^a, \q q$ - неприв. многочлен
			\[\frac{f}{g} = \frac{f}{q^a} \q\q \deg f < a \deg q\]
	\end{definition}

	\begin{definition}
			Дробь назыв. простейшей, если она имеет вид
			\[\frac{f}{q^a} \q q \text{ - неприв } a \geq 1\]
			\[\deg f < \deg q\]
	\end{definition}

	\begin{Theorem}
		\[\frac{f}{g} \in K(x) \text{ тогда } \frac{f}{g} \]
		\[\text{единственным образом (с точностью до порядка слагаемых) представима}\]
		\[\text{в виде суммы многочлена и простейших дробей}\]
	\end{Theorem}

	\begin{Lemma} [2]
		\[\frac{f}{g} \in K(x) \q \text{ Тогда } \frac{f}{g} = h + \frac{f_1}{g}, \q h \in K(x), \q \frac{f_1}{g} \text{ - прав дробь}\]
	\end{Lemma}

	\begin{Proof}
			\[\text{Делим с остатком: } f = gh + f_1, \q \deg f_1 < \deg g\]
			\[\frac{f}{g} = h + \frac{f_1}{g} \q \frac{f_1}{g} \text{ - прав. дробь}\]
	\end{Proof}

	\begin{Lemma} [3]
		\[\frac{f}{g} \text{ - прав. дробь, } g = g_1 \cdot g_2, \q \gcd(g_1, g_2) = 1\]
		\[\text{Тогда } \frac{f}{g} = \frac{f_1}{g_1} + \frac{f_2}{g_2}, \q\q \frac{f_1}{g_1}, \frac{f_2}{g_2} \text{ - прав. дроби}\]
	\end{Lemma}

	\begin{proof}
		По теореме о линейном представлении НОД в $K[x]$
		\[\exists u_1, u_2 \in K[x]\]
		\[g_1u_2 + g_2u_1 = 1 \mid \cdot f\]
		\[g_1(u_2f) + g_2(u_1f) = f\]
		\[g_2(u_1 f) = f - g_1(u_2 f)\]
		\[u_1f = g_1 h_1 + f_1 \text{ (делим с остатком)}\]
		\[f = g_1 (u_2 f) + g_2 (u_1 f) = g_1 (u_2 f) + g_2 (g_1 h_1 + f_1) = g_1 \underbrace{(u_2f + g_2 h_1)}_{= f_2} + g_2 f_1 = \]
		\[ = g_1 f_2 + g_2 f_1 \text{ - надо убедиться, что правильное}\]
		\[g_1 f_2 = f - g_2 f_1\]
		\[\deg g_1 + \deg f_2 \leq \max \{\deg f; \deg g_2 + \deg f_1\} < \deg g_1 + \deg g_2\]
		\[\deg f_2 < \deg g_2\]
		\[\frac{f}{g} = \frac{f_2}{g_2} + \frac{f_1}{g_1}\]
	\end{proof}


\section{Разложение рациональной функции в сумму многочлена и простейших \\ дробей. (единственность).}
		\begin{proof}
				Не умоляя общности можно считать, что в обоих разложениях одни и те же неприводимые
				\[\frac{f}{g} = h + \sum^{k}_{i=1} \sum^{a_i}_{j=1} \frac{f_{ij}}{q_i^j} , \deg f_{ij} < \deg q_i = \widetilde{h}
				+ \sum^{k}_{i=1} \sum^{a_i}_{j=1} \frac{\widetilde{f_{ij}}}{q_i^j}, \deg \widetilde{f_{ij}} < \deg q_i\]
				Не умоляя общности $a_i $ одни и те же в обеих суммах.
				\[h - \widetilde{h} \q\q \sum^{h}_{i=1} \sum^{a_i}_{j=1} \frac{f_{ij} - \widetilde{f_{ij}}}{q_i^j} = 0 \q (*) \]
				\[\text{Положим не все } f_{ij} - \widetilde{f_{ij} } = 0 \ \Ra \  \exists i, j \ : \ f_{ij} - \widetilde{f_{ij}} \neq 0  \]
				Для такого $i$ выберем наибольшее j из возможных.
				В $(*)$ наиб. степени $q_i$ в дроби с ненулевым числителем равна $q_i^j$\\
				Домножим $(*)$ на общее кратное знаменателей НОК $ = q_i^j \cdot ( )$ - произв. ост $q$  в каких-то степенях
				\[q_i(...) + q_i (...) + (f_{ij} - \widetilde{f_{ij} } = 0 \Ra  \]
				\[\deg (f_{ij} - \widetilde{f_{ij} }) \leq \max (\deg f_{ij}, \deg \widetilde{f_{ij} } ) < \deg q_i\]
				\[f_{ij} - \widetilde{f_{ij} } = 0 ?! \Ra \text{ в } (*) \text{ все } f_{ij} = \widetilde{f_{ij}}, \q h = \widetilde{h} \]
		\end{proof}


\section{Факториальные кольца. Содержание многочлена над факториальным \\ кольцом. Содержание произведения многочленов.}
    \begin{definition}
        R - о.ц
        \[a \not \in \{0\} \cup R^*\]
        назыв неприводимым, если
        \[a = bc \Ra b \in R^* \text{ и } c \sim a\]
        \[\text{или } c \in R^* \text{ и } b \sim a\]
        (все делители a есть либо обр. элем R либо ассоц. с a)
    \end{definition}
	\begin{definition}
			О.ц. R называется \ul{факториальным кольцом}, если в нем справедлива т-ма об однозначном разложении на множ.,
			а именно, всякий ненулевой необр. элемен R есть произведение неприводимых элементов, причем это разложение ед. с точностью
			до порядка сомножителей и ассоциированности
			\[a = p_1 \cdot ... \cdot p_n = q_1 \cdot ... \cdot q_m \q\q q_i, p_i \text{ - неприв } \Ra n = m \text{ и}\]
			\[\exists \text{ биекция } \sigma \text{ на } \{1,...,n\}\]
			\[p_i = q_{\sigma(i)} \]
			\[\Z, K[x] \text{ - факт. кольца}\]
			В факториальных кольцах можно определить НОД
			\[a = \mathcal{E}_1 \prod_{i = 1}^k q_i^{k_i} \q\q b= p_1 \prod_{i = 1}^n q_i^{l1} \q\q \mathcal{E}_1, p_1 \in R^* \q q_i
			\text{ - попарно ассоц. неприв}   \]
			\[\gcd (a,b) = \prod_{i = 1}^n q_i^{\min(k_i, l_i)}  \]
			\[ab = \mathcal{E}_1p_1 \prod_{i = 1}^n q_i^{(k_i + l_i)}  \]
	\end{definition}
	\begin{definition}
			Содержание многочлена f
			\[\text{cont}(f) = \gcd(a_1, a_2, ..., a_n)\]
	\end{definition}
	\begin{Definition}
		\[f \in R[x] \text{ называется примитивным, если  cont}(f) \sim 1\]
		В факториальном кольце $\forall$ многочлен $f \in R[x]$ можно записать как
		$f(x) = \text{cont}(f) \cdot f_1 \text{ - примитивный}$
	\end{Definition}
	\begin{Lemma} [Гаусса]
		\[\text{cont}(f\cdotg) = \text{cont}(f) \cdot \text{cont}(g)\]
	\end{Lemma}


	\section{Теорема Гаусса о факториальности кольца многочленов над факториальным кольцом.
		Факториальность колец $K[x_1, ..., x_n], \Z[x_1, ..., x_n]$}
		\begin{Theorem}
			\[R \text{ - факториальное кольцо } \Ra R[x] \text{ - факториальное}\]
		\end{Theorem}
		\begin{Lemma}[Гаусса]
			\[f, g \in R[x] \q f,g \text{ - примитивны } \Ra f \cdot g \text{ - примитивный}\]
		\end{Lemma}
		\begin{Consequence}
			\[\Z[x_1, ..., x_n], K[x_1, ..., x_n] \text{ - факториальны}\]
		\end{Consequence}


	\section{Неприводимость над $\Q $ и над $\Z$. Методы доказательства неприводимости многочленов с целыми коэффициентами
		(редукция по одному или нескольким простым модулям).}
		\[f \in \Q[x]\]
		\[\text{Хотим доказать, что } f \text{ неприв над } \Q\]
		\[\text{Не умоляя общности } f \in \Z[x] \text{ (можно домножить на знаменатель)}\]
		\[\text{cont}(f) = 1 \q \text{коэфф. в совокупности вз. просты}\]
		Идея:
		\[f = a_0 + ... + a_n x^n\]
		\[p \text{ - простое } p \nmid a_n\]
		\[\Z[x] \to \Z_{/p}\Z[x] \]
		каждый коэфф. заменяем на соотв. вычет
		\[f \to \ol{f} = [a_0] + ... + [a_n] \cdot x^n\]
		\[\text{Если } p \nmid a_n \q \deg(\ol{f}) = \deg f\]
		\[\text{Если } f \text{ приводим над } \Q \text{, то по т. Гаусса}\]
		\[f = gh \q g, h \in \Z[x]\]
		\[\deg g, \deg h < \deg f\]
		\[\ol{f} = \ol{g} \cdot \ol{h}\]
		Если $p$ не делит страш. коэфф $f$, то $p \nmid$ страш. коэфф. $g$ и $h$
		\[\deg \ol{g} = \deg g \q \text{и} \q \deg{\ol{h}} = \deg h\]
		Тогда приводимость $f$ влечет приводимость $\ol{f}$
		\begin{hypothesis}
			\[\text{Если } p \nmid a_n \q f=a_0 + ... + a_n x^n \q\q \text{cont } f = 1\]
			\[\text{и } \ol{f} \text{ - неприводим над } \Z_{/p}\Z \text{, то } f \text{ неприводим над } \Z (\Ra \text{ и над } \Q) \]
		\end{hypothesis}


	\section{Критерий неприводимости Эйзенштейна.}
			\begin{Theorem}
				\[f \in \Z[x] \q f = a_0 + a_1x + ... + a_n x^n \q \text{ cont}(f) = 1\]
				\[p \text{ - простое}\]
				\[\begin{align}
						\text{Если } &* p \nmid a_n\\
									 &* p \mid a_i\\
									 &* p^2 \nmid a_0
					\end{align} \q i=0, ..., n - 1 \text{, то } f \text{ неприводим над } \Z (\Ra \text{ и над } \Q) \]
			\end{Theorem}
			\begin{Proof}
				\[\sqsupset f = gh \q\q g, h \in \Z[x] \q\q \deg g, \deg h < n\]
				\[\ol{f} = \ol{g} \cdot \ol{h}\]
				\[\ol{f} = [a_n]x^n\]
				\[\ol{g} \sim x^m \q \ol{h} \sim x^{n - m} \q 0 < m < n\]
				\[g = b_m x^m + ... + b_0 \q \q b_m \not \vdots \  p, \q b_{m - 1}, ... , b_0 \ \vdots \ p \]
				\[h = c_{n - m}x^{n - m} + ... + c_0  \]
				\[c_{n - m} \ \not \vdots \ p \q\q c_{n-m}, ..., c_0 \ \vdots \ p\]
				\[\text{по усл. } \us{\us{p^2}{\not \dotsb}}{a_0} = \us{\us{p}{\dotsb}}{b_0} \cdot \us{\us{p}{\dotsb}}{c_0}
				\text{ - противоречие}\]
			\end{Proof}


  \section{Рациональные корни многочлена с целыми коэффициентами.}

  \begin{Theorem}
	  \[f \in \Z[x]\]
	  \[f = a_0 + ... + a_n x^n \qq a_i \in \Z\]
	  \[a_n \neq 0 \q a_0 \neq 0\]
	  \[\text{Если некор. дробь } \frac{p}{q} \text{ - корень } f \text{, то } q \mid a_n ; \q p \mid a_0\]
  \end{Theorem}

  \begin{Proof}
	  \[f\left(\frac{p}{q}\right) = 0\]
	  \[a_0 + a_1 \frac{p}{q} + ... + a_n \frac{p^n}{q^n} = 0\]
	  \[q^n a_0 + a_1 pq^{n-1} + ... + a_{n - 1} p^{n-1}q  + a_n p^n = 0\]
  \end{Proof}

  \section{Верхняя оценка модуля корня многочлена с комплексными коэффициентами.}


  \section{Симметрические функции. Коэффициенты многочлена из C[x] как симметрические функции корней.}


  \section{Алгоритм разложения на неприводимые множители многочлена с целыми коэффициентами.}


	\section{Линейные отображения векторных пространств. Линейное отображение \\полностью задается своими значениями на базисных векторах.}
			\begin{Definition}
					\[K \text{ - поле } \q\q V \text{ - в.п. над K}\]
					\[f: U \to V \q\q f \text{ - линейное, если } \forall u_1, u_2 \in U \q \forall \alpha_1, \alpha_2 \in K\]
					\begin{enumerate}
						\item \[f(\alpha u_1 + \alpha u_2) = \alpha_1 f(u_1) + \alpha_2 f(u_2)\]
						\item \begin{enumerate}
							\item \[\forall u_1, u_2 \in U \q\q f(u_1 + u_2) = f(u_1) + f(u_2)\]
							\item \[\forall u \in U \q \forall \alpha \in K \q f(\alpha u) = \alpha f(u)\]
						\end{enumerate}
					\end{enumerate}
					лин. отобр $\equiv$ гомеоморфизм вект пр-в
			\end{Definition}
			\begin{Theorem} [св-ва]
					\[f \text{ - лин. отобр. }\]
					\[f(0_u) = 0_v\]
					\[f(-u) = - f(u)\]
			\end{Theorem}
			\begin{Example}
				\[K[x] \to K[x]\]
				\[f \to f'\]
      \end{Example}

			\begin{Utv}
				\[U \text{ - в.п } \q\q \{u_i\}_{i \in I} \text{ - базис } U \]
				Достаточно задать лин. отобр. на базисных векторах
				\[f \text{ - лин. отобр } \q f: U \to V\]
				\[u \in U \q u = \sum \alpha_i u_i\]
				\[f(u) = f(\sum \alpha_i u_i) = \us{\alpha_i \neq 0}{f(\sum \alpha_i u_i)} = \us{\alpha_i \neq 0}{\sum \alpha_i f(u_i)}\]
			\end{Utv}

	\section{Сумма линейных отображений, умножение на скаляр. Пространство линейных отображений.}
    \begin{Utv}
        \[\text{пусть задано отобр. } \q h : \us{\text{базис}}{\{u_i\}_{i \in I} }\to  V \]
        \[\exists \text{ единств. лин. отобр. } f : U \to V \text{, такое что } \forall i \in I \q f(u_i) = h(u_i)\]
    \end{Utv}

    \begin{Definition}
      \[U, V \text{ - в.п. над } K\]
      \[L(U, V) \text{ - мн-во всех линейных отобр. из } U \text{ в } V\]
      \[+: L(U, V) + L(U, V) \to L(U, V)\]
      \[*: K \times L(U, V) \to L(U, V)\]
    \end{Definition}

    \begin{Theorem}
      \[L(U, V) \text{ - век. пр-во над } K\]
    \end{Theorem}


	\section{Матрица линейного отображения для данных базисов. Матрица суммы отображений. Изоморфизм
		пространства линейных отображений и \\ пространства матриц.}
    \[\dim U = m < \infty \q\q \dim V = n < \infty\]
    \[u_1, ..., u_m \text{ - базис } U; \q v_1, ..., v_n \text{ - базис }V\]
    \[f : U \to V \text{ - лин. отобр.}\]
    \[f(u_j) = \sum^{n}_{i=1} a_{ij} v_i\]
    \[A = (a_{ij}) = \begin{pmatrix}
      a_{11} & a_{1j} \\
      a_{21} & \ddots\\
           & a_{nj} 		& a_{nm}
    \end{pmatrix}\]
    \[a_{1j} \text{ - коэфф разложения } f(u_j) \text{ по базису } \{v_1, ..., v_n\}\]
    \[A \text{ - матрица лин. отобр в базисах } \{u_1, ..., u_m\}, \{v_1, ..., v_n\}\]
    \[\begin{align}
      A = [f] &\{u_j\}\\
          &\{v_j\}
    \end{align}\]
    \[f(u) = c_1 f(u_1) + ... + c_m f(u_m) = \sum^{m}_{j=1} c_j f(u_j) = \]
    \[= \sum^{m}_{j=1} c_j \sum^{n}_{i=1} a_{ij} v_i = \sum^{n}_{i=1} ( \sum^{m}_{j=1} c_j a_{ij})v_i\]
    \[\text{где } u = c_1 u_1 + ... + c_m u_m\]
    \[\begin{pmatrix}
      c_1\\
      ...\\
      c_m
    \end{pmatrix}
    = [u]_{\{u_i\}} \q\q [v]_{\{v_i\}} = A \cdot [u]_{\{u_i\}}
    \]
    \[\begin{align*}
      [f + g]  &_{\{u_j\}}& = [f]&_{\{u_j\}}& +\q [g]&_{\{u_j\}}\\
           &_{\{v_i\}}&      &_{\{v_i\}}&     &_{\{v_i\}}
    \end{align*}\]
  \begin{Definition}

        \[\begin{align}
      U,V \text{ назыв. изоморфными, если } \exists f: U \to V \q & 1) f \text{ - лин.}\\
                                      & 2) f \text{ - биекция}
    \end{align}\]
  \end{Definition}


	\section{Композиция линейных отображений. Матрица композиции.}
		\begin{definition}
      \begin{Hypothesis}
      \[u_1, ..., u_m \q v_1, ..., v_n \q w_1, ..., w_k \text{ - базисы}\]
      \[[gf]_{\us{\{w_k\}}{\{u_i\}}} = [g]_{\us{\{w_k\}}{\{v_j\}}}  [f]_{\us{\{v_j\}}{\{u_i\}}} \]
  \end{Hypothesis}

  \begin{Proof}
        \[i \text{ - ый столбец } [gf] \text{ - это коорд. } (gf)(u_i) \text{ в базисе } \{w_1, ..., w_k\}\]
          \[f(u_i) \text{ - коорд. этого вектора в базисе } v_1, ..., v_n \text{ - это }i\text{ - ый столбец матрицы } [f]\]
        \[[gf(u_i)]_{\{w\}}   \text{ - это } i \text{ - ый столбец }[gf]\]
        \[[gf(u_i)]_{\{w\}} = [g] - i \text{ - ый столбец матр. } [f] = [g][f(u_i)]_{\{v_j\}} \]
        \[\text{т.о. } [gf] = [g][f]\]
      \end{Proof}
		\end{definition}


	\section{Преобразование матрицы линейного отображения при замене базисов.}
			\begin{Definition}
					\[f: U \to V \text{ - лин}\]
					\[\begin{align}
							u_1, ..., u_m\\
							u_1', ..., u_m'
					\end{align} \text{ - базисы } U
					\q\q
					\begin{align}
							v_1, ..., v_n\\
							v_1', ..., v_n'
					\end{align}
					\text{ - базисы } V
				   \]

				   \[\begin{align}
					   A = [f]&_{\{u_i\}}& \q\q A' = [f]&_{\{u_i'\}} \\
							  &_{\{v_j\}}& 			   &_{\{v_j'\}}
				   \end{align}\]
				   \\
				   $C $ - матрица замены координат при переходе от $\{u_i\}$ к $\{u_i'\}$\\
				   $D $ - матрица замены координат при переходе от $\{v_j\}$ к $\{v_j'\}$\\
				   $i $ - ый столбец $C $ - это коорд. $u_i'$ в базисе $u_1, ..., u_m$\\
				   $i $ - ый столбец $D $ - это коорд. $v_j'$ в базисе $v_1, ..., v_k$

				   \[[u]_{\{u_i\}} = C[u]_{\{u_i'\}}\text{, аналогично для }D  \]
			\end{Definition}
			\begin{Theorem}
					\[A' = D^{-1}AC \]
			\end{Theorem}


	\section{Ядро и образ линейного отображения, их свойства. Критерий инъективности и
		сюръективности линейного отображения в терминах ядра и образа.}
			\begin{Definition}
				\[f : U \to  V \q f \text{ - лин.}\]
				\[f(U) = \{v \in V \mid \exists u \in U : v = f(u)\} = Im f \text{ (образ f)}\]
				\[f^{-1} (\{0_v\}) = \{u \in U : f(u) = 0_v\} = \ker f \text{ (ядро f)}\]
			\end{Definition}

			\begin{Hypothesis}
					\[Im f \subseteq V; \q \ker f \subseteq U\]
			\end{Hypothesis}

			\begin{Hypothesis}
				\[\text{а) лин. отобр. } f: U \to V \text{ сюръективно } \rla Im f = V\]
				\[\text{б) инъективно } \rla \ker f = \{0_u\}\]
			\end{Hypothesis}


	\section{Выбор базисов, для которых матрица линейного отображения имеет почти единичный вид. Следствие для
		матриц. Теорема о размерности ядра и образа.}
			\begin{Theorem}
					\[U, V \text{ - конечномерные; } f: U \to V \text{ - лин. Тогда } \exists \text{ базисы пр-в } U \text{ и } V,\]
					\[\text{в которых матрица f - почти единичная}\]
					\[\begin{align}
						[f]&_{\{u_i\}}\\
						   &_{\{v_j\}}
					\end{align} =
					\begin{pmatrix}
						E_2 & 0\\
						0	& 0
					\end{pmatrix}\]
			\end{Theorem}

			\begin{Consequence} [1]
				\[A \in M(n, m, K) \text{ Тогда } \exists \text{ обрат. матрицы } C \in M(m, n, K) \text{ и } \]
				\[D \in M(n, m, K) \text{, такие, что } D^{-1} AC = \begin{pmatrix}
					E_2 & 0\\
					0   & 0
				\end{pmatrix}\]
			\end{Consequence}

			\begin{Consequence} [2]
					\[\dim U < \infty; \q V \text{ - произв.}\]
					\[f: U \to V\]
					\[\text{Тогда } \dim U = \dim \ker f + \dim Im f\]
			\end{Consequence}


	\section{Критерий изоморфности конечномерных пространств}
		\begin{Definition}
			\[U, V \text{ изоморфны, есди } \exists \text{ биект. лин. отображение (изоморфизм)} f: U \to V\]
			\[U \cong V\]
		\end{Definition}
		\begin{Theorem}
				\[U, V \text{ - конечномерные в.п. над }K\]
				\[U \cong V \rla \dim U = \dim V\]
		\end{Theorem}
		\begin{Proof}
				\[\Ra f: U \to  V, \q f \text{ - биекция, лин.}\]
				\[f \text{ - инъект. } \Ra \ker f = \{0\}\]
				\[f \text{ - сюръект. } \Ra Im f = V\]
				\[\dim V = \dim Im f = \dim U - \dim \ker f = \dim U - 0 = \dim U \]
				\[\La\dim U = \dim V = n\]
				\[u_1, ..., u_n \text{ - базис } U\]
				\[v_1, ..., v_n \text{ - базис } V\]
				\[\text{Любой } u \in U \text{ единственным образом раскладывается в сумму }\]
				\[u = \alpha_1 u_1 + ... \alpha_n u_n \q \alpha_i \in K\]
				\[f(u) = \alpha_1 v_1 + ... + \alpha_n v_n\]
				\[\widetilde{u} = \widetilde{\alpha_1}u_1 + ... + \widetilde{\alpha_n}u_n\]
				\[u + \widetilde{u} = (\alpha_1 + \widetilde{\alpha})u_1 + ... + (\alpha_n + \widetilde{\alpha_n})u_n\]
				\[f(\widetilde{u}) = \widetilde{\alpha}_1 v_1 + ... + \widetilde{\alpha}_n v_n\]
				\[f(u + \widetilde{u}) = (\alpha_1 + \widetilde{\alpha}_1) v_1 + ... + (\alpha_n + \widetilde{\alpha}_n) v_n\]
				\[f(u + \widetilde{u}) = f(u) + f(\widetilde{u})\]
				\[\text{Аналогично } f(\alpha u) = \alpha f(u)\]
				\[\text{Значит } f \text{ - лин. отобр}\]
				\[\text{т.к. } v_1, ..., v_2 \text{ - сем-во образующих } \Ra f \text{ - сюръект.}\]
				\[v \in V \q v = \alpha_1 v_1 + ... + \alpha_n v_n\]
				\[u = \alpha_1 u_1 + ... + \alpha_n u_n \q f(u) = v\]
				\[\text{т.к. } v_1, ..., v_n \text{ - ЛНЗ, то f - инъект.}\]
				\[\text{достаточно проверить, что } \ker f = \{0\}\]
				\[u = \alpha_1 u_1 + ... + \alpha_n u_n\]
				\[0 = f(u) = \alpha_1 v_1 + ... + \alpha_n v_n \Ra \alpha_1, ..., \alpha_n = 0, u = 0 \Ra \ker f = \{0\}\]
				\[\Ra f \text{ - изоморфизм}\]
		\end{Proof}


	\section{Двойственное пространство. Двойственный базис. Изоморфность конечномерного пространства и его двойственного.
		Пример пространства не изоморфного своему двойственному.}
			\begin{Definition}
					\[V \text{ - в.п. над } K\]
					\[V^* = L(V, K) \text{ - двойственное пр-во к } V\]
					\[(\text{пр-во линейных отображений из V \text{ в } K } )\]
					\[\text{элементы } V^* \text{ - лин. функционалы } V \text{ (лин. отобр)}\]
			\end{Definition}

			\begin{Example}
					\[V_\R = C([0;1] \to \R)\]
					\[f \to  \int_{0}^1 f(x)dx\]
					\[a \in [0; 1] \q f \to f(a)\]
			\end{Example}

			\begin{Definition}
				\[e_1, ..., e_n \text{ - базис }V\]
				\[c_1, ..., c_n \text{ - двойственнй базис } V \text{, если}\]
				\[f(e_i, c_j) = \left\{ \begin{align}
						&1 & i = j\\
						&0 & i \neq j
				\end{align}\]
			\end{Definition}

			\begin{Theorem}
					\[\dim V = n < \infty \Ra V^* \cong V\]
			\end{Theorem}
			\begin{Proof}
					\[v_1, ..., v_n \text{ - базис } V\]
			\end{Proof}

  \section{Каноническое отождествление конечномерного пространства со вторым двойственным.}

	\section{Линейные операторы. Кольцо линейных операторов. Изоморфность кольца линейных операторов и кольца матриц.}
			\[V \text{ - в.п. над } K\]
			\[L(v, v) \text{ эл-ты этого пр-ва назыв. линейными операторами на }V\]
			\[End(V) = L(V, V)\]
			\[\text{На } End(V) \text{ определена композиция (умножение операторов)}\]
			\[\sqsupset \dim V = n\]
			\[\text{зафиксируем базис } v_1, ..., v_n \text{ пр-ва } V\]
			\[End(V) \to M_n (K) \text{ изморфизм в.п.}\]
			\[f \to [f]_{\{v_i\}} \text{ - матрица оператора в базисе} \]
			\begin{theorem}
				\[(End(V), \cdot, +) \text{ - кольцо}\]
			\end{theorem}


	\section{Многочлены от оператора. Коммутирование многочленов от одного оператора.}
	\begin{Definition}

			\[V \text{ - в.п. над } K \q\q \varphi \in End(V)\]
		\[h = a_0 + a_1 t + .... a_m t^m \in K[t]\]
		\[h(\varphi) = a_0 id + a_1 \varphi + ... + a_m \varphi^m \in End(V)\]
		Умножение = композиция операторов
		\[A \in M_n(K)\]
		\[h(A) = a_0 E + a_1 A + ... + a_m A^m \text{ - мн-н от матрицы}\]
		\[(hg)(\varphi) = h(\varphi) \cdot g(\varphi)\]
	\end{Definition}


	\section{Характеристический многочлен матрицы и оператора. Независимость характеристического многочлена оператора от выбора базиса.}
			\begin{Definition}
				\[A \in M_n(K)\]
				Характеристический многочлен A
				\[\det (A - tE) = \mathcal{X}_A(t)\]

				\[\begin{vmatrix}
					&a_{11} - t & a_{12} & ... & a_{1n}&\\
					&a_{21}     & \ddots&\\
					&\vdots     &        & \ddots&\\
					&a_{n1}    &         &     & a_{nn} - t &
				\end{vmatrix}
				= (-1)^n t^n + (-1)^{n - 1} (a_{11} + ... + a_{nn}) t^{n - 1} + ... + \det A
				\]

				\[V \text{ - в.п.} \q \dim V = n < \infty \q v_1, ..., v_n \text{ - базис } V\]
				\[f \in \text{End} \ (V) \q A = [f]_{\{v_i\}}  \text{ - матрица оператора в базисе } v_1, ..., v_n \]
				\[\mathcal{X}_f(t) = \mathcal{X}_A(t)\]
			\end{Definition}

			\begin{lemma}
				Характеристический многочлен f не зависит от выбора базиса в V
			\end{lemma}

			\begin{Proof}
			    \[\begin{align}
			    		v_1, ..., v_n\\
						v_1', ..., v'_n
			    \end{align} \text{ - базисы } V \q\q \begin{align}
				C &\text{ - матрица преобр. координат }\\
				  &\text{при переходе от } \{v_i\} к \{v_i'\}
			    \end{align}\]
				\[A = [f]_{\{v_i\}} \]
				\[A' = [f]_{\{v_i'\}} \]
				\[A' = C'AC \q (A \text{ и } A' \text{ сократимы при помощи } C)\]
				\[? \mathcal{X}_{A'}(t) = \mathcal{X}_A(t) \]
				\[\mathcal{X}_{A'}(t) = \det(C^{-1}AC - tE)  = \det (C^{-1}AC - C^{-1}(tE)C) = \]
				\[ = \det(C^{-1}(A - tE)C) = \det(C^{-1}) \cdot \det(A - tE) \cdot \det(C) = \]
				\[= \det(A - tE) = \mathcal{X}_A(t)\]
			\end{Proof}


	\section{Собственные числа и собственные векторы оператора и матрицы.\\
		Собственные числа как корни характеристического многочлена}

		\begin{Definition}
			\[f \in \text{End}(V) \q \lambda \in K\]
			\[\lambda \text{ - собственное число } f, \text{ если } \exists v \neq 0; \q v \in V : f(v) =
			\lambda \cdot v\]
			\[\text{Если } \lambda \text{ - собс. число } f \q v \in  V \q f(v) = \lambda v \text{, то } v
			\text{ - собс вектор}\]
		\end{Definition}

		\begin{Definition}
			\[\lambda \text{ - с.ч. } f \Ra V_\lambda = \{v : f(v) = \lambda v\}\]
			Поэтому удобно 0 считать с.в.
		\end{Definition}

		\begin{Definition}
			\[A \in M_n(K)\]
			\[\lambda \text{ - с.ч } A \text{, если } \exists v \neq \begin{pmatrix}
				0\\
				\vdots\\
				0
			\end{pmatrix} \in K^n : A_n = \lambda_n\]
		\end{Definition}

		\begin{Theorem}
			\[A \in M_n(K)\]
			\[\lambda \in  K \text{ - с.ч. }A \rla \lambda  \text{ - корень } \mathcal{X}_A(t)\]
		\end{Theorem}

		\begin{Proof}
		    \[\exists v \neq 0 \q Av = \lambda v\]
			\[\left(A - \lambda E\right) v = 0\]
			Рассмотрим коэф. столбца V как неизвестные
			\[\lambda \text{ - с.ч. }A \rla (A - \lambda E)v = 0 \text{ - имеет нетривиальный ранг} \]
			\[\rla \det(A - \lambda E) = 0 \rla \mathcal{X}_A(\lambda) = 0 \rla \lambda \text{ - корень }
			\mathcal{X}_A(t)\]
		\end{Proof}

		\begin{Consequence}
			\[\dim V = n < \infty \q f \in \End(V)\]
			\[\lambda \in K \text{ - с.ч. } f \rla \lambda \text{ - корень } \mathcal{X}_f(t)\]
		\end{Consequence}

		\begin{Proof}
		    \[\text{Фиксируем базис } v_1, ..., v_n\]
			\[f \ra [f] = A \q\q v \ra \begin{pmatrix}
				a_1\\
				\vdots\\
				a_n
			\end{pmatrix} = [v]\]
			\[\rla v \text{ - с.в. } f \text{, отвеч. } \lambda \q\q \begin{pmatrix}
				a_1\\
				\vdots\\
				a_n
			\end{pmatrix} \text{ - с.в. } A \text{, отвеч. } A\]
		\end{Proof}


	\section{Теорема Гамильтона-Кэли.}
			\begin{Theorem}
				\[A \in M_n(K) \q \mathcal{X}_A(A) = O_{M_{n}(K) } \]
			\end{Theorem}


	\section{Диагонализируемые операторы. Критерий диагонализируемости.\\
		Примеры недиагонализируемых операторов}
			\begin{Definition}
			    \[V \text{ - в.п. над } K \q \dim V = n < \infty\]
				\[\varphi \in \End(V)\]
				\[\varphi \text{ - диагонализируем, если } \exists \text{ базис } V, \text{ в котором матрица }
				\varphi \text{ - диагональна}\]
			\end{Definition}

			\begin{Theorem}
					\[V \text{ - в.п. } \q \dim V = n < \infty\]
					\[\varphi \in \End(V)\]
					\[\varphi \text{ - диагонализируем } \rla \exists \text{ базис } V \text{, состоящий из собс.
					векторов }\varphi\]
			\end{Theorem}

			\begin{Proof}
			    \[\Ra v_1, ..., v_n \text{ - базис}\]
				\[[\varphi] _{\{v_i\}} = \begin{pmatrix}
					\lambda_1 &        & 0\\
					          & \ddots    \\
					0         &        & \lambda_n
				\end{pmatrix} \]
				\[\varphi(v_i) = \lambda_i v_i \q v_i \neq 0 \Ra v_i \text{ - с.в.}\]
				\[\La v_1, ..., v_m \text{ - базис из с. в. } \varphi\]
				\[\varphi(v_i) = \lambda_i v_i \q \lambda \in K\]
				\[\varphi(v_i) = 0 \cdot v_1 + ... + 0 \cdot v_{i - 1} + \lambda_i v_i +
				0 \cdot v_{i + 1} + ... \]
				\[[\varphi]_{\{v_i\}} = \begin{pmatrix}
					\lambda_1 & 	  & 0\\
							  &\ddots &\\
					0 		  & 	  & \lambda_n
				\end{pmatrix} \]
			\end{Proof}

			\begin{Example}
					\[V = \CC^2\]
					\[\varphi(x) = A \cdot x \q\q A = \begin{pmatrix}
						0 & 1\\
						0 & 0
					\end{pmatrix}\]
					\[\mathcal{X}_{\varphi}(t) = \mathcal{X}_A(t) = t^2 \q \text{ c.ч } \lambda = 0 \]
					\[Ax = 0\]
					\[\rk A = 1 \q \text{2 перем } \Ra \text{ пр-во решений одномерно}\]
					$\Ra$ все с.в. лежат в одномерном пр-ве $\Ra$ непорожд $\CC^2$ \\
					$\Ra$ не диагонализ.
			\end{Example}

			\begin{Example}
				\[V = K[x]_n = \{f \in K[x];\  \deg f \leq n\}\]
				\[\Char K = 0\]
				\[\varphi = \frac{\partial }{\partial x} \q\q \varphi(f) = f'\]
				\[\text{с.ч. } \lambda = 0\]
				\[\text{с.в. пр. : константы}\]
				\[\dim V = n + 1 \q (n \geq 1 \Ra \varphi \text{ - не диагонализ})\]
			\end{Example}


  \section{Инвариантные подпространства. Матрице линейного оператора, действующего на пространстве, разложенном в прямую сумму инвариантных подпространств.}


  \section{Инвариантность ядра и образа многочлена от оператора.}


  \section{Теорема о разложении $\Ker(f g)(\phi)$ в прямую сумму инвариантных подпространств и следствия из неё.}

	\section{Жорданова форма оператора. Жорданов базис. Формулировка теоремы о жордановой форме оператора. Сведение к случаю оператора с единственным собственным числом.}
		\begin{Definition}
				\[\lambda \in K\]
				\[\mathfrak{J}(\lambda) = \begin{pmatrix}
					\lambda & & & 0\\
					1       & \ddots &\\
					        & \ddots & \ddots\\
					0 & & 1 &\lambda
				\end{pmatrix} \text{- жордан. клетка размера } n \text{ отвечающей }\lambda \]
				\[A \text{ - жорд. матрица, если }A \text{ - блочно диаг, а диг. блоки - жорд. клетки}\]
				\[\mathfrak{J}_1 = (\lambda)\]
				\[\mathfrak{J}_2 = \begin{pmatrix}
					\lambda & 0\\
					0       & \lambda
				\end{pmatrix}\]
				\[A = \begin{pmatrix}
					\mathfrak{J}_{m1}(\lambda_1) & & & 0\\
					& \mathfrak{J}_{m2}(\lambda_2)\\
					& &  \ddots &\\
					0 & & & \mathfrak{J}_{mk}(\lambda_k)
				\end{pmatrix}\]
		\end{Definition}

		\begin{Theorem} [1]
				\[K \text{ - алг. замк.} \q V, \q \dim V = n  < \infty\]
				\[\varphi \in \End(V)\]
				Тогда  $\exists$ базис пр-ва V,  в котором матрица $ \varphi$
				является жордановой матрицей.
				Причем клетки опред. однозначно с точностью до перестановки диаг. блоков
		\end{Theorem}

  \section{Относительная линейная независимость. Относительные базисы. Корневые пространства. Лемма о спуске для корневых подпространств.}


  \section{Построение жорданова базиса и жордановой формы для оператора с единственным собственным числом.}


  \section{Единственность жордановой формы оператора.}

\end{document}
