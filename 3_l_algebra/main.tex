\documentclass[11pt]{article}
\usepackage[english,russian]{babel}
\usepackage{url}
%для поддержки русского
\usepackage{graphicx,DCCN2019_ru}
%пакет с нужными мне штуками
%для красивых символов
\usepackage{upgreek}
\usepackage{dsfont}
\usepackage{amssymb}
\usepackage{tipa}
\usepackage{ wasysym }
\usepackage{gensymb} %градусы
%зачеркивание текста
\usepackage{cancel}
%для цветного текста
\usepackage[usenames]{color}
%для ссылок
\usepackage{xcolor}
\usepackage{hyperref}
% Цвета для гиперссылок
\definecolor{linkcolor}{HTML}{0000ff} % цвет ссылок
\definecolor{urlcolor}{HTML}{0000ff} % цвет гиперссылок
\hypersetup{pdfstartview=FitH,  linkcolor=linkcolor,urlcolor=urlcolor, colorlinks=true}
%Для вставки картинок
\graphicspath{{pictures/}}
\DeclareGraphicsExtensions{.pdf,.png,.jpg}

%команды
\usepackage{amsmath,amsthm,amssymb,amsfonts, enumitem, fancyhdr, color, comment, graphicx, environ}

%настройки текста
\pagestyle{fancy} 
\fancyhead{} 
\fancyfoot{} 
\usepackage[utf8]{inputenc}

%настройки страницы
\makeatletter
\fancyhead[R]{\small Павел Костин}
\fancyhead[L]{\small Матмех СПбГУ}
\fancyhead[C]{\small Билеты, мат. анализ, 2 семестр, 2019}
\pagestyle{fancy}
\fancyfoot[R]{\thepage}

%команды
\usepackage{amsmath,amsthm,amssymb,amsfonts, enumitem, fancyhdr, color, comment, graphicx, environ}
%римские цифры
\newcommand{\RNumb}[1]{\uppercase\expandafter{\romannumeral #1\relax}}
%команды для ускорения набора
\newcommand{\R}{\mathds{R}}
\newcommand{\Q}{\mathds{Q}}
\newcommand{\Z}{\mathbb{Z}}
\newcommand{\B}{\mathcal{B}}
\newcommand{\CC}{\mathds{C}}
\newcommand{\N}{\mathds{N}}
\newcommand{\ra}{\Rightarrow}
\newcommand{\la}{\Leftarrow}
\newcommand{\rla}{\Leftrightarrow}
\newcommand{\lra}{\Leftrightarrow}
\newcommand{\e}{\exists}
\newcommand{\E}{\mathcal{E}}
\newcommand{\q}{\quad}
\newcommand{\devides}{\mathop{\raisebox{-2pt}{\vdots}}}

%вёрстка
\newenvironment{solutions}[1][]
{\begin{trivlist}\item{\underline{\bfseries #1}}}{\end{trivlist}\newpage}
\newenvironment{definition}[1][Опр.]
{\begin{trivlist}\item{\underline{\bfseries #1} }}{\end{trivlist}}
\newenvironment{definition2}[2][Опр]
{\begin{trivlist}\item[\underline{{\bfseries #1}} {\bfseries #2.}]}{\end{trivlist}}
\newenvironment{instance}[1][Пример.]
{\begin{trivlist}\item{\underline{\bfseries #1} }}{\end{trivlist}}
\newenvironment{instances}[1][Примеры.]
{\begin{trivlist}\item{\underline{\bfseries #1} }}{\end{trivlist}}
\newenvironment{statement}[1][Утв.]
{\begin{trivlist}\item{\underline{\bfseries #1} }}{\end{trivlist}}
\newenvironment{lemma}[1][Лемма.]
{\begin{trivlist}\item{\underline{\bfseries #1} }}{\end{trivlist}}
\newenvironment{lemma2}[2][Лемма]
{\begin{trivlist}\item[\underline{{\bfseries #1}} {\bfseries #2.}] \hspace{0pt} \\}{\end{trivlist}}
\newenvironment{comments}[1][Замечание.]
{\begin{trivlist}\item{\underline{\bfseries #1} }}{\end{trivlist}}
\newenvironment{theorem}[1][Теорема.]
{\begin{trivlist}\item{\underline{\bfseries #1} }}{\end{trivlist}}
\newenvironment{theorem2}[2][Теорема]
{\begin{trivlist}\item[\underline{{\bfseries #1}} {\bfseries #2.}] \hspace{0pt} \\}{\end{trivlist}}
\newenvironment{reminder}[2][Напоминание]
{\begin{trivlist}\item[\underline{{\bfseries #1}} {\bfseries #2:}] \hspace{0pt} \\}{\end{trivlist}}
\newenvironment{proofs}[1][Доказательство.]
{\begin{trivlist}\item{\bfseries #1} }{\end{trivlist}}
\newenvironment{proofs2}[2][Доказательство]
{\begin{trivlist}\item[\underline{{\bfseries #1}} {\bfseries #2.}]\hspace{0pt}} {\end{trivlist}}
\newenvironment{proofByDisagreement}[1][Доказательство (от противного). ]
{\begin{trivlist}\item{\bfseries #1}}{\end{trivlist}}
\newenvironment{proofByInduction}[1][Доказательство (по индукции). ]
{\begin{trivlist}\item{\bfseries #1}}{\end{trivlist}}
\newenvironment{properties}[1][Свойство.]
{\begin{trivlist}\item{\underline{\bfseries #1} }}{\end{trivlist}}
\newenvironment{properties2}[2][Свойства]
{\begin{trivlist}\item[\underline{{\bfseries #1}} {\bfseries #2.}]\hspace{0pt}} {\end{trivlist}}
\newenvironment{consequence}[1][Cледствие.]
{\begin{trivlist}\item{\underline{\bfseries #1} }}{\end{trivlist}}
\newenvironment{consequence2}[2][Cледствие]
{\begin{trivlist}\item[\underline{\bfseries #1} {\bfseries #2.}]}{\end{trivlist}}

%фикс отступа
\usepackage{tocloft}
\setlength{\cftbeforetoctitleskip}{1em}

%сам документ
\begin{document}
\begin{center}
  \huge Лекции по алгебре
  
  (читает Демченко О. В.)
\end{center}
Данный документ неидеальный, прошу сообщать о найденных недочетах в \href{https://vk.com/drab_existence_a}{вк}
\tableofcontents
\newpage

%билеты
\section{Теория групп}

\begin{definition}
    G - мн-во, $*:G*G \ra G,\ (g_1, g_2) \ra (g_1*g_2)\ (g_1g_2)$
    \begin{enumerate}                               
    	\item $(g_1g_2)g_3 = g_1(g_2g_3) \q \forall g_1, g_2, g_3 \in G$
    	\item $\exists e \in G : eg = ge = g \q \forall g \in G$
    	\item $\forall g \in G \q \exists \widetilde{g} \in G : g\widetilde{g} = g \widetilde{g} = e$
    	\item $g_1g_2 = g_2g_1 \q \forall g_1, g_2 \in G$
	\end{enumerate} 
\end{definition}

\begin{instances}
    \begin{enumerate}  
        \item $(\Z,+)$ - группа
        \item $(\Z, \bullet)$ - не группа
        \item $(R, +)$ - группа кольца
        \item $(R^*, \bullet)$
        \item Группа самосовмещения $D_n$, например $D_4$ - квадрат, композиция - группа, $|D_n|=2n$
        \item $GL_n(K) = \{A \in M_n(K) : |A| \neq 0\}$, умножение - группа
        \item $\Z n \Z$ - частный случай п.3,4
    \end{enumerate} 
\end{instances}

\begin{properties2}{(групп)}
    \begin{enumerate}  
        \item e - единственный, $e,e'$ - нейтральные: $e=e e'=e'$
        \item $\widetilde{g}$ - единственный
        
        Пусть $\widetilde{g},\hat{g}$ - обратные, тогда $\widetilde{g}g = g\widetilde{g} = e = \hat{g}g = g\hat{g}$
        
        $\hat{g}=e \hat{g}=(\widetilde{g}g)\hat{g}=\widetilde{g}(g\hat{g})=\widetilde{g}e=\widetilde{g}$
        \item $(a b)^{-1}=b^{-1}a^{-1}$
        
        Это верно, если $(ab)(b^{-1}a^{-1})=(b^{-1}a^{-1})(ab)=e$, докажем первое:
        
        $(ab)(b^{-1}a^{-1})=((ab)b^{-1})a^{-1}=(a(bb^{-1}))a^{-1}=(ae)a^{-1}=a a^{-1}=e$
        \item $(g^{-1})^{-1}=g$
    \end{enumerate} 
\end{properties2}

\begin{definition}
    $g \in G \q n \in \Z$, тогда $g=
\left[ 
  \begin{gathered} 
    \overbrace{g...g}^n, \q n>0\\
    e, \q n=0\\ 
    \underbrace{g^{-1}...g^{-1}}_n, \q n<0
  \end{gathered} 
\right.$
\end{definition}

\begin{properties2}{(степени)}
    \begin{enumerate}                               
    	\item $g^{n+m}=g^n g^m$
    	\item $(g^n)^m=g^{n m}$
	\end{enumerate} 
\end{properties2}

\begin{definition}
    $g \in G$, $n \in N$ - порядок g $(ord g = n)$, если:
    \begin{enumerate}                               
    	\item $g^n=e$
    	\item $g^m=e$ $\ra$ $m \geqslant n$
	\end{enumerate} 
\end{definition}

\begin{instances}
    \begin{enumerate}                               
    	\item $D_4$ ord(поворот $90\degree) =4$
    	
    	$D_4$ ord(поворот $180\degree) =2$
    	\item $(\Z /6 \Z, +)$ $ord(\overline{1})=6$
    	
    	$ord(\overline{2})=3$
	\end{enumerate} 
\end{instances}

\begin{statement}
    $g^m=e \q ord(g)=n$ $\ra$ $m \devides n$ (n>0)
\end{statement}
\begin{proofs}
    $m=n q+r$, $0 \leqslant r < n$
    $e=g^m=g^{n q + r}=(g^n)^q g^r=g^r$ $\ra$ $r=0$
\end{proofs}

\begin{definition}
    $H \subset G$ называется подгруппой G (H < G) (и сама является группой), если:
    \begin{enumerate}                               
    	\item $g_1,g_2 \in H \ra g_1 g_2 \in H$
    	\item $e \in H$
    	\item $g \in H \ra g^{-1} \in H$
	\end{enumerate} 
\end{definition}

\begin{instances}
    \begin{enumerate}                               
    	\item $n\Z < \Z$
    	\item $D_4$
    	\item $SL_n(K)=\{A \in M_n(K): \q |A|=1\}$, $SL_n(K)<GL_n(K)$
	\end{enumerate} 
\end{instances}

\begin{tabular} {c|c}
	Мультипликативная запись & Аддитивная запись\\ \hline 
	$g_1 g_2$ & $g_1 + g_2$\\
	$e$ & $0$\\
	$g^{-1}$ & $-g$\\
    $g^n$ & $ng$ 
\end{tabular}

\begin{definition}
    $H<G$, $g_1,g_2 \in G$, тогда $g_1 \sim g_2$, если:
    \begin{enumerate}                               
    	\item $g_1=g_2 h$, $h \in H$ (левое)
    	\item $g_2=h g_1$, $h \in H$ (правое)
	\end{enumerate} 
\end{definition}

\begin{proofs2}{(эквивалентности)}
    \begin{enumerate}                               
    	\item (симметричность) $g_1=g_2 h \overset{*h^{-1}}{\ra} g_2 = g_1 h^{-1}$
    	\item (рефлексивность) $g=ge$
    	\item (транзитивнось) $g_1=g_2 h$, $g_2 = g_3 h$ $\ra$ $g_1=g_3(h_2 h_1)$, где $h_2 h_1 \in H$
	\end{enumerate} 
\end{proofs2}

\begin{definition}
    $[a] = \{b:a∼b\}$классы эквивалентности
\end{definition}

\begin{definition}
    $[g] = g H = \{g h, h \in H \}$ (левый класс смежности) 
    
    $g h \sim g \ra g h \in [g]$
    
    $g_1 \in [g] \ra g_1 \sim g \ra g_1 = g h$
\end{definition}

\begin{statement}
    $[e]=H$
    
    Установим биекцию: 
    
    $[g]=gh \leftarrow H$
    
    $gh \leftarrow h$
    
    Очевидно, сюръекция, почему инъекция? $g h_1 = g h_2 \overset{*g^{-1}}{\ra} h_1 = h$
\end{statement}

\begin{theorem2}{(Лагранжа)}
$H < G$, $|G| < \infty$, тогда $|G| \devides |H|$ (уже доказали!)
\end{theorem2}


\end{document}