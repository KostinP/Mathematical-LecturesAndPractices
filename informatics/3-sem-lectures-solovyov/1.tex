\documentclass[main]{subfiles}

\begin{document}
    \section{История создания методов структуризации данных. Цели и принципы структурной методологии}
    История программирования как науки и технологии --- в некотором смысле и история создания методов представления данных. Кратко --- это история постепенного повышения уровня абстракции данных, усложнения структур данных, накопления неструктурированных данных и методов выявления из них новых данных и знаний.\\

    В прошлом часто с понятием искусства программирования связывали понятие стиля программирования. Стиль программ на некотором языке определяют лежащие в его основе содержательные идеи. Это не способ внешнего оформления программ, а стройная система приемов и методов, предназначенная для решения некоторого класса задач из определенной предметной области или задач, объединенных по принципу сходства методов решения.\\

    Сначала этот стиль в основном диктовался представлениями первых компьютерных теоретиков, архитектурой первых вычислительных машин (40-е г) и состоянием техники и компьютерной технологии в целом. Первые машины обладали так называемой архитектурой фон Неймана (один из первых и ведущих разработчиков современных ЭВМ).

    \subsection{Машина фон Неймана}
    Основополагающие свойства архитектуры машины Фон Неймана сформулированы в виде принципов Фон Неймана. Эти принципы многие годы определяли основные черты архитектуры нескольких первых поколений ЭВМ. Машина Фон Неймана состоит из памяти, устройств ввода/вывода и центрального процессора (ЦП). Центральный процессор, в свою очередь, состоит из устройства управления (УУ) и арифметико-логического устройства (АЛУ).

    \subsection{Принципы фон Неймана}
    \begin{enumerate}
        \item Использование двоичной системы счисления в вычислительных машинах.
        \item Программное управление ЭВМ.
        \item Память компьютера используется не только для хранения данных, но и программ.
        \item Ячейки памяти ЭВМ имеют адреса, которые последовательно пронумерованы.
        \item Возможность условного перехода в процессе выполнения программы.
    \end{enumerate}

    \begin{enumerate}
        \item Программы как данные. Фактически, программы в архитектуре фон Неймана сразу трактуются как данные.
        \item Данные --- числовые коды\\
        1-я форма предст данных --- числовые коды.
        \item Бинарные, но почти сразу стали исп 8 и 16 разр коды
        \item "Первый"{} стиль пр-я --- императивный (кавычки --- поскольку никто о стиле тогда еще не говорил) ― ("приказной"{} --- пр-мы сост из прямых инструкций процессору), основные представители --- всевозможные машинные языки, ассемблеры.
    \end{enumerate}

    Следующий этап развития программирования --- поиск усовершенствованных языков --- результат быстрого роста объемов программ.\\

    Первое усовершенствование, остающееся в рамках того же стиля программирования --- Фортран (50е годы - Бэкус): процедурный или процедурно-ориентированный стиль.\\

    Далее, в связи с разработкой больших программных комплексов, возникла потребность в надежности и эффективности проектирования. Пришло время системных преобразований --- от случайных изобретений к разработке методов управления данными.\\

    Акцент в программировании сместился (60е-70е) в сторону усовершенствования организации данных и развития концепции модульности.

    \subsection{Двоякая цель модульности}
    Д. Парнас впервые удачно сформулировал свойства модуля: "Для написания одного модуля должно быть достаточно минимума знаний о тексте другого модуля"{}. Это относ к любой процедуре или функции как нижнему, так и верхнему уровню иерархии (реализ). Т.о. Парнас первым четко выразил идею сокрытия (инкапсуляции) информации в программировании.\\

    Модуль --- новая синтаксическая конструкция, не подверженная влиянию глобальных переменных. Это прототип соврерменных понятий класса и объекта.

    \subsection{Структурное программирование}
    Ведущие специалисты теории и практики программирования — Дейкстра, Вирт, Хоар, и др.
    \begin{enumerate}
        \item Программа — иерархическая совокупность абстрактных уровней
        \item "оператор goto нужно считать вредным"{}
    \end{enumerate}
    И все это для того, чтобы выполнялось основное требование ― контроль правильности данных на стадии создания и на стадии выполнения.\\

    Цели структурного программирования:
    \begin{enumerate}
        \item Дисциплина программирования (программист её себе навязывает сам)
        \item Читабельность текстов программ:
        \begin{enumerate}
            \item Избегание языковых конструкций с неочевидной семантикой;\\
            {if (3 < 2 < 1) printf("чепуха"{}); \\
            else printf("а паскаль это не допускает"{});}
            \item Стремление к локализации действий управляемых конструкций и используемых структур данных
            \item Разрабатываем программу: ее текст можно читать от начала до конца без переходов на другую страницу.
        \end{enumerate}
        \item Эффективность разработки (структурированный код легче отладить)
        \item Надежность программ
        \item Время и стоимость
    \end{enumerate}

    Основные принципы структурной методологии:
    \begin{enumerate}
        \item Абстракция --- представление решения без некоторых деталей; позволяет представить весь пр-й проект в виде нескольких уровней абстракции: верхний --- весь проект (крупн детали), последний уровень --- постепенная детализация;
        \item Формализация
        \item Фрагментация --- принц "разделяй и властвуй"{}
        \item Иерархия --- важен не только факт разбиения, но и его структура; вообще иерархическая структура помогает эффективно \textbf{\textcolor{red}{упр}}; в частности, в программировании иерархическая структура помогает достигать целей структурного программирования.
    \end{enumerate}
    Структурное программирование --- крик души обеспокоенных специалистов, однако рядовые программисты не любят следовать советам.\\

    Структурное программирование помогло на определенном этапе. Однако одна идея не может решать всех проблем в \textbf{\textcolor{red}{экон и прогр}} проектировании --- объемы и сложность ПО только растут.\\

    В то же время структурное программирование, как таблицу умножения, изобрели и используют.
\end{document}
