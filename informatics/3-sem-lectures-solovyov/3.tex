\documentclass[main]{subfiles}

\begin{document}
    \section{Представление полиномов вектором коэффициентов и оценки сложности их вычислений. Представление полиномов вектором значений и оценки сложности их вычислений}

    \begin{definition}
        $K$ --- поле, $a_1,..., a_n \in K$. Тогда $A(x) = \us{i=0}{\os{n-1}{\sum}} a_i x^i$ называется полиномом
    \end{definition}
    Любое целое число, строго большее степени полинома, называется границей степени (degree-bound) данного полинома. Следовательно, степенью полинома с границей степени $n$ может быть любое целое число от $0$ до $n–1$ включительно.\\

    Границу $n$ или количество коэффициентов будем называть длиной (размером) представления.\\

    \subsection{Операции над полиномами}
    \begin{enumerate}
        \item Сложение полиномов
        \item Умножение полиномов
    \end{enumerate}

\end{document}
