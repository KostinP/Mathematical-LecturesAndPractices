\documentclass[12pt, fleqn]{article}

\usepackage{../../../template/template}

 
\begin{document}

\begin{lect}{2019-11-22}
    \begin{Reminder}
        \[K[x]\bigg/_{<f(x)>} \text{ - поле, если } f(x) \text{ - неприв}  \]
        \[a(x) \qq \deg a(x) < \deg f(x)\]
        \[(a(x), f(x)) = 1\]
        \[a(x)b(x) + f(x)c(x) = 1\]
        Если $f$ - приводим, то
        \[f(x) = m(x)n(x)\]
        \[m(x)n(x) = f(x)\]
    \end{Reminder}

    \begin{Example}
        \[R \text{ - поле}\]
        \[\CC - ?\]
        \[\CC \to i^2 = -1\]
        \[\R \times \R \qq a + bi\]
        \[(a + bi)(c + di) = ...\]
        \[x^2 + 1 = 0\]
        \[\text{В } \R \text{ нет корней}\]
        \[\R[x] \bigg/_{<x^2 + 1>} \]
        \[ax + b \ \leftrightarrow \ ai + b\]
        Переход - подставить $i$ вместо $x$

        \[x^2 + 1 \us{x^2 + 1}{\equiv} 0\]
        \[K(\alpha_1) \text{ где } \alpha_1 \text{ корень }f\]
    \end{Example}

    Задачи с предыдущей пары
    \begin{Task}
        \[\Z/_3\Z[x] \bigg/_{<x^2 + 1>} \to \Z/_3\Z(i) \text{, где } i^2 = -1\]
        \[\R \to \CC\]
        \[\begin{tabular}{c|c|c|c|c|c|c|c|c|c}
            & 0 & 1 & 2 & x & 2x & 1 + x & 2 + x & 1 + 2x & 2 + 2x \\
            0 & 0 & 0 & 0 & 0 & 0 & 0 & 0 & 0 & 0\\
            1 & 0 & 1 & 2 & x & 2x & 1 + x & 2 + x & 1 + 2x & 2 + 2x\\
            2 & 0 & 2 & 1 & 2x\\
            x & 0 & x & 2x\\
            2x & 0 & 2x\\
            1 + x & 0 & 1 + x\\
            2 + x & 0 & 2 + x\\
            1 + 2x & 0 & 1 + 2x\\
            2 + 2x & 0 & 2 + 2x
        \end{tabular}\]
    \end{Task}

    \begin{Task}
        \[Q[x] \bigg/_{<x^3 + 2x^2 + 2x + 1>} \]
        доказать, что это поле и найти обратный к $x^2 + x + 1$
    \end{Task}

    \begin{Task}
        \[\text{В поле характеристики $p$ есть подполе изоморфное } \Z/_p\Z\]
        \[K \text{ - есть } 1\]
        \[0 \q 1 \q 1 + 1 \q 1 + 1 + 1 \q 1 + 1 + 1 + 1 \q ... \q \underbrace{1 + ... + 1}_{p - 1} \]
        \[\underbrace]{1 + ... + 1}_m = \underbrace{1 + ... + 1}_n \q m > n\]
        \[\underbrace{1 + ... + 1}_{m - n} = 0 ?! \text{ такого быть не может} \]
        \[(1 + ... + 1) + (1 + ... + 1) = \underbrace{1 + ... + 1}_p + 1 + ... + 1 =0 +  1 + ... + 1 \]
        Проверим замкнутость
        \[\underbrace{(1 + ... + 1)}_m \underbrace{(1 + ... + 1)}_n = \underbrace{1 + ... + 1}_{m + n}  \]
        \[\underbrace{(1 + ... + 1)}_m \underbrace{(1 + ... + 1)}_{\frac{1}{m}}  = 1\]
        по модулю $p$
    \end{Task}

    \begin{Task}
        \[\exists \text{ поле из } pq \text{ эл-тов } p \neq q, \qq p, q \in \mathcal{P}\]
        Характеристика $K$ либо $p$ либо $q$\\
        рассм $(K, +)$ как группу $\abs{K} = pq$
        \[\underbrace{(a + ... + a)}_{pq}  = 0\]
        \[\underbrace{(1 + ... + 1)}_q \underbrace{(a + ... + a)}_p\]
        В $K$ есть подполе из $p$ элементов\\
        $K$ - век пр-во над $\Z/_p\Z$
        $\abs{K} = p^n$
    \end{Task}

    \begin{Reminder}
        \[K \text{ - поле}\]
        \[K^* \qq (K \setminus \{0\}, \cdot) \text{ - группа}\]
    \end{Reminder}

    \begin{utv}
        Мультипликативная группа конечного поля - циклическая
    \end{utv}

    \begin{Proof}
        \[\abs{K} = n + 1 \qq n = p_1^{\alpha_1} \cdot ... \cdot p_n^{\alpha_n}  \]
        \[\text{рассмотрим многочлен } x^{\frac{n}{p_1}} = 0 \Ra \text{ есть не корень } b_1\]
        \[b_1^{\frac{n}{p_1}} \neq 1 \]
        \[b_1^{\frac{n}{p_1^{\alpha_1} }} \text{ - порядок } p_1^{\alpha_1}  \]
        \[b_1^n = 1\]
        \[c_1 \text{ - порядка } p_1^{\alpha_1} \]
        \[c_2 \text{ - порядка } p_2^{\alpha_2} \]
        \[\vdots\]
        \[c_n \text{ - порядка } p_n^{\alpha_n} \]
        Рассмотрим $c_1 \cdot ... \cdot c_n$ - порядок $n$
    \end{Proof}

    \begin{task}
        Разложить на мн-ли в $\Z/_7\Z \q x^8 - x$
    \end{task}

    \begin{task}
        в $\Z/_3\Z[x]\bigg/_{x^3 - x  + 1} $ найти обратный к $x^2 + 1$
    \end{task}

    \begin{task}
        Сколько прим. элементов в поле из $p^k$ элементов\\
        Примитивный элемент - порождающие мультипликативной группы
    \end{task}

    \begin{task}
        Найти прим. элементы в 
        \begin{enumerate}
            \item $\Z/_7\Z$
            \item $\Z/_{11}\Z $
            \item $\Z/_3\Z[x] \bigg/_{<x^2 + 1>} $
        \end{enumerate}
    \end{task}

    \begin{task}
        $(!) \q \Z/_p\Z \q (x + y)^{p^m}  = x^{p^m} + y^{p^m} $
    \end{task}
\end{lect}

\end{document}
