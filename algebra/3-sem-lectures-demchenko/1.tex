\documentclass[main]{subfiles}

\begin{document}
    \section{Теория групп}
    \subsection{Определение группы. Примеры. Простейшие свойства групп. Аддитивная и мультипликативная записи. Прямое произведение групп}

    \begin{definition}
        G --- мн-во, $*: \q G*G \ra G,\ (g_1, g_2) \ra (g_1*g_2)\q (g_1g_2)$
        \begin{enumerate}
        	\item $(g_1g_2)g_3 = g_1(g_2g_3) \q \forall g_1, g_2, g_3 \in G$
        	\item $\exists e \in G : eg = ge = g \q \forall g \in G$
        	\item $\forall g \in G \q \exists \widetilde{g} \in G : g\widetilde{g} = \widetilde{g}g = e$\\
            Тогда $G$ называется \ul{группой}\\
        	\item $g_1g_2 = g_2g_1 \q \forall g_1, g_2 \in G$\\
            Если выполняется п.4, то группа называется \ul{абелевой}
    	\end{enumerate}
    \end{definition}

    \begin{examples}
        \begin{enumerate}
            \item $(\Z,+)$ - группа
            \item $(\Z, \cdot)$ - не группа
            \item $(R, +)$ - группа кольца
            \item $(R^*, \cdot)$
            \item Группа самосовмещения $D_n$, например $D_4$ --- квадрат, композиция --- группа, $|D_n|=2n$
            \item $GL_n(K) = \{A \in M_n(K) : |A| \neq 0\}$
            \[(GL_n(K),\ *) \text{ --- группа}\]
            \item $\Z/ n\Z$ --- частный случай п. 3,4
        \end{enumerate}
    \end{examples}

    \begin{theorem}[простейшие св-ва групп]
        \begin{enumerate}
            \item e --- единственный, \\
                $e,e'$ --- нейтральные: $e=e e'=e'$
            \item $\widetilde{g}$ --- единственный
                Пусть $\widetilde{g},\hat{g}$ --- обратные, тогда:
                \[\widetilde{g}g = g\widetilde{g} = e = \hat{g}g = g\hat{g}\]
                \[\hat{g}=e \hat{g}=(\widetilde{g}g)\hat{g}=\widetilde{g}(g\hat{g})=\widetilde{g}e=\widetilde{g}\]
                \item $(a b)^{-1}=b^{-1}a^{-1}$

                Это верно, если $(ab)(b^{-1}a^{-1})=(b^{-1}a^{-1})(ab)=e$, докажем первое:
                \[(ab)(b^{-1}a^{-1})=((ab)b^{-1})a^{-1}=(a(bb^{-1}))a^{-1}=(ae)a^{-1}=a a^{-1}=e\]
            \item $(g^{-1})^{-1}=g$
        \end{enumerate}
    \end{theorem}

    \begin{tabular} {c|c}
    	Мультипликативная запись & Аддитивная запись\\ \hline
    	$g_1 g_2$ & $g_1 + g_2$\\
    	$e$ & $0$\\
    	$g^{-1}$ & $-g$\\
        $g^n$ & $ng$
    \end{tabular}\\
    Пользуемся аддитивной записью, если знаем, что группа абелева

    \begin{definition}
		Пусть G, H --- группы, рассмотрим
		\[G \times H = \{(g,h): g\in G,\ h\in H\}\]
		Введем операцию
		\[(g,\ h)*(g',\ h')\overset{def}{=}(g*_G g',\ h*_H h')\]
		Докажем, что это группа
	\end{definition}

	\begin{proof}
		Ассоциативность:
        \[((g,\ h)(g',\ h'))(g'',\ h'') \overset{?}{=} (g,\ h)((g',\ h')(g'',\ h''))\]
		\[(gg',\ hh')(g'',\ h'') \overset{?}{=} (g,\ h)(g' g'',\ h' h'')\]
        \[((gg')g'',\ (hh')h'') \overset{?}{=} (g(g'g''),\ h(h'h'')) \text{ --- очевидно}\]
		Нейтральный:
		\[(e_G,\ e_H)(g,\ h) = (e_G g,\ e_H h) = (g,\ h)\]
		Обратный:
		\[(g,\ h)(g^{-1},\ h^{-1}) = (g g^{-1},\ h h^{-1}) = (e_G,\ e_H)\]
	\end{proof}

    \newpage
    \subsection{Определение степени элемента группы и её свойства. Порядок элемента, примеры. Свойства порядка}

    \begin{definition}
        $g \in G \q n \in \Z$, тогда $g^n=
        \left[
          \begin{gathered}
            \overbrace{g...g}^n, \q n>0\\
            e, \q n=0\\
            \underbrace{g^{-1}...g^{-1}}_n, \q n<0
          \end{gathered}
        \right.$
    \end{definition}

    \begin{theorem}[св-ва степени]
        \begin{enumerate}
        	\item $g^{n+m}=g^n g^m$
        	\item $(g^n)^m=g^{n m}$
    	\end{enumerate}
    \end{theorem}

    \begin{definition}
        $g \in G$, $n \in N$ --- \ul{порядок} g $(\ord g = n)$, если:
        \begin{enumerate}
        	\item $g^n=e$
        	\item $g^m=e$ $\RA$ $m \geqslant n$
    	\end{enumerate}
      Порядок может быть бесконечным
    \end{definition}

    \begin{examples}
        \begin{enumerate}
            \item $D_4 \ \ord(\text{поворот } 90\degree) = 4$
                \[D_4$ ord(поворот $180\degree) =2\]
        	\item $(\Z /6 \Z, +)$
          \[\ord(\overline{1})=6,\q \ord(\overline{2})=3\]
    	\end{enumerate}
    \end{examples}

    \begin{Utv}
        \[g^m=e \q ord(g)=n$ $\RA$ $m \devides n \q (n>0)\]
    \end{Utv}

    \begin{Proof}
        \[m=n q+r$,\q $0 \leqslant r < n\]
        \[e=g^m=g^{n q + r}=(g^n)^q g^r=g^r \RA r=0\]
    \end{Proof}

    \newpage
    \subsection{Подгруппы. Примеры. Отношения по подгруппе. Смежные классы. Теорема Лагранжа}

    \begin{definition}
        $H \subset G$ называется \ul{подгруппой} G (H < G) (и сама является группой), если:
        \begin{enumerate}
        	\item $g_1,g_2 \in H \RA g_1 g_2 \in H$
        	\item $e \in H$
        	\item $g \in H \RA g^{-1} \in H$
    	\end{enumerate}
    \end{definition}

    \begin{examples}
        \begin{enumerate}
        	\item $n\Z < \Z$
        	\item $D_4$
        	\item $SL_n(K)=\{A \in M_n(K): \q |A|=1\}$, $SL_n(K)<GL_n(K)$
    	\end{enumerate}
    \end{examples}

    \begin{definition}
        $H<G$,\q $g_1,g_2 \in G$,\q тогда $g_1 \sim g_2$, если:
        \begin{enumerate}
        	\item $g_1=g_2 h$, $h \in H$ (левое отношение)
        	\item $g_2=h g_1$, $h \in H$ (правое отношение)
    	\end{enumerate}
    \end{definition}

    \begin{proof}[эквивалентность]
        \begin{enumerate}
        	\item (симметричность) $g_1=g_2 h \overset{*h^{-1}}{\Ra} g_2 = g_1 h^{-1}$
        	\item (рефлексивность) $g=ge$
        	\item (транзитивнось) $g_1=g_2 h_1$, $g_2 = g_3 h_2$ $\Ra$ $g_1=g_3(h_2 h_1)$, где $h_2 h_1 \in H$
    	\end{enumerate}
    \end{proof}

    \begin{definition}
        $[a] = \{b:a  \sim b\}$ классы эквивалентности
    \end{definition}

    \begin{definition}
        $[g] = g H = \{g h, h \in H \}$ (левый класс смежности)
        \[g h \sim g \RA g h \in [g]\]
        \[g_1 \in [g] \RA g_1 \sim g \RA g_1 = g h\]
    \end{definition}

    \begin{utv}
        $[e]=H$\\
        Установим биекцию:
        \[[g]=gh \leftarrow H\]
        \[gh \leftarrow h \ (*)\]
        Очевидно, сюръекция, почему инъекция?
        \[g h_1 = g h_2 \overset{*g^{-1}}{\ra} h_1 = h_2\]
    \end{utv}

    \begin{theorem}[Лагранжа]
        $H < G$, \q $|G| < \infty$,\q тогда $|G| \devides |H|$ (уже доказали!)
    \end{theorem}

    \begin{proof}
        Действительно, левое $\sim$, получаем классы экв-ты, они равномощны, не пересек. покрыв. всю группу

        (n ящиков по 20 апельсинов $\ra$ $n \devides 20$)

        Из (*) в кл. экв-ти лежит столько же элем., сколько в подгруппе

        Порядок группы --- кол-во эл-ов в ней
    \end{proof}
\end{document}
