\documentclass[main]{subfiles}

\begin{document}
    \begin{Utv}
        \[x^{p^n} - x = \prod_{d \mid n} \text{унитарные непр. мн-ны над } \Z/p\Z \text{ степени } d  \]
    \end{Utv}

    \begin{Example}
        \[p = 2 \qq n = 4\]
        \[x^{16} - x = x(x + \overline{1})(x^2 + x + \overline{1})(x^4 + x^3 + x^2 + x + \overline{1})
        (x^4  +x^3 + \overline{1})(x^4 + x + \overline{1})\]
    \end{Example}

    \begin{Definition}
        \[m_p(d) \text{ --- кол-во непр. унит. мн-нов степ. } d \text{ над } \Z/p\Z \]
        \[p^n = \sum_{d\mid n}m_p(d)d \qq m_p(1) = p \]
    \end{Definition}

    \begin{consequence}
        $\text{Все } \ M_p(d) \text{ --- полож.}$
    \end{consequence}

    \begin{Proof}[следствия]
       \[M_p(d) = m_p(d)d\]
        \[p^n = \sum_{d\mid n} M_p(d) \qq M_p(d) \leq p^d \]
        \[M_p(n) = p^n - \sum_{\os{d|n}{d \neq n}} M_p(d) \geq p^n - \sum_{\os{d|n}{d \neq n}}p^d \geq
        p^n - (p^{n - 1} + p^{n - 2} + ... + p  ) = \]
        \[= p^n - \frac{p^n - p}{p - 1} = \frac{p^{n + 1} - 2p^n + p }{p - 1} > 0\]
    \end{Proof}

    \begin{Utv}[предложение]%это предложение
        \[f \in \Z/p\Z[x] \text{ --- непр. } \deg f = d \]
        \[x^{p^n} - x \devides f \rla n \devides d\]
    \end{Utv}


    \begin{proof}
        Хотим, чтобы $g_1 ... g_k = \us{d|k}{\prod}...$\\
        $(\Ra)$:\\
        $x^{p^n} - x$ разложим в произв. непр. по осн. теореме арифм. Домножим на $\const$, чтоы были унит.
        \[x^{p^n} - x = g_1 \cdot ... \cdot g_k,\q \text{$g_i$ --- унит. непрв.}\]
        Возьмем произв. $g_i$
        \[x^{p^n} - x \devides g_i \RA \deg g_i | n \text{ (по предп.)}\]
        $g_i$ будет в проиведении $\us{d|n}{\prod}...$\\
        $(\La)$:\\
        Возьмём произв. множитель $\us{d|n}{\prod}$. Его степень - делитель $n$, он явл. делителем $x^{p^n} - x$ (по предп.) $\Ra$ он есть в разложении.

        Если степени есть в разложении, то:
        \[h \devides g^2 \RA h' \devides g\]
        \[(x^{p^n} - x)' = \us{\text{характ. $ = p$}}{p^n (x^{p^n - 1}) - 1} = -1 \text{, но } -1 \not \devides g\]
    \end{proof}

    \begin{Lemma}[для док-ва предложения]
        \[(x^{p^n} - x, x^{p^d} - x  ) = x^{p^{(n, d)} } - x \]
    \end{Lemma}

    \begin{Proof}[предложения]
        \[\La n \devides d \]
        \[F = \Z/p\Z[x]\Big/_{(f)} \qq \abs{F} = p^d  \]
        \[\overline{x} \text{ --- класс по модулю}\]
        \[\abs{F^*} = p^d - 1\]
        \[\forall \alpha \in F^* \q \ord \alpha = t \q ts = p^d - 1\]
        \[\alpha^t = 1\]
        \[\alpha^{p^d - 1} = \alpha^{st} = 1  \]
        \[\alpha^{p^d} = \alpha \]
        \[\overline{x}^{p^d} = \overline{x}  \ \text{ в } F \]
        \[x^{p^d} - x \devides f \]
        \[(x^{p^n} - x, x^{p^d} - x ) = x^{p^{(n, d)} } - x = x^{p^d} - x \devides f   \]
        Если НОД делится на $f \Ra$ каждый делится

        \[\Ra x^{p^n} - x \devides f \]
        \[x^{p^d} - x \devides f \]
        \[\Ra (x^{p^n} - x, x^{p^d} - x  ) = x^{p^{(n, d)} } - x \devides f \]
        \[d' = (n, d)\]
        \[\overline{x}^{p^{d'} }  = \overline{x} \q \text{ в } F\]
        \[\varphi(\overline{x})^{p^{d'} } =  \varphi(\overline{x}) \qq \varphi \in \Z_{/p}\Z[t]  \]
        \[1) \q \begin{matrix}
            \lambda^{p^{d'} } = \lambda\\
            \eta^{p^{d'} } = \eta
        \end{matrix} \bigg| \Ra (\lambda + \eta)^{p^{d'} }  = (\lambda + \eta) \qq \lambda, \eta \in F\]
        \[2) \q \lambda^{p^{d'} } = \lambda \Ra (a\lambda)^{p^{d'} } = a\lambda
        \q a \in \Z_{/p}\Z, \q \lambda \in F \]

        \begin{Proof}
            \[1) \q (x + y)^p = x + y \q \text{ в поле хар-ки } p\]
            По формуле Бинома Ньютона (промежуточные слагаемые $\devides p$)
            \[2) \q a \in \Z/p\Z \q a^p = a \text{ по малой т. Ферма}\]
        \end{Proof}

        \[\lambda^{p^{d'} } = \lambda \q \forall \lambda \in F \]
        \[\Ra \lambda^{p^{d'}- 1 } = 1 \]
       % Мы знаем, что в мульт. группе $$
        Это происходит с любым элементом поля $F$, в котором $p^d$ элементов.\\
        В $\abs{F^*} \q p^d - 1$ элемент, а мы получили, что $\forall $ элемента
        $\lambda^{p^{d'} - 1} = 1 $, это возможно, только, если $d' = d$ (т.к. $d' \leq d$)
        \[\Ra n \devides d\]
    \end{Proof}

    \begin{Proof}[леммы]
        \[n = dq + r\]
        \begin{multline*}
            x^{p^n} - x = x^{p^{dq + r} } - x = \\
            = \underbrace{(x^{p^d} - x )^{p^{d(q - 1) + r} } + (x^{p^d} - x )^{p^{d(q - 2) + r} }
            + ... + (x^{p^d} - x )^{p^r}}_{\devides x^{p^d} - x } + (x^{p^r} - x )
        \end{multline*}
        \[\begin{matrix}
            n = dq + r & x^{p^n} - x = (x^{p^d} - x )g + x^{p^r} - x  \\
            d = rq_1 + r_1 & x^{p^d} - x =  (x^{p^r} - x )g_1 + x^{p^{r_1} } - x \\
            r = r_1q_2 + r_2 & ...
        \end{matrix}\]
    \end{Proof}

\end{document}
