\documentclass[12pt]{article}
\usepackage[english,russian]{babel}
\usepackage{url}
\usepackage{graphicx,DCCN2019_ru}
\usepackage{ulem}

\pagestyle{fancy} 
\fancyhead{} 
\fancyfoot{} 

\usepackage[utf8]{inputenc}

\makeatletter
\fancyhead[R]{\small Матмех СПбГУ\\ {29 May 2019}}
\fancyhead[L]{\small Павел Костин\\ {Homework, 141 группа}}
\fancyhead[C]{\small Homework 4}

\usepackage{amsmath,amsthm,amssymb,amsfonts, enumitem, fancyhdr, color, comment, graphicx, environ}
\newenvironment{problem}[2][Problem]
{\begin{trivlist}\item[{\bfseries #1} {\bfseries #2.}]}{\end{trivlist}}
\newenvironment{solutions}[2][Solutions]
{\begin{trivlist}\item[{\bfseries #1} {\bfseries #2.}]}{\end{trivlist}}

\begin{document}

\begin{problem}{1} 
Доказать неприводимость в $\mathds{Q}[x]$ многочлена
\begin{center}
$x^5 + 2x^3 + 3x^2 - 6x - 5$
\end{center}
воспользовавшись редукцией по какому-то модулю.
\end{problem}

\begin{solutions}{1} 
$f \stackrel{\mod 2}{=} x^5+x^2+1$, $f(0)=1 (mod\ 2)$, $f(1)=1(mod\ 2)$. 
Очевидно f не делится на линейный в $\mathds{Z}/2$. Неприводимый в $\mathds{Z}/2, \deg = 2$: $x^2+x+1$, но $x^5+x^2+1 = (x^2+x+1)(x^3+x^2)+1 \Rightarrow$ f не делится на квадратный 

$\Rightarrow$ f неприводим в $\mathds{Z}$ $\Leftrightarrow$ неприводим в $\mathds{Q}$
\end{solutions}

\begin{problem}{2} 
Доказать неприводимость в $\mathds{Q}[x]$ многочлена
\begin{center}
$x^5 - 6x^3 + 2x^2 - 4x +5$
\end{center}
\end{problem}

\begin{solutions}{2} 
Докажем, что f можно разложить в $\mathds{Z}[2]$ и $\mathds{Z}[3]$ на произведение многочленов разных степеней. $f \stackrel{\mod 2}{=} x^5+1$, $f(0)=1(mod\ 2)$, $f(1)=0(mod\ 2)$. $x^5+1=(x-1)(x^4+x^3+x^2+x+1)$, $(x^4+x^3+x^2+x+1)(0)=1(mod\ 2)$, $(x^4+x^3+x^2+x+1)(1)=1(mod\ 2)$, $(x^4+x^3+x^2+x+1)(0)=(x^2+1)(x^2+x)+1 (mod\ 2)$. Значит он неприводим. По модулю $\mathds{Z}[3]$ достаточно предъявить разложение, $x^5+1 = (x^2+1)(x^3+2x+2)$. Ч.т.д.

$\Rightarrow$ f неприводим в $\mathds{Z}$ $\Leftrightarrow$ неприводим в $\mathds{Q}$
\end{solutions}

\begin{problem}{3} 
Доказать неприводимость в $\mathds{Q}[x]$ многочлена
\begin{center}
$x^5 - 12x^4 + 36x - 12$
\end{center}
\end{problem}

\begin{solutions}{3} 
1 \xout{$\mathop{\raisebox{-2pt}{\vdots}}$} 3, 12 $\mathop{\raisebox{-2pt}{\vdots}}$ 3, 36 $\mathop{\raisebox{-2pt}{\vdots}}$ 3, 12 $\mathop{\raisebox{-2pt}{\vdots}}$ 3, 12 \xout{$\mathop{\raisebox{-2pt}{\vdots}}$} 9 $\Rightarrow$ по признаку Эйзенштейна неприводим.
\end{solutions}

\begin{problem}{4} 
Доказать неприводимость многочлена $x^5-x+1$ над полем $\mathds{F}_5$
\end{problem}

\begin{solutions}{4} 
$f(0)=1$, $f(1)=1$, $f(2)=1$, $f(3)=1$, $f(4)=1$, f не делится на линейные в $\mathds{F}_5$, Неприводимые в $\mathds{F}_5, \deg = 2$: 

1) $x^2+2$, но $x^5-x+1 = (x^2+2)(x^3+3x) -2x+1$

2) $x^2+3$, но $x^5-x+1 = (x^2+3)(x^3+2x) -2x+1$

3) $x^2+x+1$, но $x^5-x+1 = (x^2+x+1)(x^3+4x^2+1) -2x $

4) $x^2+x+2$, но $x^5-x+1 = (x^2+x+2)(x^3+4x^2-4x+3) -2x$

5) $x^2+2x+3$, но $x^5-x+1 = (x^2+2x+3)(x^3+3x^2+x+4) -2x-1$

6) $x^2+2x+4$, но $x^5-x+1 = (x^2+2x+4)(x^3+3x^2+3) -2x-1$

7) $x^2+3x+3$, но $x^5-x+1 = (x^2+3x+3)(x^3+2x^2+x+1) -2x-2$

8) $x^2+3x+4$, но $x^5-x+1 = (x^2+3x+4)(x^3+2x^2+2) -2x-2$

9) $x^2+4x+1$, но $x^5-x+1 = (x^2+4x+1)(x^3+x^2+4) -2x-3$

10) $x^2+4x+2$, но $x^5-x+1 = (x^2+4x+2)(x^3+x^2+4x+2) -2x-3$

$\Rightarrow$ f неприводим в $\mathds{Z}$ $\Leftrightarrow$ неприводим в $\mathds{Q}$
\end{solutions}

\begin{problem}{5} 
Покажите, что многочлен $f=(x-a_1)...(x-a_n) - 1$ неприводим над $\mathds{Z}$ при различных целых $a_i$
\end{problem}

\begin{solutions}{5} 
От противного, пусть многочлен приводим при всех целых $a_i$ над $\mathds{Z}$. Тогда он представим в виде $f=g h$, где $g,h\in\mathds{Z}$. $f(a_i)=-1$, значит $g(a_i)*h(a_i)=-1 \Rightarrow g(a_i)+h(a_i)=0$, значит $g(a_i)+h(a_i) \mathop{\raisebox{-2pt}{\vdots}} f$, но $\deg (g(a_i)+h(a_i)) < n$, противоречие.
\end{solutions}

\begin{problem}{6} 
Покажите, что многочлен $f=x^{105}-9$ неприводим над $\mathds{Z}$
\end{problem}

\begin{solutions}{6} 
От противного, пусть многочлен приводим над $\mathds{Z}$. Тогда он представим в виде $f=g h$, где $g,h\in\mathds{Z}$. Пусть $\deg g = k < 105$, так как $\deg h > 0$, а $\deg g + \deg h = 105$. $g=(x-a_1\sqrt[105]{9})...(x-a_k\sqrt[105]{9})$, где $a_i^{105} = 1$ - решения уравнения $x^{105}-9=0$, заметим, что $g(0)=(\sqrt[105]{9})^k\in\mathds{Z}$, но $k<105$, противоречие.
\end{solutions}
\end{document}