\documentclass[12pt]{article}
\usepackage[english,russian]{babel}
\usepackage{url}
\usepackage{graphicx,DCCN2019_ru}
\usepackage{polynom}

\pagestyle{fancy} 
\fancyhead{} 
\fancyfoot{} 

\usepackage[utf8]{inputenc}

\makeatletter
\fancyhead[R]{\small Матмех СПбГУ\\ {29 May 2019}}
\fancyhead[L]{\small Павел Костин\\ {Homework, 141 группа}}
\fancyhead[C]{\small Homework 3}

\usepackage{amsmath,amsthm,amssymb,amsfonts, enumitem, fancyhdr, color, comment, graphicx, environ}
\newenvironment{problem}[2][Problem]
{\begin{trivlist}\item[{\bfseries #1} {\bfseries #2.}]}{\end{trivlist}}
\newenvironment{solutions}[2][Solutions]
{\begin{trivlist}\item[{\bfseries #1} {\bfseries #2.}]}{\end{trivlist}}
\newenvironment{ihw}[2][ИДЗ]
{\begin{trivlist}\item[{\bfseries #1} {\bfseries #2.}]}{\end{trivlist}}

\begin{document}

\begin{problem}{1} 
Найти НОД и его линейное разложение для многочленов 

$f=x^5+x^4+1$ и $g=x^6+x^5+x^4+x^2+x+1$ в кольце $\mathds{F}_2[x]$.
\end{problem}

\begin{solutions}{1} Выполним деление:

$\polylongdiv{x^6+x^5+x^4+x^2+x+1}{x^5+x^4+1}$

$\polylongdiv{x^5+x^4+1}{x^4+x^2+1}$

$\polylongdiv{x^4+x^2+1}{x^3+x^2+x}$

$\polylongdiv{x^3+x^2+x}{x^2+x+1}$

Значит НОД: $x^2+x+1$. Знаем: 

$x^4+x^2+1 =  (x^3+x^2+x)(x+1) + x^2+x+1$

$x^5+x^4+1 = (x^4+x^2+1)(x+1) + x^3+x^2+x$

$x^6+x^5+x^4+x^2+x+1 = (x^5+x^4+1)x + x^4+x^2+1$

Значит:

$x^2+x+1 = x^4+x^2+1 + (x^3+x^2+x)(x+1)$

$x^2+x+1 = x^4+x^2+1 + (x^5+x^4+1 + (x^4+x^2+1)(x+1))(x+1)$

$x^2+x+1 = (x^5+x^4+1)(x+1) + (x^4+x^2+1)x^2$

$x^2+x+1 = (x^5+x^4+1)(x+1) + (x^6+x^5+x^4+x^2+x+1 + (x^5+x^4+1)x)x^2$

$x^2+x+1 = (x^6+x^5+x^4+x^2+x+1)x^2 + (x^5+x^4+1)(x^3+x+1)$

$x^2+x+1 = f x^2 + g (x^3+x+1)$
\end{solutions}

\begin{problem}{2} 
Доказать, что полином $1+x+\frac{1}{2}x^2+...+\frac{1}{n!}x^n$ не имеет кратных корней в \mathds{C}.
\end{problem}

\begin{solutions}{2} 
$\sqsupset x_0$ - корень кратности k, $k \geqslant 2$. Тогда $f(x) \mathop{\raisebox{-2pt}{\vdots}} (x-x_0)^k$, $f(x)\textquoteright \mathop{\raisebox{-2pt}{\vdots}} (x-x_0)^{k-1}$, но $f(x)=f\textquoteright(x) + \frac{1}{n!}x^n$, правая часть не делится на $(x-x_0)^k$ (кроме случая $x=0$) $\Rightarrow$ получено противоречие.
При $x=0$ мы видим $f(0)=1$ и тут тоже не может быть корней.
\end{solutions}

\begin{problem}{3} 
Избавится от кратных множителей и найти разложение на неприводимые для многочлена с коэффициентами в \mathds{Z}/3 
\begin{center}
$f(x) = x^4+x^3+2x^2+x+1$
\end{center}
\end{problem}

\begin{solutions}{3}
Знаем, что НОД(f, $f \textquoteright$) = $\hat{f}$ - произведение всех кратных корней. $f \textquoteright (x) = x^3+x+1$, найдем $\hat{f}$: 

$\polylongdiv{x^4+x^3+2x^2+x+1}{x^3+x+1}$


$\polylongdiv{x^3+x+1}{x^2+2x}$

$\polylongdiv{x^2+2x}{-x}$ \sim x+2

$\polylongdiv{x^4+x^3+2x^2+x+1}{x+2}$

$\hat{f} = x^3+2x^2+x+2$

$\polylongdiv{x^3+2x^2+x+2}{x-1}$

$\polylongdiv{x^4+x^3+2x^2+x+1}{x^3+2x^2+x+2}$

$\hat{f} = x^3+2x^2+x+2 =(x-1)(x^2+1)$

$f = x^3+2x^2+x+2 =(x-1)^2(x^2+1)$
\end{solutions}

\begin{problem}{4} 
Не раскрывая скобки докажите тождество
\begin{center}
$\frac{a}{(a-b)(a-c)}+\frac{b}{(b-c)(b-a)}+\frac{c}{(c-a)(c-b)} = 0$
\end{center}
\end{problem}

\begin{solutions}{4} 
$\frac{a^2(x-b)(x-c)}{(a-b)(a-c)}+\frac{b^2(x-a)(x-c)}{(b-c)(b-a)}+\frac{c^2(x-a)(x-b)}{(c-a)(c-b)} = x^2$, подставим x=0:

$\frac{a a b c}{(a-b)(a-c)}+\frac{b a b c}{(b-c)(b-a)}+\frac{c a b c}{(c-a)(c-b)} = 0$
если все множетели не ноль, можно сократить на $a b c$ и получить искомое равенство, если один из них ноль, равенство имеет вид (боо $a$):
$\frac{b}{(b-c)(b)}+\frac{c}{c(c-b)} = 0$ $\Rightarrow$ $\frac{1}{(b-c)}+\frac{1}{(c-b)} = 0$ $\Rightarrow$ $0=0$


\end{solutions}

\begin{problem}{5} 
Найти сумму $\sum\limits_{k\equiv1(3)}(-1)^k C^k_n$
\end{problem}

\begin{solutions}{5} 
$\sqsupset f(x) = (x-1)^n = \sum\limits_{k=0}^n(-1)^k x^{n-k}$, рассмотрим $f(x)$ по $\mod{x^3-1}$: $\bar{f}(x) = \sum\limits_{k\equiv0(3)}^n(-1)^k x^{n-k} + x\sum\limits_{k\equiv1(3)}^n(-1)^k x^{n-k} + x^2\sum\limits_{k\equiv2(3)}^n(-1)^k x^{n-k}$, домножим на $x^2$, чтобы было удобнее: 

$x^2 \bar{f}(x) = x^2\sum\limits_{k\equiv0(3)}^n(-1)^k x^{n-k} + \sum\limits_{k\equiv1(3)}^n(-1)^k x^{n-k} + x\sum\limits_{k\equiv2(3)}^n(-1)^k x^{n-k}$, теперь у нужной нам суммы нулевая степень x. С другой стороны, остаток от деления $f(x)$ на $x^3-1$ - это решение интерполяционной задачи:

$x^3-1=(x-1)(x-\epsilon_3)(x-\epsilon^2_3)$, то есть:

$x^2 \bar{f}(x)=\frac{\epsilon^2_3(1-\epsilon_3)^n(x-1)(x-\epsilon^2_3)}{3\epsilon^2_3} + \frac{\epsilon_3(1-\epsilon^2_3)^n(x-1)(x-\epsilon^2_3)}{3\epsilon_3} = \frac{(1-\epsilon_3)^n(x-1)(x-\epsilon^2_3)}{3} + \frac{(1-\epsilon^2_3)^n(x-1)(x-\epsilon^2_3)}{3}$, значит:

$\sum\limits_{k\equiv1(3)}^n(-1)^k C^k_n = \frac{\epsilon^2_3(1-\epsilon_3)^n}{3} + \frac{\epsilon_3(1-\epsilon^2_3)^n}{3}$, подставим $\epsilon_3 = \frac{-1-i\sqrt3}{6}$ и $\epsilon^2_3 = \frac{-1+i\sqrt3}{2}$, получим:

$\sum\limits_{k\equiv1(3)}^n(-1)^k C^k_n = \frac{\frac{-1+i\sqrt3}{2}(1-\frac{-1-i\sqrt3}{6})^n}{3} + \frac{\frac{-1-i\sqrt3}{6}(1-\frac{-1+i\sqrt3}{2})^n}{3}$, сгруппипуем:

$\sum\limits_{k\equiv1(3)}^n(-1)^k C^k_n = \frac{1}{3} (\frac{-1-i\sqrt3}{2}(\frac{3}{2}-\frac{i\sqrt{3}}{2})^n + \frac{-1+i\sqrt3}{2}(\frac{3}{2}+\frac{i\sqrt{3}}{2})^n)$, вынесем за скобки корень:

$\sum\limits_{k\equiv1(3)}^n(-1)^k C^k_n = \frac{1}{3} (\frac{-1-i\sqrt3}{2}(\sqrt{3})^n(\cos{\frac{-\pi}{6}}+i\sin{\frac{-\pi}{6}})^n + \frac{-1+i\sqrt3}{2}(\sqrt{3})^n(\cos{\frac{\pi}{6}}+i\sin{\frac{\pi}{6}})^n)$, перейдем для удобства в exp: 

$\sum\limits_{k\equiv1(3)}^n(-1)^k C^k_n = \frac{1}{3} (\frac{-1-i\sqrt3}{2}(\sqrt{3})^n e^(\frac{-\pi n}{6}) + \frac{-1+i\sqrt3}{2}(\sqrt{3})^n e^(\frac{-\pi n}{6}))$, немного упростим выражение: 

$\sum\limits_{k\equiv1(3)}^n(-1)^k C^k_n = \frac{1}{3} (\sqrt{3})^n (- \frac{e^(\frac{-\pi n}{6})+e^(\frac{\pi n}{6})}{2} - -i\sqrt3 \frac{e^(\frac{-\pi n}{6}) - e^(\frac{\pi n}{6})}{2i})$, окончательный ответ:

$\sum\limits_{k\equiv1(3)}^n(-1)^k C^k_n = -\frac{1}{3} (\sqrt{3})^n (\cos{\frac{\pi n}{6}} + \sqrt{3} \sin{\frac{\pi n}{6}} )$
\end{solutions}

\begin{problem}{6} 
Разложите на простейшие над \mathds{С} и над \mathds{R} $\frac{x^2}{x^6+27}$
\end{problem}

\begin{solutions}{6} 
Разложим $x^6+27 = (x^2)^3+3^2 = (x^2+3)(x^4-3x^2+9) = (x^2+3)(x^2 - i\sqrt3 x - 3)(x^2 + i\sqrt3x - 3)$, решения: x=$i\sqrt3$, $-i\sqrt3$, $\frac{3 + i\sqrt3}{2}$, $\frac{-3 + i\sqrt3}{2}$, $\frac{3 - i\sqrt3}{2}$, $\frac{-3 - i\sqrt3}{2}$, значит по формуле Лагранжа $\sum\limits_{k=1}^{6} \frac{x_k^2}{6x_i^5 (x-x_i)}$ - над \mathds{С}, подставляя и группируя получаем ответ над \mathds{R}: $\frac{1}{18 (x^2-3x+3)} + \frac{1}{18 (x^2+3x+3)} - \frac{1}{9 (x^2+3)}$
\end{solutions}

\begin{ihw}{КП} 
Найдите интерполяционный многочлен принимающий значения 
\begin{center}
7, -3, -1, 1, 15
\end{center}
в точках -2, -1, 0, 1, 2 соответственно.
\end{ihw}

\begin{solutions}{ИДЗ} 
$f=-1+2x^3$, подходит $x=-1,\ 0,\ 1,\2$, по методу Ньютона $f_{new}=f+c\phi$, где $\phi = (x-2)(x-1)x(x+1)=(x^3-x)(x-2)$, $c=\frac{7-f(-2)}{\phi(-2)}=\frac{7+17}{24}=1$, значит $f_{new}=-1+2x^3+(x^3-x)(x-2)$ - ответ
\end{solutions}

\end{document}