\documentclass[12pt]{article}
\usepackage[english,russian]{babel}
\usepackage{url}
\usepackage{graphicx,DCCN2019_ru}

\pagestyle{fancy} 
\fancyhead{} 
\fancyfoot{} 

\usepackage[utf8]{inputenc}

\makeatletter
\fancyhead[R]{\small Матмех СПбГУ\\ {29 May 2019}}
\fancyhead[L]{\small Павел Костин\\ {Homework, 141 группа}}
\fancyhead[C]{\small Homework 2}

\usepackage{amsmath,amsthm,amssymb,amsfonts, enumitem, fancyhdr, color, comment, graphicx, environ}
\newenvironment{problem}[2][Problem]
{\begin{trivlist}\item[{\bfseries #1} {\bfseries #2.}]}{\end{trivlist}}
\newenvironment{solutions}[2][Solutions]
{\begin{trivlist}\item[{\bfseries #1} {\bfseries #2.}]}{\end{trivlist}}

\begin{document}

\begin{problem}{1} 
Найдите базис суммы и пересечения для подпространств 

$V = \mathds{R}[t]_\leqslant6$ вида $U_1 = \{ f \in V\ |\ f \mathop{\raisebox{-2pt}{\vdots}} t^2-4t+3\}$ и $U_2 = \{ f \in V\ |\ f \mathop{\raisebox{-2pt}{\vdots}} t^2-5t+4\}$
\end{problem}

\begin{solutions}{1} 
$t^2-4t+3 = (t-3)(t-1)$, $t^2-5t+4 = (t-4)(t-1)$ $\Rightarrow$ каждый вектор $v$ из базиса $U_1\cap U_2$ делится на $(t-4)(t-3)(t-1)$ и $\deg v \leqslant6$, в качестве базиса можно взять $(t-4)(t-3)(t-1)$, $t(t-4)(t-3)(t-1)$, $t^2(t-4)(t-3)(t-1)$, $t^3(t-4)(t-3)(t-1)$. Аналогично каждый вектор $v$ из базиса $U_1+U_2$ должен делиться на $(t-1)$ и $\deg v \leqslant6$, в качестве базиса можно взять $(t-1)$, $t(t-1)$, $t^2(t-1)$, $t^3(t-1)$, $t^4(t-1)$, $t^5(t-1)$. Действительно, нетрудно видеть, что любой вектор из суммы попадает в наш базис, ровно как и то что мы указали ЛНЗ набор. По формуле Грассмана в пересечении дродно быть $5+5-6 = 4$ ЛНЗ вектора что мы и предоставили. Любой вектор из нашего базиса лежит в заданных пространствах, ровно как и наоборот $\Rightarrow$ базис выбран корректно ч.т.д.
\end{solutions}

\begin{problem}{2} 
Пусть $U_1$, $U_2$, $U_3$ – подпространства конечномерного пространства V. Доказать, что подпространство $(U_1\cap U_2)+(U_2\cap U_3)+(U_3\cap U_1)$ содержится в подпространстве $(U_1+U_2)\cap(U_2+U_3)\cap(U_3+U_1)$ и разность размерностей
этих подпространств есть четное число.
\end{problem}

\begin{solutions}{2} 
Каждое слоагаемое суммы содержится в пересечении пространств, значит сумма тоже содержится.

$\dim(U+V) = \dim U+\dim V-\dim(U\cap V)$ $\Rightarrow$ $\dim((U_2+U_3)\cap (U_3+U_1)) = \dim(U_2+U_3)+\dim(U_3+U_1)-\dim(U_1+U_2+U_3)$. Обозначим через U подпространство, размерность которого мы только что нашли. Ясно, что оно содержит $U_3$. Поэтому его сумма с $U_1+U_2$ будет равна $U_1+U_2+U_3$. Отсюда, применяя ту же формулу, имеем $\dim(U\cap (U_1+U_2)=\dim(U)+\dim(U_1+U_2)-\dim(U_1+U_2+U_3)$.

Таким образом, размерность пересечения трёх подпространств, то есть $\dim((U_2+U_3)\cap(U_3+U_1)\cap(U_1+U_2))$, равна $\dim(U_1+U_2)+\dim(U_2+U_3)+\dim(U_3+U_1)-2\dim(U_1+U_2+U_3)$.

Теперь вычислим размерность суммы трёх подпространств, указанных в начале. Заметим, что $(U_2\cap U_3)+(U_3\cap U_1)\leqslant U_3$ в пересечении с $U_1\cap U_2$ даёт $U_1\cap U_2\cap U_3$. Поэтому $\dim((U_2\cap U_3)+(U_3\cap U_1)+(U_1\cap U_2))$ равна $\dim((U_2\cap U_3)+(U_3\cap U_1))+\dim(U_1\cap U_2)-\dim(U_1\cap U_2\cap U_3)$, а это в свою очередь равно $\dim(U_1\cap U_2)+\dim(U_2\cap U_3)+\dim(U_3\cap U_1)-2\dim(U_1\cap U_2\cap U_3)$. Наконец, выражая размерности пересечений, получаем $2(\dim U_1+\dim U_2+\dim U_3)-\dim(U_1+U_2)-\dim(U_2+U_3)-\dim(U_3+U_1)-2\dim(U_1\cap U_2\cap U_3)$.

Видно, что разность размерностей одного и другого подпространства чётна.
\end{solutions}

\begin{problem}{3} 
Найти базис пересечения и суммы подпространств $U=\langle u_1, u_2, u_3\rangle$, $V=\langle v_1, v_2, v_3\rangle$ в $\mathds{R}^5$ если 
\begin{center}
$
u_1=
\begin{pmatrix}
1\\
3\\
2\\
0\\
1
\end{pmatrix},\ 
u_2=
\begin{pmatrix}
1\\
2\\
0\\
1\\
1
\end{pmatrix},\ 
u_3=
\begin{pmatrix}
2\\
0\\
1\\
1\\
3
\end{pmatrix},\ 
v_1=
\begin{pmatrix}
1\\
1\\
1\\
1\\
2
\end{pmatrix},\ 
v_2=
\begin{pmatrix}
5\\
2\\
5\\
2\\
8
\end{pmatrix},\ 
v_3=
\begin{pmatrix}
4\\
1\\
1\\
4\\
7
\end{pmatrix}
$
\end{center}
\end{problem}

\begin{solutions}{3}
$\begin{pmatrix}
1 & 1 & 2 & 1 & 5 & 4\\
3 & 2 & 0 & 1 & 2 & 1\\
2 & 0 & 1 & 1 & 5 & 1\\
0 & 1 & 1 & 1 & 2 & 4\\
1 & 1 & 3 & 2 & 8 & 7
\end{pmatrix}\sim
\begin{pmatrix}
1 & 1 & 2 & 1 & 5 & 4\\
0 & -1 & -6 & -2 & -13 & -11\\
0 & -2 & -3 & -1 & -5 & -7\\
0 & 1 & 1 & 1 & 2 & 4\\
0 & 0 & 1 & 1 & 3 & 3
\end{pmatrix}\sim$

$\begin{pmatrix}
1 & 1 & 2 & 1 & 5 & 4\\
0 & 1 & 1 & 1 & 2 & 4\\
0 & -2 & -3 & -1 & -5 & -7\\
0 & 1 & 6 & 2 & 13 & 11\\
0 & 0 & 1 & 1 & 3 & 3
\end{pmatrix}\sim
\begin{pmatrix}
1 & 1 & 2 & 1 & 5 & 4\\
0 & 1 & 1 & 1 & 2 & 4\\
0 & 0 & -1 & 1 & -1 & 1\\
0 & 0 & 5 & 1 & 11 & 7\\
0 & 0 & 1 & 1 & 3 & 3
\end{pmatrix}\sim$

$\begin{pmatrix}
1 & 1 & 2 & 1 & 5 & 4\\
0 & 1 & 1 & 1 & 2 & 4\\
0 & 0 & 1 & 1 & 3 & 3\\
0 & 0 & 5 & 1 & 11 & 7\\
0 & 0 & 1 & -1 & 1 & -1
\end{pmatrix}\sim
\begin{pmatrix}
1 & 1 & 2 & 1 & 5 & 4\\
0 & 1 & 1 & 1 & 2 & 4\\
0 & 0 & 1 & 1 & 3 & 3\\
0 & 0 & 0 & -4 & -4 & -8\\
0 & 0 & 0 & -2 & -2 & -4
\end{pmatrix}\sim$
$\begin{pmatrix}
1 & 1 & 2 & 1 & 5 & 4\\
0 & 1 & 1 & 1 & 2 & 4\\
0 & 0 & 1 & 1 & 3 & 3\\
0 & 0 & 0 & 2 & 2 & 4\\
0 & 0 & 0 & 0 & 0 & 0
\end{pmatrix}$. 

Линейно независимых столбцов 4, в качестве базиса суммы можно взять четыре первых вектора, они ЛНЗ.

2 линейно зависимых столбца $\Rightarrow$ должно получиться два вектора в пересечении. Для пары $(t_6,t_5) = (1,0)$ и пары $(t_6,t_5) = (0,1)$ получаем $t_4 = -2$ и $t_4 = -1$, по формуле $-(t_6v_3+t_5v_2+t_4v_1)$ имеем:

$\begin{pmatrix}
-2\\
1\\
1\\
-2\\
-3
\end{pmatrix}$ и 
$\begin{pmatrix}
4\\
1\\
4\\
1\\
6
\end{pmatrix}
$
\end{solutions}

\begin{problem}{4} 
Найти определитель
\begin{center}
$A =\begin{vmatrix}
x & 0 & 0 & ... & 0 & c_n\\
-1 & x & 0 & ... & 0 & c_{n-1}\\
0 & -1 & x & ... & 0 & c_{n-1}\\
... & ... & ... & ... & ... & ...\\
0 & 0 & 0 & ... & x & c_2\\
0 & 0 & 0 & ... & -1 & x+c_1
\end{vmatrix}$
\end{center}
\end{problem}

\begin{solutions}{4} 
Разложим определитель по последнему столбцу:

$\det A = c_n(-1)^{n+1}A_1 + c_{n-1}(-1)^{n+2}A_2 +...+ c_2(-1)^{2n-1}A_{n-1} + (x+c_1)(-1)^{2n}A_{n}$
Где $A_i = x^{n-1-(i-1)}(-1)^{i-1}$, так как это матрица с нулевыми наддиагональными элементами $\Rightarrow$ $\det A = \sum\limits_{i=0}^{n-1} c_{n-i} x^i+x^n$
\end{solutions}

\begin{problem}{5} 
Обратите матрицу и посчитайте её определитель
\begin{center}
$A=
\begin{pmatrix}
1 & -1 & 0 & 0 & ... & 0\\
-1 & 2 & -1 & 0 & ... & 0\\
0 & -1 & 2 & -1 & ... & 0\\
... & ... & ... & ... & ... & ...\\
0 & 0 & 0 & ... & 2 & -1\\
0 & 0 & 0 & ... & -1 & 2
\end{pmatrix}
$
\end{center}
\end{problem}

\begin{solutions}{5} 
А $\sim \begin{pmatrix}
1 & -1 & 0 & 0 & ... & 0\\
0 & 1 & -1 & 0 & ... & 0\\
0 & 0 & 1 & -1 & ... & 0\\
... & ... & ... & ... & ... & ...\\
0 & 0 & 0 & ... & 1 & -1\\
0 & 0 & 0 & ... & 0 & 1
\end{pmatrix} \sim$
$\begin{pmatrix}
1 & 0 & 0 & 0 & ... & 0\\
0 & 1 & 0 & 0 & ... & 0\\
0 & 0 & 1 & 0 & ... & 0\\
... & ... & ... & ... & ... & ...\\
0 & 0 & 0 & ... & 1 & 0\\
0 & 0 & 0 & ... & 0 & 1
\end{pmatrix} \Rightarrow$ 
проделывая то же:
$\begin{pmatrix}
1 & 0 & 0 & 0 & ... & 0\\
0 & 1 & 0 & 0 & ... & 0\\
0 & 0 & 1 & 0 & ... & 0\\
... & ... & ... & ... & ... & ...\\
0 & 0 & 0 & ... & 1 & 0\\
0 & 0 & 0 & ... & 0 & 1
\end{pmatrix} \sim$
$\begin{pmatrix}
1 & 0 & 0 & 0 & ... & 0\\
1 & 1 & 0 & 0 & ... & 0\\
1 & 1 & 1 & 0 & ... & 0\\
... & ... & ... & ... & ... & ...\\
1 & 1 & 1 & ... & 1 & 0\\
1 & 1 & 1 & ... & 1 & 1
\end{pmatrix} \sim$ 

$A^{-1}=
\begin{pmatrix}
n & n-1 & n-2 & ... & 2 & 1\\
n-1 & n-1 & n-2 & ... & 2 & 1\\
n-1 & n-1 & n-2 & ... & 2 & 1\\
... & ... & ... & ... & 2 & 1\\
2 & 2 & 2 & 2 & 2 & 1\\
1 & 1 & 1 & 1 & 1 & 1
\end{pmatrix}, 
\det A = 1$
\end{solutions}
\end{document}