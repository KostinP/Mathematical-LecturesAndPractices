\documentclass[11pt, fleqn]{article}
\usepackage{../../../../template/template}

%сам документ
\begin{document}
\begin{center}
  \huge ДЗ по алгебре №1, 3 сем

  \Large (преподаватель Демченко О. В.)

  \large Записал Костин П.А.
\end{center}

\begin{task}
  На пространстве $V = \{ X \in M_{3,2}(\R): c X d^T = 0\}$, где $c \in \R^3$ задан оператор:
  \[LX = AXB + mX,\q X \in V\]
  где $A \in M_3(\R),\q B \in M_2(\R),\q m \in \R$
  \begin{enumerate}
    \item Найти жорданов базис и жорданову матрицу L
    \item Для собственного вектора из жорданового базиса проверить непосредственно, что он является таковым
    \item Найти $L^{(n)}X_0$ для данных $X_0 \in V$ и $n \in \N$
  \end{enumerate}
  Полезно воспользоваться тем, что $L$ имеет только одно собственное число
\end{task}

\begin{Remark}[мои данные]
  \[c = (-1,1,4) \qq d = (-2,1) \qq m = -2 \qq n = 31\]
  \[A = \begin{pmatrix}
    -3 & 12 & 16\\
    0 & 1 & 0\\
    -1 & 3 & 5
  \end{pmatrix} \qq B = \begin{pmatrix}
    -1 & -4\\
    1 & 3
  \end{pmatrix} \qq X_0 = \begin{pmatrix}
    -1 & 3\\
    -1 & -1\\
    5 & 11
  \end{pmatrix}\]
\end{Remark}

\begin{proof}
  \begin{enumerate}
    \item Найдем базис V:
    \[c X d^T = \begin{pmatrix}
          -1 & 1 & 4
      \end{pmatrix} \begin{pmatrix}
        x_{11} & x_{12}\\
        x_{21} & x_{22}\\
        x_{31} & x_{32}
      \end{pmatrix} \begin{pmatrix}
        -2\\
        1
      \end{pmatrix} =\]
      \[= \begin{pmatrix}
        -x_{11} + x_{21} + 4 \cdot x_{31} & -x_{12} + x_{22} + 4 \cdot x_{32}
      \end{pmatrix} \begin{pmatrix}
        -2\\
        1
      \end{pmatrix}\]
      \[= \begin{pmatrix}
        2 \cdot x_{11} - 2 \cdot x_{21} - 8 \cdot x_{31} - x_{12} + x_{22} + 4 \cdot x_{32}
      \end{pmatrix}\]
      Подставляем стандартный базис для уравнения $c X d^T = 0$ для всех векторов кроме $x_{12}$ (потому что так проще считать), чтобы выбрать свой базис в пространстве V:
      \[
      \left.
      \begin{matrix}
        x_{11} = 1 \Ra x_{12} = 2\\
        x_{21} = 1 \Ra x_{12} = -2\\
        x_{31} = 1 \Ra x_{12} = -8\\
        x_{22} = 1 \Ra x_{12} = 1\\
        x_{32} = 1 \Ra x_{12} = 4
      \end{matrix} \q \right|
      \begin{matrix}
        e_1 =
        \begin{pmatrix}
          1 & 2\\
          0 & 0\\
          0 & 0
        \end{pmatrix} \q
        e_2 =
        \begin{pmatrix}
          0 & -2\\
          1 & 0\\
          0 & 0
        \end{pmatrix} \q
        e_3 =
        \begin{pmatrix}
          0 & -8\\
          0 & 0\\
          1 & 0
        \end{pmatrix}\\
        e_4 =
        \begin{pmatrix}
          0 & 1\\
          0 & 1\\
          0 & 0
        \end{pmatrix} \q
        e_5 =
        \begin{pmatrix}
          0 & 4\\
          0 & 0\\
          0 & 1
        \end{pmatrix}
      \end{matrix}
      \]
    \item Найдем координаты $L(e_i)$ в $\{e_i\}_{i \in 1:5}$:
    \[L(X) = AXB + mX = \begin{pmatrix}
      -3 & 12 & 16\\
      0 & 1 & 0\\
      -1 & 3 & 5
    \end{pmatrix} X \begin{pmatrix}
      -1 & -4\\
      1 & 3
    \end{pmatrix} -2 X\]
    \[L(e_1) = \begin{pmatrix}
      -3 & 12 & 16\\
      0 & 1 & 0\\
      -1 & 3 & 5
    \end{pmatrix} \begin{pmatrix}
      1 & 2\\
      0 & 0\\
      0 & 0
    \end{pmatrix} \begin{pmatrix}
      -1 & -4\\
      1 & 3
    \end{pmatrix} - 2 \begin{pmatrix}
      1 & 2\\
      0 & 0\\
      0 & 0
    \end{pmatrix} = \]
    \[= \begin{pmatrix}
      -3 & -6 \\
      0 & 0 \\
      -1 & -2
    \end{pmatrix} \begin{pmatrix}
      -1 & -4\\
      1 & 3
    \end{pmatrix} - \begin{pmatrix}
      2 & 4\\
      0 & 0\\
      0 & 0
    \end{pmatrix} = \begin{pmatrix}
      -3 & -6\\
      0 & 0\\
      -1 & -2
    \end{pmatrix} - \begin{pmatrix}
      2 & 4\\
      0 & 0\\
      0 & 0
    \end{pmatrix} = \begin{pmatrix}
      -5 & -10\\
      0 & 0\\
      -1 & -2
    \end{pmatrix}\]
    \[L(e_2) = \begin{pmatrix}
      -3 & 12 & 16\\
      0 & 1 & 0\\
      -1 & 3 & 5
    \end{pmatrix} \begin{pmatrix}
      0 & -2\\
      1 & 0\\
      0 & 0
    \end{pmatrix} \begin{pmatrix}
      -1 & -4\\
      1 & 3
    \end{pmatrix} - 2 \begin{pmatrix}
      0 & -2\\
      1 & 0\\
      0 & 0
    \end{pmatrix} =\]
    \[= \begin{pmatrix}
      12 & 6 \\
      1 & 0 \\
      3 & 2
    \end{pmatrix} \begin{pmatrix}
      -1 & -4\\
      1 & 3
    \end{pmatrix} - \begin{pmatrix}
      0 & -4\\
      2 & 0\\
      0 & 0
    \end{pmatrix} = \begin{pmatrix}
      -6 & -30\\
      -1 & -4\\
      -1 & -6
    \end{pmatrix} - \begin{pmatrix}
      0 & -4\\
      2 & 0\\
      0 & 0
    \end{pmatrix} = \begin{pmatrix}
      -6 & -26\\
      -3 & -4\\
      -1 & -6
    \end{pmatrix}\]
    \[L(e_3) = \begin{pmatrix}
      -3 & 12 & 16\\
      0 & 1 & 0\\
      -1 & 3 & 5
    \end{pmatrix} \begin{pmatrix}
      0 & -8\\
      0 & 0\\
      1 & 0
    \end{pmatrix} \begin{pmatrix}
      -1 & -4\\
      1 & 3
    \end{pmatrix} - 2 \begin{pmatrix}
      0 & -8\\
      0 & 0\\
      1 & 0
    \end{pmatrix} =\]
    \[= \begin{pmatrix}
      16 & 24\\
      0 & 0\\
      5 & 8
    \end{pmatrix} \begin{pmatrix}
      -1 & -4\\
      1 & 3
    \end{pmatrix} - \begin{pmatrix}
      0 & -16\\
      0 & 0\\
      2 & 0
    \end{pmatrix} = \begin{pmatrix}
      8 & 8\\
      0 & 0\\
      3 & 4
    \end{pmatrix} - \begin{pmatrix}
      0 & -16\\
      0 & 0\\
      2 & 0
    \end{pmatrix} = \begin{pmatrix}
      8 & 24\\
      0 & 0\\
      1 & 4
    \end{pmatrix}\]
    \[L(e_4) = \begin{pmatrix}
      -3 & 12 & 16\\
      0 & 1 & 0\\
      -1 & 3 & 5
    \end{pmatrix} \begin{pmatrix}
      0 & 1\\
      0 & 1\\
      0 & 0
    \end{pmatrix} \begin{pmatrix}
      -1 & -4\\
      1 & 3
    \end{pmatrix} - 2 \begin{pmatrix}
      0 & 1\\
      0 & 1\\
      0 & 0
    \end{pmatrix} =\]
    \[= \begin{pmatrix}
      0 & 9\\
      0 & 1\\
      0 & 2
    \end{pmatrix} \begin{pmatrix}
      -1 & -4\\
      1 & 3
    \end{pmatrix} - \begin{pmatrix}
      0 & 2\\
      0 & 2\\
      0 & 0
    \end{pmatrix} = \begin{pmatrix}
      9 & 27\\
      1 & 3\\
      2 & 6
    \end{pmatrix} - \begin{pmatrix}
      0 & 2\\
      0 & 2\\
      0 & 0
    \end{pmatrix} = \begin{pmatrix}
      9 & 25\\
      1 & 1\\
      2 & 6
    \end{pmatrix}\]
    \[L(e_5) = \begin{pmatrix}
      -3 & 12 & 16\\
      0 & 1 & 0\\
      -1 & 3 & 5
    \end{pmatrix} \begin{pmatrix}
      0 & 4\\
      0 & 0\\
      0 & 1
    \end{pmatrix} \begin{pmatrix}
      -1 & -4\\
      1 & 3
    \end{pmatrix} - 2 \begin{pmatrix}
      0 & 4\\
      0 & 0\\
      0 & 1
    \end{pmatrix} =\]
    \[= \begin{pmatrix}
      0 & 4\\
      0 & 0\\
      0 & 1
    \end{pmatrix} \begin{pmatrix}
      -1 & -4\\
      1 & 3
    \end{pmatrix} - \begin{pmatrix}
      0 & 8\\
      0 & 0\\
      0 & 2
    \end{pmatrix} = \begin{pmatrix}
      4 & 12\\
      0 & 0\\
      1 & 3
    \end{pmatrix} - \begin{pmatrix}
      0 & 8\\
      0 & 0\\
      0 & 2
    \end{pmatrix} = \begin{pmatrix}
      4 & 4\\
      0 & 0\\
      1 & 1
    \end{pmatrix}\]
    \item Составим матрицу $[L]_{\{e\}}$
    \[\Ra [L]_{\{e\}} = \begin{pmatrix}
      -5 & -6 & 8 & 9 & 4\\
      0 & -3 & 0 & 1 & 0\\
      -1 & -1 & 1 & 2 & 1\\
      0 & -4 & 0 & 1 & 0\\
      -2 & -6 & 4 & 6 & 1
    \end{pmatrix}\]
    \item Докажем, что $\forall A,B \in M_n(R)$, где $R$ - комм. кольцо $\Tr(AB) = \Tr(BA)$
    \begin{Proof}
      \[A = \begin{pmatrix}
      a_{11} & a_{12} & \ldots & a_{1n}\\
      a_{21} & a_{22} & \ldots & a_{2n}\\
      \vdots & \vdots & \ddots & \vdots\\
      a_{n1} & a_{n2} & \ldots & a_{nn}
    \end{pmatrix} \qq B = \begin{pmatrix}
      b_{11} & b_{12} & \ldots & b_{1n}\\
      b_{21} & b_{22} & \ldots & b_{2n}\\
      \vdots & \vdots & \ddots & \vdots\\
      b_{n1} & b_{n2} & \ldots & b_{nn}
      \end{pmatrix}\]
      \[\Tr(AB) = \sum_{i=1}^n \sum_{j=1}^n a_{ij} b_{ji} \qq \Tr(BA) = \sum_{i=1}^n \sum_{j=1}^n b_{ij} a_{ji}\]
      Так как $a_{lm}b_{ml} = b_{ml}a_{lm} \Ra \Tr(AB) = \Tr(BA)$
    \end{Proof}
    \[\Ra \Tr(AB) \os{\text{т. о Ж.ф.}}{=} \Tr(C^{-1}JC) = \Tr((C^{-1}J)C) = \Tr(C(C^{-1}J)) = \Tr(EJ) = \Tr(J)\]
    Т.к. по условию J имеет одно собственное число:
    \[\Tr([L]_{\{e\}}) = 5 \lambda \Ra 5 \lambda = -5 \Ra \lambda = -1\]
    \item Найдем с.в., отвечающий с.ч. $\lambda = -1$:
    \[[L]_{\{e\}}v = \lambda v \Ra \Ra ([L]_{\{e\}}-\lambda E)v = 0 \qq \deg V = 5 \Ra v^T = \begin{pmatrix}
      a & b & c & d & e
    \end{pmatrix}\]
    \[[L]_{\{e\}}-\lambda E = \begin{pmatrix}
      -4 & -6 & 8 & 9 & 4\\
      0 & -2 & 0 & 1 & 0\\
      -1 & -1 & 2 & 2 & 1\\
      0 & -4 & 0 & 2 & 0\\
      -2 & -6 & 4 & 6 & 2
    \end{pmatrix} \sim \begin{pmatrix}
      -4 & -6 & 8 & 9 & 4\\
      0 & -2 & 0 & 1 & 0\\
      0 & 0 & 0 & 0 & 0\\
      0 & 0 & 0 & 0 & 0\\
      0 & 0 & 0 & 0 & 0
    \end{pmatrix}\]
    Порядок действий:\\
    $(1) \os{-\frac{1}{4}}{\ra} (3) \qq (1) \os{-\frac{1}{2}}{\ra} (5) \qq (2) \os{\frac{1}{4}}{\ra} (3) \qq (2) \os{-2}{\ra} (4) \qq (2) \os{-\frac{3}{2}}{\ra} (5)$
    \[\Ra \begin{cases}
      -4a - 6b + 8c + 9d + 4e = 0\\
      -2b + d = 0
    \end{cases} \Ra \begin{cases}
      a = 3b + 2c + e\\
      d = 2b
    \end{cases}\]
    \[\Ra v = \begin{pmatrix}
      3\\
      1\\
      0\\
      2\\
      0\\
    \end{pmatrix}b + \begin{pmatrix}
      2\\
      0\\
      1\\
      0\\
      0
    \end{pmatrix} c + \begin{pmatrix}
      1\\
      0\\
      0\\
      0\\
      1
    \end{pmatrix}e = \begin{pmatrix}
      3b + 2c + e\\
      b\\
      c\\
      2b\\
      e
    \end{pmatrix}\]
    \[(1) \os{-\frac{1}{4}}{\ra} (3): \begin{pmatrix}
      3b + 2c + e\\
      b\\
      -\frac{3}{4}b - \frac{1}{2}c - \frac{1}{4}e\\
      2b\\
      e
    \end{pmatrix}\]
    $$(1) \os{-\frac{1}{2}}{\ra} (5): \begin{pmatrix}
      3b + 2c + e\\
      b\\
      -\frac{3}{4}b - \frac{1}{2}c - \frac{1}{4}e\\
      2b\\
      -\frac{3}{2}b - c + \frac{1}{2}e
    \end{pmatrix}$$
    \[(2) \os{\frac{1}{4}}{\ra} (3): \begin{pmatrix}
      3b + 2c + e\\
      b\\
      -\frac{1}{2}b - \frac{1}{2}c - \frac{1}{4}e\\
      2b\\
      -\frac{3}{2}b - c + \frac{1}{2}e
    \end{pmatrix}\]
    $$(2) \os{-2}{\ra} (4): \begin{pmatrix}
      3b + 2c + e\\
      b\\
      -\frac{1}{2}b - \frac{1}{2}c - \frac{1}{4}e\\
      0\\
      -\frac{3}{2}b - c + \frac{1}{2}e
    \end{pmatrix}$$
    \[(2) \os{-\frac{3}{2}}{\ra} (5): \begin{pmatrix}
      3b + 2c + e\\
      b\\
      -\frac{1}{2}b - \frac{1}{2}c - \frac{1}{4}e\\
      0\\
      -3b - c + \frac{1}{2}e
    \end{pmatrix}\]
    Запишем условие для нахождения просоедененного вектора:
    \[\begin{pmatrix}
      -4 & -6 & 8 & 9 & 4\\
      0 & -2 & 0 & 1 & 0\\
      0 & 0 & 0 & 0 & 0\\
      0 & 0 & 0 & 0 & 0\\
      0 & 0 & 0 & 0 & 0
    \end{pmatrix} \begin{pmatrix}
      a_1\\
      b_1\\
      c_1\\
      d_1\\
      e_1
    \end{pmatrix} = \begin{pmatrix}
      3b + 2c + e\\
      b\\
      -\frac{1}{2}b - \frac{1}{2}c - \frac{1}{4}e\\
      0\\
      -3b - c + \frac{1}{2}e
    \end{pmatrix}\]
    \[\begin{cases}
      -4a_1 - 6b_1 + 8c_1 + 9d_1 + 4e_1 = 3b + 2c + e\\
      -2b_1 + d_1 = b\\
      0 = -\frac{1}{2}b - \frac{1}{2}c - \frac{1}{4}e\\
      0 = 0\\
      0 = -3b - c + \frac{1}{2}e
    \end{cases}\]
    \item Найдем присоединенные вектора
    \item Ищем следующий
    \item Найдем следующий
    \item Построим жорданов базис
    \item Нужно ещё два...
    \item !!!
  \end{enumerate}
\end{proof}


\end{document}
