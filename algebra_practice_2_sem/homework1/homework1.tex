\documentclass[12pt]{article}
\usepackage[english,russian]{babel}
\usepackage{url}
\usepackage{graphicx,DCCN2019_ru}

\pagestyle{fancy} 
\fancyhead{} 
\fancyfoot{} 

\usepackage[utf8]{inputenc}

\makeatletter
\fancyhead[R]{\small Матмех СПбГУ\\ {29 May 2019}}
\fancyhead[L]{\small Павел Костин\\ {Homework, 141 группа}}
\fancyhead[C]{\small Homework 1}

\usepackage{amsmath,amsthm,amssymb,amsfonts, enumitem, fancyhdr, color, comment, graphicx, environ}
\newenvironment{problem}[2][Problem]
{\begin{trivlist}\item[{\bfseries #1} {\bfseries #2.}]}{\end{trivlist}}
\newenvironment{solutions}[2][Solutions]
{\begin{trivlist}\item[{\bfseries #1} {\bfseries #2.}]}{\end{trivlist}}

\begin{document}

\begin{problem}{1} 
Пусть пространство $V = \{ f \in K[x]\ |\ deg\ f \leqslant n \} $. Покажите, что любой набор многочленов $p_0(x), p_1(x), . . . , p_n(x)$,
что $deg\ pi(x) = i$, является базисом V
\end{problem}

\begin{solutions}{1} 
\textbf{ЛНЗ}. $\sqsupset p_i = a_i_i x^i + a_i_i_-_1 x^{i-1} +...a_i_0$, $a_i_i \neq 0$. Вычтем $p_n_-_1$ из $p_n$ с коэффициентом $a_{n n-1}/a_{n-1 n-1}$. Аналогично обнулим остальные коэффициенты $p_n$. Проделаем такую же операцию с остальными $p_i$ по убыванию $i$. Получим новую систему векторов $p_i = a_{ii} x^i$ (такую же ЛЗ или ЛНЗ). Запишем матрицу из векторов. Её определитель равен произведению коэффициентов (по условию ненулевых), их произведение не ноль $\Rightarrow$ вектора ЛНЗ.

\textbf{Порождаемость}. Очевидно из первого пункта.
\end{solutions}

\begin{problem}{2} 
Рассмотрим множество векторов $cos^i x * sin^j x$, что $i, j \geqslant 0$ и $i + j \leqslant 3$. Найдите максимальную линейно
независимую систему и выразите все остальные векторы через эту систему.
\end{problem}

\begin{solutions}{2} 
?????????????????????????????????????????????????????????????
\end{solutions}

\begin{problem}{3} 
Покажите, что множество векторов $cos^k x$ линейно независимы $k \geqslant 0$.
\end{problem}

\begin{solutions}{3} 
Рассмотрим промежуток $(0;\frac{\pi}{2})$ и выберем оттуда k различных значений $x_k$. Им поставим в соответствие линейные комбинации  из $y_i=cos x_i$. Составим из сиситемы матрицу

$\begin{pmatrix}
1 & y_1 ^ 1 & ... & y_1 ^k\\
1 & y_2 ^ 1 & ... & y_2 ^k\\
... & ... & ... & ...\\
1 & y_k ^ 1 & ... & y_k ^k
\end{pmatrix}$
- матрица Вандермонта $det \neq 0$ $\Rightarrow$ система ЛНЗ.
\end{solutions}

\begin{problem}{4} 
Выяснить, являются ли вектора линейно зависимыми и если да, то найти хотя бы одну их нетривиальную линейную комбинацию, равную нулю

\begin{center}
$\begin{pmatrix}
1\\
-1\\
2\\
-3
\end{pmatrix}
\begin{pmatrix}
-2\\
1\\
1\\
2
\end{pmatrix}
\begin{pmatrix}
1\\
-3\\
1\\
0
\end{pmatrix}
\begin{pmatrix}
0\\
-3\\
4\\
-1
\end{pmatrix}$
\end{center}
\end{problem}

\begin{solutions}{4} 
Заметим, что если составить систему из линейных комбинаций строк, то соответствующая ей матрица будет иметь ненулевой определитель $\Rightarrow$ вектора ЛЗ. Если подставить в систему два числа, остальные два найдутся, одно из решений:

\begin{center}$\lambda_1=1$ $\lambda_2=1$ $\lambda_3=1$ $\lambda_4=-1$\end{center}
\end{solutions}
\end{document}