\documentclass[discrete.tex]{subfiles}

\begin{document}
  \section{Биномиальные кучи}

  \begin{definition}[биноминальное дерево]
    $B_0$ (б.д. порядка 0) - 1 вершина\\
    $B_1$ - 1 вершина с одним сыном\\
    $B_2$ - $B_1$, к корню которого подвешано ещё одно $B_1$\\
    ...\\
    $B_k$  состоит из двух биномиальных деревьев $B_{k−1}$, связанных вместе таким образом, что корень одного из них является дочерним узлом корня второго дерева
  \end{definition}

  \begin{properties}
    \begin{enumerate}
      \item Биномиальное дерево $B_k$ с $n$ вершинами имеет $2^k$ узлов.
      \item Биномиальное дерево $Bk$ с $n$ вершинами имеет высоту $k$.
      \item Биномиальное дерево $B_k$ с $n$ вершинами имеет высоту $k$.
      \item Биномиальное дерево $B_k$ с $n$ вершинами имеет ровно $k\choose i$ узлов на высоте $i$
      \item Биномиальное дерево $B_k$ с $n$ вершинами имеет корень степени $k$; степень всех остальных вершин меньше степени корня биномиального дерева;
      \item В биномиальном дереве $B_k$ с $n$ вершинами максимальная степень произвольного узла равна $\log n$
    \end{enumerate}
  \end{properties}

  \begin{proof}
    \begin{enumerate}
      \item База $k = 1$ - верно. Пусть для некоторого $k $ условие верно, то докажем, что для $k + 1$ это также верно:

      Так как в дереве порядка $k+1$ вдвое больше узлов, чем в дереве порядка $k$, то дерево порядка $k+1$ имеет $2^k \cdot 2 = 2^{k+1}$ узлов. Переход доказан, то биномиальное дерево $B_k$ с $n$ вершинами имеет $2^k$ узлов.
      \item Докажем по индукции:

      База $k = 1$ - верно. Пусть для некоторого $k $ условие верно, то докажем, что для $k + 1$ это также верно:

      Так как в дереве порядка $k+1$ высота больше на $1$ (так как мы подвешиваем к текущему дереву дерево того же порядка), чем в дереве порядка $k$, то дерево порядка $k+1$ имеет высоту $k + 1$. Переход доказан, то биномиальное дерево $B_k$ с $n$ вершинами имеет высоту $k$.
      \item Докажем по индукции:

      База $k = 1$ - верно. Пусть для некоторого $k $ условие верно, то докажем, что для $k + 1$ это также верно:

      Рассмотрим $i$ уровень дерева $B_{k+1}$. Дерево $B_{k+1}$ было получено подвешиванием одного дерева порядка $k$ к другому. Тогда на $i$ уровне дерева $B_{k+1}$ всего узлов ${k\choose i} + {k\choose {i - 1}}$, так как от подвешенного дерева в дерево порядка $k+1$ нам пришли узлы глубины $i-1$. То для $i$-го уровня дерева $B_{k+1}$ количество узлов ${k\choose i} + {k\choose {i - 1}} = {{k + 1}\choose i}$. Переход доказан, то биномиальное дерево $B_k$ с $n$ вершинами имеет ровно $k\choose i$ узлов на высоте $i$.
      \item Так как в дереве порядка $k+1$ степень корня больше на $1$, чем в дереве порядка $k$, а в дереве нулевого порядка степень корня $0$, то дерево порядка $k$ имеет корень степени $k$. И так как при таком увеличении порядка (при переходе от дерева порядка $k$  к $k+1$) в полученном дереве лишь степень корня возрастает, то доказываемый инвариант, то есть степень корня больше степени остальных вершин, не будет нарушаться.
      \item Докажем это утверждение для корня. Степень остальных вершин меньше по предыдущему свойству. Так как степень корня дерева порядка $k$ равна $k$, а узлов в этом дереве $n = 2^k$, то прологарифмировав обе части получаем, что $k=O(\log n)$, то степень произвольного узла не более $\log n$.
    \end{enumerate}
  \end{proof}

  \begin{definition}[биноминальная куча]
    Представляет собой множество биномиальных деревьев, которые удовлетворяют следующим свойствам:
    \begin{enumerate}
      \item Каждое биномиальное дерево в куче подчиняется свойству неубывающей кучи: ключ узла не меньше ключа его родительского узла (упорядоченное в соответствии со свойством неубывающей кучи дерево),
      \item Для любого неотрицательного целого $k$ найдется не более одного биномиального дерева, чей корень имеет степень $k$.
    \end{enumerate}
  \end{definition}

  \begin{remark}
    Григорьева ещё упорядочивала деревья
  \end{remark}

\end{document}
