\documentclass[discrete.tex]{subfiles}

\begin{document}
  \section{Шифрование с открытым ключом}
  В системах с шифрованием с открытым ключом используется два ключа: один для шифрования, другой для дешифровки

  У каждой стороны есть два ключа: публичный и секретный, который знает только его владелец. Для того чтобы A послал зашифрованное сообщение стороне B необходимо использовать публичный ключ B, потому что лишь B обладает секретным ключом для дешифровки этого сообщения

  Примеры:
  \begin{enumerate}
    \item "Рюкзак"{} Меркла-Хеллмана
    \item El Camol
    \item RSA - один из самых популярных
  \end{enumerate}

  \begin{alg}[RSA]
    n - большое целое число, пр-ние двух простых сомножителей p и q
    \[\varphi = p \cdot q - p - q + 1\]
  \end{alg}

  \begin{Lemma}[1]
    \[\text{a взаимно просто с n} \Ra a^{\varphi} = 1 \mod n\]
  \end{Lemma}

  \begin{proof}
    $\Phi(n)$ - мн-во всех чисел, не превосходящих n и вз. простых с ним
    \[|\Phi(n)| = \varphi\]
    Если $ba \os{\mod n}{=} ca$, то $c-b \devide n$. Таким образом
    \[\prod_{b \in \Phi(n)} b = \prod_{b \in \Phi(n)} (ba) = a^{\varphi} \prod_{b \in \Phi(n)} b \mod n\]
  \end{proof}

  \begin{lemma}[2]
    $\forall$e вз. простого с $\varphi$ $\e! d \in 1:n$, для которого $e d = 1 \mod \varphi$
  \end{lemma}

  \begin{proof}
    Если $\gcd(e,\varphi) = 1$, то $\e d,k$: $de + k\varphi = 1$ (по алгоритму Евклида). Верно $\forall$ вз. простых $e$ и $\varphi$. Однозначность определения d получается док-ом от противного
  \end{proof}

  \begin{alg}[RSA]
    Получатель:
    \begin{enumerate}
      \item Выбирает два больших простых $p$ и $q$
      \item Перемножает $n=pq$
      \[\varphi(n) = (p-1)(q-1)\]
      \item Выбирает число e, в.п. с $\varphi(n)$
      \item Определяет число d
      \item Отправляет отправителю открытые ключи $n$ и $e$
    \end{enumerate}
    Отправитель:
    \begin{enumerate}
      \item Получает открытые ключи $e$ и $n$
      \item Делит такст на кусочки $<n$ (на в.п. с n длины), представляет их в двоичном виде
      \item Каждый кусочек x возводит в степень e
      \[x^e \mod n \ra c\]
      (ex. $e=11001001 \lra x \cdot x^8 \cdot x^{64} \cdot x^{128}$)
      \item Отправляет результат получателю
    \end{enumerate}
    Дешифрование (получатель):
    \begin{enumerate}
      \item Возводит полученное число в степень $d$ по $\mod n$
      \[\Ra c^d = (x^e)^d = (x^{k \varphi(n) + 1}) \mod n\]
      \[\lra x^{k \varphi(n)} x \mod n = x\]
    \end{enumerate}
  \end{alg}

  \begin{Example}
    \[n=1093709 = \os{p}{997} \cdot \os{q}{1097}\]
    \[\varphi = 1091616\]
    \[e = 397 \qq d = 145777\]
    n и e сообщаются отправителям\\

    Кодируем текст:
    \[x^397 = x^{110001101_2} = x^{256} x^{128} x^{8} x^{4} x^1 = y\]

    Отправляем:
    \[\text{Получатель вычисляет $y^d$ по $\mod n = x$}\]
  \end{Example}

\end{document}
