%для красивых символов
\usepackage{upgreek}
\usepackage{dsfont}
\usepackage{amssymb}
\usepackage{tipa}
\usepackage{ wasysym }
\usepackage{gensymb} %градусы
%зачеркивание текста
\usepackage{cancel}
%для цветного текста
\usepackage[usenames]{color}
%для ссылок
\usepackage{xcolor}
\usepackage{hyperref}
% Цвета для гиперссылок
\definecolor{linkcolor}{HTML}{0000ff} % цвет ссылок
\definecolor{urlcolor}{HTML}{0000ff} % цвет гиперссылок
\hypersetup{pdfstartview=FitH,  linkcolor=linkcolor,urlcolor=urlcolor, colorlinks=true}
%Для вставки картинок
\graphicspath{{pictures/}}
\DeclareGraphicsExtensions{.pdf,.png,.jpg}

%команды
\usepackage{amsmath,amsthm,amssymb,amsfonts, enumitem, fancyhdr, color, comment, graphicx, environ}

%настройки текста
\pagestyle{fancy} 
\fancyhead{} 
\fancyfoot{} 
\usepackage[utf8]{inputenc}

%настройки страницы
\makeatletter
\fancyhead[R]{\small Павел Костин}
\fancyhead[L]{\small Матмех СПбГУ}
\fancyhead[C]{\small Практика, мат. анализ, 3 семестр, 2019}
\pagestyle{fancy}
\fancyfoot[R]{\thepage}

%команды
\usepackage{amsmath,amsthm,amssymb,amsfonts, enumitem, fancyhdr, color, comment, graphicx, environ}
%римские цифры
\newcommand{\RNumb}[1]{\uppercase\expandafter{\romannumeral #1\relax}}
%команды для ускорения набора
\newcommand{\R}{\mathds{R}}
\newcommand{\Q}{\mathds{Q}}
\newcommand{\Z}{\mathbb{Z}}
\newcommand{\B}{\mathcal{B}}
\newcommand{\CC}{\mathds{C}}
\newcommand{\N}{\mathds{N}}
\newcommand{\ra}{\Rightarrow}
\newcommand{\la}{\Leftarrow}
\newcommand{\rla}{\Leftrightarrow}
\newcommand{\lra}{\Leftrightarrow}
\newcommand{\e}{\exists}
\newcommand{\E}{\mathcal{E}}
\newcommand{\q}{\quad}
\newcommand{\devides}{\mathop{\raisebox{-2pt}{\vdots}}}

%вёрстка
\newenvironment{solutions}[1][]
{\begin{trivlist}\item{\underline{\bfseries #1}}}{\end{trivlist}\newpage}
\newenvironment{definition}[1][Опр.]
{\begin{trivlist}\item{\underline{\bfseries #1} }}{\end{trivlist}}
\newenvironment{definition2}[2][Опр]
{\begin{trivlist}\item[\underline{{\bfseries #1}} {\bfseries #2.}]}{\end{trivlist}}
\newenvironment{instance}[1][Пример.]
{\begin{trivlist}\item{\underline{\bfseries #1} }}{\end{trivlist}}
\newenvironment{instances}[1][Примеры.]
{\begin{trivlist}\item{\underline{\bfseries #1} }}{\end{trivlist}}
\newenvironment{statement}[1][Утв.]
{\begin{trivlist}\item{\underline{\bfseries #1} }}{\end{trivlist}}
\newenvironment{exercise}[1][Упр.]
{\begin{trivlist}\item{\underline{\bfseries #1} }}{\end{trivlist}}
\newenvironment{lemma}[1][Лемма.]
{\begin{trivlist}\item{\underline{\bfseries #1} }}{\end{trivlist}}
\newenvironment{lemma2}[2][Лемма]
{\begin{trivlist}\item[\underline{{\bfseries #1}} {\bfseries #2.}] \hspace{0pt} \\}{\end{trivlist}}
\newenvironment{comments}[1][Замечание.]
{\begin{trivlist}\item{\underline{\bfseries #1} }}{\end{trivlist}}
\newenvironment{theorem}[1][Теорема.]
{\begin{trivlist}\item{\underline{\bfseries #1} }}{\end{trivlist}}
\newenvironment{theorem2}[2][Теорема]
{\begin{trivlist}\item[\underline{{\bfseries #1}} {\bfseries #2.}] \hspace{0pt} \\}{\end{trivlist}}
\newenvironment{reminder}[2][Напоминание]
{\begin{trivlist}\item[\underline{{\bfseries #1}} {\bfseries #2:}] \hspace{0pt} \\}{\end{trivlist}}
\newenvironment{reshenie}[1][Решение.]
{\begin{trivlist}\item{\bfseries #1} }{\end{trivlist}}
\newenvironment{proofs}[1][Доказательство.]
{\begin{trivlist}\item{\bfseries #1} }{\end{trivlist}}
\newenvironment{proofs2}[2][Доказательство]
{\begin{trivlist}\item[\underline{{\bfseries #1}} {\bfseries #2.}]\hspace{0pt}} {\end{trivlist}}
\newenvironment{proofByDisagreement}[1][Доказательство (от противного). ]
{\begin{trivlist}\item{\bfseries #1}}{\end{trivlist}}
\newenvironment{proofByInduction}[1][Доказательство (по индукции). ]
{\begin{trivlist}\item{\bfseries #1}}{\end{trivlist}}
\newenvironment{properties}[1][Свойство.]
{\begin{trivlist}\item{\underline{\bfseries #1} }}{\end{trivlist}}
\newenvironment{properties2}[2][Свойства]
{\begin{trivlist}\item[\underline{{\bfseries #1}} {\bfseries #2.}]\hspace{0pt}} {\end{trivlist}}
\newenvironment{consequence}[1][Cледствие.]
{\begin{trivlist}\item{\underline{\bfseries #1} }}{\end{trivlist}}
\newenvironment{consequence2}[2][Cледствие]
{\begin{trivlist}\item[\underline{\bfseries #1} {\bfseries #2.}]}{\end{trivlist}}

%фикс отступа
\usepackage{tocloft}
\setlength{\cftbeforetoctitleskip}{1em}