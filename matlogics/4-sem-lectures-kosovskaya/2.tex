\documentclass[12pt, fleqn]{article}

\usepackage{../../../template/template}

 
\begin{document}
 
\begin{lect} {2020-02-18}
    \begin{theorem}
        по всякой функ. формуле можно построить ее равносильную ДНФ.
    \end{theorem}

    \begin{Proof}
        \[\begin{tabular}{c c c|c}
            x_1 & ... & x_n & P\\ \hline
            \text{и} & \vdots & \text{и}\\
                     & &  \\
            \alpha^i_1 & ... & \alpha_n^i & \text{и}\\
                       & \vdots &\\
            \text{л} & & \text{л}
        \end{tabular}\]
        \[k_i = x_i^{\alpha_1^i} \& \ ...\  \& x_n^{\alpha_n^i}  \]
        \[(i = 1, ..., m)\]
        \[k_i = \text{ и } \rla x_j = \alpha_j^i \qq (j = 1, ..., n)\]
        \[\text{т.к. } \alpha_j^i^{\alpha_j^i}  = \text{ и }\]
        \[\alpha_j^i^\beta = \text{ л } \qq (\beta \neq \alpha_j^i)\]
        \[P \rla k_1 \vee ... \vee k_m\]
    \end{Proof}

    \begin{remark}
        ДНФ, в которой каждая элем. конъюк. содержит все переменные называется совершенной ДНФ
    \end{remark}

    \begin{Definition}
        \[\alpha_i - const, \qq x_i \text{ - разл перем.}\]
        Элементарной дизъюнкцикй называется ф-ла вида 
        \[D = x_1^{\alpha_1} \vee ... \vee x_m^{\alpha_m}  \]
    \end{Definition}

    \begin{definition}
        Ф-ла вида 
        \[(D_1) \ \& \ ... \ \& \ (D_k)\]
        наз. конъюктивной нормальной формулой (КНФ)
    \end{definition}

    \begin{theorem} [о КНФ]
        по всякой позициональной формуле не явл. т-логией можно построить равносильную ей КНФ
    \end{theorem}

    \begin{proof}
        $\forall \ P$ - не тавт. $\RA$ $\rceil P - $ не противоречит \\
        по т. о ДНФ \qq $\rceil P \rla k_1 \vee ... \vee k_m$
        \[P \rla \rceil \rceil P \rla \rceil (k_1 \vee ... \vee k_m)\]
        внеся $\rceil$ до переменных получ. КНФ
    \end{proof}

    \begin{Definition}
        \[\letus k_1, ..., k_m \text{ - элем-ные конъюкции без отрицаний, } c \in  \{\text{и, л}\}\]
        выражение вида\\
        \[k_1 \oplus ... \oplus k_m \oplus c\]
        называется полиномом Жегалкина
    \end{Definition}

    \begin{theorem}
        по всякой позиоциональной формуле можно построить равносильный ей полином Жегалкина\\
        Если $P$ - противореч., то полином Жег.: л\\
        В противном случае строим СОВЕРШЕННУЮ ДНФ для $P $\\
        \[k_1 \vee ... \vee k_m\]
        В СОВЕРШЕННОЙ ДНФ \ $\vee$ можно заменить на $\oplus$\\
        т.к. \q $\rceil x \rla x \oplus 1$, то раскрыв скобки и приведя подобные члены
        \[A \oplus A = 0\]
        получим полином Жегалкина
    \end{theorem}

    \begin{remark}
        для всякой позициональной формулы существует ровно один полином Жегалкина с точностью до перестановки 
        слагаемых
    \end{remark}

    \subsection{Исчисления высказываний}

    Первые исчисления основывались на 
    \[\begin{tabular}{c  c}
        \text{посылки } \ A & A \to B \\ \hline
        \text{заключение }\ \qq B
    \end{tabular}\]
    \qq\qq\qq(modus ponens)
    \[\begin{tabular} {c  c}
        A & \rceil A \vee B\\ \hline
        \qq B
    \end{tabular}\]
    Для того, чтобы задать исчисление достаточно задать
    \begin{itemize}
        \item Алфавит (конечный набор символов)
        \item формулы (эффективно проверяемое мн-во слов в алфавите)
        \item Аксиомы (эффективно проверяемое мн-во формул)
        \item Правила вывода 
            \[\begin{tabular} {c c c}
                P_1 & ... & P_n\\\hline
                    &Q
            \end{tabular}, \text{ где } P_1, ..., P_n, Q - \text{ ф-лы}\]
        \[P_1, ..., P_n \text{ - посылки правила } \qq Q \text{ - заключение} \]
    \end{itemize}

    \begin{definition}
        Ф-ла $P$ наз. выводимой ($\vDash_I$ P) в исчислении I, если существует конечная послед. ф-л 
        \[A_1, ..., A_n, P, \text{ т. ч. каждая формула этой послед либо является аксиомой,}\]
        либо может быть получена из предыдущих по одному из правил вывода.
        При этом сама эта послед называется выводом в исчисл. I, а формулу $P$ называют теоремой исчисления.
    \end{definition}

    \subsection{Секвенциальное исчисление высказываний (seqnent)}

    \begin{enumerate}
        \item Алфавит: \q алфавит для записи проп. переменн, $\rceil, \&, $ (, ), $\vdash$
        \item Ф-лы: секвенции, т.е $P_1 ... P_n \ \vdash \ Q_1 ... Q_m$, где 
            \[P_i, Q_j \text{ - проп. ф-лы со связками } \rceil, \&\]
        (Как читать) \q Из того, что верны все формулы $P_1, ..., P_n$ следует, что верна хоть одна из формул 
        $Q_1, ..., Q_m$\\    \\
        Формульный образ секвенции: 
        \[\Phi(P_1 ... P_n \vdash Q_1 ... Q_m) \ \os{\text{def}}{=}\ P_1 \  \&\ ...\ \&\ P_n \to 
        (Q_1 \vee ... \vee Q_m)\]
        \item Аксиомы: любая секв. вида 
            \[\Gamma_1\ A\ \Gamma_2 \vdash \Delta_1 \ A \ \Delta_2, \text{ где } \]
            \[\Gamma_1, \Gamma_2, \Delta_1, \Delta_2 \text{ - списки ф-л,  } A \text{ - ф-ла ИВ}\]
            В секв. $\Gamma \vdash \Delta$ список $\Gamma$ наз. антецедент\\
            список $\Delta$ наз. супцедент (консеквент.)
    \end{enumerate}
    
    \subsection{Правила вывода}

    Если *-лы связка, то \\
    ($\vdash *$) правило введения * в супцедент\\
    ($*\vdash$) --- || --- || --- || - антецедент
    \[(\vdash \rceil) \q \frac{\Gamma_1 A \Gamma_2 \vdash \Delta_1 \Delta_2}
    {\Gamma_1 \Gamma_2 \ \vdash \Delta_1 \rceil A \Delta_2}\]
    
    \[(\rceil \vdash) \q \frac{\Gamma_1 \Gamma_2 \vdash \Delta_1 A \Delta_2}
    {\Gamma_1 \rceil A \Gamma_2 \vdash \Delta_1 \Delta_2}\]

    \[(\vdash\&)\q \frac{\Gamma \vdash \Delta_1 B \Delta_2}{\Gamma \vdash \Delta_1 A\ \&\ B \Delta_2}\]

    \[(\& \vdash)\q \frac{\Gamma_1 A \ B\Gamma_2\ \vdash\ \Delta}{\Gamma_1 A\ \& \ B \Gamma_2 \vdash \Delta}\]

    \begin{definition}
        Правило вывода называется допустимым в исчислении, если по всякому выводу, содержащему применение 
        этого правила, можно построить вывод с той же последней формулой (секвенцией), не содержащий применение 
        этого правила.
    \end{definition}

    \[(\vdash \vee) \frac{\Gamma \vdash \Delta_1 A B \Delta_2}{\Gamma \vdash \Delta_1 A \vee B \Delta_2}\]

    \[(\vee \vdash) \frac{\Gamma_1 A \Gamma_2 \vdash \Delta \Gamma_1 B \Gamma_2 \vdash \Delta
    }{\Gamma_1 A \vee B \Gamma_2 \vdash \Delta}\]

    \[(\vdash \to ) \frac{\Gamma_1 A \Gamma_2 \vdash \Delta_1 B \Delta_2}{\Gamma_1 \Gamma_2 \vdash 
    \Delta_1 A \to B \Delta_2}\]

    \[(\to \vdash) \frac{\Gamma_1 \Gamma_2 \vdash \Delta_1 A \Delta_2 \Gamma_1 B \Gamma_2 
    \vdash \Delta_1 \Delta_2}{\Gamma_1 A \to B \Gamma_2 \vdash \Delta_1 \Delta_2}\]
    
    \[(\vdash \leftrightarrow) \frac{\Gamma_1 B \Gamma_2 \vdash \Delta_1 A \Delta_2}
    {\Gamma_1 \Gamma_2 \vdash \Delta_1 A \leftrightarrow B \Delta_2}\]

    \[(\leftrightarrow \vdash) \frac{\Gamma_1 AB \Gamma_2 \vdash \Delta_1 \Delta_2}{\Gamma_1 A \leftrightarrow 
    B \Gamma_2 \vdash \Delta_1 \Delta_2}\]

    \begin{theorem}[правила] 
        $(\vdash \vee), (\vee \vdash), (\vdash \to ), (\to \vdash), (\vdash 
        \leftrightarrow), (\leftrightarrow \vdash), (\vdash \oplus), (\oplus \vdash), (\vdash |), 
        (| \vdash), (\vdash \downarrow), (\downarrow \vdash)$
        допустимы в секв. исч-нии высказываний
    \end{theorem}

    \begin{Proof}
        \[(\to \vdash)\]
        \[A \to B \rla \rceil A \vee B \rla \rceil (A \& \rceil B)\]
    \end{Proof}
\end{lect}

\end{document}
