\documentclass[main]{subfiles}

\begin{document}
\begin{lect} {2020-03-03}
    \subsection{Предметные переменные}
    \subsection{Предметные константы}
    \subsection{Функциональные символы}
    \subsection{Предметные символы}

    \begin{definition}[терм]
        \begin{enumerate}
            \item Предметные переменные и предметные const являются термами
            \item Если $t_1,...,t_n$ --- термы, $t$ --- n-местный ф-ый символ, то $f(t_1,...,t_n)$ --- терм
        \end{enumerate}
    \end{definition}

    \begin{definition}[атомарная ф-ла]
        Если $t_1,...,t_n$ --- термы, $P$ --- n-местный предикатный символ, то $P(t_1,...,t_n)$ --- атомарная ф-ла
    \end{definition}

    \begin{definition}[предикатная ф-ла]
        \begin{enumerate}
            \item Атомарная ф-ла является предикатной ф-лой
            \item Если $F$ --- предикатная ф-ла, то $\rceil F$ --- пред. ф-ла
            \item Если $F, R$ --- предикатная ф-ла, * --- бинарная логическая связка, то $(F*R)$ --- пред. ф-лац
            \item Если F --- пред. ф-ла, x --- предметн. перем., то $\forall x F$ и $\e x F$ --- предметн. ф-лы
        \end{enumerate}

        При атом. $\forall x$ и $\e x$ наз. {\bf кванторными комплексами}

        Ф-ла F называется {\bf областью действия} квантора
    \end{definition}

    \begin{definition}
        Вхождение предметной переменной в ф-лу называется {\bf связанным}, если оно находится в квантовом комплексе или в области действия квантора по этой переменной
    \end{definition}

    \begin{definition}
        Вхождение предметной переменной называется {\bf свободным}, если оно не является связанным
    \end{definition}

    \begin{Example}
        \[\forall x (P(x,y) \ra \e z Q(x,y,z)) \vee \forall y\ Q(x,y,z)\]
    \end{Example}

    \begin{definition}
        Ф-ла, у которой ни одна переменная не имеет как свободных так и связанных вхождений называется {\bf чистой}
    \end{definition}

    \begin{definition}
        Терм $T$ называется {\bf свободным для подстановки} в ф-лу $F$ вместо свободных вхождений предметной переменной $x$, если он не содержит переменных, в области действия кванторов по которым имеется своб. вхождение переменной $x$

        Будем писать терм св. в $F$ вм. $x$
    \end{definition}

    \begin{tabular}{c|c|c|c|c}
        пер./терм & x & y & z & u\\
        \hline
        a &  &  &  & \\
        f(x) &  & - &  & \\
        f(y) & - &  & - & \\
        f(z) &  & - &  & \\
        f(u) &  &  &  & \\
        g(x,y) & - & - & - & \\
        g(x,z) &  & - &  & \\
        g(y,z) & - & - & - & \\
        g(x,y,z) & - & - & - &
    \end{tabular}

    Альтернативное определение: если эта подстановка не увеличивает кол-во связанных вхождений других переменных

    \begin{consequence}
        \begin{enumerate}
            \item const св. для подстановки в любую ф-лу вместо свободного вхождения любой переменной
            \item Всякая переменная свободна для подстановки в любую ф-лу вместо свободных вхождений себя
            \item Если терм $t$ не содержит переменных не имеющих связанных вхождений в ф-лу, то он свободный для подстановки в неё вместо свободных вхождений любой переменной
            \item Если ф-ла не сод. своб. вхождений предм. перем. $x$, то всякий терм свободный для подстановки в неё вместо свободных вхождений $x$
        \end{enumerate}
    \end{consequence}

    \subsection{Секвенциальные исчисления предикатов}

    Алф: алфавит для предм. перем, алфавит для записи ф-ных символов, алфавит для предм. символов, , , (,), $\&, \rceil, \forall, \e,$ палкавертикальнаяизнеёпалкагоризонтальная

    Ф-лы: Секвенции из предикатных ф-л

    Акс: $\Gamma_1\ A\ \Gamma_2\ палкавертикальнаяизнеёпалкагоризонтальная\ \triangle_1\ A\ \triangle_2$

    Правила вывода: $(\rceil палкавертикальнаяизнеёпалкагоризонтальная),\ (палкавертикальнаяизнеёпалкагоризонтальная \rceil),\ (\& палкавертикальнаяизнеёпалкагоризонтальная),\ (палкавертикальнаяизнеёпалкагоризонтальная, \&),$
    $(палкавертикальнаяизнеёпалкагоризонтальная \forall)$ \begin{tabular}{c}
        $\Gamma \q палкавертикальнаяизнеёпалкагоризонтальная \q \triangle_1 \q [A]_y^x \q \triangle_2$
        \hline
        $\Gamma \q палкавертикальнаяизнеёпалкагоризонтальная \q \triangle_1 \q \forall x A \q \triangle_2$
    \end{tabular}
    $(\e палкавертикальнаяизнеёпалкагоризонтальная)$ \begin{tabular}{c}
        $\Gamma_1 \q [A]_y^x \q \Gamma_2 \q палкавертикальнаяизнеёпалкагоризонтальная \q \triangle$
        \hline
        $\Gamma_1 \q \e x A \q \Gamma_2 \q палкавертикальнаяизнеёпалкагоризонтальная \q \triangle$
    \end{tabular}
    Если
    \begin{enumerate}
        \item Переменная $y$ не входит свободно в заключение правила
        \item Переменная $y$ своб. для подстановки в ф-лу A вм. своб. вх-ий перем. $x$
    \end{enumerate}

    $[A]_t^x$ --- рез-тат прдстановки терма $t$ вместо всех своб. вхожд. предм. перем. $x$

    $[A(x)]^x_t двепалкигоризонтальныеамеждунимикружочек A(t)$, где $x$ -- единств. воб. перем. ф-лы

    $[A(\us{i, n \geq 1}{x_1,...,x_i,...,x_n})]_t^{x_i} двепалкигоризонтальныеамеждунимикружочек A(x_1,...,t,...,x_n)$

    Если терм $t$ своб д. подст. в ф-лу $A$ вм. своб. вх-ий предм. перем. $x$

    \subsection{Интерпретации}
    Чтобы задать интерпретацию достаточно
    \begin{enumerate}
        \item Задать область интерпретации $D$, т.е. мн-во const, которые могут быть взяты в качестве значений для переменных
        \item Каждому n-местному функциональному символу поставив в соответствие конкретную ф-ию $f: D^n \ra D$
        \item Кажому n-местному предикатному символу поставив в соответствие конкретное отношение $P \subset D^n$
    \end{enumerate}

    \begin{Example}
        \[\e x P(x, f(y)\]
        \[I_1: D = \N\]
        \[\q f(x) = x + 1\]
        \[\q P(x,y) \lra x + z < y\]
        \[\e x (x + z < y + 1)\]
        \[\e x (x + 1 < y)$ --- А при $y = 0,\ y = 1\]
        \[I_2: D = \N\]
        \[\q f(x) = x\]
        \[\q P(x,y) \lra x = y\]
        \[\q \e x (x = y) \text{ И при всех }y\]
    \end{Example}

    \begin{definition}
        Ф-ла называется {\bf истинной (ложной) в заданной интерпретации}, если она истинна (ложна) на всех наборах значений свободных переменных из области интерпретации
    \end{definition}

    \begin{definition}
        Ф-ла называется {\bf выполнимой в заданной интерпертации}, если она истинна хоть на одном наборе значений свободных переменных для области интерпретации
    \end{definition}

    \begin{definition}
        Ф-ла называется {\bf общезначимой} (противоречием), если она истинна (ложна) в любой интерпретации на любом наборе значений свободных переменных
    \end{definition}

    \begin{definition}
        Ф-ла называется {\bf выполнимой}, если она истинна хоть в одной интерпретации, хоть на одном наборе значений свободных переменных
    \end{definition}

    \begin{theorem}
        Чистая секвенция выводима в секвенциальном исчислении предикатов тогда и только тогда, когда её формульный образ
    \end{theorem}

    \begin{consequence}
        Секвенциальное инсчисление предикатов, в котором исп. только чистые формулы полно и непротиворечиво
    \end{consequence}
\end{lect}
\end{document}
